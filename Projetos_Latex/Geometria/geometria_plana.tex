% !TEX program = lualatex
\documentclass[a4paper,12pt]{article}

% ===== Idioma / Layout =====
\usepackage[portuguese]{babel}
\usepackage[margin=2.5cm]{geometry}
\usepackage{float}

% ===== Matemática / Teoremas =====
\usepackage{amsmath,amsthm,amssymb}
\newtheorem{exercicio}{Exercício}[subsection]

% ===== TikZ =====
\usepackage{xcolor}
\usepackage{tikz}
\usetikzlibrary{calc,angles,quotes,intersections}

% --- Cores ---
\definecolor{azulmarinho}{RGB}{0,0,139}
\definecolor{angVerde}{RGB}{0,140,0}
\definecolor{angLaranja}{RGB}{220,120,0}
\definecolor{angRoxo}{RGB}{120,0,180}

% --- Macro setor (tem ; interno, não esquece) ---
% uso: \Setor[<estilo>]{(x,y)}{raio}{ang_ini}{ang_fim}
\newcommand{\Setor}[5][]{%
  \filldraw[#1] #2 -- ++(#4:#3) arc (#4:#5:#3) -- cycle;
}

\begin{document}

\begin{exercicio}
Exercício de geometria básica.

Considere a figura abaixo.

\begin{figure}[H]
  \centering
  \begin{tikzpicture}
    % Quadrado
    \draw[azulmarinho, ultra thick, name path=quad] (0,0) rectangle (6,6);

    % Segmentos
    \draw[blue, thick] (0,0) -- (-3,0);
    \draw[blue, thick] (6,6) -- (10,6);
    \draw[red, thick] (-3,0) -- (10,6);
    \draw[red, dashed, thick] (0,1.4) -- (6,1.4);

    % Ângulos como setores preenchidos
    \Setor[fill=red!50, draw=red!70!black]{(-3,0)}{1}{0}{25}
    \Setor[fill=red!50, draw=red!70!black]{(0,1.4)}{1}{0}{25}
    \Setor[fill=red!50, draw=red!70!black]{(10,6)}{1}{-180}{-155}

    % Nós / rótulos
    \node[anchor=north east] at(-1.5,1.1){1};
    \node[anchor=north east] at (3,3.2){3};
    \node[anchor=north east] at (7.5,5.3){2};
    \node[anchor=north east] at (3.5,1.9){\(6x\)};
    \node[anchor=north east] at (0,0) {A};
    \node[anchor=north east] at (0,6.5){B};
    \node[anchor=north east] at (6.5,6.5){C};
    \node[anchor=north east] at (6.5,0){D};
    \node[anchor=north east] at (0.5,2.2){E};
    \node[anchor=north east] at (6.5,2){F};
    \node[anchor=north east, rotate=90] at (5.4,1){\(x\)};
    \node[anchor=north east, rotate=90] at (5.4,3.2){\(3x\)};
    \node[anchor=north east, rotate=90] at (5.4,5.2){\(2x\)};
  \end{tikzpicture}
\end{figure}

Se considerarmos a proporção entre o triângulo de hipotenusa 1 e o triângulo interno ao quadrado, temos:
\[
  \frac{1}{x}=\frac{3}{3x}
\]

Da mesma forma, considerando a proporção dos triângulos externos, temos:
\[
  \frac{1}{x}=\frac{2}{2x}
\]

Então, o lado \(L\) do quadrado é igual a \(x+2x+3x=6x\).
Por Pitágoras, temos:
\begin{align*}
  9 &= 36x^2+9x^2\\
  9 &= 45x^2\\
  x^2 &= \frac{1}{5}\therefore\, x=\sqrt{\frac{1}{5}}\\
  L &= 6\sqrt{\frac{1}{5}}\\
  A &= \frac{36}{5}\approx \boxed{7{,}2\text{ u.a.}}
\end{align*}

\end{exercicio}
\end{document}
