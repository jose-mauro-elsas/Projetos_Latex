A revisão bibliográfica abordará os fundamentos conceituais e técnicos do
descomissionamento offshore, incluindo práticas consolidadas em bacias maduras,
como o Mar do Norte. Serão discutidos os princípios da engenharia offshore
relacionados ao ciclo de vida das plataformas, bem como aspectos da oceanografia
física relevantes para operações marítimas, tais como regimes de correntes, ondas,
processos sedimentares e sua influência sobre a estabilidade e o impacto ambiental
das estruturas.

Adicionalmente, serão analisados estudos sobre impactos ambientais do
descomissionamento, recuperação de áreas marinhas e o papel das estruturas offshore
como habitats artificiais, com foco nas lacunas existentes no contexto brasileiro.