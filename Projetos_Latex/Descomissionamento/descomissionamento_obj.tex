\subsection{Objetivo Geral}

Analisar o descomissionamento de plataformas offshore no Brasil sob uma perspectiva
integrada, considerando aspectos técnicos, ambientais e oceanográficos associados ao
fim de vida das instalações.

Vale lembrar que a distância entre as locações e os estaleiros de desmantelamento não
é exatamente um problema, uma vez que  outros tipos de embarcações, tais  como graneleiros e 
porta-contêineres, são em sua maioria descomissionados fora do Brasil (em particular na 
Turquia) e que muitas UEPs são produzidas em locais diferentes nas suas diversas fases.

Há exemplos de cascos construídos na China, com equipamentos montados na Holanda para serem
instalados na costa de Sergipe. Portanto, em um horizonte de 30 anos, as plataformas que forem
descomissionadas na nova fronteira da Margem Equatorial poderão ser desmanteladas nos estaleiros
do Rio de Janeiro. Essa avaliação também é objetivo desse trabalho.

\subsection{Objetivos Específicos}

\begin{itemize}
    \item Caracterizar o parque offshore brasileiro em termos de tipo de instalação,
          idade, bacia sedimentar e lâmina d’água; identificar unidades potencialmente próximas 
          do fim de sua vida útil nominal conforme normas como ANP, API e DNV.
    \item Avaliar a relação entre ambiente oceanográfico, profundidade de instalação e
          complexidade das operações de descomissionamento; analisar o arcabouço regulatório 
          nacional aplicável ao descomissionamento offshore; discutir os principais impactos 
          ambientais associados às diferentes estratégias de descomissionamento.
\end{itemize}

