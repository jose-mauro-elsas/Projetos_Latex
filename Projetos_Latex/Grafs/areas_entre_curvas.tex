

\begin{figure}[H]
  \centering
  \begin{tikzpicture}
    \begin{axis}[
      xmin=0.2, xmax=1.3,
      ymin=-0.5, ymax=3.25,
      axis x line=middle,
      axis y line=middle,
      axis line style={-},
      tick align=outside,
      minor tick num=1,
      xlabel={$x$},
      ylabel={$y$},
      title={$f(x)=|x|$}
    ]
      % Curva 1 (seno), com "name path" no próprio plot:
      \addplot[domain=0.4:1.2, samples=200, name path=A]
        {sin(deg(8*x))+2}
        node[pos=0.5, above, xshift=-48pt] {$y=\sin(8x)+2$};

      % Curva 2 (a reta), também como plot (em vez de \draw):
      \addplot[domain=0.4:1.2, samples=200, name path=B]
        {0.625*x+0.25} node[pos=.6, above, yshift=-24pt] {$y=0.625x+0.25$};
       

      % Sombreado ENTRE as duas curvas:
      \addplot[gray!20, draw=none] fill between[of=A and B];

      % Suas marcações verticais (podem ficar como estão):
      \draw[line width=0.6pt] (0.4,0) -- (0.4,1.95);
      \draw[line width=0.6pt] (1.2,0) -- (1.2,1.85);

      % Se você quiser manter a reta desenhada também, pode (opcional),
      % mas ela já está representada pelo plot acima.
      %\draw[line width=0.6pt] (0.4,0.5) -- (1.2,1);

    \end{axis}
  \end{tikzpicture}
  \caption{Área entre curvas.}
\end{figure}

\begin{figure}[H]
  \centering
  \begin{tikzpicture}
    \begin{axis}[
      xmin=0, xmax=2,
      ymin=-3, ymax=4,
      axis x line=middle,
      axis y line=middle,
      axis line style={-},
      tick align=outside,
      minor tick num=1,
      xlabel={$x$},
      ylabel={$y$},
      title={$f(x)$}
    ]
      % Curva 1 (seno), com "name path" no próprio plot:
      \addplot[domain=0:1.5, samples=200, name path=A]
        {cos(deg(2*x))+2}
        node[pos=0.5, above, xshift=60pt] {$y=cos(2x)+2$};

      % Curva 2 (a reta), também como plot (em vez de \draw):
      \addplot[domain=0:1.5, samples=200, name path=B]
        {-x^2-.8} 
        node[pos=.6, above, yshift=-48pt] {$y=x^2-2$};
       

     % Sombreado ENTRE as duas curvas:
    \addplot[gray!20, draw=none, fill opacity=0.35] fill between[of=A and B];



      % Suas marcações verticais (podem ficar como estão):
      %\draw[line width=0.6pt] (0.4,0) -- (0.4,1.95);
      %\draw[line width=0.6pt] (1.2,0) -- (1.2,1.85);

      % Se você quiser manter a reta desenhada também, pode (opcional),
      % mas ela já está representada pelo plot acima.
      %\draw[line width=0.6pt] (0.4,0.5) -- (1.2,1);

    \end{axis}
  \end{tikzpicture}
  \caption{Área entre curvas.}
\end{figure}

\begin{figure}[H]
  \centering
  \begin{tikzpicture}
    \begin{axis}[
      xmin=0.2, xmax=1.3,
      ymin=-0.5, ymax=3.25,
      axis x line=middle,
      axis y line=middle,
      axis line style={-},
      tick align=outside,
      minor tick num=1,
      xlabel={$x$},
      ylabel={$y$},
      title={$f(x)=|x|$}
    ]
      % Curva 1 (seno), com "name path" no próprio plot:
      \addplot[domain=0.4:1.2, samples=200, name path=A]
        {sin(deg(8*x))+2}
        node[pos=0.5, above, xshift=-48pt] {$y=\sin(8x)+2$};

      % Curva 2 (a reta), também como plot (em vez de \draw):
      \addplot[domain=0.4:1.2, samples=200, name path=B]
        {0.625*x+0.25} node[pos=.6, above, yshift=-24pt] {$y=0.625x+0.25$};
       

      % Sombreado ENTRE as duas curvas:
      \addplot[gray!20, draw=none] fill between[of=A and B];

      % Suas marcações verticais (podem ficar como estão):
      \draw[line width=0.6pt] (0.4,0) -- (0.4,1.95);
      \draw[line width=0.6pt] (1.2,0) -- (1.2,1.85);

      % Se você quiser manter a reta desenhada também, pode (opcional),
      % mas ela já está representada pelo plot acima.
      %\draw[line width=0.6pt] (0.4,0.5) -- (1.2,1);

    \end{axis}
  \end{tikzpicture}
  \caption{Área entre curvas.}
\end{figure}

\begin{figure}[H]
  \centering
  \begin{tikzpicture}
    \begin{axis}[
      xmin=0, xmax=2,
      ymin=-3, ymax=4,
      axis x line=middle,
      axis y line=middle,
      axis line style={-},
      tick align=outside,
      minor tick num=1,
      xlabel={$x$},
      ylabel={$y$},
      title={$f(x)$}
    ]
      % Curva 1 (seno), com "name path" no próprio plot:
      \addplot[domain=0:1.1, samples=200, name path=A]
        {cos(deg(2*x))+1}
        node[pos=0.5, above, xshift=60pt] {$y=cos(2x)+2$};

      % Curva 2 (a reta), também como plot (em vez de \draw):
      \addplot[domain=0:1.1, samples=200, name path=B]
        {x^2-.8} 
        node[pos=.6, above, yshift=-48pt] {$y=x^2-2$};
       

     % Sombreado ENTRE as duas curvas:
    \addplot[gray!20, draw=none, fill opacity=0.45] fill between[of=A and B];



      % Suas marcações verticais (podem ficar como estão):
      %\draw[line width=0.6pt] (0.4,0) -- (0.4,1.95);
      %\draw[line width=0.6pt] (1.2,0) -- (1.2,1.85);

      % Se você quiser manter a reta desenhada também, pode (opcional),
      % mas ela já está representada pelo plot acima.
      %\draw[line width=0.6pt] (0.4,0.5) -- (1.2,1);

    \end{axis}
  \end{tikzpicture}
  \caption{Área entre curvas.}
\end{figure}
