\begin{figure}[H]
\centering
\begin{tikzpicture}
\begin{axis}[
    xmin=0.5, xmax=4,
    ymin=0.5, ymax=3,
    axis x line=middle,
    axis y line=middle,
    samples=200,
    xlabel={$x$},
    ylabel={$y$}
]
    % 1. Definimos a curva e damos um nome a ela
    \addplot[blue, thick, domain=1:3.75, name path=curva] {0.5*(x-2)^2+1};
    
    % 2. reta 
    \addplot[red, thick, domain=1:3.75, name path=reta]{.5*x+0.5};

    % 3. Preenchimento
    %\addplot[blue!20] fill between[of=curva and reta];
\end{axis}
\end{tikzpicture}
\caption{Função positiva em todo o intervalo.}
\end{figure}

\begin{figure}[H]
\centering
\begin{tikzpicture}
\begin{axis}[
    xmin=0.5, xmax=4,
    ymin=-3, ymax=.5,
    axis x line=middle,
    axis y line=middle,
    samples=200,
    xlabel={$x$},
    ylabel={$y$}
]
    % 1. Parábola
    \addplot[blue, thick, domain=1:3.75, name path=curva] {-(0.5*(x-2)^2+1)};
    
    % 2. Reta 
    \addplot[red, thick, domain=1:3.75, name path=reta]{-(.5*x+0.5)};

    % 3. Preenchimento hachurado apenas entre as intersecções
    \addplot [
        % Define o padrão de hachurado
        pattern=north east lines, 
        pattern color=gray,
    ] fill between [
        of=curva and reta,
        split, % Divide a área nos pontos de cruzamento
        % Estiliza apenas o segmento desejado (o do meio)
        every segment no 0/.style={transparent},
        every segment no 1/.style={pattern=north east lines},
        every segment no 2/.style={transparent}
    ];
\end{axis}
\end{tikzpicture}
\caption{Área hachurada entre as intersecções.}
\end{figure}