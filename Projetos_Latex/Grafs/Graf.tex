\documentclass[a4paper,12pt]{article}

% ===== Básico =====
\usepackage[T1]{fontenc}
%\usepackage[utf8]{inputenc}
\usepackage[brazil]{babel}
\usepackage[margin=3cm,tmargin=3cm,rmargin=2cm,bmargin=2cm]{geometry}
\usepackage{indentfirst}
\usepackage{setspace}
\usepackage{enumitem}
\usepackage{booktabs}

% ===== Matemática / fontes =====
\usepackage{lmodern}
\usepackage{amsmath,amsthm,amsfonts,amssymb,dsfont,mathtools}
\usepackage{cancel}

% ===== Figuras / Gráficos =====
\usepackage{graphicx}
\usepackage{float}
\usepackage{tabularx}
% PGFPlots (já carrega o TikZ)
\usepackage{pgfplots}
\pgfplotsset{compat=1.18, width=11cm}
\usepgfplotslibrary{fillbetween}
\usetikzlibrary{arrows.meta,intersections}
\usepackage{float}
\usepackage{tabularx}
\usetikzlibrary{calc}

\usepackage{pgfplots}          % carrega TikZ indiretamente
\pgfplotsset{compat=1.18, width=11cm}

\usetikzlibrary{patterns}      % agora OK
\usepgfplotslibrary{fillbetween}

\renewcommand{\CancelColor}{\color{red}}
\newcommand{\dif}{\,\mathrm{d}}
\newcommand{\sen}{\operatorname{sen}}



\usetikzlibrary{patterns}      % agora OK
\usepgfplotslibrary{fillbetween}


% ... (seu preâmbulo permanece igual) ...

\begin{document}
\begin{titlepage}
  \centering
  \vspace*{\fill}
  {\Huge \textbf{Gráficos de funções e desenhos}}
  \vfill
  \today
\end{titlepage}

\clearpage

\section{Gráficos de funções elementares}

% ... (outras subseções) ...

\subsection{Área entre as curvas I}

% Mova o código da figura para ANTES do \end{document}
% Ou garanta que este código esteja dentro do arquivo area_entre_funcoes.tex

\begin{figure}[H]
\centering
\begin{tikzpicture}
\begin{axis}[
    xmin=0.5, xmax=4,
    ymin=0.5, ymax=3,
    axis x line=middle,
    axis y line=middle,
    samples=200,
    xlabel={$x$},
    ylabel={$y$}
]
    % 1. Definimos a curva e damos um nome a ela
    \addplot[blue, thick, domain=1:3.75, name path=curva] {0.5*(x-2)^2+1};
    
    % 2. reta 
    \addplot[red, thick, domain=1:3.75, name path=reta]{.5*x+0.5};

    % 3. Preenchimento
    %\addplot[blue!20] fill between[of=curva and reta];
\end{axis}
\end{tikzpicture}
\caption{Função positiva em todo o intervalo.}
\end{figure}

\begin{figure}[H]
\centering
\begin{tikzpicture}
\begin{axis}[
    xmin=0.5, xmax=4,
    ymin=0.5, ymax=3,
    axis x line=middle,
    axis y line=middle,
    samples=200,
    xlabel={$x$},
    ylabel={$y$}
]
    % 1. Parábola
    \addplot[blue, thick, domain=1:3.75, name path=curva] {0.5*(x-2)^2+1};
    
    % 2. Reta 
    \addplot[red, thick, domain=1:3.75, name path=reta]{.5*x+0.5};

    % 3. Preenchimento hachurado apenas entre as intersecções
    \addplot [
        % Define o padrão de hachurado
        pattern=north east lines, 
        pattern color=gray,
    ] fill between [
        of=curva and reta,
        split, % Divide a área nos pontos de cruzamento
        % Estiliza apenas o segmento desejado (o do meio)
        every segment no 0/.style={transparent},
        every segment no 1/.style={pattern=north east lines},
        every segment no 2/.style={transparent}
    ];
\end{axis}
\end{tikzpicture}
\caption{Área hachurada entre as intersecções.}
\end{figure}

\subsection{Completar Quadrados.}
\begin{figure}[hbt]
  \centering
  \begin{tikzpicture}
    \begin{axis}[
      xmin=0, xmax=6,
      ymin=0, ymax=6,
      axis x line=middle,
      axis y line=middle,
      axis line style={-},
      tick align=outside,
      grid=none,
      minor tick num=1,
      xlabel={$x$},
      ylabel={$x$},
      xtick=\empty,
      ytick=\empty,
    ]
    \draw [red, thick,] (1,1) -- (1,4);
    \draw [red, thick,] (1,4) -- (4,4);
    \draw [red, thick,] (4,4) -- (4,1);
    \draw [red, thick,] (4,1) -- (1,1);
    \draw [blue, thick,] (1,4) -- (1,5);
    \draw [blue, thick,] (1,5) -- (4,5);
    \draw [blue, thick,] (4,5) -- (4,4);
    \draw [black, thick,] (4,1) -- (5,1);
    \draw [black, thick,] (5,1) -- (5,4);
    \draw [black, thick,] (5,4) -- (4,4);
    \draw [black, dashed, thick,] (5,4) -- (5,5);
    \draw [black, dashed, thick,] (5,5) -- (4,5);

    \fill[pattern=north east lines]
    (1,1) -- (1,4) -- (1,4) -- (4,4) -- (4,4) -- (4,1) -- cycle;
    \node at (2.6,2.6) {\Large{$x^2$}};
    \node at (2.6,5.2) {\Large{$x$}};
    \node at (5.2,2.6) {\Large{$x$}};
    \node at (4.5,0.5) {\Large{$\frac{b}{2}$}};
    \node at (0.5,4.5) {\Large{$\frac{b}{2}$}};
    \end{axis}
  \end{tikzpicture}
  \caption{Completar Quadrados.}
\end{figure}

\begin{figure}[H]
\centering
\begin{tikzpicture}
\begin{axis}[
    xmin=0.5, xmax=4,
    ymin=0.5, ymax=3,
    axis x line=middle,
    axis y line=middle,
    samples=200,
    xlabel={$x$},
    ylabel={$y$}
]
    % 1. Definimos a curva e damos um nome a ela
    \addplot[blue, thick, domain=1:3.75, name path=curva] {0.5*(x-2)^2+1};
    
    % 2. reta 
    \addplot[red, thick, domain=1:3.75, name path=reta]{.5*x+0.5};

    % 3. Preenchimento
    %\addplot[blue!20] fill between[of=curva and reta];
\end{axis}
\end{tikzpicture}
\caption{Função positiva em todo o intervalo.}
\end{figure}

\begin{figure}[H]
\centering
\begin{tikzpicture}
\begin{axis}[
    xmin=0.5, xmax=4,
    ymin=-3, ymax=.5,
    axis x line=middle,
    axis y line=middle,
    samples=200,
    xlabel={$x$},
    ylabel={$y$}
]
    % 1. Parábola
    \addplot[blue, thick, domain=1:3.75, name path=curva] {-(0.5*(x-2)^2+1)};
    
    % 2. Reta 
    \addplot[red, thick, domain=1:3.75, name path=reta]{-(.5*x+0.5)};

    % 3. Preenchimento hachurado apenas entre as intersecções
    \addplot [
        % Define o padrão de hachurado
        pattern=north east lines, 
        pattern color=gray,
    ] fill between [
        of=curva and reta,
        split, % Divide a área nos pontos de cruzamento
        % Estiliza apenas o segmento desejado (o do meio)
        every segment no 0/.style={transparent},
        every segment no 1/.style={pattern=north east lines},
        every segment no 2/.style={transparent}
    ];
\end{axis}
\end{tikzpicture}
\caption{Área hachurada entre as intersecções.}
\end{figure}
\begin{figure}[hbt]
\centering
\begin{tikzpicture}
\begin{axis}[
  axis x line=middle, 
  axis y line=middle, 
  axis line style={->},
  xlabel={$x$}, ylabel={$y$}, 
  xmin=-2, xmax=2, ymin=-1.5, ymax=1.5,
  title={Caso 3: mudança de sinal em $[a,b]$},
    % --- CONFIGURAÇÃO DA SETA ---
  % Definimos que a seta (>) terá comprimento (length) de 4.5pt 
  % e largura (width) de 1.5pt (proporção 3:1)
  axis line style={-{Stealth[length=9pt, width=3pt]}},
  % ----------------------------
  axis on top
]
  \def\a{-1.5} 
  \def\b{1.5}
  
  % 1. Caminhos para hachura (invisíveis)
  \addplot[name path=curva, draw=none, domain=\a:\b, samples=200] {-x^2+1};
  \addplot[name path=eixo, draw=none, domain=\a:\b] {0};

  % 2. Aplicar hachuras por domínio
  \addplot[pattern=north east lines, pattern color=gray!60] 
    fill between[of=curva and eixo, soft clip={domain=\a:-1}];
    
  \addplot[pattern=north west lines, pattern color=blue!40] 
    fill between[of=curva and eixo, soft clip={domain=-1:1}];
    
  \addplot[pattern=north east lines, pattern color=gray!60] 
    fill between[of=curva and eixo, soft clip={domain=1:\b}];

  % 3. Desenhar a curva real
  \addplot[thick, blue, domain=\a:\b, samples=200] {-x^2+1};
  
  % Pontos de interseção
  \addplot[only marks, mark=*, mark size=1.5pt] coordinates {(-1,0) (1,0)};
  
  % --- CORREÇÃO DAS LINHAS VERTICAIS ---
  % Linha em a: desce do eixo (0) até o valor negativo da função
  \draw[dash pattern=on 5pt off 3pt, line width=0.6pt, gray] 
    (axis cs:\a,0) -- (axis cs:\a,{-(\a)^2+1});
    
  % Linha em b: desce do eixo (0) até o valor negativo da função
  \draw[dash pattern=on 5pt off 3pt, line width=0.6pt, gray] 
    (axis cs:\b,0) -- (axis cs:\b,{-(\b)^2+1});

  % Rótulos ajustados para não sobrepor
  \node[above left] at (axis cs:\a,0) {$a$};
  \node[above right] at (axis cs:\b,0) {$b$};
  \node[above right, blue] at (axis cs:0,1) {$f(x)$};
\end{axis}
\end{tikzpicture}
\caption{Áreas em lados alternados do eixo x}
\end{figure}

\subsection{Área abaixo do Eixo X}
Quando a função $f(x)$ é negativa ($f(x) < 0$), tratamos o eixo $x$ ($y=0$) como o teto e a curva como o chão.
\begin{equation}
A = \int_{a}^{b} [0 - f(x)] \, dx = -\int_{a}^{b} f(x) \, dx
\end{equation}

\begin{figure}[hbt]
  \centering
  \begin{tikzpicture}
    \begin{axis}[
      xmin=-1, xmax=4,
      ymin=-1, ymax=3,
      axis x line=middle,
      axis y line=middle,
      axis line style={-},
      tick align=outside,
      grid=both,
      minor tick num=1,
      xlabel={$x$},
      ylabel={$y$},
      title={Triangulo}
    ]
      % Triângulo (fechando o caminho voltando ao primeiro ponto)
      \addplot[
        thick
      ] coordinates {
        (0,0)
        (3,0)
        (1,2)
        (0,0)
      };

      % Preenchimento opcional
      \addplot[
        fill=blue!15,
        draw=none
      ] coordinates {
        (0,0)
        (3,0)
        (1,2)
        (0,0)
      };

      % Pontos marcados
      \addplot[only marks, mark=*] coordinates {
        (0,0)
        (3,0)
        (1,2)
      };

    \end{axis}
  \end{tikzpicture}
  \caption{Ciclo Termodinâmico}
\end{figure}

%--------------------------------------------------
%somente por coordenadas
%--------------------------------------------------
\begin{tikzpicture}
  % vértices
  \coordinate (A) at (0,0);
  \coordinate (B) at (3,0);
  \coordinate (C) at (1.5,2);

  % triângulo preenchido
  \fill[blue!15] (A) -- (B) -- (C) -- cycle;

  % contorno
  \draw[thick] (A) -- (B) -- (C) -- cycle;

  % marcar vértices
  \fill (A) circle (1.5pt) node[below left] {$A(0,0)$};
  \fill (B) circle (1.5pt) node[below right] {$B(3,0)$};
  \fill (C) circle (1.5pt) node[above] {$C(1{,}5,2)$};
\end{tikzpicture}

\begin{figure}[hbt]
  \centering
  \begin{tikzpicture}
    \begin{axis}[
      xmin=0, xmax=5,
      ymin=0, ymax=35,
      axis x line=middle,
      axis y line=middle,
      axis line style={-},
      tick align=outside,
      grid=both,
      minor tick num=1,
      xlabel={$V$},
      ylabel={$P$},
      %title={Triângulo com vértices $(0,0)$, $(3,0)$ e $(1,2)$}
    ]
      % Triângulo (fechando o caminho voltando ao primeiro ponto)
      \addplot[
        thick
      ] coordinates {
        (1,10)
        (1,30)
        (4,30)
        (1,10)
      };

      % Preenchimento opcional
      % \addplot[
      %   fill=blue!15,
      %   draw=none
      % ] coordinates {
      %   (0,0)
      %   (3,0)
      %   (1,2)
      %   (0,0)
      % };

      % Pontos marcados
      \addplot[only marks, mark=*] coordinates {
        (1,10)
        (1,30)
        (4,30)
      };

    \end{axis}
  \end{tikzpicture}
  \caption{Triângulo definido pelas coordenadas dos vértices.}

  
\end{figure}
\end{document}
