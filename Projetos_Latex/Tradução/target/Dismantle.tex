%!TEX program = lualatex
% Arquivo: Dismantle.tex
\documentclass[a4paper,12pt]{article}

% ===== Idioma =====
\usepackage[portuguese]{babel}
\addto\captionsportuguese{%
 \renewcommand{\contentsname}{Sumário}%
}

% ===== Fonte (LuaLaTeX) =====
\usepackage{fontspec}
\setmainfont{Latin Modern Roman}

% ===== Layout =====
\usepackage[top=3cm,bottom=2cm,left=3cm,right=2cm]{geometry}
\usepackage{indentfirst}
\usepackage{setspace}
\onehalfspacing
\setlength{\parindent}{0pt}

% ===== Tabelas / listas =====
\usepackage{booktabs}
\usepackage{enumitem}
\setlist[itemize]{noitemsep,topsep=3pt}

% ===== Figuras gerais =====
\usepackage{graphicx}
\usepackage{float}
\usepackage{pdflscape}
\graphicspath{{./}{./imagens/}{./figuras/}}

% ===== Links (no final) =====
\usepackage{hyperref}
\usepackage{xurl}
\Urlmuskip=0mu plus 1mu
\hypersetup{
  colorlinks=true,
  linkcolor=black,
  urlcolor=blue,
  citecolor=black
}

\begin{document}
Navios de casco de madeira eram, historicamente, simplesmente incendiados ou
``convenientemente'' queimados ao final de sua vida útil. Durante a dinastia
Tudor (1485--1603), os navios eram desmantelados de forma sistemática, com a
madeira reaproveitada em novas embarcações ou em outras construções. Esse
procedimento, contudo, deixou de ser aplicável com o advento dos navios de
casco metálico no século XIX.

Em 1880, a Denny Brothers, de Dumbarton, passou a utilizar aço naval proveniente
de sucata em seu estaleiro. A partir do final do século XIX, diversas nações
começaram a adquirir navios britânicos para demolição, incluindo Alemanha,
Itália, Holanda e Japão. A indústria italiana de desmantelamento iniciou-se em
1892, enquanto a indústria japonesa ganhou impulso após a aprovação de uma lei de
subsídio à indústria naval local.

Após acidentes ou desastres, operadores de navios de passageiros frequentemente
evitavam manter o nome original de embarcações danificadas, a fim de preservar
a imagem de suas companhias. Muitos navios da era vitoriana realizaram suas
viagens finais com a última letra do nome removida ou danificada, numa tentativa
de dissociar o navio de sua identidade comercial anterior.

A partir da década de 1930, tornou-se economicamente mais vantajoso encalhar
navios diretamente em praias, em vez de docá-los para desmontagem. Para isso,
o navio deveria estar o mais leve possível e ser lançado à praia em velocidade
máxima, durante a preamar. As operações exigiam marés da ordem de 3\,m, bem como
proximidade com indústrias metalúrgicas. As principais ferramentas utilizadas
incluíam cortadores elétricos, bolas de demolição e maçaricos de corte.

Empresas como a Thos.\ Ward Ltd., uma das maiores demolidoras do Reino Unido,
recondicionavam e revendiam mobiliário e equipamentos provenientes dos navios
desmantelados. Muitos artefatos históricos eram vendidos em leilões públicos.
O navio de passageiros RMS \textit{Mauritania}, por exemplo, foi vendido como
sucata por GBP~78.000, mas diversas peças alcançaram valores elevados em ofertas
internacionais. Em contrapartida, armamentos e informações militares, ainda que
obsoletos, eram cuidadosamente removidos por pessoal da Marinha antes do
desmantelamento.

Ao longo da segunda metade do século XX, países do Leste Asiático, beneficiados
por custos trabalhistas mais baixos, passaram a dominar a indústria de
desmantelamento naval. Com o aumento progressivo dos custos, a atividade migrou
do Japão e de Hong Kong para a Coreia do Sul e, posteriormente, para a China.
O porto ao sul da cidade de Kaohsiung, em Taiwan, destacou-se como líder mundial
no final das décadas de 1960 e 1970, tendo desmontado 220 navios em 1972,
totalizando 1,6 milhão de toneladas. Em 1977, Taiwan detinha mais da metade do
mercado mundial, seguido por Espanha e Paquistão. À época, Bangladesh ainda não
possuía capacidade significativa nesse setor.

O método de reciclagem por encalhe em praias ganhou relevância no Sul da Ásia
após 1960, quando o navio grego \textit{M/D Alpine} encalhou em Sitakunda,
Chittagong (então parte do Paquistão), após um ciclone severo. Incapaz de ser
reflutuado, o navio permaneceu no local por anos, até ser adquirido e desmontado
pela Chittagong Steel House em 1965. Esse episódio é frequentemente apontado como
o marco inicial da indústria de desmantelamento em Bangladesh. Até 1980, o
Gadani Ship Breaking Yard, no Paquistão, era o maior estaleiro de desmantelamento
do mundo.

O endurecimento das normas ambientais em países industrializados elevou os
custos de descarte de resíduos perigosos, incentivando a exportação de navios
para regiões com regulamentação mais branda, especialmente no Sul da Ásia.
Esse processo levou à adoção da Convenção da Basileia, em 1989. Em 2004, uma
decisão dessa convenção passou a classificar navios antigos como ``lixo tóxico'',
restringindo sua exportação sem autorização prévia do país importador, o que
resultou no ressurgimento de estaleiros ambientalmente conformes em alguns países.

Um caso emblemático foi o do porta-aviões francês \textit{Clemenceau}, que deixou
o porto de Toulon em 31 de dezembro de 2005 com destino ao estaleiro de Alang, na
Índia. Após protestos relacionados ao descarte inadequado de resíduos perigosos,
a Suprema Corte indiana suspendeu temporariamente o acesso do navio, e o Conselho
de Estado francês ordenou seu retorno. Posteriormente, o estaleiro Able UK, em
Hartlepool, assumiu o desmantelamento segundo práticas ambientalmente aceitáveis.
A desmontagem teve início em novembro de 2009 e foi concluída no final de 2010,
sendo considerada um ponto de inflexão na gestão ambiental do setor.

Em 2009, a Associação dos Advogados Ambientais de Bangladesh obteve decisão
judicial que proibiu temporariamente o desmantelamento fora de padrões
regulatórios. Durante 14 meses, a indústria ficou noteiramente paralisada,
resultando na perda de milhares de empregos, até a revogação do banimento.
No mesmo período, a recessão econômica global e a queda da demanda por transporte
marítimo provocaram um aumento no número de navios destinados ao
descomissionamento, evidenciando a relação inversa entre a taxa de
desmantelamento e o preço do frete marítimo.

\end{document}