A metodologia adotada neste trabalho se caracterizará como uma pesquisa de natureza
exploratória e descritiva, baseada na análise de dados secundários e revisão
bibliográfica.

Inicialmente, será utilizada uma base de dados consolidada a partir de informações
públicas disponibilizadas pela Agência Nacional do Petróleo, Gás Natural e Biocombustíveis (ANP), contendo dados sobre plataformas offshore em operação no Brasil, tais como tipo de unidade, bacia sedimentar, operador, lâmina d’água e ano de início de operação.

Em seguida poderá ser escolhido uma região como, por exemplo, os campos de Marlim e Voador que já têm o PDI feito, aprovado pela ANP e publicado. Dessa forma, pode-se analisar a documentação disponível e fazer uma abrangência de caráter mais geral.

Os dados serão organizados e analisados para identificar padrões relacionados
à idade das instalações, à distribuição espacial e inferir os resultados para casos semelhantes na Bacia de Campos. A análise será complementada por uma abordagem conceitual dos aspectos oceanográficos e ambientais relevantes ao descomissionamento, com base na literatura especializada.