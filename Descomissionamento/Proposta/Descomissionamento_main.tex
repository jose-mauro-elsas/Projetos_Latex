%!TEX program = lualatex
% Arquivo: Descomissionamento_main.tex
\documentclass[a4paper,12pt]{article}

% ===== Idioma =====
\usepackage[portuguese]{babel}
\addto\captionsportuguese{%
  \renewcommand{\contentsname}{Sumário}%
}

% ===== Fonte (LuaLaTeX) =====
\usepackage{fontspec}
\setmainfont{Latin Modern Roman}

% ===== Layout =====
\usepackage[top=3cm,bottom=2cm,left=3cm,right=2cm]{geometry}
\usepackage{indentfirst}
\usepackage{setspace}
\onehalfspacing
\setlength{\parindent}{1.25cm}
\setlength{\parskip}{0pt}

% ===== Tabelas / listas =====
\usepackage{booktabs}
\usepackage{enumitem}
\setlist[itemize]{noitemsep,topsep=3pt}

% ===== Figuras gerais =====
\usepackage{graphicx}
\usepackage{float}
\usepackage{pdflscape}
\graphicspath{{./}{./imagens/}{./figuras/}}

% ===== Links =====
\usepackage[hidelinks]{hyperref}
\usepackage{xurl}
\Urlmuskip=0mu plus 1mu

% ===== TikZ (fluxogramas/diagramas) =====
\usepackage[dvipsnames]{xcolor}
\usepackage{tikz}
\usetikzlibrary{shapes.geometric, arrows.meta, positioning}
% =========================================================
% Arquivo: flu1_styles.tex
% Requer no main:
%   \usepackage[dvipsnames]{xcolor}
%   \usepackage{tikz}
%   \usetikzlibrary{shapes.geometric, arrows.meta, positioning}
% =========================================================

\tikzset{
  process/.style = {
    rectangle,
    rounded corners,
    minimum width=6.0cm,
    minimum height=1.05cm,
    text centered,
    text width=5.7cm,
    draw=blue!70!black,
    fill=blue!5,
    line width=0.9pt
  },
  process_final/.style = {
    process,
    fill=green!6,
    draw=green!45!black
  },
  arrow/.style = {
    thick,
    ->,
    >=Stealth,
    line width=1.1pt,
    draw=gray!80
  }
}
  % define process/process_final/arrow

\begin{document}

% ================== FOLHA DE ROSTO ==================
\begin{titlepage}
  \begin{center}
    \vspace*{-1cm}
    \includegraphics[width=3cm]{uerj.jpg}\\[0.8cm]

    \textbf{UNIVERSIDADE DO ESTADO DO RIO DE JANEIRO}\\[0.2cm]
    \textbf{FACULDADE DE OCEANOGRAFIA}\\[0.2cm]

    \textbf{Proposta de estudo para TCC}\\[0.5cm]
    \textbf{Descomissionamento de UEPs e sistemas Subsea de Produção.}\\[8cm]
  \end{center}

  \vfill

  \begin{flushright}
  Aluno: José Mauro Xavier Elsas\par
  Disciplina: Oceanografia\par
  Professor: David Zee
  \end{flushright}

  \vfill

  \begin{center}
    Rio de Janeiro\\
    Janeiro de 2026
  \end{center}
\end{titlepage}

\tableofcontents
\newpage

\section{Abreviaturas e Siglas}
% ============================
% ABREVIATURAS E SIGLAS
% ============================
%\section*{ABREVIATURAS E SIGLAS}
%\addcontentsline{toc}{section}{ABREVIATURAS E SIGLAS}

\begin{description}[leftmargin=3.2cm,style=nextline]
  \item[ANM] Árvore de Natal Molhada
  \item[ANP] Agência Nacional do Petróleo, Gás Natural e Biocombustíveis.
  \item[API] \textit{American Petroleum Institute}.
  \item[CGBS] \textit{Concrete Gravity Based Structure}.
  \item[DNV] \textit{Det Norske Veritas}.
  \item[FPSO] \textit{Floating Production, Storage and Offloading}.
  \item[FSO] \textit{Floating Storage and Offloading}.
  \item[ISO] \textit{International Organization for Standardization}.
  \item[IMO] \textit{International Maritime Organization}.
  \item[PDI] Plano de Descomissionamento de Instalações.
  \item[REVAMP] {Modernizar instalações para aumentar a produtividade e vida útil}
  \item[PETROBRAS]Petróleo Brasileiro S.A.
  \item[UEP]  Unidade Estacionária de Produção
\end{description}

\newpage

\section{Introdução}
Historicamente, as plataformas offshore são projetadas com uma vida útil 
nominal da ordem de 25 a 30 anos \cite{PSANorwayLifeExtension2007}, definida a 
partir de critérios de projeto estrutural, análises de fadiga, cargas ambientais 
extremas e hipóteses conservadoras de degradação ao longo do tempo. Tais critérios 
consideram, em maior ou menor grau, parâmetros oceanográficos como altura significativa
de ondas, correntes de fundo e de superfície, eventos extremos associados a 
frentes frias e ciclones subtropicais, bem como processos de corrosão influenciados 
pela hidrodinâmica local. No entanto, a experiência operacional demonstra que 
a vida produtiva 
dos campos pode exceder a vida originalmente projetada das instalações, seja 
por revisões de reservas, melhorias tecnológicas, estratégias de recuperação
avançada ou mudanças no contexto econômico. Nessas situações, é comum a adoção
de programas de extensão de vida útil, incluindo \textbf{REVAMPs}, reforços
estruturais, substituição de sistemas críticos e avaliações aprofundadas de 
integridade estrutural e ambiental.

Essa transição estrutural é caracterizada por um conjunto de transformações 
simultâneas, entre as quais se destacam a introdução de fontes de energia 
renovável, a maturação de campos produtores de hidrocarbonetos mais antigos 
e a entrada em operação de novas fronteiras exploratórias em águas profundas e 
ultraprofundas.

Esse processo ocorre, na maioria das vezes, em campos de petróleo localizados
em ambientes marinhos caracterizados por elevada complexidade física,
biogeoquímica e ecológica. Nos ambientes oceânicos, as interações entre
estruturas artificiais, ventos, ondas e correntes podem afetar processos
de sedimentação e suspensão de sedimentos finos, influenciando também
ecossistemas bentônicos e pelágicos. 

Assim, as interações entre o oceano e as estruturas offshore
desempenham papel central na avaliação de impactos ambientais ao longo
do ciclo de vida das instalações.

Nesse contexto, impõem-se desafios técnicos, regulatórios, ambientais e 
econômicos associados ao descomissionamento e desmantelamento de plataformas e demais 
instalações de produção, tema que vem ganhando centralidade na agenda do setor energético 
nacional. O descomissionamento offshore não se limita à remoção física de estruturas, 
mas envolve a compreensão dos processos oceanográficos locais, tais como correntes, 
regimes de ondas, ressuspensão de sedimentos e conectividade ecológica, que influenciam 
diretamente a  dispersão de contaminantes, a estabilidade do leito marinho e a resiliência 
dos ecossistemas afetados.


Por outro lado, quando a continuidade operacional deixa de ser técnica ou
economicamente viável, impõe-se a necessidade de planejamento e execução do
descomissionamento, processo que envolve a retirada total ou parcial das 
instalações, a destinação adequada de materiais, a recuperação ambiental e o 
atendimento a requisitos regulatórios específicos. Do ponto de vista ambiental 
e oceanográfico, esse processo demanda a avaliação dos impactos associados à 
remoção ou permanência de estruturas no ambiente marinho, considerando efeito 
sobre habitats artificiais, comunidades recifais, dinâmica sedimentar e 
qualidade da água. No Brasil, o descomissionamento é regulado principalmente 
pela Agência Nacional do Petróleo, Gás Natural e Biocombustíveis (ANP), por

meio da Resolução ANP n.º 817/2020, \cite {ANP8172020}
que estabelece diretrizes para o descomissionamento de instalações de exploração 
e produção e o Plano de Descomissionamento de Instalações (PDIs) para a devolução 
de áreas a serem recuperadas.

Outra norma que estabelece parâmetros para o descomissionamento e abandono de estruturas é a Resolução A.672(16): Guidelines and Standards for the Removal of Offshore Installations and Structures on the Continental Shelf and in the Exclusive Economic Zone de 1989 \cite{IMOA672}

Apesar de existir um marco regulatório relativamente recente, o
descomissionamento de embarcações offshore no Brasil ainda apresenta lacunas técnicas
e operacionais. Sobretudo quando comparado à experiência internacional em bacias 
maduras, como o Mar do Norte, onde estudos ambientais de longo prazo e a 
integração com a oceanografia operacional têm orientado decisões sobre remoção 
parcial, a criação de recifes artificiais e monitoramento pós-descomissionamento.

Entre os principais desafios destaca-se a diversidade tipológica das instalações
(plataformas fixas, semissubmersíveis, FPSOs e FSOs), a variabilidade das condições
ambientais ao longo da extensa margem continental brasileira, a complexidade
logística das operações offshore e a necessidade de integração entre requisitos
técnicos, ambientais e econômicos. Destaca-se ainda a heterogeneidade dos ambientes
oceanográficos entre bacias sedimentares, com diferenças significativas em lâmina
d'água, regime de correntes, produtividade biológica e sensibilidade ambiental.

Há, no âmbito da indústria, a necessidade de adequar eventuais aumentos no número de
barris por dia produzidos à longevidade dos equipamentos. A integridade estrutural e,
consequentemente, a durabilidade dos sistemas de produção estão diretamente associadas
à vida econômica de um campo de petróleo. 

Nesse contexto, um campo maduro só se mantém viável se os sistemas de produção
(topside e subsea) puderem operar de forma segura e eficiente, sem a necessidade
de investimentos elevados e recorrentes.


Nesse contexto, a análise sistemática do parque instalado offshore 
brasileiro torna-se fundamental para compreender a escala, a distribuição espacial 
e o horizonte temporal dos desafios de descomissionamento. A planilha consolidada 
utilizada neste trabalho, elaborada a partir de dados públicos da ANP, reúne 
informações sobre plataformas atualmente em operação no Brasil, incluindo nome da
instalação, bacia sedimentar, tipo de unidade, operador, lâmina d’água e ano de 
início de operação. Esse conjunto de dados permite identificar unidades 
potencialmente próximas do fim de sua vida útil nominal, bem como avaliar
tendências associadas à concentração geográfica, profundidade de instalação e 
tipo de tecnologia empregada, aspectos diretamente
relacionados à complexidade ambiental das operações de descomissionamento.

A partir dessa base empírica, é possível discutir criticamente questões 
como:
\begin{enumerate}
    \item a coerência econômica entre a idade das instalações e as estratégias de 
    extensão de vida adotadas;
    \item o impacto esperado do descomissionamento sobre determinadas bacias 
    produtoras e áreas costeiras onde são realizados os desmantelamentos de UEPs
    considerando suas especificidades oceanográficas e ambientais;
    \item a adequação do arcabouço normativo nacional frente à diversidade de 
    cenários técnicos e ambientais; e
    \item as oportunidades de aprimoramento em planejamento, governança e gestão de 
    risco associadas ao ciclo final de vida das plataformas offshore, com ênfase na 
    mitigação de impactos ambientais.
\end{enumerate}

Dessa forma, o presente anteprojeto propõe-se a investigar o descomissionamento de
plataformas de produção no Brasil sob uma perspectiva integrada, combinando análise
de dados reais das instalações em operação, revisão normativa e fundamentos da
engenharia offshore e da oceanografia física. Busca-se, assim, contribuir para o
entendimento técnico-científico do tema e para o aprimoramento das práticas de
planejamento do fim de vida de ativos offshore no país, à luz dos desafios 
ambientais impostos pelo ambiente marinho.


\section{Justificativa e Motivação}
O descomissionamento offshore representa uma etapa crítica do ciclo de vida das
instalações de produção, que têm implicações diretas sobre o meio ambiente marinho,
a segurança operacional e a governança dos recursos naturais, muito embora estruturas
de jaquetas sejam projetadas conforme normas como a API 2AW-SD \cite{API2AWSD}.
No Brasil, o aumento do número de plataformas próximas
ao fim de sua vida útil nominal impõe a necessidade de planejamento antecipado e de avaliações
técnicas e ambientais robustas.

Depois que o sistema de produção atinge o fim da vida útil, pode passar por obras de
revitalização que o enquadram nos requisitos normativos adequados. Como, por exemplo, o
\textit{Requirements for Life Extension of Ageing Offshore Production Installations},
norma essa que propõe, para casos de extensão da vida útil de uma instalação de 
produção, \cite{PSANorwayLifeExtension2007}:

\begin{quotation}
    \textit{\(\dots\) to demonstrate and document that the installation is fit for
    continued safe operation during the life extension period. The application should describe
    the barrier systems that underwrite safety and contain an analysis of their current and
    predicted integrity and performance. The information can be presented in a synthesized
    and evaluated form, and reference made to other more detailed reviews and assessments
    as appropriate.} (2008, p.37)
\end{quotation} 

Quando termina o tempo determinado pelo projeto de aumento da vida em produção,
é iniciado o processo de descomissionamento, que deve cumprir algumas etapas:

\begin{enumerate}
    \item Estudos que precedem o descomissionamento, tais como estudos de
    engenharia, de impacto ambiental e entrega da documentação.
    \item Abandono dos poços conectados ao sistema.
    \item Desmobilização dos equipamentos \textit{topside} (planta de produção,
    módulo de habitação, compressão de gás, geração de energia.)
    \item Remoção da plataforma, propriamente dita (jaqueta, sistemas de ancoragem,
    ou \textit{Concrete Gravity Based Structure}, se houver.)
    \item Remoção de linhas rígidas e flexíveis, manifolds, válvulas e demais equipamentos.
    \item Recolhimento de resíduos, limpeza e recuperação da área.
    \item Desmantelamento e reciclagem do sistema executado em terra.
\end{enumerate}


A regulamentação e procedimentação da atividade é resultado de um histórico de muitos
acidentes e impactos ambientais, como os ocorridos ao longo da segunda metade do século XX,
quando países do Leste Asiático, beneficiados por custos trabalhistas mais baixos,
passaram a dominar a indústria de desmantelamento naval. Com o aumento progressivo dos custos,
a atividade migrou do Japão e de Hong Kong para a Coreia do Sul e, posteriormente, para a China.
O porto ao sul da cidade de Kaohsiung, em Taiwan, destacou-se como líder mundial
no final das décadas de 1960 e 1970, tendo desmontado 220 navios em 1972,
totalizando 1,6 milhão de toneladas. Em 1977, Taiwan detinha mais da metade do
mercado mundial, seguido por Espanha e Paquistão. À época, Bangladesh ainda não
possuía capacidade significativa nesse setor.

O método de reciclagem por encalhe em praias ganhou relevância no Sul da Ásia
após 1960, quando o navio grego \textit{M/D Alpine} encalhou em Sitakunda,
Chittagong (então parte do Paquistão), após um ciclone severo. Incapaz de ser
reflutuado, o navio permaneceu no local por anos, até ser adquirido e desmontado
pela Chittagong Steel House em 1965. Esse episódio é frequentemente apontado como
o marco inicial da indústria de desmantelamento em Bangladesh. Até 1980, o
Gadani Ship Breaking Yard, no Paquistão, era o maior estaleiro de desmantelamento
do mundo.

O endurecimento das normas ambientais em países industrializados elevou os
custos de descarte de resíduos perigosos, incentivando a exportação de navios
para regiões com regulamentação mais branda, especialmente no Sul da Ásia.
Dentro desse modelo de negócio, os armadores passaram a vender as embarcações para
intermediários que mudavam o nome da embarcação, o porto de registro e a bandeira para
desvincular o nome dos proprietários originais de eventuais problemas jurídicos
futuros.

O fato é que, da forma como era feito o desmantelamento de embarcações, vemos que:
\begin{itemize}
    \item desigualdades sociais se aprofundavam, 
    \item um grupo de intermediários e donos de áreas de desmantelamento se tornaram
    milionários,
    \item os perigos de acidentes pessoais vitimavam  pessoas comuns e,
    \item o impacto ambiental atingia diretamente o litoral da região. 
 \end{itemize}

No ano de 1989 a convenção da Basileia, \cite{Basileia1989} que trata de transporte transnacional de
cargas perigosas definiu de como de deveria se lidar com materiais perigosos. 
Embora a Convenção da Basileia não citasse diretamente navios, embarcações ou
plataformas, alguns materiais usados na construção naval são listados no Anexo VIII
da Convenção.

Como os materiais são descritos no anexo VIII citado, mas o termo "navio"  ou "embarcação"
(destinada ao descomissionamento) não é explicitamente citado, a IMO publicou em 2009
com total clareza a Convenção de Hong Kong. Como efeito geral, tivemos a implantação
de estaleiros ambientalmente conformes em alguns países.

A legislação brasileira (Resolução CONAMA n.º 452, de 2 de julho de 2012.) \cite{CONAMA4522012} aplica a 
Basileia para navios porque o STF e os tribunais superiores brasileiros já 
consolidaram o entendimento de que a mistura de materiais perigosos (amianto, óleos)
torna o objeto inteiro um resíduo perigoso.

O caso de referência é o do porta-aviões São Paulo, exemplo perfeito dessa discrepância entre o texto da lei e a prática. O navio que deveria ter sido desmantelado na Turquia, não teve autorização de atracação porque o governo turco (via Ministério do Meio Ambiente) cancelou a sua autorização de entrada. O motivo foi a divergência no inventário de materiais perigosos (especialmente amianto). Eles alegaram que o Brasil não provou que o navio estava "limpo " o suficiente, ferindo o princípio do Consentimento Prévio Informado. Por isso foi devolvido e teve de cruzar o Atlântico de volta. Na volta, o Porto de Santos e a Mari nha não permitiram seu reingresso em águas interiores por risco ambiental e de naufrágio.

Tendo ficado por muito tempo fundeado ao largo da costa de Pernambuco, hoje não está mais lá. Após vários meses na costa de Pernambuco, a Marinha do Brasil optou pelo afundamento controlado em fevereiro de 2023, a cerca de 350 km da costa, em águas profundas.
Organizações como a Basel Action Network (BAN) denunciaram o Brasil, afirmando que o afundamento violou a Convenção de Basileia, já que o navio (cheio de resíduos tóxicos) foi "descartado" no oceano em vez de ser reciclado.

Essa atividade, intensiva em mão de obra que representa, certamente, um mercado em
expansão, deve atender a regulações de diferentes áreas, como a ambiental e a trabalhista. Tais regulações, cada uma em seu respectivo domínio, têm por finalidade 
evitar acidentes com vítimas e/ou a ocorrência de impactos ambientais significativos.
Algumas áreas costeiras da Ásia e da África operaram por longo período — e, em lguns 
casos, ainda operam — sob condições que não atendem às disposições da Convenção Internacional de Hong Kong para a Reciclagem Segura e Ambientalmente Adequada de navios, de 
2009 (IMO) \cite{HKC2009}, a qual estabelece, como obrigação primária de cada Estado signatário:

\begin{quotation}
    $\dots$ desta Convenção se compromete a pôr em execução, de maneira plena e
    completa, suas disposições, de modo a prevenir, reduzir, minimizar e, dentro do possível, eliminar acidentes, ferimentos e outros efeitos adversos sobre a saúde 
    humana e o meio ambiente causados pela reciclagem de navios, e a aumentar a segurança das embarcações e a proteção da saúde humana e do meio ambiente durante toda a vida útil de 
    um navio. (2009, p.2)
\end{quotation}

A resolução da ANP \cite{ANP8172020} indica matrizes de competências e de normas
\cite{ANPDescomissionamentoPortal} que consideram, inclusive, normas para tratamento de
materiais radioativos. Uma vez que pode haver radioatividade no poço ou no reservatório,
que acaba por contaminar não só os materiais e equipamentos introduzidos no
poço, bem como todos os equipamentos que integram o sistema de produção, como ANMs.

Do ponto de vista oceanográfico, a remoção ou manutenção de estruturas offshore
interfere em processos como a ressuspensão de sedimentos, a dispersão de
contaminantes, a modificação de habitats bentônicos e a conectividade ecológica.
Entretanto, tais aspectos ainda são pouco explorados integradamente no contexto
nacional, especialmente quando comparados a bacias maduras internacionais. Justifica-se, portanto, 
a preocupação com os impactos ambientais, uma vez que a maioria dos equipamentos está instalada no 
assoalho marinho, e não na UEP, que pode ser removida para desmonte e reciclagem em terra.

Toda a parte \textit{subsea} do sistema de produção precisa ser limpa e inertizada
antes de sua desconexão da UEP e dos equipamentos de fundo. As operações de recolhimento
de linhas flexíveis, rígidas, risers e manifolds precisam ser executadas para
minimizar riscos de acidentes pessoais ou ambientais.

Assim, justifica-se a realização deste estudo como contribuição técnico-científica
para o entendimento do descomissionamento offshore no Brasil, com ênfase nos
condicionantes ambientais e oceanográficos que influenciam o planejamento e a
execução dessas operações.


% ===== Fluxograma (entra aqui, depois da Justificativa) =====
\newpage
Então graficamente temos:

\begin{figure}[H]
  \centering
  % Arquivo: flu1_body.tex
% Somente o ambiente tikzpicture (sem \begin{document})

\begin{tikzpicture}[
  node distance=1.0cm and 0.9cm
]
% ===== Linha 1 (esq -> dir) =====
\node (s1) [process] {1. Estudos de engenharia, impacto ambiental e documentação};
\node (s2) [process, right=of s1] {2. Abandono dos poços conectados};

% ===== Linha 2 (dir -> esq) =====
\node (s4) [process, below=of s2] {4. Remoção da estrutura (jaqueta, ancoragem ou GBS)};
\node (s3) [process, left=of s4] {3. Desmobilização do \textit{topside} (produção, habitação, energia)};

% ===== Linha 3 (esq -> dir) =====
\node (s5) [process, below=of s3] {5. Remoção de linhas, manifolds e válvulas};
\node (s6) [process, right=of s5] {6. Resíduos, limpeza e recuperação da área};

% ===== Linha 4 (final centralizado) =====
\node (s7) [process_final, below=of s6, xshift=-3.45cm] {7. Desmantelamento e reciclagem em terra};

% ===== Setas (fluxo serpente) =====
\draw [arrow] (s1) -- (s2);

\draw [arrow] (s2) -- (s4);
\draw [arrow] (s4) -- (s3);

\draw [arrow] (s3) -- (s5);
\draw [arrow] (s5) -- (s6);

\draw [arrow] (s6) -- (s7);

\end{tikzpicture}

  \caption{Fluxograma do processo de descomissionamento.}
  \label{fig:flu1}
\end{figure}

\section{Objetivos}
\subsection{Objetivo Geral}

Analisar o descomissionamento de plataformas offshore no Brasil sob uma perspectiva
integrada, considerando aspectos técnicos, ambientais e oceanográficos associados ao
fim de vida das instalações.

Vale lembrar que a distância entre as locações e os estaleiros de desmantelamento não
é exatamente um problema, uma vez que  outros tipos de embarcações, tais  como graneleiros e 
porta-contêineres, são em sua maioria descomissionados fora do Brasil (em particular na 
Turquia) e que muitas UEPs são produzidas em locais diferentes nas suas diversas fases.

Há exemplos de cascos construídos na China, com equipamentos montados na Holanda para serem
instalados na costa de Sergipe. Portanto, em um horizonte de 30 anos, as plataformas que forem
descomissionadas na nova fronteira da Margem Equatorial poderão ser desmanteladas nos estaleiros
do Rio de Janeiro. Essa avaliação também é objetivo desse trabalho.

\subsection{Objetivos Específicos}

\begin{itemize}
    \item Caracterizar o parque offshore brasileiro em termos de tipo de instalação,
          idade, bacia sedimentar e lâmina d’água; identificar unidades potencialmente próximas 
          do fim de sua vida útil nominal conforme normas como ANP, API e DNV.
    \item Avaliar a relação entre ambiente oceanográfico, profundidade de instalação e
          complexidade das operações de descomissionamento; analisar o arcabouço regulatório 
          nacional aplicável ao descomissionamento offshore; discutir os principais impactos 
          ambientais associados às diferentes estratégias de descomissionamento.
\end{itemize}



\section{Revisão Bibliográfica}
A revisão bibliográfica abordará os fundamentos conceituais e técnicos do
descomissionamento offshore, incluindo práticas consolidadas em bacias maduras,
como o Mar do Norte. Serão discutidos os princípios da engenharia offshore
relacionados ao ciclo de vida das plataformas, bem como aspectos da oceanografia
física relevantes para operações marítimas, tais como regimes de correntes, ondas,
processos sedimentares e sua influência sobre a estabilidade e o impacto ambiental
das estruturas.

Adicionalmente, serão analisados estudos sobre impactos ambientais do
descomissionamento, recuperação de áreas marinhas e o papel das estruturas offshore
como habitats artificiais, com foco nas lacunas existentes no contexto brasileiro.

\section{Metodologia}
A metodologia adotada neste trabalho se caracterizará como uma pesquisa de natureza
exploratória e descritiva, baseada na análise de dados secundários e revisão
bibliográfica.

Inicialmente, será utilizada uma base de dados consolidada a partir de informações
públicas disponibilizadas pela Agência Nacional do Petróleo, Gás Natural e Biocombustíveis (ANP), contendo dados sobre plataformas offshore em operação no Brasil, tais como tipo de unidade, bacia sedimentar, operador, lâmina d’água e ano de início de operação.

Em seguida poderá ser escolhido uma região como, por exemplo, os campos de Marlim e Voador que já têm o PDI feito, aprovado pela ANP e publicado. Dessa forma, pode-se analisar a documentação disponível e fazer uma abrangência de caráter mais geral.

Os dados serão organizados e analisados para identificar padrões relacionados
à idade das instalações, à distribuição espacial e inferir os resultados para casos semelhantes na Bacia de Campos. A análise será complementada por uma abordagem conceitual dos aspectos oceanográficos e ambientais relevantes ao descomissionamento, com base na literatura especializada.

\section{Resultados esperados}

Espera-se que o trabalho produza um diagnóstico do parque offshore brasileiro sob a
ótica do descomissionamento, identificando tendências e desafios associados ao fim
de vida das plataformas. Espera-se, ainda, contribuir para a compreensão da
importância dos condicionantes oceanográficos e ambientais no planejamento dessas
operações, fornecendo subsídios para o aprimoramento das práticas de gestão e
governança do descomissionamento offshore no Brasil.

\section{Proposta para execução}
O desenvolvimento do trabalho está previsto para ocorrer conforme as seguintes
etapas: 
\begin{itemize}
    \item levantamento e revisão bibliográfica;
    \item Organização e análise da base de dados;
    \item integração dos aspectos técnicos, ambientais e oceanográficos; 
     redação e revisão do trabalho final.
\end{itemize}

% ===== Referências =====
\newpage
\begingroup
\sloppy
\setlength{\emergencystretch}{2em}
\raggedright

\phantomsection
\addcontentsline{toc}{section}{Referências}
\begin{thebibliography}{99}

\bibitem{ANP8172020}
AGÊNCIA NACIONAL DO PETRÓLEO, GÁS NATURAL E BIOCOMBUSTÍVEIS (ANP).
\textbf{\label{Resolução ANP nº 817, de 24 de abril de 2020}}.
Dispõe sobre o descomissionamento de instalações de exploração e produção de petróleo e gás natural, devolução de áreas e reversão de bens.
Diário Oficial da União: Brasília–DF, 27 abr. 2020.
Disponível em: \url{https://www.in.gov.br/web/dou/-/resolucao-n-817-de-24-de-abril-de-2020-254001378}.
Acesso em: 12 jan. 2026.

% \bibitem{ANP8542021}
% AGÊNCIA NACIONAL DO PETRÓLEO, GÁS NATURAL E BIOCOMBUSTÍVEIS (ANP).
% \textbf{Resolução ANP nº 854, de 27 de setembro de 2021}.
% Regulamenta os procedimentos para apresentação de garantias financeiras que assegurem recursos para o descomissionamento de instalações.
% Diário Oficial da União: Brasília, DF, 29 set. 2021.
% Disponível em: \url{https://www.in.gov.br/en/web/dou/-/resolucao-anp-n-854-de-27-de-setembro-de-2021-348273146}.
% Acesso em: 24 jan. 2026.

\bibitem{ANPDescomissionamentoPortal}
AGÊNCIA NACIONAL DO PETRÓLEO, GÁS NATURAL E BIOCOMBUSTÍVEIS (ANP).
\textbf{Descomissionamento de instalações}.
Portal gov.br, [s.l.], 2020.
Disponível em: \url{https://www.gov.br/anp/pt-br/assuntos/exploracao-e-producao-de-oleo-e-gas/seguranca-operacional/descomissionamento-de-instalacoes}.
Acesso em: 12 jan. 2026.

\bibitem{HKC2009}
ORGANIZAÇÃO MARÍTIMA INTERNACIONAL (IMO).
\textbf{Convenção Internacional de Hong Kong para a Reciclagem Segura e Ambientalmente Adequada de Navios, 2009}.
Hong Kong, China, 15 maio 2009.
Disponível em: \url{https://www.imo.org/en/About/Conventions/Pages/The-Hong-Kong-International-Convention-for-the-Safe-and-Environmentally-Sound-Recycling-of-Ships.aspx}.
Acesso em: 14 jan. 2026.

% \bibitem{ANPGarantiasPortal}
% AGÊNCIA NACIONAL DO PETRÓLEO, GÁS NATURAL E BIOCOMBUSTÍVEIS (ANP).
% \textbf{Garantias financeiras de descomissionamento}.
% Portal gov.br, [s.l.], 2021.
% Disponível em: \url{https://www.gov.br/anp/pt-br/assuntos/exploracao-e-producao-de-oleo-e-gas/desenvolvimento-e-producao/garantias-financeiras-de-descomissionamento}.
% Acesso em: 24 jan. 2026.

% \bibitem{ANPDadosAbertosPlataformasOperacao}
% AGÊNCIA NACIONAL DO PETRÓLEO, GÁS NATURAL E BIOCOMBUSTÍVEIS (ANP).
% \textbf{Dados abertos: plataformas em operação} (arquivo CSV).
% Rio de Janeiro: ANP, [s.d.].
% Disponível em: \url{https://www.gov.br/anp/pt-br/centrais-de-conteudo/dados-abertos/arquivos/arquivos-fase-de-desenvolvimento-e-producao/lpo/dados-abertos-plataformas-operacao.csv}.
% Acesso em: 24 jan. 2026.

% \bibitem{ISO19900}
% INTERNATIONAL ORGANIZATION FOR STANDARDIZATION (ISO).
% \textbf{ISO 19900:2019 -- Petroleum and natural gas industries -- General requirements for offshore structures}.
% Geneva: ISO, 2019.
% Disponível em: \url{https://www.iso.org/standard/69761.html}.
% Acesso em: 24 jan. 2026.

% \bibitem{ISO19901-1}
% INTERNATIONAL ORGANIZATION FOR STANDARDIZATION (ISO).
% \textbf{ISO 19901-1:2015 -- Petroleum and natural gas industries -- Specific requirements for offshore structures -- Part 1: Metocean design and operating considerations}.
% Geneva: ISO, 2015.
% Disponível em: \url{https://www.iso.org/standard/60183.html}.
% Acesso em: 24 jan. 2026.

% \bibitem{ISO19902}
% INTERNATIONAL ORGANIZATION FOR STANDARDIZATION (ISO).
% \textbf{ISO 19902:2020 -- Petroleum and natural gas industries -- Fixed steel offshore structures}.
% Geneva: ISO, 2020.
% Disponível em: \url{https://www.iso.org/standard/65688.html}.
% Acesso em: 24 jan. 2026.

\bibitem{API2AWSD}
AMERICAN PETROLEUM INSTITUTE (API).
\textbf{API Recommended Practice 2A-WSD: Planning, Designing, and Constructing 
Fixed Offshore Platforms — Working Stress Design}.
22. ed. Washington, DC: API, nov. 2014.
Disponível em: \url{https://www.api.org/~/media/files/publications/whats%20new/2a-wsd_e22%20pa.pdf}.
Acesso em: 20 jan. 2026.

\bibitem{IMOA672}
INTERNATIONAL MARITIME ORGANIZATION (IMO).
\textbf{Resolution A.672(16): Guidelines and Standards for the Removal of Offshore
 Installations and Structures on the Continental Shelf and in the Exclusive Economic 
 Zone}.
Adopted on 19 Oct. 1989.
London: IMO, 1989.
Disponível em: \url{https://wwwcdn.imo.org/localresources/en/KnowledgeCentre/IndexofIMOResolutions/AssemblyDocuments/A.672%2816%29.pdf}.
Acesso em: 24 jan. 2026.

\bibitem{PSANorwayLifeExtension2007}
PETROLEUM SAFETY AUTHORITY NORWAY (PSA).
\textbf{Requirements for Life Extension of Ageing Offshore Production Installations}.
Norway: PSA, 2007.
Disponível em: \url{https://www.havtil.no/contentassets/24974571fd8442bea4c21d3679d93a8e/requirements-for-life-extension-of-ageing-offshore-production-installations.pdf}.
Acesso em: 10 jan. 2026.

% \bibitem{IOGP584}
% INTERNATIONAL ASSOCIATION OF OIL \& GAS PRODUCERS (IOGP).
% \textbf{Overview of International Offshore Decommissioning Regulations -- Volume 1: Facilities}.
% IOGP Report 584. London: IOGP, jul. 2017.
% Disponível em: \url{https://www.iogp.org/bookstore/product/overview-of-international-offshore-decommissioning-regulations-volume-1-facilities/}.
% Acesso em: 24 jan. 2026.

% \bibitem{IOGP585}
% INTERNATIONAL ASSOCIATION OF OIL \& GAS PRODUCERS (IOGP).
% \textbf{Overview of International Offshore Decommissioning Regulations -- Volume 2: Wells Plugging \& Abandonment}.
% IOGP Report 585. London: IOGP, maio 2023.
% Disponível em: \url{https://www.iogp.org/bookstore/product/overview-of-international-offshore-decommissioning-regulations-volume-2-wells-plugging-abandonment/}.
% Acesso em: 24 jan. 2026.

% \bibitem{IOGP484}
% INTERNATIONAL ASSOCIATION OF OIL \& GAS PRODUCERS (IOGP).
% \textbf{Decommissioning of Offshore Concrete Gravity Based Structures (CGBS) in the OSPAR Maritime Area / Other Global Regions}.
% Report 484-1. London: IOGP, 2017.
% Disponível em: \url{https://www.iogp.org/bookstore/wp-content/uploads/sites/2/woocommerce_uploads/2017/01/484-1.pdf}.
% Acesso em: 24 jan. 2026.

\bibitem{Basileia1989}
ORGANIZAÇÃO DAS NAÇÕES UNIDAS (ONU); PROGRAMA DAS NAÇÕES UNIDAS PARA O MEIO AMBIENTE (PNUMA).
\textbf{Convenção da Basileia sobre o Controle de Movimentos Transfronteiriços de Resíduos Perigosos e seu Depósito, 1989}.
Basileia, Suíça, 22 mar. 1989.
Disponível em: \url{https://www.basel.int/Portals/4/Basel%20Convention/docs/text/con-e-rev.pdf}.
Acesso em: 29 jan. 2026.

\bibitem{CONAMA4522012}
BRASIL. Ministério do Meio Ambiente. Conselho Nacional do Meio Ambiente (CONAMA).
\textbf{Resolução CONAMA nº 452, de 2 de julho de 2012}.
Dispõe sobre os procedimentos de controle da importação de resíduos, conforme as normas adotadas pela Convenção 
da Basileia sobre o Controle de Movimentos Transfronteiriços de Resíduos Perigosos e seu Depósito.
Diário Oficial da União: Brasília–DF, 4 jul. 2012.
Disponível em: \url{https://conama.mma.gov.br/?id=656&option=com_sisconama&task=arquivo.download}.
Acesso em: 29 jan. 2026.


\end{thebibliography}
\endgroup


\end{document}
