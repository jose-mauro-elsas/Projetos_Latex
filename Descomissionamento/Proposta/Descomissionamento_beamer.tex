%!TEX program = lualatex
% Arquivo: Descomissionamento_beamer.tex
\documentclass[aspectratio=169,11pt]{beamer}

% ===== Tema / aparência =====
\usetheme{Madrid}
% Aumenta o tamanho dos números do enumerate (bolinhas numeradas)
\setbeamerfont{enumerate item}{size=\large}
\setbeamerfont{enumerate subitem}{size=\normalsize}
\setbeamerfont{enumerate subsubitem}{size=\normalsize}

\usecolortheme{default}
\setbeamertemplate{navigation symbols}{}

% ===== Fix: evita "Referências |" quando não há subseção =====
\setbeamertemplate{frametitle}{%
  \nointerlineskip%
  \begin{beamercolorbox}[wd=\paperwidth,ht=2.6ex,dp=1.2ex,leftskip=.6em,rightskip=.6em]{frametitle}%
    \usebeamerfont{frametitle}\insertframetitle%
    \ifx\insertsubsectionhead\empty\else\ \textbar\ \insertsubsectionhead\fi%
  \end{beamercolorbox}%
}

% ===== Idioma =====
\usepackage[portuguese]{babel}

% ===== Fonte (LuaLaTeX) =====
\usepackage{fontspec}
\setmainfont{Latin Modern Roman}

% ===== Figuras / tabelas =====
\usepackage{graphicx}
\usepackage{booktabs}

% ===== Links =====
\usepackage{xurl}
\Urlmuskip=0mu plus 1mu

% ===== Metadados =====
\title[Descomissionamento offshore]{Descomissionamento de UEPs e Sistemas Subsea\\de Produção}
\subtitle{Proposta de estudo para TCC (Oceanografia)}
\author[José Mauro X. Elsas]{José Mauro Xavier Elsas}
\institute[UERJ]{Universidade do Estado do Rio de Janeiro (UERJ)\\Faculdade de Oceanografia}
\date{Janeiro de 2026}

% ===== Atalhos =====
\newcommand{\UEP}{Unidade Estacionária de Produção}
\newcommand{\Subsea}{\textit{subsea}}

% ===== TikZ =====
\usepackage{tikz}
\usetikzlibrary{positioning,arrows.meta,calc,backgrounds}

\tikzset{
  process/.style={
    draw,
    rounded corners=2pt,
    thick,
    align=left,
    text width=4.9cm,
    minimum height=1.25cm,
    inner xsep=6pt,
    inner ysep=6pt,
    font=\small
  },
  process_final/.style={
    process,
    font=\small\bfseries
  },
  arrow/.style={
    -{Stealth[length=3mm,width=2.2mm]},
    thick
  },
  megaArrow/.style={
    draw,
    line width=3.2pt,
    rounded corners=10pt,
    -{Stealth[length=5mm,width=3.6mm]},
    opacity=0.65
  }
}

% ===== Citações: modo tolerante (não para a compilação se faltar bibitem) =====
% Você pode remover isto quando as referências estiverem OK.
\usepackage{cite}
\makeatletter
\renewcommand{\cite}[1]{\textsuperscript{[\@for\@tempa:=#1\do{\@tempa,}\@gobble]}}
\makeatother

\begin{document}

% ================== CAPA ==================
\begin{frame}
  \titlepage
\end{frame}

% ================== SUMÁRIO ==================
\begin{frame}{Sumário}
  \tableofcontents
\end{frame}

% ================== 1. CONTEXTO ==================
\section{Contexto e motivação}

\begin{frame}{Por que o tema importa?}
\begin{itemize}
  \item Parte relevante das instalações offshore se aproxima do fim de vida útil nominal.
  \item Decisões de descomissionamento impactam:
    \begin{itemize}
      \item segurança operacional e integridade estrutural;
      \item meio ambiente marinho e riscos de impactos ambientais;
      \item produtividade, vida útil dos ativos e impactos sociais.
    \end{itemize}
  \item O planejamento precisa ser antecipado e baseado em evidências (técnicas, ambientais e 
  regulatórias).
\end{itemize}
\end{frame}

\begin{frame}{Problema}
\begin{block}{Questão central}
Como avaliar se uma operação de descomissionamento pode ser feita com minimo impacto ambiental 
seguindo a documentação regulatória em \UEP{} e sistemas de produção \Subsea{}, no contexto brasileiro, 
conciliando requisitos normativos e boas práticas internacionais?
\end{block}

\begin{block}{Pano de fundo regulatório}
No Brasil, a Resolução ANP n.º 817/2020 estabelece diretrizes e obrigações para o descomissionamento 
\cite{ANP8172020}. Em paralelo, há referências internacionais relevantes, como diretrizes da IMO para 
remoção de instalações \cite{IMOA672}.
\end{block}
\end{frame}

% ================== 2. OBJETIVOS ==================
\section{Objetivos}

\begin{frame}{Objetivo geral}
\begin{itemize}
  \item Estruturar um anteprojeto de TCC que consolide um 
  \textbf{arcabouço técnico-ambiental-regulatório} para o descomissionamento offshore (topside+\Subsea{}), com foco no contexto brasileiro.
\end{itemize}
\end{frame}

\begin{frame}{Objetivos específicos}
\begin{enumerate}
  \item Mapear requisitos regulatórios nacionais aplicáveis (ANP, IBAMA, ICMbio, Inea).
  \item Sistematizar referências e diretrizes internacionais (IMO, boas práticas, etc.).
  \item Identificar principais impactos e receptores ambientais marinhos ao longo das etapas do descomissionamento.
  \item Propor um roteiro metodológico de análise (dados, premissas, critérios e entregáveis).
  \item Avaliar a relação entre ambiente oceanográfico, profundidade e complexidade do descomissionamento;
  \end{enumerate}
\end{frame}

% ================== 3. MARCO REGULATÓRIO ==================
\section{Marco regulatório e referências}

\begin{frame}{Regulação nacional (exemplo de base)}
\begin{itemize}
  \item \textbf{ANP}:
    \begin{itemize}
      \item Resolução nº 817/2020: regras para descomissionamento, devolução de áreas e reversão de bens \cite{ANP8172020}.
      \item Portal institucional sobre descomissionamento (visão geral e documentos) \cite{ANPDescomissionamentoPortal}.
    \end{itemize}
\end{itemize}

\begin{block}{Nota}
Neste anteprojeto, a regulação nacional entra como \textbf{restrição} e \textbf{critério de conformidade} para qualquer alternativa técnica.
\end{block}
\end{frame}

\begin{frame}{Referências internacionais (exemplos)}
\begin{itemize}
  \item \textbf{IMO}:
    \begin{itemize}
      \item Diretrizes para remoção de instalações no mar (Res. A.672(16)) \cite{IMOA672}.
      \item Convenção de Hong Kong (reciclagem de navios) — relevante para destinos/rotas de sucateamento e aspectos ambientais \cite{HKC2009}.
    \end{itemize}
  \item \textbf{PSA Norway}: requisitos para extensão de vida útil (quando aplicável antes do descomissionamento) \cite{PSANorwayLifeExtension2007}.
  \item \textbf{API}: prática recomendada para plataformas fixas (base para integridade/estrutura em certos contextos) \cite{API2AWSD}.
\end{itemize}
\end{frame}

% ================== 4. ESCOPO E ETAPAS ==================
\section{Escopo e etapas do descomissionamento}

\begin{frame}{Escopo técnico}
\begin{itemize}
  \item Integração topside + \Subsea{} no planejamento e execução.
  \item Cadeia de decisão: engenharia \(\rightarrow\) conformidade \(\rightarrow\) execução \(\rightarrow\) verificação.
  \item Considerar:
  \begin{itemize}
    \item logística offshore e janelas meteoceanográficas;
    \item gestão de resíduos e limpeza;
    \item monitoramento e critérios de aceitação pós-descomissionamento.
  \end{itemize}
\end{itemize}
\end{frame}

% === Frame SÓ do diagrama (sem frame dentro de frame) ===
\begin{frame}[t]{Fluxo típico de descomissionamento offshore}
\centering

\resizebox{0.6\linewidth}{!}{%
\begin{tikzpicture}[node distance=1.05cm and 1cm]

\node (s1) [process] {1. Estudos de engenharia, impacto ambiental e documentação};
\node (s2) [process, right=of s1] {2. Abandono dos poços conectados};

\node (s4) [process, below=of s2] {4. Remoção da estrutura (jaqueta, ancoragem ou CGBS)};
\node (s3) [process, left=of s4] {3. Desmobilização do \textit{topside} (produção, habitação, energia)};

\node (s5) [process, below=of s3] {5. Remoção de linhas, manifolds, ANMs e válvulas};
\node (s6) [process, right=of s5] {6. Resíduos, limpeza e recuperação da área};

\node (s7) [process_final, below=of s6, xshift=-3.45cm] {7. Desmantelamento e reciclagem em terra};

\draw [arrow] (s1) -- (s2);
\draw [arrow] (s2) -- (s4);
\draw [arrow] (s4) -- (s3);
\draw [arrow] (s3) -- (s5);
\draw [arrow] (s5) -- (s6);
\draw [arrow] (s6) -- (s7);

% \begin{scope}[on background layer]
%   \draw [megaArrow]
%     ($(s1.south west)+(-0.15,-0.35)$)
%       -- ($(s2.south east)+( 0.15,-0.35)$)
%       -- ($(s4.south east)+( 0.15,-0.35)$)
%       -- ($(s3.south west)+(-0.15,-0.35)$)
%       -- ($(s5.south west)+(-0.15,-0.35)$)
%       -- ($(s6.south east)+( 0.15,-0.35)$)
%       -- ($(s7.south east)+( 0.15,-0.35)$);
% \end{scope}

\end{tikzpicture}%
}

\vspace{0.4em}
{\footnotesize Fluxo típico de descomissionamento offshore (visão resumida).}

\end{frame}

% ================== 5. METODOLOGIA ==================
\section{Metodologia proposta}

\begin{frame}{Metodologia}
\begin{enumerate}
  \item \textbf{Levantamento documental}: normas, guias, legislação e publicações técnicas.
  IMO, ANP, e outras.
  \item \textbf{Estruturação do processo}: etapas, decisões-chave, atores e evidências requeridas.
  \item \textbf{Matriz de aspectos e impactos}: por etapa (topside/\Subsea{}) e por receptor ambiental.
  \item \textbf{Critérios de avaliação}: conformidade, risco, custo, prazo, viabilidade logística.
  \item \textbf{Síntese}: proposta de roteiro/fluxo para estudo de caso de um ou mais campos. Exemplo: 
  Marlim e Voador.
\end{enumerate}
\end{frame}

%=====================Entregáveis=====================

\begin{frame}{Entregáveis esperados}
\begin{itemize}
  \item Quadro de localização dos ativos a serem descomissionados em 5 anos.
  \item Quadro comparativo: requisitos nacionais vs. diretrizes internacionais.
  \item Análise das as operações de descomissionamento de plataformas “offshore” no Brasil sob uma perspetiva
integrada, considerando aspectos técnicos, ambientais e oceanográficos.
  \item Análise sobre futuras possibilidades de descomissionamento com menor impacto ambiental.
\end{itemize}
\end{frame}

% ================== 6. CRONOGRAMA ==================
\section{Proposta de execução}

\begin{frame}{Cronograma (macro)}
\begin{tabular}{p{0.22\linewidth} p{0.70\linewidth}}
\toprule
\textbf{Fase} & \textbf{Atividades}\\
\midrule
1. Base & Revisão bibliográfica; mapeamento regulatório; organização de referências\\
2. Modelo & Definição de escopo; estudos de impactos; critérios e roteiro\\
3. Aplicação & Estudo de caso (se aplicável); consolidação de resultados\\
4. Escrita & Redação final; revisão; formatação.\\
\bottomrule
\end{tabular}
\end{frame}

% % ================== REFERÊNCIAS ==================
% \section{Referências}

% \begin{frame}[allowframebreaks]{Referências}
% \tiny

% % Se você tem este arquivo, ele deve conter \bibitem{...}
% % Se não tiver, comente este \input por enquanto.
% \begingroup
%   \let\addcontentsline\relax
%   \let\section\relax
%   \let\subsection\relax
%   \let\subsubsection\relax
%   \begingroup
\sloppy
\setlength{\emergencystretch}{2em}
\raggedright

\phantomsection
\addcontentsline{toc}{section}{Referências}
\begin{thebibliography}{99}

\bibitem{ANP8172020}
AGÊNCIA NACIONAL DO PETRÓLEO, GÁS NATURAL E BIOCOMBUSTÍVEIS (ANP).
\textbf{\label{Resolução ANP nº 817, de 24 de abril de 2020}}.
Dispõe sobre o descomissionamento de instalações de exploração e produção de petróleo e gás natural, devolução de áreas e reversão de bens.
Diário Oficial da União: Brasília–DF, 27 abr. 2020.
Disponível em: \url{https://www.in.gov.br/web/dou/-/resolucao-n-817-de-24-de-abril-de-2020-254001378}.
Acesso em: 12 jan. 2026.

% \bibitem{ANP8542021}
% AGÊNCIA NACIONAL DO PETRÓLEO, GÁS NATURAL E BIOCOMBUSTÍVEIS (ANP).
% \textbf{Resolução ANP nº 854, de 27 de setembro de 2021}.
% Regulamenta os procedimentos para apresentação de garantias financeiras que assegurem recursos para o descomissionamento de instalações.
% Diário Oficial da União: Brasília, DF, 29 set. 2021.
% Disponível em: \url{https://www.in.gov.br/en/web/dou/-/resolucao-anp-n-854-de-27-de-setembro-de-2021-348273146}.
% Acesso em: 24 jan. 2026.

\bibitem{ANPDescomissionamentoPortal}
AGÊNCIA NACIONAL DO PETRÓLEO, GÁS NATURAL E BIOCOMBUSTÍVEIS (ANP).
\textbf{Descomissionamento de instalações}.
Portal gov.br, [s.l.], 2020.
Disponível em: \url{https://www.gov.br/anp/pt-br/assuntos/exploracao-e-producao-de-oleo-e-gas/seguranca-operacional/descomissionamento-de-instalacoes}.
Acesso em: 12 jan. 2026.

\bibitem{HKC2009}
ORGANIZAÇÃO MARÍTIMA INTERNACIONAL (IMO).
\textbf{Convenção Internacional de Hong Kong para a Reciclagem Segura e Ambientalmente Adequada de Navios, 2009}.
Hong Kong, China, 15 maio 2009.
Disponível em: \url{https://www.imo.org/en/About/Conventions/Pages/The-Hong-Kong-International-Convention-for-the-Safe-and-Environmentally-Sound-Recycling-of-Ships.aspx}.
Acesso em: 14 jan. 2026.

% \bibitem{ANPGarantiasPortal}
% AGÊNCIA NACIONAL DO PETRÓLEO, GÁS NATURAL E BIOCOMBUSTÍVEIS (ANP).
% \textbf{Garantias financeiras de descomissionamento}.
% Portal gov.br, [s.l.], 2021.
% Disponível em: \url{https://www.gov.br/anp/pt-br/assuntos/exploracao-e-producao-de-oleo-e-gas/desenvolvimento-e-producao/garantias-financeiras-de-descomissionamento}.
% Acesso em: 24 jan. 2026.

% \bibitem{ANPDadosAbertosPlataformasOperacao}
% AGÊNCIA NACIONAL DO PETRÓLEO, GÁS NATURAL E BIOCOMBUSTÍVEIS (ANP).
% \textbf{Dados abertos: plataformas em operação} (arquivo CSV).
% Rio de Janeiro: ANP, [s.d.].
% Disponível em: \url{https://www.gov.br/anp/pt-br/centrais-de-conteudo/dados-abertos/arquivos/arquivos-fase-de-desenvolvimento-e-producao/lpo/dados-abertos-plataformas-operacao.csv}.
% Acesso em: 24 jan. 2026.

% \bibitem{ISO19900}
% INTERNATIONAL ORGANIZATION FOR STANDARDIZATION (ISO).
% \textbf{ISO 19900:2019 -- Petroleum and natural gas industries -- General requirements for offshore structures}.
% Geneva: ISO, 2019.
% Disponível em: \url{https://www.iso.org/standard/69761.html}.
% Acesso em: 24 jan. 2026.

% \bibitem{ISO19901-1}
% INTERNATIONAL ORGANIZATION FOR STANDARDIZATION (ISO).
% \textbf{ISO 19901-1:2015 -- Petroleum and natural gas industries -- Specific requirements for offshore structures -- Part 1: Metocean design and operating considerations}.
% Geneva: ISO, 2015.
% Disponível em: \url{https://www.iso.org/standard/60183.html}.
% Acesso em: 24 jan. 2026.

% \bibitem{ISO19902}
% INTERNATIONAL ORGANIZATION FOR STANDARDIZATION (ISO).
% \textbf{ISO 19902:2020 -- Petroleum and natural gas industries -- Fixed steel offshore structures}.
% Geneva: ISO, 2020.
% Disponível em: \url{https://www.iso.org/standard/65688.html}.
% Acesso em: 24 jan. 2026.

\bibitem{API2AWSD}
AMERICAN PETROLEUM INSTITUTE (API).
\textbf{API Recommended Practice 2A-WSD: Planning, Designing, and Constructing 
Fixed Offshore Platforms — Working Stress Design}.
22. ed. Washington, DC: API, nov. 2014.
Disponível em: \url{https://www.api.org/~/media/files/publications/whats%20new/2a-wsd_e22%20pa.pdf}.
Acesso em: 20 jan. 2026.

\bibitem{IMOA672}
INTERNATIONAL MARITIME ORGANIZATION (IMO).
\textbf{Resolution A.672(16): Guidelines and Standards for the Removal of Offshore
 Installations and Structures on the Continental Shelf and in the Exclusive Economic 
 Zone}.
Adopted on 19 Oct. 1989.
London: IMO, 1989.
Disponível em: \url{https://wwwcdn.imo.org/localresources/en/KnowledgeCentre/IndexofIMOResolutions/AssemblyDocuments/A.672%2816%29.pdf}.
Acesso em: 24 jan. 2026.

\bibitem{PSANorwayLifeExtension2007}
PETROLEUM SAFETY AUTHORITY NORWAY (PSA).
\textbf{Requirements for Life Extension of Ageing Offshore Production Installations}.
Norway: PSA, 2007.
Disponível em: \url{https://www.havtil.no/contentassets/24974571fd8442bea4c21d3679d93a8e/requirements-for-life-extension-of-ageing-offshore-production-installations.pdf}.
Acesso em: 10 jan. 2026.

% \bibitem{IOGP584}
% INTERNATIONAL ASSOCIATION OF OIL \& GAS PRODUCERS (IOGP).
% \textbf{Overview of International Offshore Decommissioning Regulations -- Volume 1: Facilities}.
% IOGP Report 584. London: IOGP, jul. 2017.
% Disponível em: \url{https://www.iogp.org/bookstore/product/overview-of-international-offshore-decommissioning-regulations-volume-1-facilities/}.
% Acesso em: 24 jan. 2026.

% \bibitem{IOGP585}
% INTERNATIONAL ASSOCIATION OF OIL \& GAS PRODUCERS (IOGP).
% \textbf{Overview of International Offshore Decommissioning Regulations -- Volume 2: Wells Plugging \& Abandonment}.
% IOGP Report 585. London: IOGP, maio 2023.
% Disponível em: \url{https://www.iogp.org/bookstore/product/overview-of-international-offshore-decommissioning-regulations-volume-2-wells-plugging-abandonment/}.
% Acesso em: 24 jan. 2026.

% \bibitem{IOGP484}
% INTERNATIONAL ASSOCIATION OF OIL \& GAS PRODUCERS (IOGP).
% \textbf{Decommissioning of Offshore Concrete Gravity Based Structures (CGBS) in the OSPAR Maritime Area / Other Global Regions}.
% Report 484-1. London: IOGP, 2017.
% Disponível em: \url{https://www.iogp.org/bookstore/wp-content/uploads/sites/2/woocommerce_uploads/2017/01/484-1.pdf}.
% Acesso em: 24 jan. 2026.

\bibitem{Basileia1989}
ORGANIZAÇÃO DAS NAÇÕES UNIDAS (ONU); PROGRAMA DAS NAÇÕES UNIDAS PARA O MEIO AMBIENTE (PNUMA).
\textbf{Convenção da Basileia sobre o Controle de Movimentos Transfronteiriços de Resíduos Perigosos e seu Depósito, 1989}.
Basileia, Suíça, 22 mar. 1989.
Disponível em: \url{https://www.basel.int/Portals/4/Basel%20Convention/docs/text/con-e-rev.pdf}.
Acesso em: 29 jan. 2026.

\bibitem{CONAMA4522012}
BRASIL. Ministério do Meio Ambiente. Conselho Nacional do Meio Ambiente (CONAMA).
\textbf{Resolução CONAMA nº 452, de 2 de julho de 2012}.
Dispõe sobre os procedimentos de controle da importação de resíduos, conforme as normas adotadas pela Convenção 
da Basileia sobre o Controle de Movimentos Transfronteiriços de Resíduos Perigosos e seu Depósito.
Diário Oficial da União: Brasília–DF, 4 jul. 2012.
Disponível em: \url{https://conama.mma.gov.br/?id=656&option=com_sisconama&task=arquivo.download}.
Acesso em: 29 jan. 2026.


\end{thebibliography}
\endgroup

% \endgroup

% \end{frame}

\end{document}
