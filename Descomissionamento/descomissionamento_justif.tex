O descomissionamento offshore representa uma etapa crítica do ciclo de vida das
instalações de produção, que têm implicações diretas sobre o meio ambiente marinho,
a segurança operacional e a governança dos recursos naturais, muito embora estruturas
de jaquetas sejam projetadas conforme normas como a API 2AW-SD \cite{API2AWSD}.
No Brasil, o aumento do número de plataformas próximas
ao fim de sua vida útil nominal impõe a necessidade de planejamento antecipado e de avaliações
técnicas e ambientais robustas.

Depois que o sistema de produção atinge o fim da vida útil, pode passar por obras de
revitalização que o enquadram nos requisitos normativos adequados. Como, por exemplo, o
\textit{Requirements for Life Extension of Ageing Offshore Production Installations},
norma essa que propõe, para casos de extensão da vida útil de uma instalação de 
produção, \cite{PSANorwayLifeExtension2007}:

\begin{quotation}
    \textit{\(\dots\) to demonstrate and document that the installation is fit for
    continued safe operation during the life extension period. The application should describe
    the barrier systems that underwrite safety and contain an analysis of their current and
    predicted integrity and performance. The information can be presented in a synthesized
    and evaluated form, and reference made to other more detailed reviews and assessments
    as appropriate.} (2008, p.37)
\end{quotation} 

Quando termina o tempo determinado pelo projeto de aumento da vida em produção,
é iniciado o processo de descomissionamento, que deve cumprir algumas etapas:

\begin{enumerate}
    \item Estudos que precedem o descomissionamento, tais como estudos de
    engenharia, de impacto ambiental e entrega da documentação.
    \item Abandono dos poços conectados ao sistema.
    \item Desmobilização dos equipamentos \textit{topside} (planta de produção,
    módulo de habitação, compressão de gás, geração de energia.)
    \item Remoção da plataforma, propriamente dita (jaqueta, sistemas de ancoragem,
    ou \textit{Concrete Gravity Based Structure}, se houver.)
    \item Remoção de linhas rígidas e flexíveis, manifolds, válvulas e demais equipamentos.
    \item Recolhimento de resíduos, limpeza e recuperação da área.
    \item Desmantelamento e reciclagem do sistema executado em terra.
\end{enumerate}


A regulamentação e procedimentação da atividade é resultado de um histórico de muitos
acidentes e impactos ambientais, como os ocorridos ao longo da segunda metade do século XX,
quando países do Leste Asiático, beneficiados por custos trabalhistas mais baixos,
passaram a dominar a indústria de desmantelamento naval. Com o aumento progressivo dos custos,
a atividade migrou do Japão e de Hong Kong para a Coreia do Sul e, posteriormente, para a China.
O porto ao sul da cidade de Kaohsiung, em Taiwan, destacou-se como líder mundial
no final das décadas de 1960 e 1970, tendo desmontado 220 navios em 1972,
totalizando 1,6 milhão de toneladas. Em 1977, Taiwan detinha mais da metade do
mercado mundial, seguido por Espanha e Paquistão. À época, Bangladesh ainda não
possuía capacidade significativa nesse setor.

O método de reciclagem por encalhe em praias ganhou relevância no Sul da Ásia
após 1960, quando o navio grego \textit{M/D Alpine} encalhou em Sitakunda,
Chittagong (então parte do Paquistão), após um ciclone severo. Incapaz de ser
reflutuado, o navio permaneceu no local por anos, até ser adquirido e desmontado
pela Chittagong Steel House em 1965. Esse episódio é frequentemente apontado como
o marco inicial da indústria de desmantelamento em Bangladesh. Até 1980, o
Gadani Ship Breaking Yard, no Paquistão, era o maior estaleiro de desmantelamento
do mundo.

O endurecimento das normas ambientais em países industrializados elevou os
custos de descarte de resíduos perigosos, incentivando a exportação de navios
para regiões com regulamentação mais branda, especialmente no Sul da Ásia.
Dentro desse modelo de negócio, os armadores passaram a vender as embarcações para
intermediários que mudavam o nome da embarcação, o porto de registro e a bandeira para
desvincular o nome dos proprietários originais de eventuais problemas jurídicos
futuros.

O fato é que, da forma como era feito o desmantelamento de embarcações, vemos que:
\begin{itemize}
    \item desigualdades sociais se aprofundavam, 
    \item um grupo de intermediários e donos de áreas de desmantelamento se tornaram
    milionários,
    \item os perigos de acidentes pessoais vitimavam  pessoas comuns e,
    \item o impacto ambiental atingia diretamente o litoral da região. 
 \end{itemize}

No ano de 1989 a convenção da Basileia, \cite{Basileia1989} que trata de transporte transnacional de
cargas perigosas definiu de como de deveria se lidar com materiais perigosos. 
Embora a Convenção da Basileia não citasse diretamente navios, embarcações ou
plataformas, alguns materiais usados na construção naval são listados no Anexo VIII
da Convenção.

Como os materiais são descritos no anexo VIII citado, mas o termo "navio"  ou "embarcação"
(destinada ao descomissionamento) não é explicitamente citado, a IMO publicou em 2009
com total clareza a Convenção de Hong Kong. Como efeito geral, tivemos a implantação
de estaleiros ambientalmente conformes em alguns países.

A legislação brasileira (Resolução CONAMA n.º 452, de 2 de julho de 2012.) \cite{CONAMA4522012} aplica a 
Basileia para navios porque o STF e os tribunais superiores brasileiros já 
consolidaram o entendimento de que a mistura de materiais perigosos (amianto, óleos)
torna o objeto inteiro um resíduo perigoso.

O caso de referência é o do porta-aviões São Paulo, exemplo perfeito dessa discrepância entre o texto da lei e a prática. O navio que deveria ter sido desmantelado na Turquia, não teve autorização de atracação porque o governo turco (via Ministério do Meio Ambiente) cancelou a sua autorização de entrada. O motivo foi a divergência no inventário de materiais perigosos (especialmente amianto). Eles alegaram que o Brasil não provou que o navio estava "limpo " o suficiente, ferindo o princípio do Consentimento Prévio Informado. Por isso foi devolvido e teve de cruzar o Atlântico de volta. Na volta, o Porto de Santos e a Mari nha não permitiram seu reingresso em águas interiores por risco ambiental e de naufrágio.

Tendo ficado por muito tempo fundeado ao largo da costa de Pernambuco, hoje não está mais lá. Após vários meses na costa de Pernambuco, a Marinha do Brasil optou pelo afundamento controlado em fevereiro de 2023, a cerca de 350 km da costa, em águas profundas.
Organizações como a Basel Action Network (BAN) denunciaram o Brasil, afirmando que o afundamento violou a Convenção de Basileia, já que o navio (cheio de resíduos tóxicos) foi "descartado" no oceano em vez de ser reciclado.

Essa atividade, intensiva em mão de obra que representa, certamente, um mercado em
expansão, deve atender a regulações de diferentes áreas, como a ambiental e a trabalhista. Tais regulações, cada uma em seu respectivo domínio, têm por finalidade 
evitar acidentes com vítimas e/ou a ocorrência de impactos ambientais significativos.
Algumas áreas costeiras da Ásia e da África operaram por longo período — e, em lguns 
casos, ainda operam — sob condições que não atendem às disposições da Convenção Internacional de Hong Kong para a Reciclagem Segura e Ambientalmente Adequada de navios, de 
2009 (IMO) \cite{HKC2009}, a qual estabelece, como obrigação primária de cada Estado signatário:

\begin{quotation}
    $\dots$ desta Convenção se compromete a pôr em execução, de maneira plena e
    completa, suas disposições, de modo a prevenir, reduzir, minimizar e, dentro do possível, eliminar acidentes, ferimentos e outros efeitos adversos sobre a saúde 
    humana e o meio ambiente causados pela reciclagem de navios, e a aumentar a segurança das embarcações e a proteção da saúde humana e do meio ambiente durante toda a vida útil de 
    um navio. (2009, p.2)
\end{quotation}

A resolução da ANP \cite{ANP8172020} indica matrizes de competências e de normas
\cite{ANPDescomissionamentoPortal} que consideram, inclusive, normas para tratamento de
materiais radioativos. Uma vez que pode haver radioatividade no poço ou no reservatório,
que acaba por contaminar não só os materiais e equipamentos introduzidos no
poço, bem como todos os equipamentos que integram o sistema de produção, como ANMs.

Do ponto de vista oceanográfico, a remoção ou manutenção de estruturas offshore
interfere em processos como a ressuspensão de sedimentos, a dispersão de
contaminantes, a modificação de habitats bentônicos e a conectividade ecológica.
Entretanto, tais aspectos ainda são pouco explorados integradamente no contexto
nacional, especialmente quando comparados a bacias maduras internacionais. Justifica-se, portanto, 
a preocupação com os impactos ambientais, uma vez que a maioria dos equipamentos está instalada no 
assoalho marinho, e não na UEP, que pode ser removida para desmonte e reciclagem em terra.

Toda a parte \textit{subsea} do sistema de produção precisa ser limpa e inertizada
antes de sua desconexão da UEP e dos equipamentos de fundo. As operações de recolhimento
de linhas flexíveis, rígidas, risers e manifolds precisam ser executadas para
minimizar riscos de acidentes pessoais ou ambientais.

Assim, justifica-se a realização deste estudo como contribuição técnico-científica
para o entendimento do descomissionamento offshore no Brasil, com ênfase nos
condicionantes ambientais e oceanográficos que influenciam o planejamento e a
execução dessas operações.
