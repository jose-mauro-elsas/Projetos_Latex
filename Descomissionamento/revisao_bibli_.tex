%!TEX program = lualatex
\documentclass[a4paper,12pt]{article}

% =========================================================
% Idioma
% =========================================================
\usepackage[portuguese]{babel}

% =========================================================
% Fonte e acentuação (LuaLaTeX)
% =========================================================
\usepackage{fontspec}
\setmainfont{Latin Modern Roman}

% =========================================================
% Layout da página
% =========================================================
\usepackage[
  top=3cm,
  bottom=2cm,
  left=3cm,
  right=2cm
]{geometry}

% =========================================================
% Parágrafos e espaçamento
% =========================================================
\usepackage{indentfirst}   % indenta o primeiro parágrafo
\usepackage{setspace}
\onehalfspacing
\setlength{\parindent}{1.25cm}

% =========================================================
% Matemática básica
% =========================================================
\usepackage{amsmath,amssymb}

% =========================================================
% Figuras e tabelas
% =========================================================
\usepackage{graphicx}
\usepackage{float}
\usepackage{booktabs}

% =========================================================
% Controle fino de listas (opcional)
% =========================================================
\usepackage{enumitem}
\setlist[itemize]{noitemsep,topsep=3pt}
\setlist[enumerate]{noitemsep,topsep=3pt}

% =========================================================
% Hiperlinks (último pacote)
% =========================================================
\usepackage{hyperref}
\hypersetup{
  colorlinks=true,
  linkcolor=black,
  citecolor=black,
  urlcolor=blue
}

\begin{document}
\begin{enumerate}
    \item Definição do Tema e Pergunta de Pesquisa: Escolha do tema e formulação de
     uma pergunta clara e objetiva que guiará a revisão.
    \item Levantamento de Literatura (Busca): Identificação de fontes relevantes em 
    bases de dados (ex: Google Acadêmico, Scielo), utilizando palavras-chave.
    \item Seleção de Fontes: Aplicação de critérios de inclusão e exclusão 
    (idioma, período, tipo de estudo)
 para filtrar o material levantado. 
    \item Leitura e Fichamento (Coleta): Leitura técnica (resumo, introdução e 
    conclusão) e organização dos textos principais, fichando os pontos-chave. 
    \item Análise Crítica e Síntese: Avaliação dos estudos, identificando lacunas, 
    padrões e semelhanças, organizando as informações para responder à pergunta de pesquisa.
    \item Redação da Revisão: Escrita do texto estruturado, com introdução (objetivo), desenvolvimento 
(discussão dos resultados) e conclusão, integrando as ideias dos autores. 
\end{enumerate}

\end{document}