Para a investigação de um fenômeno, temos a nossa disposição dois métodos: experimental e estatístico. Resumidamente, o \textbf{MÉTODO EXPERIMENTAL} consiste em manter constantes as causas (fatores), com exceção de uma, variada para que seus efeitos sejam descobertos. Contudo, nem sempre poderemos aplicar o método experimental, pois os fatores que afetam um fenômeno podem não permanecer constantes enquanto variamos a causa que nos interessa.
Por exemplo, para analisarmos uma queda nas vendas de uma empresa nacional que produz chocolates finos, teríamos que considerar vários fatores que não necessariamente permanecerão constantes durante toda a investigação do fenômeno, tais como a região, o fluxo de turistas na localidade; a temperatura média; o preço do concorrente; o mês de férias, etc.

Assim, diante da impossibilidade de manter as causas ou fatores constantes, o \textbf{MÉTODO ESTATÍSTICO} admite e registra todas as possíveis variações das causas presentes, procurando determinar a influência de cada fator no resultado. Dessa forma, o método estatístico descobrirá relações entre os fatores, como, por exemplo, a influência da temperatura média e do fluxo de turistas na venda de chocolates finos.