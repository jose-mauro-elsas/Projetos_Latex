
Muitas vezes, certos elementos de um conjunto de dados possuem relevância maior que os demais. Nessa situação, para calcular a média de tais conjuntos, devemos encontrar uma média ponderada. Uma média ponderada é a média de um conjunto de dados cujos valores possuem pesos variados.

Ela é calculada pela igualdade a seguir, em que \(p\) é o peso de cada valor de \(x\):
\medskip
\[
    \bar{x}=\frac{\sum_{i=1}^{n}(x_i\times p_1)}{\sum_{i=1}^{n}p_i}
\]

Observe que no numerador cada valor será multiplicado pelo seu respectivo peso, enquanto no denominador teremos a soma de todos os pesos.

Suponha que um candidato tenha prestado um concurso público para o cargo de Auditor Fiscal, alcançando as seguintes notas:

\begin{table}[H]
    \centering
\begin{tabular}{lr}
\toprule
Disciplina                  &Nota \(x_i\)\\
\midrule
Língua Portuguesa           &4,0\\
Direito Administrativo      &4,0\\
Direito Constitucional      &4,0\\
Direito Tributário          &7,0\\
Legislação Tributária       &7,0\\
Contabilidade               &8,0\\
Auditoria                   &8,0\\
\bottomrule
\end{tabular}
\caption{Notas.}
\end{table}

Considere, também, que o edital desse concurso previa que algumas disciplinas teriam importância maior do que outras, por isso foram atribuídos pesos diferentes às várias disciplinas. Digamos que os
pesos tenham sido distribuídos da seguinte forma:

\begin{table}[H]
    \centering
\begin{tabular}{lr}
\toprule
Disciplina                  &Peso \(p_i\)\\
\midrule
Língua Portuguesa           &1\\
Direito Administrativo      &2\\
Direito Constitucional      &2\\
Direito Tributário          &3\\
Legislação Tributária       &3\\
Contabilidade               &3\\
Auditoria                   &4\\
\bottomrule
\end{tabular}
\caption{Pesos.}
\end{table}

Agora, admita que o candidato deveria alcançar uma nota 7,0 ou superior na prova objetiva para que fosse convocado para a etapa discursiva. Se você fosse um dos avaliadores desse concurso, você consideraria o candidato aprovado na prova objetiva?

Para responder a esse questionamento, devemos calcular a média aritmética ponderada desse candidato, considerando os pesos de cada disciplina. Dessa forma, devemos multiplicar cada nota pelo seu respectivo peso, somar esses produtos e dividir pela soma dos pesos.

\begin{table}[H]
    \centering
\begin{tabular}{lccc}
\toprule
Disciplina                  &Notas \(x_i\)& Pesos \(p_i\) &\(x_i\times p_i\)\\
\midrule
Língua Portuguesa           &4,0         &1 & \(4,0\times 1\)=4,0\\
Direito Administrativo      &4,0         &2 & \(4,0\times 2\)=8,0\\
Direito Constitucional      &4,0         &2 & \(4,0\times 2\)=8,0\\
Direito Tributário          &6,0         &3 & \(7,0\times 3\)=18,0\\
Legislação Tributária       &6,0         &3 & \(7,0\times 3\)=18,0\\
Contabilidade               &7,0         &3 & \(8,0\times 3\)=21,0\\
Auditoria                   &7,0         &3 & \(8,0\times 3\)=21,0\\
\bottomrule
\end{tabular}
\caption{Notas.}
\end{table}

Nesse ponto, temos uma lista contendo todos os produtos de notas e pesos. Então, a média aritmética
ponderada é dada por:

\[
    \bar{x}=\frac{\sum_{i=1}^{n}(x_i\times p_1)}{\sum_{i=1}^{n}}=\frac{4,0+8,0+8,0+18,0+18,0+21,0+21,0}{1+2+2+3+3+3+3}=\frac{98}{17}=5,76
\]