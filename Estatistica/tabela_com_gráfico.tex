\begin{table}[H]
    \centering
    \begin{tabular}{cc}
        \toprule
        \shortstack{Nível de \\Satisfação \(X_i\)} & \shortstack{Frequência \\\(f_i\)}\\
        \midrule
        0   & 5\\
        1   & 5\\
        2   & 8\\
        3   & 10\\
        4   & 13\\
        5   & 10\\
        \bottomrule
    \end{tabular}
    \caption{Frequência de níveis de Satisfação}
\end{table}

%================================
% ===== Fonte única de dados =====
\pgfplotstableread{
Xi fi
0  5
1  5
2  8
3  10
4  13
5  10
}\Satisf

% ===== TABELA (gerada dos dados) =====
\begin{table}[H]
  \centering
  \pgfplotstabletypeset[
    columns={Xi,fi},
    columns/Xi/.style={column name=\shortstack{Nível de\\Satisfação \(X_i\)}},
    columns/fi/.style={column name=\shortstack{Frequência\\\(f_i\)}},
    every head row/.style={before row=\toprule, after row=\midrule},
    every last row/.style={after row=\bottomrule},
    col sep=space,
  ]{\Satisf}
  \caption{Frequência de níveis de Satisfação}
\end{table}

% ===== GRÁFICO (gerado dos mesmos dados) =====
\begin{figure}[H]
  \centering
  \begin{tikzpicture}
    \begin{axis}[
      ybar,
      width=11cm,
      height=6.5cm,
      bar width=28pt,
      grid=both,
      xlabel={Nível de Satisfação (\(X_i\))},
      ylabel={Frequência (\(f_i\))},
      xtick=data,
      ymin=0,
    ]
      \addplot table[x=Xi, y=fi] {\Satisf};
    \end{axis}
  \end{tikzpicture}
  \caption{Gráfico de barras da frequência por nível de satisfação}
\end{figure}