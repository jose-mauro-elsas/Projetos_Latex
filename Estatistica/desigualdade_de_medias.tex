Dada uma lista de .. números positivos, \({x_1, x_2, \dots, x_n}\), podemos afirmar que:

\[
    \boxed{\bar{x}\geq G \geq H}
\]

em que \(\bar{x}\) é a média aritmética; \(G\) é a média geométrica e \(H\) é a média harmônica.
Significa dizer que a média aritmética será sempre maior ou igual à média geométrica que, por sua vez, será sempre maior ou igual à média harmônica. A igualdade ocorrerá quando os números da lista forem todos iguais.

Tomemos como exemplo os números 4, 12 e 20. Como sabemos, a média aritmética será:

\[
    \bar{x}=\frac{4+12+20}{3}=12
\]

A média geométrica é

\[
    G=\sqrt[3]{4\times 12\times 20}\approx9,86
\]

E a média harmônica é:

\[
    H=\frac{3}{\frac{1}{4}+\frac{1}{12}+\frac{1}{20}}\approx2,61
\]
Obtivemos, portanto, uma média aritmética \(\bar{x} = 12\) maior que a média geométrica \(G = 9,86\) que, por sua vez, é maior que a média harmônica \(H = 2,61\).

Agora, analisaremos um caso em que as três médias são iguais: considere uma lista composta pelos números 5, 5 e 5. Nesse caso, temos que \(\bar{x}, G e H\) são, respectivamente:

\[
    \bar{x}=\frac{5+5+5}{3}=5
\]

A média geométrica é

\[
    G=\sqrt[3]{5\times 5\times 5}=5
\]

E a média harmônica é:

\[
    H=\frac{3}{\frac{1}{5}+\frac{1}{5}+\frac{1}{5}}=5
\]

Portanto, quando todos os números da lista são iguais, as média aritmética \(\bar{x}\), geométrica \(G\) e harmônica \(H\) também são iguais.