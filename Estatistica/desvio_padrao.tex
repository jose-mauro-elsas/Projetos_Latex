O desvio padrão \((s ou \sigma)\) é definido como sendo a raiz quadrada da média aritmética dos quadrados dos
desvios e, dessa forma, é determinado pela raiz quadrada da variância. É uma das medidas de variabilidade
mais utilizadas porque consegue apontar de forma mais precisa a dispersão dos valores em relação à
média aritmética.

Valores muito próximos da média resultarão em um desvio-padrão pequeno, enquanto valores mais espalhados levarão a desvios maiores. Essa medida será sempre maior ou igual a zero. Ela será igual a zero quando todos os elementos do conjunto forem iguais.

O desvio padrão é utilizado para comparar a variabilidade de dois conjuntos de dados diferentes quando as médias forem aproximadamente iguais e quando as unidades de medidas para os dois conjuntos forem idênticas.

A fórmula para o cálculo do desvio padrão populacional é:
\[
\sigma^2 =
\frac{\sum_{i=1}^{n}(x_i-\mu)^2}{n}
\]
Para o desvio padrão amostral, a fórmula é a seguinte:

\[
s = \sqrt{\frac{\sum_{i=1}^{m}(x_i-\bar{x})^2}{n-1}}
\]
Como vimos no tópico anterior, a utilização do divisor $(n-1)$ resulta em uma melhor estimativa do
parâmetro populacional. Além disso, como a soma dos desvios em relação à média aritmética é sempre nula, somente $(n-1)$ dos desvios $(x_i-\bar{x})$ são independentes, uma vez que esses $(n-1)$ desvios determinam automaticamente o valor desconhecido.

Por fim, o desvio-padrão é expresso nas mesmas unidades dos dados originais. Tanto o desvio padrão como a variância são usados como medidas de dispersão ou variabilidade. O uso de uma medida ou de outra dependerá da finalidade que se tiver em mente.

O desvio-padrão será igual a zero quando todos os elementos forem iguais. Se todos os elementos forem iguais, a média aritmética do conjunto será igual ao valor dos elementos e todos os desvios
também serão iguais a zero. Logo, o desvio-padrão também será zero.

O desvio-padrão é sempre maior ou igual a zero, isto é, sempre tem valor positivo.

\subsection{Desvio-padrão para dados não-agrupados}

Para dados não agrupados, o desvio-padrão pode ser expresso por meio das seguintes fórmulas:

\begin{enumerate}[label=\alph*.]
    \item para populações;
\[
\sigma=\sqrt{\frac{\sum_{i=1}^{n}d_i^2}{n}}
=\sqrt{\frac{\sum_{i=1}^{n}(x_i-\mu)^2}{n}}
\]
    \item para amostras;
\[
s=\sqrt{\frac{\sum_{i=1}^{n}d_i^2}{n-1}}
=\sqrt{\frac{\sum_{i=1}^{n}(x_i-\bar{x})^2}{n-1}}
\]
\end{enumerate}

\begin{example}
    Vamos calcular o desvio-padrão amostral do conjunto de números mostrado a seguir:
    \begin{center}
        \{1, 2, 3, 5, 9\}
    \end{center}
    Iniciaremos pelo cálculo da média aritmética:
    \begin{align*}
        \bar{x}=\frac{1+2+3+5+9}{5}=\frac{20}{5}=4
    \end{align*}
    Em seguida, montaremos uma tabela para facilitar o cálculo do desvio padrão:

    \begin{table}[ht]
        \centering
            \begin{tabular}{cc}
                \toprule
            $x_i$& $(x_i-\bar{x})$ \\
            \midrule
            1 & \((1-4)^2=9\) \\
            3 & \((1-4)^2=4\) \\
            5 & \((1-4)^2=1\) \\
            7 & \((1-4)^2=1\) \\
            9 & \((1-4)^2=25\)\\
            \midrule 
              & \(\sum (x_i-\bar{x})^2=40\)\\
                \bottomrule
            \end{tabular}
        \caption{}
        \label{tab:despav_amost}
        \end{table}
    Por fim, aplicando a fórmula do desvio padrão temos:
\[
    s=\sqrt{\frac{\sum_{i=1}^{n}(x_i-\bar{x})^2}{n-1}}=\sqrt{\frac{40}{5-1}}=\sqrt{10}\cong3.16
\]
    \end{example}
\subsection{Desvio-padrão para dados agrupados sem intervalo de Classe}

Quando os valores vierem dispostos em uma tabela de frequências, o desvio-padrão será calculado por meio
de uma das seguintes fórmulas:
\begin{enumerate}[label=\alph*.]
    \item para populações;
    \[
\sigma=\sqrt{\frac{\sum_{i=1}^{n}(d_i\times f_i)^2}{n}}
=\sqrt{\frac{\sum_{i=1}^{n}[(X_i-\mu)^2\times f_i]}{n}}
    \]
    \item para amostras:
    \[
s=\sqrt{\frac{\sum_{i=1}^{n}(d_i\times f_i)^2}{n-1}}
=\sqrt{\frac{\sum_{i=1}^{n}[(X_i-\bar{x})^2\times f_i]}{n-1}}
\]  
em que
\[
    n=\sum_{i=1}^{m} \;\text{e}\; \bar{x}=\frac{\sum_{i=1}^{m X_i\bar{x}}}{n}
\]
\begin{example}
    Durante a mesma pesquisa sobre a quantidade de filhos dos professores de uma escola, produziu-se a tabela de frequências apresentada a seguir. Vamos calcular o desvio-padrão amostral dessa distribuição.

    \begin{table}[H]
                    \centering
            \begin{tabular}{ccc}
                \toprule
            \begin{tabular}[c]{@{}l@{}}Nº de   filhos \\ por professor\end{tabular} & $f_i$ & $x_i\times f_1$\\
                \midrule
            0 & 4 & 0 \times 4 = 0 \\
            1 & 8 & 1 \times 8 = 8 \\
            2 & 4 & 2 \times 4 = 8 \\
            3 & 2 & 3 \times 2 = 6 \\
            4 & 2 & 4 \times 2 = 8 \\
            \midrule
            \begin{tabular}[c]{@{}l@{}}*  Pesquisa \\ populacional\end{tabular} & $f_i$  = 20 & $x_i\times f_i  = 30$\\
            \bottomrule
            \end{tabular}
            \caption{}
            \label{tab:filhos__por_prof}
        \end{table}
    
        Iniciaremos pelo cálculo da média aritmética:
        
\end{example}
\end{enumerate}