O desvio padrão \((s ou \sigma)\) é definido como sendo a raiz quadrada da média aritmética dos quadrados dos
desvios e, dessa forma, é determinado pela raiz quadrada da variância. É uma das medidas de variabilidade
mais utilizadas porque consegue apontar de forma mais precisa a dispersão dos valores em relação à
média aritmética.

Valores muito próximos da média resultarão em um desvio-padrão pequeno, enquanto valores mais espalhados levarão a desvios maiores. Essa medida será sempre maior ou igual a zero. Ela será igual a zero quando todos os elementos do conjunto forem iguais.

O desvio padrão é utilizado para comparar a variabilidade de dois conjuntos de dados diferentes quando as médias forem aproximadamente iguais e quando as unidades de medidas para os dois conjuntos forem idênticas.

A fórmula para o cálculo do desvio padrão populacional é:
\[
\sigma^2 =
\frac{\sum_{i=1}^{n}(x_i-\mu)^2}{n}
\]
Para o desvio padrão amostral, a fórmula é a seguinte:

\[
s = \sqrt{\frac{\sum_{i=1}^{m}(x_i-\bar{x})^2}{n-1}}
\]
Como vimos no tópico anterior, a utilização do divisor $(n-1)$ resulta em uma melhor estimativa do
parâmetro populacional. Além disso, como a soma dos desvios em relação à média aritmética é sempre nula, somente $(n-1)$ dos desvios $(x_i-\bar{x})$ são independentes, uma vez que esses $(n-1)$ desvios determinam automaticamente o valor desconhecido.

Por fim, o desvio-padrão é expresso nas mesmas unidades dos dados originais. Tanto o desvio padrão como a variância são usados como medidas de dispersão ou variabilidade. O uso de uma medida ou de outra dependerá da finalidade que se tiver em mente.

O desvio-padrão será igual a zero quando todos os elementos forem iguais. Se todos os elementos forem iguais, a média aritmética do conjunto será igual ao valor dos elementos e todos os desvios
também serão iguais a zero. Logo, o desvio-padrão também será zero.

O desvio-padrão é sempre maior ou igual a zero, isto é, sempre tem valor positivo.

\subsection{Desvio-padrão para dados não-agrupados}

Para dados não agrupados, o desvio-padrão pode ser expresso por meio das seguintes fórmulas:

\begin{enumerate}[label=\alph*.]
    \item para populações;
\[
\sigma=\sqrt{\frac{\sum_{i=1}^{n}d_i^2}{n}}
=\sqrt{\frac{\sum_{i=1}^{n}(x_i-\mu)^2}{n}}
\]
    \item para amostras;
\[
s=\sqrt{\frac{\sum_{i=1}^{n}d_i^2}{n-1}}
=\sqrt{\frac{\sum_{i=1}^{n}(x_i-\bar{x})^2}{n-1}}
\]
\end{enumerate}

\begin{example}
    Vamos calcular o desvio-padrão amostral do conjunto de números mostrado a seguir:
    \begin{center}
        \{1, 2, 3, 5, 9\}
    \end{center}
    Iniciaremos pelo cálculo da média aritmética:
    \begin{align*}
        \bar{x}=\frac{1+2+3+5+9}{5}=\frac{20}{5}=4
    \end{align*}
    Em seguida, montaremos uma tabela para facilitar o cálculo do desvio padrão:

    \begin{table}[ht]
        \centering
            \begin{tabular}{cc}
                \toprule
            $x_i$& $(x_i-\bar{x})$ \\
            \midrule
            1 & \((1-4)^2=9\) \\
            3 & \((1-4)^2=4\) \\
            5 & \((1-4)^2=1\) \\
            7 & \((1-4)^2=1\) \\
            9 & \((1-4)^2=25\)\\
            \midrule 
              & \(\sum (x_i-\bar{x})^2=40\)\\
                \bottomrule
            \end{tabular}
        \caption{}
        \label{tab:despav_amost}
        \end{table}
    Por fim, aplicando a fórmula do desvio padrão temos:
\[
    s=\sqrt{\frac{\sum_{i=1}^{n}(x_i-\bar{x})^2}{n-1}}=\sqrt{\frac{40}{5-1}}=\sqrt{10}\cong3.16
\]
    \end{example}
\subsection{Desvio-padrão para dados agrupados sem intervalo de Classe}

Quando os valores vierem dispostos em uma tabela de frequências, o desvio-padrão será calculado por meio
de uma das seguintes fórmulas:
\begin{enumerate}[label=\alph*.]
    \item para populações;
    \[
\sigma=\sqrt{\frac{\sum_{i=1}^{n}(d_i\times f_i)^2}{n}}
=\sqrt{\frac{\sum_{i=1}^{n}[(X_i-\mu)^2\times f_i]}{n}}
    \]
    \item para amostras:
    \[
s=\sqrt{\frac{\sum_{i=1}^{n}(d_i\times f_i)^2}{n-1}}
=\sqrt{\frac{\sum_{i=1}^{n}[(X_i-\bar{x})^2\times f_i]}{n-1}}
\]  
em que
\[
    n=\sum_{i=1}^{m} \;\text{e}\; \bar{x}=\frac{\sum_{i=1}^{m X_i\bar{x}}}{n}
\]
\end{enumerate}
\begin{example}
    Durante a mesma pesquisa sobre a quantidade de filhos dos professores de uma escola, produziu-se a tabela de frequências apresentada a seguir. Vamos calcular o desvio-padrão amostral dessa distribuição.

    \begin{table}[H]
                    \centering
            \begin{tabular}{ccc}
                \toprule
            \begin{tabular}[c]{@{}l@{}}Nº de   filhos \\ por professor\end{tabular} & $f_i$ & $x_i\times f_1$\\
                \midrule
            0 & 4 & 0 \times 4 = 0 \\
            1 & 8 & 1 \times 8 = 8 \\
            2 & 4 & 2 \times 4 = 8 \\
            3 & 2 & 3 \times 2 = 6 \\
            4 & 2 & 4 \times 2 = 8 \\
            \midrule
            \begin{tabular}[c]{@{}l@{}}*  Pesquisa \\ populacional\end{tabular} & $f_i$  = 20 & $x_i\times f_i  = 30$\\
            \bottomrule
            \end{tabular}
            \caption{}
            \label{tab:filhos__por_prof}
        \end{table}
    
        Iniciaremos pelo cálculo da média aritmética:
        
    \begin{align*}
        \bar{x}=\frac{sum X_i \times f_i}{\sum f_i}=\frac{30}{20}=1,5\;\text{filhos por professor}
    \end{align*}

    Em seguida, adicionaremos uma nova coluna à tabela anterior, em que calcularemos os produtos dos quadrados dos desvios por suas respectivas frequências:

    \begin{table}[H]
    \centering
    \begin{tabular}{cccr}
    \toprule    
        \begin{tabular}[c]{@{}c@{}}Nº de   filhos \\ por professor\end{tabular} & $f_1$ & $x_i\times f_i$&$(x_i-\bar{x})^2\times f_i$\\
    \midrule
        0 & 4 & 0 \times 4 = 0 & $(0-1,5)^2\times 4=9$ \\
        1 & 8 & 1 \times 8 = 8 & $(1-1,5)^2\times 8=2$ \\
        2 & 4 & 2 \times 4 = 8 & $(2-1,5)^2\times 4=1$ \\
        3 & 2 & 3 \times 2 = 6 & $(3-1,5)^2\times 2=4,5$ \\
        4 & 2 & 4 \times 2 = 8 & $(4-1,5)^2\times 2=12,5$ \\
    \midrule
        \begin{tabular}[c]{@{}c@{}}*  Pesquisa \\ populacional\end{tabular} & $f_i$  = 20 & $x_i\times f_i=30$ & \multicolumn{1}{l}{}$\Big(\sum x_i-\bar{x}\Big)^2\times f_i$=29\\
        \bottomrule
    \end{tabular}
    \caption{}
    \label{tab:desvpad_filhos}
\end{table}
Por fim, aplicando a fórmula do desvio padrão amostral, temos:
\[
    s=\sqrt{\frac{\sum_{i=1}^{n}(d_i\times f_i)^2}{n-1}=\sqrt{\frac{29}{19}=\sqrt{1,52}}}=\cong 1,23    
\]
\end{example}

\subsection{Desvio-padrão para dados agrupados em classes}

Quando tivermos que calcular o desvio-padrão para dados agrupados em classes, usaremos as mesmas fórmulas para dados sem intervalos de classes, utilizando para $x_i$ os pontos médios de cada classe, mas adotando os mesmos procedimentos.

\begin{example}
    Durante uma pesquisa, feito em um grupo de estudadntes registrou as estaturas de 40 alunos, obtendo a distribuição de frequências apresentada a seguir. Calcular o desvio padrão amostral da distribuição,

        \begin{table}[H]
        \centering
        \begin{tabular}{lc}
        \hline
        Estaturas & Frequência \(f_i\) \\ \hline
        150 \vdash 154 & 4  \\
        154 \vdash 158 & 9  \\
        158 \vdash 162 & 11 \\
        162 \vdash 166 & 8  \\
        166 \vdash 170 & 5  \\
        170 \vdash 174 & 3  \\ \hline
        \shortstack{*Pesquisa \\amostral} &\(\sum f_i = 40\) \\ \hline
        \end{tabular}
        \caption{}
        \label{tab:despav__classes}
\end{table}
Inicialmente, construiremos uma tabela como a mostrada a seguir:
\begin{table}[H]
\begin{tabular}{lcccccc}
\hline
Estaturas & \begin{tabular}[c]{@{}c@{}}Frequência \\ $(f_i)$\end{tabular} & $x_i$ & $x_i \times f_i$ & $(x_i-\bar{x})$ & $(x_i \bar{x})^2$ & $(x_i - \bar{x})^2 \times f_i$ \\ \hline
150 \vdash 154 & 4  & 152 & 608 & -9   & 81  & 324 \\
154 \vdash 158 & 9  & 156 & 1.404 & -5 & 25  & 225 \\
158 \vdash 162 & 11 & 160 & 1.760 & -1 & 1   & 11 \\
162 \vdash 166 & 8  & 164 & 1.312 & 3  & 9   & 72 \\
166 \vdash 170 & 5  & 168 & 840   & 7  & 49  & 245 \\
170 \vdash 174 & 3  & 172 & 516   & 11 & 121 & 363 \\ \hline
\begin{tabular}[c]{@{}l@{}}* Pesquisa \\ amostral\end{tabular} & $\sum f_i = 40$ &  & $\sum x_i \times f_i = 6440$ &  &  & \begin{tabular}[c]{@{}c@{}}$\sum (x_i - \bar{x}) \times f_i$\\ $= 1240$\end{tabular}
\end{tabular}
\caption{}
\label{tab:desv_padr}
\end{table}
Feito isso, podemos calcular a média da distribuição por meio da seguinte fórmula:
\[
    \bar{x}=\frac{\sum x_i\times f_i}{\sum f_i}=\frac{6440}{40}=161
\]
Conhecendo a média, completamos a tabela com as diferenças e os produtos necessários para o cálculo da variância. Agora, aplicando a fórmula da variância amostral, temos:
\[
    s=\sqrt{\frac{\sum_{i=1}^{k}(x_i-\bar{x})^2}{n-1}\times f_i}
    = \sqrt{\frac{\sum_{i=1}^{6}(x_i-161)^2}{40-1}\times f_i}
    =\sqrt{\frac{1240}{39}}=\SI{5,64}{cm}
\]
O desvio-padrão das estaturas é 5,64 𝑐𝑚. Vimos anteriormente que o desvio médio, para essa mesma distribuição, foi de 4,63 cm.
\end{example}

\subsection{Propriedades do Desvio-padrão}

Nesse tópico, vamos estudar as principais propriedades do desvio-padrão.

\begin{enumerate}
    \item Somando-se (ou subtraindo-se) uma constante 𝒄 a todos os valores de uma variável, o desvio-padrão do conjunto não é alterado.
    \item Multiplicando-se (ou dividindo-se) todos os valores de uma variável por uma constante $c$, o desvio-padrão do conjunto fica multiplicado (ou dividido) por essa constante.
\end{enumerate}

\subsection{Coeficiente de variação}

O desvio-padrão pode ser utilizado para a comparação de duas ou mais séries de valores, no que diz respeito à variabilidade e dispersão, quando os conjuntos possuem a mesma média e estão expressos na mesma unidade de medida (p.ex., os dois conjuntos em centímetros). Porém, quando os conjuntos de dados estão expressos em unidades diferentes (p.ex., quilogramas e centímetros), precisamos de outra medida.

Para contornar essa limitação do desvio-padrão, podemos caracterizar a dispersão ou variabilidade dos dados de maneira relativa ao seu valor médio. Nesse sentido, o coeficiente de variação é uma medida de    dispersão relativa que fornece a variação dos dados em relação à média, podendo ser calculado como:

\begin{enumerate}[label=\alph*.]
    \item para populações;
    \begin{align*}
        CV=\frac{\sigma}{\mu}\times 100\%
    \end{align*}
    \item para amostras:
        \begin{align*}
        CV=\frac{s}{\bar{x}}\times 100\%
    \end{align*}
\end{enumerate}
em que: $\sigma$ é o desvio-padrão populacional; $\mu$ é a média populacional; $s$ é o desvio-padrão amostral; e $\bar{x}$ é a
média amostral.
\begin{example}
    Em uma empresa de tecnologia, o salário médio dos homens é de R\$ 1800,00 com desvio-padrão de R\$ 810,00 e o salário médio das mulheres é de R\$ 1500,00 com desvio padrão de R\$ 705,00. A dispersão relativa dos salários dos homens é maior que a das mulheres?

    Vamos identificar os dados do problema:
    \begin{enumerate}[label=\alph*.]
        \item para homens:
            \begin{align*}
            \begin{cases}
                \mu_H=1800\\
                \sigma_H=810
            \end{cases}
        \end{align*}
        \item para mulheres:
            \begin{align*}
            \begin{cases}
                \mu_H=1500\\
                \sigma_H=705
            \end{cases}
        \end{align*}
    \end{enumerate}
Agora, vamos calcular os respectivos coeficientes de variação:
    \begin{enumerate}[label=\alph*.]
        \item para homens;
            \begin{align*}
                CV=\frac{\sigma_H}{\mu_H}\times 100\%=\frac{810}{1800}=45\%
            \end{align*}
        \item para mulheres
            \begin{align*}
                CV=\frac{\sigma_M}{\mu_M}\times 100\%=\frac{705}{1500}=47\%
            \end{align*}
    \end{enumerate}
    Portanto, os salários das mulheres apresentam uma dispersão relativa maior que os salários dos homens. Além disso, as duas distribuições possuem uma alta dispersão (CV > 30\%).
\end{example}
\subsection{Assimetria}

Assimetria é o grau de desvio ou afastamento da simetria de uma distribuição.

Quando a curva é simétrica, a média, a mediana e a moda coincidem,
num mesmo ponto, de ordenada máxima, havendo um perfeito equilíbrio na distribuição. Quando o equilíbrio não acontece, isto é, a média, a mediana e a moda recaem em pontos diferentes da distribuição esta será assimétrica; enviesada a direita ou esquerda.