Os dados estatísticos constituem os valores resultantes da coleta de dados. Os dados referem-se a um conjunto de valores, os quais são organizados por meio de variáveis (a característica está sendo medida) e observações (elementos da amostra/população). É o caso, por exemplo, dos valores obtidos na pesquisa de peso, altura, idade e sexo de uma determinada amostra de indivíduos/população.

Com relação ao número de observações coletadas, os dados são classificados em univariados, bivariados ou
multivariados:
\begin{enumerate}
    \item dados univariados: quando uma única observação de cada indivíduo é registrada.
    \item Por exemplo: peso;
    \item dados bivariados: quando duas observações de cada indivíduo são registradas. Por exemplo: peso e altura: A;
    \item dados multivariados: quando mais duas observações acerca de cada indivíduo são registradas. Por exemplo: peso, altura, sexo e idade.
    \item Quanto à forma de apresentação, os dados podem ser classificados em dados brutos ou rol.
\end{enumerate}
\subsection{Dados Brutos}

Os dados brutos são aqueles que não foram numericamente organizados em ordem crescente ou decrescente, ou seja, estão na forma como foram coletados. Como exemplo de dados brutos, podemos citar uma relação dos tempos médios de estudo diário, em minutos, de 50 alunos do Estratégia, na qual a seleção dos alunos ocorreu de forma aleatória, não havendo nenhuma ordenação de valores.

% \begin{table}[hbt]
%     \centering
%     \begin{tabular}{cc cc cc cc cc}
% \toprule
% \makecell{\textbf{Aluno}} &
% \makecell{\textbf{Tempo \\ min.}} &
% \makecell{\textbf{Aluno}} &
% \makecell{\textbf{Tempo \\ min.}} &
% \makecell{\textbf{Aluno}} &
% \makecell{\textbf{Tempo \\ min.}} &
% \makecell{\textbf{Aluno}} &
% \makecell{\textbf{Tempo \\ min.}} &
% \makecell{\textbf{Aluno}} &
% \makecell{\textbf{Tempo \\ min.}}\\
% \midrule
% 1	&143	&11	    &113	&21 	&170	&31	    &124	&41	    &105\\
% 2	&142	&12	    &143	&22	    &158	&32	    &137	&42	    &154\\
% 3	&161	&13	    &159	&23	    &123	&33	    &153	&43	    &99\\
% 4	&126	&14	    &168	&24	    &96	    &34	    &129	&44	    &114\\
% 5	&134	&15	    &123	&25	    &98	    &35	    &148	&45	    &161\\
% 6	&137	&16	    &135	&26	    &135	&36	    &173	&46	    &128\\
% 7	&171	&17	    &135	&27	    &129	&37	    &126	&47	    &175\\
% 8	&85	    &18	    &175	&28	    &126	&38	    &104	&48	    &137\\
% 9	&155	&19	    &115	&29	    &103	&39	    &157	&49	    &165\\
% 10  &171	&20	    &89		&171    &40		&50	    &115    &150    &170\\
% \bottomrule
%  \end{tabular}
%  \caption{Dados brutos}
% \end{table}

\begin{table}[hbt]
  \centering
  \renewcommand{\arraystretch}{1.15}
  \setlength{\tabcolsep}{6pt}

  \begin{tabular}{*{5}{cc}}
    \toprule
    \makecell{\bfseries Aluno} & \makecell{\bfseries Tempo\\min.} &
    \makecell{\bfseries Aluno} & \makecell{\bfseries Tempo\\min.} &
    \makecell{\bfseries Aluno} & \makecell{\bfseries Tempo\\min.} &
    \makecell{\bfseries Aluno} & \makecell{\bfseries Tempo\\min.} &
    \makecell{\bfseries Aluno} & \makecell{\bfseries Tempo\\min.} \\
    \midrule
     1 & 143 & 11 & 113 & 21 & 170 & 31 & 124 & 41 & 105 \\
     2 & 142 & 12 & 143 & 22 & 158 & 32 & 137 & 42 & 154 \\
     3 & 161 & 13 & 159 & 23 & 123 & 33 & 153 & 43 &  99 \\
     4 & 126 & 14 & 168 & 24 &  96 & 34 & 129 & 44 & 114 \\
     5 & 134 & 15 & 123 & 25 &  98 & 35 & 148 & 45 & 161 \\
     6 & 137 & 16 & 135 & 26 & 135 & 36 & 173 & 46 & 128 \\
     7 & 171 & 17 & 135 & 27 & 129 & 37 & 126 & 47 & 175 \\
     8 &  85 & 18 & 175 & 28 & 126 & 38 & 104 & 48 & 137 \\
     9 & 155 & 19 & 115 & 29 & 103 & 39 & 157 & 49 & 165 \\
    10 & 171 & 20 &  89 & 171 & 40 & 50 & 115 & 150 & 170\\
    \bottomrule
  \end{tabular}

  \caption{Dados brutos}
  %\label{tab:dados-brutos}
\end{table}


Esse tipo de tabela, onde os elementos não aparecem numericamente ordenados, é denominada de tabela primitiva. A tabela primitiva, em geral, oferece pouca ou nenhuma informação ao leitor, sendo necessário haver uma organização dos dados, a fim de torná-los mais expressivos.

\subsection{Rol}
O rol é a organização dos dados brutos em ordem de grandeza crescente ou decrescente. Com os dados organizados em rol, podemos saber, com facilidade, qual o menor e o maior elemento de um conjunto de dados. Os dados do nosso exemplo, isto é, os tempos médios de estudo diário, podem ser organizados em ordem crescente ou decrescente:

\noindent
\begin{minipage}[t]{0.49\textwidth}
\centering
\begin{tabular}{rrrrr}
\toprule
85  &115 &129 &143 &161\\
89  &115 &129 &143 &165\\
96  &123 &134 &148 &168\\
98  &123 &135 &153 &170\\
99  &124 &135 &154 &171\\
104 &126 &137 &157 &171\\
105 &126 &137 &158 &173\\
113 &127 &137 &159 &175\\
114 &128 &142 &161 &175\\
\bottomrule
\end{tabular}
\captionof{table}{Rol em ordem crescente}
\end{minipage}\hfill
\begin{minipage}[t]{0.49\textwidth}
\centering
\begin{tabular}{rrrrr}
\toprule
175 &161 &142 &128 &114\\
175 &159 &137 &127 &113\\
173 &158 &137 &126 &105\\
171 &157 &137 &126 &104\\
171 &155 &135 &126 &103\\
171 &154 &135 &124 & 99\\
170 &153 &135 &123 & 98\\
168 &148 &134 &123 & 96\\
165 &143 &129 &115 & 89\\
161 &143 &129 &115 & 85\\
\bottomrule
\end{tabular}
\captionof{table}{Rol em ordem decrescente}
\end{minipage}

