\section{Medidades de Variabilidade}
Já estudamos mecanismos para encontrar valores (média, mediana e moda) que sintetizam o comportamento dos elementos de um conjunto de dados. Esses valores fornecem parâmetros
significativos para uma análise dos dados, porém, também é importante identificarmos como variam ou
como se diferenciam as características dos elementos de um conjunto.

Imagine, por exemplo, que você precise avaliar três grupos de alunos, cada um com cinco elementos, no que
diz respeito ao domínio de uma determinada matéria. Os testes evidenciaram os seguintes resultados:

\begin{align*}
    A=& 7,7,7,7,7\\
    B=& 5, 6, 7, 8, 9\\
    C=& 1, 4, 10, 10, 10\\
\end{align*}

Para analisar esses dados, podemos, inicialmente, calcular a média aritmética dos três grupos. Concluímos, então, que todos possuem a mesma média aritmética \(\bar{x} = 7\). Contudo, ao observarmos a variação dos dados, percebemos que os grupos se comportam de maneira diferente, apesar de todos possuírem a mesma
média.

A média de A, de B e de C são iguais a 7.

Nesse caso, a média, embora seja uma medida representativa do conjunto, não indica o grau de homogeneidade ou heterogeneidade existente entre os valores que compõem o conjunto. Desse modo,
precisamos recorrer a procedimentos matemáticos que possibilitem a compreensão da discrepância existente entre os valores do conjunto.

As medidas de dispersão (ou variabilidade) são justamente métricas que evidenciam a variação dos dados de um conjunto. Elas podem ser divididas em dois grupos:

\begin{enumerate}[label=\alph*.]
    \item medidas de dispersão absoluta:
    \begin{itemize}[label=--]
        \item amplitude total;
        \item amplitude interquartílica;
        \item desvio médio;
        \item variância; e
        \item desvio-padrão.
    \end{itemize}
    \item medidas de variação relativa:
    \begin{itemize}[label=--]
        \item coeficiente de variação (de Pearson); e
        \item variância relativa.
    \end{itemize}
\end{enumerate}

Agora, aprenderemos a medir o grau de concentração ou dispersão dos dados em torno da média. Para isso, estudaremos as principais medidas de dispersão, sendo: amplitude total, amplitude interquartílica, desvio médio, variância, desvio padrão, coeficiente de variação e variância relativa.

A amplitude total (ou simplesmente amplitude) é a diferença entre os valores extremos de um conjunto de observações, ou seja, a diferença entre o maior e o menor elemento desse conjunto:
 \[
    A_T=\bar{x}_max-\bar{x}_min
 \]
 Essa medida de dispersão chama atenção por ser extremamente simples e muito fácil de se calcular.
Contudo, há uma certa restrição quanto ao seu uso por conta de sua grande instabilidade, vez que considera somente os valores extremos da série.

Por exemplo, vamos comparar os conjuntos A e B da tabela a seguir:

\begin{table}[ht]
    \centering
\begin{tabular}{lll}
    \toprule
Conjunto & Média \(\bar{x}\) & Amplitude total (AT)\\
\midrule
\(A= 5,7,8,9,10,11,55\) & \(\bar{x} = 15\) & \(55 - 5 = 50\) \\
\(B= 12,13,14,15,16,17,18\) & \(\bar{x} = 15\) & \(18 - 12 = 6\)\\
\bottomrule
\end{tabular}
\caption{Médias e Amplitudea}
\label{tab:media_amplitude}
\end{table}

As médias aritméticas dos dois conjuntos são iguais a 15. Portanto, no que diz respeito a essa medida de posição, podemos considerá-los idênticos. Porém, ao calcularmos a amplitude total, verificamos que os valores do conjunto A apresentam um grau de dispersão bem maior que os do conjunto B.

Isso acontece porque, no cálculo da amplitude total, desconsideramos os valores da série que se encontram entre os extremos, o que pode conduzir a interpretações equivocadas. Com frequência, um valor discrepante pode afetar a medida de maneira acentuada. É o caso, por exemplo, do último valor (55) do conjunto A, sensivelmente maior que seu antecessor (11), que elevou a magnitude da amplitude total para 50.

Além disso, a amplitude total também é sensível ao tamanho de amostra. Normalmente, a amplitude total tende a aumentar com o incremento da dimensão da amostra, ainda que não proporcionalmente. Ainda, a amplitude total pode apresentar muita variação de uma amostra para outra, ainda que extraídas de uma
mesma população.

Apesar das limitações dessa medida, há situações em que ela pode ser aplicada de forma satisfatória. É o caso, por exemplo, da variação da temperatura em um dia. Também é o caso de quando uma compreensão rápida dos dados é mais relevante que a exatidão de um procedimento complexo.

\subsection{Amplitude para dados não agrupados}

Para dados não agrupados, o cálculo da amplitude total pode ser expresso pela fórmula:
 \[
    A_T=x_{max}-x_{min}
 \]

 em que $x_max$ é o maior elemento; e $x_min$ é o menor elemento do conjunto.
 \begin{example}
    Calcular a amplitude total dos conjuntos apresentados a seguir:
\begin{align*}
    &A = 50, 50, 50, 50, 50, 50, 50\\
    &B = 47, 48, 49, 50, 51, 52, 53\\
    &C = 20, 30, 40, 50, 60, 70, 80\\
\end{align*}

Aplicando a fórmula anterior para esses dados, obtemos os seguintes resultados:
\begin{align*}
    &AT A = x_{max} - x_{min} = 50 - 50 = 0\\
    &AT B = x_{max} - x_{min} = 53 - 47 = 6\\
    &AT C = x_{max} - x_{min} = 80 - 20 = 60\\
   \end{align*}

Nesse caso, podemos observar que o conjunto A obteve uma amplitude total igual a 0, ou seja, uma dispersão nula. Então, significa que os valores não variam entre si. O conjunto B, por sua vez, obteve uma amplitude igual a 6. Já a variável C teve uma amplitude total igual a 60.
\medskip
Embora o valor da amplitude total seja diferente para os conjuntos A, B e C, todos possuem a mesma média aritmética (50). Independentemente da média, verificamos que o conjunto A possui
elementos mais homogêneos do que os conjuntos B e C. E, também, que os elementos do conjunto B são mais homogêneos do que os do conjunto C.
 \end{example}

 \subsection{Amplitude Total para dados agrupados sem intervalos de classes}

 Para dados agrupados SEM intervalos de classe, a fórmula usada para a identificação da amplitude total é similar à adotada para dados não-agrupados. A única diferença consiste na identificação dos valores mínimo e máximo, que agora ocorre por meio de uma tabela de frequências.

 \begin{example}
    Calcular a amplitude total da tabela de frequências apresentada a seguir.

    \begin{table}[ht]
        \centering
\begin{tabular}{ll}
    \toprule
$x_i$& $f_i$ \\
    \midrule
1 & 10 \\
3 & 15 \\
5 & 10 \\
7 & 8 \\
9 & 7\\
    \bottomrule
\end{tabular}
\caption{Frequências}
\label{tab:frequencias}
\end{table}

Nesse caso, como 1 e 9 são os valores mínimo e máximo da variável $x_i$, temos o seguinte resultado:

\begin{align*}
    &AT = x_{max} - x_{min}\\
    &AT = 9 - 1=8\\
\end{align*}

É importante ressaltar que esses valores foram selecionados independentemente da frequência
associada a eles.
 \end{example}

 \subsection{Amplitude Total para dados agrupados em classes}

 Para dados agrupados em intervalos de classe, podemos definir a amplitude total de duas formas:
\begin{enumerate}
    \item  pela diferença entre o limite superior da última classe \(l_{inf}\) e o limite inferior da primeira classe \(l_{sup}\), conforme expresso na fórmula a seguir:

    \[
        A = (l_{sup}) - (l_{inf})
    \]

    \item pela diferença entre o ponto médio da última classe \(PM_{Utl}\) e o ponto médio da primeira classe \(PM_{pri}\), conforme expresso na fórmula a seguir:
    
    \[
        A=PM_{ult}-PM_{pri}
    \]

\end{enumerate}

\begin{example}
    Calcular a amplitude total da distribuição de frequências apresentada a seguir:
\begin{table}[h!]
    \centering
\begin{tabular}{lll}
\hline
Classes  &$PM_i$ & $f_i$ \\ \hline
1  \vdash 5   & 3  & 5  \\
5  \vdash 9   & 7  & 10 \\
9  \vdash 13  & 11 & 15 \\
13 \vdash 17  & 15 & 10 \\
17 \vdash 21  & 19 & 5  \\ \hline
Total &  & 45 \\ \hline
\end{tabular}
\caption{Amplitudes agrupadas por classes}
\label{tab:amplitude_classes}
\end{table}

Pelo primeiro método, temos que o limite superior da última classe é 21, enquanto o limite inferior da primeira classe é 1. Portanto, temos a seguinte amplitude:

\begin{align*}
    A= L_{sup} - L_{inf}
    A = 21 - 1 = 20
\end{align*}
Pelo segundo método, temos que o ponto médio da última classe é 19, enquanto o ponto médio da primeira classe é 3. Portanto, temos a seguinte amplitude:

\begin{align*}
    A= PM_{ult} - PM_{pri}
    A = 19 - 3 = 16
\end{align*}

Observe que a amplitude é menor pelo segundo método, porque os extremos da distribuição são desconsiderados.
\end{example}

\subsection{Propriedades da Amplitude Total}
Nesse tópico, estudaremos as principais propriedades da amplitude total:
\begin{enumerate}
    \item Somando-se (ou subtraindo-se) uma constante 𝒄 a todos os valores de uma variável, a amplitude do conjunto não é alterada.
    \item Multiplicando-se (ou dividindo-se) todos os valores de uma variável por uma constante c, a amplitude do conjunto fica multiplicada (ou dividida) por essa constante.
\end{enumerate}

Como já sabemos, denominamos de quartis os valores de uma série que a dividem em quatro partes iguais, isto é, quatro partes contendo o mesmo número de elementos (25\%). A imagem a seguir mostra os quartis de uma distribuição hipotética:

\subsection{Amplitude interquartílica}

\begin{figure}[h]
    \centering
\begin{tikzpicture}
\begin{axis}[
  width=12cm, height=6cm,
  axis lines=left,
  xlabel={$x$}, ylabel={$f(x)$},
  axis line style={-{Stealth[length=3mm,width=2mm]}},
  domain=-4:4, samples=300,
  ymin=0,
  ymax=0.6,
  grid=both
]

% linhas verticais
\addplot [blue, thick] {(1/sqrt(2*pi) * exp(-x^2/2))+0.05};
\draw [red, thick] (-2,0) -- (-2,0.103);
\draw [red, thick] (0,0) -- (0,0.448);
\draw [red, thick] (2,0) -- (2,0.103);


% rótulos embaixo do eixo x
\node[font=\itshape] at (0,0.5) {Md=\(Q_2\)};
\node[font=\itshape] at (2,0.16) {\(Q_3\)};
\node[font=\itshape] at (-2,0.16) {\(Q_1\)};
\end{axis}
\end{tikzpicture}
\caption{Amplitude interquartílica}
\label{Graf:amplitide_quartis}
\end{figure}

Temos, então, 3 quartis (\(Q1, Q2, Q3\)) para dividir uma série em quatro partes iguais:

\begin{itemize}[label=--]
    \item Q1: o primeiro quartil corresponde à separação dos primeiros 25\% de elementos da série;
    \item Q2: o segundo quartil corresponde à separação de metade dos elementos da série, coincidindo com a mediana \(Q2 = Md\);
    \item Q3: o terceiro quartil corresponde à separação dos primeiros 75\% de elementos da série, ou dos últimos 25\% de elementos da série.
\end{itemize}

A amplitude interquartílica (ou distância interquartílica, ou intervalo interquartílico) é o resultado da subtração entre o terceiro quartil e o primeiro quartil:
\[
    A_{IQ}= Q_1 - Q_3
\]
A amplitude semi-interquartílica (ou desvio quartílico) é definida como a metade desse valor, sendo calculada pela expressão apresentada a seguir:
\[
    D_Q=\frac{Q_3-Q_1}{2}
\]
A fórmula da amplitude interquartílica (ou distância interquartílica) é muito parecida com a fórmula da amplitude semi-interquartílico (ou desvio quartílico), podendo ser facilmente confundida.

\subsection{Propriedades da Amplitude Interquartílica}

\begin{enumerate}
    \item Somando-se (ou subtraindo-se) uma constante 𝒄 a todos os valores de uma variável, a amplitude interquartílica (e o desvio quartílico) do conjunto não é alterada.
    \item Multiplicando-se (ou dividindo-se) todos os valores de uma variável por uma constante c, a amplitude interquartílica (e o desvio quartílico) do conjunto fica multiplicada (ou dividida) por essa constante.
\end{enumerate}

\subsection{Desvios em relação à média aritmética e mediana}

Antes de apresentarmos as fórmulas para o cálculo do desvio médio e da variância, precisamos compreender qual o conceito de desvio em estatística. Um desvio é a distância entre qualquer observação do conjunto de dados e uma medida descritiva desse conjunto:
\begin{center}
    devio = observação - medida
\end{center}

Em especial, os desvios em relação à média aritmética e em relação à mediana:
\[
    di= x - \bar{x}\, \text{(média)}
\]
ou
\[
    di= x - Md \quad\text{(mediana)}
\]
Quando os desvios em relação a uma medida descritiva são pequenos, as observações estão concentradas em torno dessa medida e, portanto, a variabilidade dos dados é pequena.
Agora, quando os desvios são maiores, significa que as observações estão dispersas e, portanto, a variabilidade dos dados é grande.

\subsection{Propriedades dos desvios em relação à Média Aritmética e
Mediana}

\begin{enumerate}
    \item A soma algébrica dos desvios em relação à média é nula.
    \item A soma dos quadrados dos desvios da sequência de números $x_i$, em relação a um número a, é mínima se a for a média aritmética dos números.
    \item A soma dos desvios absolutos de uma sequência de números, em relação a um número a, é mínima quando a é a mediana dos números.
\end{enumerate}

\subsection{Desvio absoluto médio}

O desvio absoluto médio, ou simplesmente desvio médio, mede a dispersão entre os valores da distribuição e a média dos dados coletados. Para compreender essa medida, vamos supor que o Estratégia Concursos tenha realizado uma semana de revisão para estudantes da área fiscal, obtendo os seguintes números de
visualizações:

\begin{table}[H]
    \centering
\begin{tabular}{lr}
\hline
\multicolumn{1}{l}{\begin{tabular}[c]{@{}c@{}}Dia da   \\ semana\end{tabular}} & \multicolumn{1}{c}{\begin{tabular}[c]{@{}c@{}}Número de\\ visualizações\end{tabular}} \\
\toprule
Domingo     & 2.000 \\
Segunda     & 4.000 \\
Terça       & 5.200 \\
Quarta      & 6.300 \\
Quinta      & 5.400 \\
Sexta       & 4.100 \\
Sábado      & 2.400 \\
\midrule
Total       & $\sum f_i = 29.400$\\
\bottomrule
\end{tabular}
\caption{Desvio Absoluto médio}
\label{tab:desvio_medio}
\end{table}

Isso significa que a semana de revisão teve uma média diária de 4.200 visualizações. Esse resultado, porém, não retrata a realidade com fidedignidade, pois alguns dias tiveram mais visualizações do que a média; enquanto outros não. Por isso, é importante sabermos o quão distante a média está em relação aos valores
reais por ela representados.

Para calculá-los, basta subtrairmos o valor da média de cada observação, conforme mostrado a seguir:

\begin{table}[H]
    \centering
\begin{tabular}{lclr}
\hline
\multicolumn{1}{c}{\begin{tabular}[c]{@{}c@{}}Dia da \\ semana\end{tabular}} & \begin{tabular}[c]{@{}c@{}}Número de \\ visualizações\end{tabular} & \multicolumn{2}{c}{$x_i-\bar{x}$} \\ \hline
Domingo & 2.000 & 2.000 - 4.200 = & -2200 \\
Segunda & 4.000 & 4.000 - 4.200 = & -200 \\
Terça   & 5.200 & 5.200 - 4.200 = & 1000 \\
Quarta  & 6.300 & 6.300 - 4.200 = & 2100 \\
Quinta  & 5.400 & 5.400 - 4.200 = & 1200 \\
Sexta   & 4.100 & 4.100 - 4.200 = & -100 \\
Sábado  & 2.400 & 2.400 - 4.200 = & -1800 \\ \hline
\end{tabular}
\caption{Cálculo do desvio}
\label{tab:desvio}
\end{table}

Ao calcularmos o desvio médio, obtemos resultados positivos e negativos, que se anulam ao serem somados. Percebam que existem valores de observações que estão muito próximos da média, enquanto outros estão mais distantes.

Como a soma de todos os desvios médios é sempre igual a zero para qualquer conjunto de dados (Primeira propriedade dos desvios), sabemos que \(\sum (x - \bar{x}) n_i\) não nos fornecerá nenhuma informação relevante nem nos ajudará a compreender o que está acontecendo com essa variável. 

Para superar essa dificuldade, podemos utilizar somente os resultados positivos dos desvios calculados. A fórmula do cálculo do desvio médio se apresenta da seguinte maneira:
\begin{align*}
    D_m = \frac{\sum_{n=1}^n |x_i- \bar{x}|}{n}
\end{align*}

em que $D_m$ representa o desvio médio, \(|x_i - \bar{x}|\) representa o módulo da diferença entre uma determinada observação e a média calculada, $f_i$ representa a frequência de um determinado valor para a variável da distribuição, e $n$ representa o total de elementos formados pela distribuição. 

O desvio médio é uma medida de dispersão mais robusta do que a amplitude total e a amplitude interquartílica, pois considera todos os valores do conjunto. O inconveniente dessa medida é a operação de módulo, que, por conta de suas características matemáticas, dificulta o estudo de suas propriedades.

\subsection{Desvio Médio para dados não-agrupados}

O desvio absoluto médio $(D_m)$, de um conjunto de $n$ observações $x_11, \dots , x_n$, é a média dos valores absolutos das diferenças entre as observações e a média. Isto é,

\begin{align*}
    D_m = \frac{\sum_{n=1}^n |x_i- \bar{x}|}{n}
\end{align*}

As barras verticais indicam a operação de módulo, responsável por transformar qualquer número negativo em um número positivo, isto é, retornar o valor absoluto.

\begin{example}
    Calcular o desvio médio do conjunto mostrado a seguir:
    \begin{align*}
        {1, 2, 3, 5, 9}
    \end{align*}
    Iniciaremos pelo cálculo da média aritmética:
    \begin{align*}
        \bar{x}=\frac{1+2+3+5+9}{5}=4
    \end{align*}

\begin{table}[H]
\centering
\begin{tabular}{l l c}
\hline
\multicolumn{1}{c}{$x_i$} &
\multicolumn{1}{c}{$x_i-\bar{x}$} &
\multicolumn{1}{c}{$|x_i-\bar{x}|$} \\
\hline
1 & $(1-4)=-3$ & 3 \\
2 & $(2-4)=-2$ & 2 \\
3 & $(3-4)=-1$ & 1 \\
5 & $(5-4)=1$  & 1 \\
9 & $(9-4)=5$  & 5 \\
\hline
\end{tabular}
\caption{Desvio médio}
\label{tab:cal_desv_med}

\end{table}
Aplicando a fórmula do desvio médio, temos:
\begin{align*}
    D_m = \frac{\sum_{n=1}^n |x_i-4|}{n}=\frac{\sum_{n=1}^n |x_i- \bar{x}|}{n}=\frac{12}{5}=2,4
\end{align*}
\end{example}

\subsection{Desvio Médio para dados agrupados sem intervalo de classe}

Quando os valores vierem dispostos em uma tabela de frequências, o desvio médio será calculado por meio da seguinte fórmula:

\begin{align*}
    D_m = \frac{\sum_{i=1}^{m} |x_i-\bar{x}|\times f_i}{\sum f_i}
\end{align*}

Em que $m$ indica o número de grupos que os dados estão organizados; e \(|x_1 - \bar{x}|\) representa o módulo da diferença entre uma determinada observação e a média calculada.

Durante uma pesquisa, o Estratégia Concursos registrou a quantidade de filhos de seus professores, obtendo a tabela de frequências apresentada a seguir. Vamos calcular o desvio médio dessa distribuição.

\begin{table}[ht]
    \centering
\begin{tabular}{ccc}
    \toprule
\begin{tabular}[c]{@{}c@{}}Nº de filhos \\ por professor\end{tabular} & $f_i$& $x_i \times f_i$\\
\midrule
0 & 4 & 0 \times 4 = 0 \\
1 & 8 & 1 \times 8 = 8 \\
2 & 4 & 2 \times 4 = 8 \\
3 & 2 & 3 \times 2 = 6 \\
4 & 2 & 4 \times 2 = 8 \\
\midrule
*Pesquisa populacional &$\sum f_i = 20$ & $\sum x_i\times f_i=30$\\
\hline
\end{tabular}
\caption{Distribuição de frequências}
\label{tab:distrib}
\end{table}

Iniciamos pelo cálculo da média aritmética:

\begin{align*}
    &D_m = \frac{\sum_{i=1}^{m} |x_i-\bar{x}|\times f_i}{\sum f_i}=\frac{30}{20}=1,5\,\text{Filhos por professor}
\end{align*}

Em seguida, adicionaremos uma nova coluna à tabela anterior, em que calcularemos os produtos dos desvios absolutos por suas respectivas frequências:

\begin{table}[H]
    \centering
\begin{tabular}{cccc}
    \toprule
\begin{tabular}[c]{@{}c@{}}Nº de filhos\\  por professor\end{tabular} & $f_i$ & $x_i\times f_i$ & $|x_i - \bar{x}| \times x f_i$ \\ \hline
0 & 4 & 0 & |0 - 1,5| \times 4 = 6 \\
1 & 8 & 8 & |1 - 1,5| \times 8 = 4 \\
2 & 4 & 8 & |2 - 1,5| \times 4 = 2 \\
3 & 2 & 6 & |3 - 1,5| \times 2 = 3 \\
4 & 2 & 8 & |4 - 1,5| \times 2 = 5 \\
\midrule
\morecmidrules * Pesquisa populacional & $\sum f_i = 20 $ & $\sum x_i \times f_i = 30$ & $\sum |x_i - \bar{x}| \times f_i = 20$ \\ \hline
\end{tabular}
\caption{}
\label{tab:my-table}
\end{table}

Por fim, aplicando a fórmula do desvio médio, temos:

\begin{align*}
    D_m = \frac{\sum_{i=1}^{m} |x_i-\bar{x}|\times f_i}{\sum f_i}=\frac{20}{20}=1
\end{align*}

\subsection{Desvio Médio para dados agrupados em classes}

Se os dados estiverem agrupados em classe, deveremos adotar a mesma convenção que tomamos para o cálculo da média: vamos assumir que todos os valores coincidem com os pontos médios das suas respectivas classes.

\begin{example}
    Durante uma pesquisa, uma escola registrou as estaturas de 40 alunos, obtendo a distribuição de frequências apresentada a seguir. Calcule o desvio médio dessa distribuição
    \begin{table}[H]
        \centering
        \begin{tabular}{lc}
        \hline
        Estaturas & Frequência \(f_i\) \\ \hline
        150 \vdash 154 & 4  \\
        154 \vdash 158 & 9  \\
        158 \vdash 162 & 11 \\
        162 \vdash 166 & 8  \\
        166 \vdash 170 & 5  \\
        170 \vdash 174 & 3  \\ \hline
        *Pesquisa amostral &\(\sum f_i = 40\) \\ \hline
        \end{tabular}
        \caption{}
        \label{tab:estaturas_classes}
\end{table}

Inicialmente, construiremos uma tabela como a mostrada a seguir:
\begin{table}[H]
    \centering
\begin{tabular}{lcccccc}
\hline
Estaturas & \begin{tabular}[c]{@{}c@{}}Frequência \\ $(f_i)$\end{tabular} & $x_i$ & $x_i \times f_i$ & $(x_i-\bar{x})$ & $|x_i \bar{x}|$ & $|x_i - \bar{x}| \times f_i$ \\ \hline
150 \vdash 154 & 4  & 152 & 608     & -9 & 9 & 36  \\
150 \vdash 154 & 9  & 156 & 1.404   & -5 & 5 & 45  \\
150 \vdash 154 & 11 & 160 & 1.760   & -1 & 1 & 11  \\
150 \vdash 154 & 8  & 164 & 1.312   & 3 & 3 & 24   \\
150 \vdash 154 & 5  & 168 & 840     & 7 & 7 & 35   \\
150 \vdash 154 & 3  & 172 & 516     & 11 & 11 & 33 \\ \hline
\begin{tabular}[c]{@{}l@{}}* Pesquisa \\ amostral\end{tabular} & $\sum f_i = 40$ &  & $\sum x_i \times f_i = 6440$ &  &  & $\sum |x_i - \bar{x}| \times f_i = 184$
\end{tabular}
\caption{Desvio Médio}
\label{tab:desvio__medio}
\end{table}
\end{example}

Feito isso, podemos calcular a média da distribuição por meio da seguinte fórmula:

\begin{align*}
    \bar{x} = \frac{\sum x_i\times f_i}{\sum f_i}=\frac{6640}{40}=161
\end{align*}

Conhecendo a média, completamos a tabela com as diferenças e os produtos necessários para o cálculo do desvio médio. Assim, aplicando a fórmula do desvio médio, temos:

\begin{align*}
    D_m = \frac{\sum_{i=1}^{6} |x_i-\bar{x}|\times f_i}{\sum f_i}=\frac{184}{40}=4,6
\end{align*}
Portanto, o desvio médio para essa distribuição de estaturas é 4,6 cm.