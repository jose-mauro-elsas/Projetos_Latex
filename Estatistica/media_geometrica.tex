

A média geométrica é uma medida estatística muito utilizada em situações de acumulação de percentuais, fato muito comum em problemas financeiros. Também é encontrada na geometria plana,
quando, por exemplo, devemos fazer com que a área de um quadrado seja igual à área de um retângulo.

Essa medida é definida, para o conjunto de números positivos, como a raiz \textit{n}-ésima do produto de \textit{n} elementos de um conjunto de dados. A propriedade principal dessa média é preservar o
produto dos elementos de um conjunto de dados.

O raciocínio para encontrarmos a fórmula da média geométrica é análogo ao adotado para a média aritmética. Dada uma lista de \textit{n} números, \({x_1, x_2, \dots, x_n}\), o produto de seus termos é igual a:

\[
    \underbrace{{x_1\times x_2\times \dots\times x_n}}_{\text{n fatores}}
\]

A média geométrica dessa lista é um número \textit{G}, tal que, se todos os elementos forem substituídos por textit{G}, o produto da lista permanecerá preservado. Assim, substituindo todos os elementos por \textit{G}, teremos
uma nova lista, \({G_1, G_2, \dots, G_n}\), cujo produto é:

\[
    \underbrace{{G\times G\times \dots\times G}}_{\text{n fatores}}=G^n
\]

Como os produtos das duas listas são iguais, temos:

\[
    G^n=x_1\times x_2\times \dots\times x_n=\sqrt[n]{x_1\times x_2\times \dots\times x_n}
\]

Note que a raiz varia conforme a quantidade de elementos da lista de números, isto é, se a lista contém dois números, teremos uma raiz quadrada; se a lista contém três números, teremos uma raiz cúbica.