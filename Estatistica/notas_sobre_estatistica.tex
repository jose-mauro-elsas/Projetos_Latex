% !TEX program = lualatex
\documentclass[a4paper,12pt]{book}
\usepackage[brazil]{babel}
\usepackage{geometry}
\usepackage{float}
\usepackage{booktabs}
\usepackage{caption} % para \captionof
\geometry{top=2.5cm,bottom=2.0cm,left=2.5cm,right=2.0cm}
\usepackage{setspace}
\usepackage{array} %multicolunas
\usepackage{multicol}
\usepackage[hypcap=false]{caption}
\usepackage{enumitem} %para enumerar com letras.
\usepackage{graphicx}
\usepackage{multirow} %permite várias linhas na cabeça da tabela.
\usepackage{amsmath}
\usepackage{fontspec}
\setmainfont{Times New Roman}
\usepackage{amsthm}
\usepackage{siunitx}

\usepackage{microtype}
\usepackage{hyperref}
\usepackage{pdfpages}
\usepackage{enumitem} %com esse dá pra usar letras na enumeração

% ===== Figuras / Gráficos =====
\usepackage{graphicx}
\usepackage{float}
\usepackage{tabularx}
% PGFPlots (já carrega o TikZ)
\usepackage{pgfplots}
\usepackage{pgfplotstable}
\pgfplotsset{compat=1.18, width=11cm}
\usepgfplotslibrary{fillbetween}
\usetikzlibrary{arrows.meta,intersections}
\usepackage{float}
\usepackage{makecell}
\usepackage[table]{xcolor} % ou \usepackage{colortbl}
\newtheorem{example}{Exemplo}[section]

% ===============================
% Ambientes teóricos
% ===============================
% \theoremstyle{plain}      % itálico
% \newtheorem{theorem}{Teorema}[section]

% \theoremstyle{definition} % texto normal
% \newtheorem{definition}{Definição}[section]

% \theoremstyle{remark}     % texto normal, sem itálico
% \newtheorem{remark}{Observação}[section]


\theoremstyle{definition}
\newtheorem{propriedade}{Propriedade}[section]




%=======================================
%         Header
%=======================================
\usepackage{fancyhdr}
\pagestyle{fancy}
\fancyhead[R]{\nouppercase{\scshape\rightmark}}
\lhead{\textsc{Notas sobre Estatística}}
\setlength{\headheight}{15pt}
\addtolength{\topmargin}{-3pt}

%======================================
%títulos das seções
%=====================================
\usepackage{titlesec}

% Section em versal-versalete (small caps)
\titleformat{\section}
  {\normalfont\Large\scshape}
  {\thesection}
  {1em}
  {}

% (opcional) Subsection também
\titleformat{\subsection}
  {\normalfont\large\scshape}
  {\thesubsection}
  {1em}
  {}


\usetikzlibrary{decorations.pathreplacing,arrows.meta,calc}

\pgfplotsset{compat=1.18, width=11cm}
\usetikzlibrary{arrows.meta}

\setlist[itemize]{noitemsep,topsep=4pt}
\begin{document}
\tableofcontents
\newpage
\chapter{Fundamentos}
\section{Introdução à estatística}
\raggedbottom

A origem da estatística remonta às civilizações antigas, em que vários povos coletavam e registravam dados populacionais e econômicos de interesse do Estado. 

Nessa época, também eram realizadas estimativas das riquezas individuais e familiares, as quais eram utilizadas para determinar o montante de impostos a serem pagos pela população.

O termo estatística se originou da palavra \textit{status}, que significa Estado em latim. O termo era utilizado para designar um conjunto de dados, relativos aos Estados, que os governantes utilizavam para controle fiscal e segurança nacional. O primeiro a utilizar a palavra foi Schneider, ainda no século XVII, em latim. Depois, foi adotada pelo acadêmico alemão Godofredo Achenwall.

A Estatística pode ser definida como a ciência que estuda os processos de coleta, organização, análise e interpretação de dados numéricos variáveis referentes a qualquer fenômeno. Ou ainda, podemos conceituá-la como um conjunto de técnicas de coleta, organização, análise e interpretação de dados, aplicáveis a várias áreas do conhecimento, que auxiliam no processo de tomada de decisão.

Os avanços computacionais tornaram a Estatística mais acessível e permitiram aplicações mais sofisticadas em diferentes áreas do conhecimento. Nesse cenário, os softwares estatísticos passaram a disponibilizar ferramentas antes inimagináveis, voltadas para planejamento de experimentos, teste de hipóteses, cálculos de confiabilidade, criação de gráficos complexos e elaboração de modelos preditivos.
A Estatística pode ser dividida em três grandes ramos: Estatística Descritiva (ou dedutiva), Estatística Probabilística e Estatística Inferencial (ou indutiva). Alguns autores, porém, consideram a Estatística Probabilística como parte da Estatística Inferencial.

A Estatística Descritiva (ou Dedutiva) é responsável pela coleta, organização, descrição e resumo dos dados observados. A partir de um determinado conjunto de dados, a Estatística Descritiva busca organizá-los em tabelas (ou gráficos) e estabelecer um sumário por meio de medidas descritivas como a média, os valores mínimo e máximo, o desvio padrão, entre outras.
A Estatística Probabilística é responsável por estabelecer o modelo matemático probabilístico adotado para explicar os fenômenos aleatórios investigados pela Estatística. Os resultados desses fenômenos aleatórios podem variar de uma observação para outra, dificultando muito a previsão de um resultado futuro. Por isso, a Teoria da Probabilidade é usada para medir a chance de ocorrência de determinados eventos.

A Estatística Inferencial (ou Indutiva) é responsável pela análise e interpretação dos dados. A partir da análise de dados de uma amostra, a Estatística Indutiva estabelece inferências e previsões sobre a população, auxiliando na tomada decisões. Além disso, busca generalizar conclusões a respeito da população a partir de uma amostra, analisando a representatividade, a significância e a confiabilidade dos resultados obtidos.

\subsection{Conceitos Iniciais}
Neste tópico, apresentaremos alguns conceitos iniciais da estatística que costumam ser abordados em provas de concursos públicos, dentre os quais podemos citar: população, amostra, censo, amostragem, parâmetros e estatísticas.

\subsection{População}
Uma POPULAÇÃO é um conjunto que contém TODOS OS INDIVÍDUOS, OBJETOS OU ELEMENTOS a serem estudados, que apresentam uma ou mais características em comum. A população pode ser finita, quando apresenta um número pequeno ou limitado de observações; ou infinita, quando apresenta um número muito grande ou ilimitado de observações.

\subsection{Amostra}
Uma AMOSTRA é um SUBCONJUNTO EXTRAÍDO DA POPULAÇÃO para análise, devendo ser representativo daquele grupo. A partir das informações colhidas da amostra, os resultados obtidos podem ser utilizados para generalizar, inferir ou tirar conclusões acerca da população. Como exemplo, podemos citar as pesquisas eleitorais, em que uma amostra de eleitores deve ser extraída conforme a proporcionalidade de gênero, idade, grau de instrução e classe social.

\subsection{Censo}
O CENSO, ou recenseamento, é um estudo dos dados relativos a TODOS os elementos de uma população. O censo pode custar muito caro e demandar um tempo considerável, de forma que um estudo considerando somente uma parcela da população pode ser uma alternativa mais simples, rápida e menos onerosa. Como exemplos, podemos citar a pesquisa sobre o grau de escolaridade dos habitantes brasileiros, o estudo sobre a renda dos brasileiros e a pesquisa de emprego.

\subsection{Amostragem}
A AMOSTRAGEM é um processo que consiste na SELEÇÃO CRITERIOSA dos elementos a serem submetidos à investigação. Se forem cometidos erros no processo de seleção da amostra, muito provavelmente, o estudo ficará comprometido e os resultados serão tendenciosos. Portanto, devemos garantir que a amostra seja representativa da população. Isso significa que, com exceção de pequenas discrepâncias inerentes à aleatoriedade existente no processo de amostragem, uma amostra deve possuir as mesmas características básicas da população, no que diz respeito às variáveis que desejamos pesquisar.

\subsection{Parâmetros}
Os \textbf{PARÂMETROS} são \textbf{DESCRIÇÕES NUMÉRICAS} de \textbf{CARACTERÍSTICAS POPULACIONAIS} que raramente são
conhecidas. Em geral, é muito caro ou demorado obter os dados da população inteira. Assim, algumas medidas precisam ser estimadas a partir de critérios ou métodos definidos pelo pesquisador, para representar características desconhecidas de uma população (por exemplo, a proporção de homens e mulheres na população brasileira). Normalmente, os parâmetros populacionais são constantes para uma população.

\subsection{Estatística (ou estimador)}
As ESTATÍSTICAS são MEDIDAS NUMÉRICAS OBTIDAS DE AMOSTRAS representativas extraídas da população. A partir das informações colhidas da amostra, as estatísticas amostrais obtidas podem ser utilizadas para inferir ou tirar conclusões acerca dos parâmetros populacionais, como a proporção de homens e mulheres na população brasileira. Resumidamente, as estatísticas (ou estimadores) são descrições numéricas de características amostrais. Normalmente, as estatísticas amostrais diferem de uma amostra para outra.

\begin{figure}[hbt]
    \centering


\begin{tikzpicture}[
    xmark/.style={blue!70!black, font=\small},
    pop/.style={draw=blue!70!black, fill=blue!20, thick},
    sample/.style={draw=blue!70!black, fill=blue!10, thick},
    arrow/.style={-Stealth, very thick, blue!70!black}
]

% =========================
% População (círculo grande)
% =========================
\draw[pop] (0,0) circle (4);
\node[blue!70!black, font=\large\bfseries] at (0,4.6) {População};

% X aleatórios na população
\foreach \x/\y in {
 -2.5/2, -1/2.8, 1.5/2.5, 2.8/1.2,
 -3/0.5, -1.8/1.5, 0/2, 2/0.8,
 -2.2/-1, -0.5/-2.5, 1.5/-2,
 3/-0.8, -1/-0.2, 0.8/1,
 2.2/-1.8, -3/-2, 0/-1.2
}{
  \node[xmark] at (\x,\y) {x};
}

% =========================
% Subconjunto destacado
% =========================
\draw[sample] (1.2,-0.8) circle (1.8);

\foreach \x/\y in {
 0.8/-0.2, 1.5/-0.5, 1/-1.2, 1.8/-1,
 0.9/-1.6
}{
  \node[xmark] at (\x,\y) {x};
}

% =========================
% Amostragem (seta)
% =========================
\draw[arrow] (2.5,-2.2) .. controls (4,-3.5) and (6,-3.5) .. (7.5,-1.8);
\node[blue!70!black, font=\large\bfseries] at (5,-4.2) {Amostragem};

% =========================
% Amostra (círculo menor)
% =========================
\draw[sample] (9,0) circle (1.8);
\node[blue!70!black, font=\large\bfseries] at (9,2.6) {Amostra};

\foreach \x/\y in {
 8.5/0.5, 9.2/0.8, 9.5/-0.3, 8.8/-0.7, 9.1/-1.1
}{
  \node[xmark] at (\x,\y) {x};
}

\end{tikzpicture}
\end{figure}



%parâmetros
\section{Noções Iniciais sobre Estatística}
\section{Método experimental vs. Método estatístico}
Para a investigação de um fenômeno, temos a nossa disposição dois métodos: experimental e estatístico. Resumidamente, o \textbf{MÉTODO EXPERIMENTAL} consiste em manter constantes as causas (fatores), com exceção de uma, variada para que seus efeitos sejam descobertos. Contudo, nem sempre poderemos aplicar o método experimental, pois os fatores que afetam um fenômeno podem não permanecer constantes enquanto variamos a causa que nos interessa.
Por exemplo, para analisarmos uma queda nas vendas de uma empresa nacional que produz chocolates finos, teríamos que considerar vários fatores que não necessariamente permanecerão constantes durante toda a investigação do fenômeno, tais como a região, o fluxo de turistas na localidade; a temperatura média; o preço do concorrente; o mês de férias, etc.

Assim, diante da impossibilidade de manter as causas ou fatores constantes, o \textbf{MÉTODO ESTATÍSTICO} admite e registra todas as possíveis variações das causas presentes, procurando determinar a influência de cada fator no resultado. Dessa forma, o método estatístico descobrirá relações entre os fatores, como, por exemplo, a influência da temperatura média e do fluxo de turistas na venda de chocolates finos.
\section{Dados Estatísticos}
Os dados estatísticos constituem os valores resultantes da coleta de dados. Os dados referem-se a um conjunto de valores, os quais são organizados por meio de variáveis (a característica está sendo medida) e observações (elementos da amostra/população). É o caso, por exemplo, dos valores obtidos na pesquisa de peso, altura, idade e sexo de uma determinada amostra de indivíduos/população.

Com relação ao número de observações coletadas, os dados são classificados em univariados, bivariados ou
multivariados:
\begin{enumerate}
    \item dados univariados: quando uma única observação de cada indivíduo é registrada.
    \item Por exemplo: peso;
    \item dados bivariados: quando duas observações de cada indivíduo são registradas. Por exemplo: peso e altura: A;
    \item dados multivariados: quando mais duas observações acerca de cada indivíduo são registradas. Por exemplo: peso, altura, sexo e idade.
    \item Quanto à forma de apresentação, os dados podem ser classificados em dados brutos ou rol.
\end{enumerate}
\subsection{Dados Brutos}

Os dados brutos são aqueles que não foram numericamente organizados em ordem crescente ou decrescente, ou seja, estão na forma como foram coletados. Como exemplo de dados brutos, podemos citar uma relação dos tempos médios de estudo diário, em minutos, de 50 alunos do Estratégia, na qual a seleção dos alunos ocorreu de forma aleatória, não havendo nenhuma ordenação de valores.

% \begin{table}[hbt]
%     \centering
%     \begin{tabular}{cc cc cc cc cc}
% \toprule
% \makecell{\textbf{Aluno}} &
% \makecell{\textbf{Tempo \\ min.}} &
% \makecell{\textbf{Aluno}} &
% \makecell{\textbf{Tempo \\ min.}} &
% \makecell{\textbf{Aluno}} &
% \makecell{\textbf{Tempo \\ min.}} &
% \makecell{\textbf{Aluno}} &
% \makecell{\textbf{Tempo \\ min.}} &
% \makecell{\textbf{Aluno}} &
% \makecell{\textbf{Tempo \\ min.}}\\
% \midrule
% 1	&143	&11	    &113	&21 	&170	&31	    &124	&41	    &105\\
% 2	&142	&12	    &143	&22	    &158	&32	    &137	&42	    &154\\
% 3	&161	&13	    &159	&23	    &123	&33	    &153	&43	    &99\\
% 4	&126	&14	    &168	&24	    &96	    &34	    &129	&44	    &114\\
% 5	&134	&15	    &123	&25	    &98	    &35	    &148	&45	    &161\\
% 6	&137	&16	    &135	&26	    &135	&36	    &173	&46	    &128\\
% 7	&171	&17	    &135	&27	    &129	&37	    &126	&47	    &175\\
% 8	&85	    &18	    &175	&28	    &126	&38	    &104	&48	    &137\\
% 9	&155	&19	    &115	&29	    &103	&39	    &157	&49	    &165\\
% 10  &171	&20	    &89		&171    &40		&50	    &115    &150    &170\\
% \bottomrule
%  \end{tabular}
%  \caption{Dados brutos}
% \end{table}

\begin{table}[hbt]
  \centering
  \renewcommand{\arraystretch}{1.15}
  \setlength{\tabcolsep}{6pt}

  \begin{tabular}{*{5}{cc}}
    \toprule
    \makecell{\bfseries Aluno} & \makecell{\bfseries Tempo\\min.} &
    \makecell{\bfseries Aluno} & \makecell{\bfseries Tempo\\min.} &
    \makecell{\bfseries Aluno} & \makecell{\bfseries Tempo\\min.} &
    \makecell{\bfseries Aluno} & \makecell{\bfseries Tempo\\min.} &
    \makecell{\bfseries Aluno} & \makecell{\bfseries Tempo\\min.} \\
    \midrule
     1 & 143 & 11 & 113 & 21 & 170 & 31 & 124 & 41 & 105 \\
     2 & 142 & 12 & 143 & 22 & 158 & 32 & 137 & 42 & 154 \\
     3 & 161 & 13 & 159 & 23 & 123 & 33 & 153 & 43 &  99 \\
     4 & 126 & 14 & 168 & 24 &  96 & 34 & 129 & 44 & 114 \\
     5 & 134 & 15 & 123 & 25 &  98 & 35 & 148 & 45 & 161 \\
     6 & 137 & 16 & 135 & 26 & 135 & 36 & 173 & 46 & 128 \\
     7 & 171 & 17 & 135 & 27 & 129 & 37 & 126 & 47 & 175 \\
     8 &  85 & 18 & 175 & 28 & 126 & 38 & 104 & 48 & 137 \\
     9 & 155 & 19 & 115 & 29 & 103 & 39 & 157 & 49 & 165 \\
    10 & 171 & 20 &  89 & 171 & 40 & 50 & 115 & 150 & 170\\
    \bottomrule
  \end{tabular}

  \caption{Dados brutos}
  %\label{tab:dados-brutos}
\end{table}


Esse tipo de tabela, onde os elementos não aparecem numericamente ordenados, é denominada de tabela primitiva. A tabela primitiva, em geral, oferece pouca ou nenhuma informação ao leitor, sendo necessário haver uma organização dos dados, a fim de torná-los mais expressivos.

\subsection{Rol}
O rol é a organização dos dados brutos em ordem de grandeza crescente ou decrescente. Com os dados organizados em rol, podemos saber, com facilidade, qual o menor e o maior elemento de um conjunto de dados. Os dados do nosso exemplo, isto é, os tempos médios de estudo diário, podem ser organizados em ordem crescente ou decrescente:

\noindent
\begin{minipage}[t]{0.49\textwidth}
\centering
\begin{tabular}{rrrrr}
\toprule
85  &115 &129 &143 &161\\
89  &115 &129 &143 &165\\
96  &123 &134 &148 &168\\
98  &123 &135 &153 &170\\
99  &124 &135 &154 &171\\
104 &126 &137 &157 &171\\
105 &126 &137 &158 &173\\
113 &127 &137 &159 &175\\
114 &128 &142 &161 &175\\
\bottomrule
\end{tabular}
\captionof{table}{Rol em ordem crescente}
\end{minipage}\hfill
\begin{minipage}[t]{0.49\textwidth}
\centering
\begin{tabular}{rrrrr}
\toprule
175 &161 &142 &128 &114\\
175 &159 &137 &127 &113\\
173 &158 &137 &126 &105\\
171 &157 &137 &126 &104\\
171 &155 &135 &126 &103\\
171 &154 &135 &124 & 99\\
170 &153 &135 &123 & 98\\
168 &148 &134 &123 & 96\\
165 &143 &129 &115 & 89\\
161 &143 &129 &115 & 85\\
\bottomrule
\end{tabular}
\captionof{table}{Rol em ordem decrescente}
\end{minipage}


\section{Variáveis Estatísticas}
\subsection{Variáveis Estatística.}

A variável estatística consiste no conjunto de características que desejamos averiguar estatisticamente. Ela também pode ser definida como o objeto da pesquisa estatística. Por exemplo, se nosso interesse é conhecer quantas horas os alunos do Estratégia estudam diariamente, então nossa variável é o número de horas estudadas por dia. As variáveis estatísticas podem ser classificadas, inicialmente, em duas categorias: qualitativas e quantitativas.

\subsection{Variáveis qualitativas}
As variáveis qualitativas são as características que não podem ser descritas de forma numérica, mas que podem ser definidas por meio de qualidades (atributos ou categorias) do indivíduo pesquisado. Elas podem ser classificadas em nominais ou ordinais:
\begin{enumerate}[label=\alph*.]
  \item variável qualitativa nominal (ou categórica), as possíveis categorias não podem ser ordenadas. Por exemplo, a cor dos olhos dos moradores de uma determinada cidade (pretos, castanhos, azuis e verdes).
  \item Variável qualitativa ordinal, as possíveis categorias podem ser ordenadas de alguma forma. Por exemplo, o grau de instrução dos funcionários de um determinado órgão (fundamental, médio, superior).
\end{enumerate}

\subsection{Variáveis qualitativas}
As variáveis quantitativas são características que podem ser descritas em termos de quantidades (valores numéricos), obtidas por meio de contagem ou mensuração. Elas podem ser classificadas em discretas e contínuas:
\begin{enumerate}[label=\alph*.]
    \item variáveis quantitativas discretas, os possíveis valores formam um conjunto finito ou enumerável de números e, geralmente, resultam de um processo de contagem. O número de ocorrências da característica em análise pode ser contado. Por exemplo, o número de leitos abertos em hospitais de uma determinada cidade;
    \item variáveis quantitativas contínuas, os possíveis valores formam um intervalo de números reais e, normalmente, resultam de um processo de mensuração. A característica pode ser medida em uma escala contínua, a qual podem ser associados um número infinito de possíveis valores, de modo a não haver lacunas ou interrupções. Por exemplo, a altura dos moradores de uma determinada cidade.
\end{enumerate}

\section{Séries Estatísticas}

Uma série estatística consiste em um conjunto de dados organizado com base em uma característica comum, ou seja, uma mesma variável. Normalmente, uma série estatística é representada por meio de uma tabela ou de um gráfico, conforme ficar melhor representado, a fim de sintetizar os dados estatísticos observados e torná-los mais compreensivos.

\begin{enumerate}[label=\alph*.]
    \item Corpo — conjunto de linhas e colunas com as informações sobre a variável em estudo;
    \item cabeçalho — parte superior que especifica o conteúdo das colunas;
    \item coluna indicadora — parte que indica o conteúdo das linhas;
    \item linhas — traços que facilitam a leitura dos dados;
    \item célula — espaço onde os dados são armazenados;
    \item título — identificação da tabela, contendo as informações sobre seu conteúdo;
    \item fonte — referência de onde os dados foram obtidos, localizada no rodapé.
\end{enumerate}
\begin{figure}[H]
    \begin{center}
        \includegraphics[width=.8\linewidth]{Estat_01.png}

    \end{center}
    \caption{Série estatística}
\end{figure}

Um gráfico é uma forma clara e objetiva de apresentar uma série estatística. O objetivo é proporcionar uma compreensão mais rápida do fenômeno em estudo. Para isso, o gráfico deve ser destituído de detalhes sem importância (ser simples); permitir a correta intepretação dos valores representativos do fenômeno (ser
claro); e transmitir a verdade sobre o fenômeno (ser verossímil). A série estatística apresentada na tabela anterior pode ser representada graficamente da seguinte forma:
\begin{figure}[hbt]
  \centering
  \begin{tikzpicture}
    \begin{axis}[
    xmin=1965, xmax=2015,
    ymin=0, ymax=250000,
    axis x line=middle,
    axis y line=middle,
    width=0.8 \textwidth,
    height=7cm,
    scale only axis,
    scaled y ticks=true,
    ytick scale label code/.code={$\times 10^{#1}$},
    xtick distance=10, % ajuste: 5, 10, 20...
      axis line style={-},
      tick align=outside,
      grid=both,
      minor tick num=1,
      xlabel={$Ano$},
      ylabel={$N$},
    ]
           \addplot[
        thick
      ] coordinates {
        (1970,93143)
        (1980,119011)
        (1991,146825)
        (2000,169799)
        (2010,190755)
      };
    \end{axis}
  \end{tikzpicture}
  \caption{Variação da População.}
\end{figure}

  Tabela.
  \begin{itemize}
    \item Quadro que resume um conjunto de observações.
    \item Composta de cabeçalho, corpo, coluna indicadora, linhas, células, título e fonte.
  \end{itemize}
  Gráfico.
  \begin{itemize}
    \item Forma simples e clara de apresentar uma série estatística.
    \item Proporcionar uma compreensão mais rápida do fenômeno em estudo.
    \item Deve ser simples, claro e verossímil.
  \end{itemize}
Finalmente, podemos verificar a presença de três elementos nas séries estatísticas: o tempo, o espaço e a espécie. Conforme os elementos variem, a série pode ser classificada em temporal (ou cronológica), geográfica (ou territorial) e específica.

\subsection{Séries Temporais (cronológicas)}

É a série cujos dados são dispostos segundo a época de ocorrência. Isto é, enquanto o tempo varia, o fato e o local permanecem constantes. Também são chamadas de séries históricas ou evolutivas. A principal característica é o fator cronológico variável.

A seguir temos a série histórica da população residente no Brasil no período de 1970 a 2010, com frequência decenal. Percebam que o tempo varia; contudo, o fato que está sendo analisado (quantidade populacional) e o local alvo da pesquisa (Brasil), continuam iguais.

\begin{table}[H]
    \centering
    \begin{tabular}{ll}
        \toprule
      Anos  &   População\\
      1970  &   93143\\
      1980  &   119011\\
      1991  &   146825\\
      2000  &   169799\\
      2010  &   190755\\
        \midrule
        \bottomrule
          \end{tabular}
    \caption{Censo 2010}
\end{table}

\subsection{Séries Geográficas (ou Territoriais)}
É a série cujos dados são dispostos segundo a localidade de ocorrência. Isto é, enquanto o local varia, o fato e o tempo permanecem constantes. Também são chamadas de séries espaciais ou de localização. A principal característica é o fator geográfico variável.

A seguir temos a série geográfica da população urbana residente em cada uma das regiões brasileiras no ano de 2010. Percebam que o local (região) varia; contudo, o fato que está sendo analisado (quantidade populacional) e o tempo (ano de 2010) permanecem constantes.


\begin{table}[H]
    \centering
    \begin{tabular}{ll}
        \toprule
      Região    &       População\\
      norte         &   11664\\
      nordeste      &   32821\\
      sudeste       &   74696\\
      sul           &   23260\\
      centro-oeste  &   12482\\
        \midrule
        \bottomrule
          \end{tabular}
    \caption{Censo 2010}
\end{table}

\subsection{Séries Específicas}

É a série cujos dados são dispostos segundo a modalidade de ocorrência. Isto é, enquanto o fato varia, a época e o local permanecem constantes. Também são chamadas de séries categóricas. A principal característica é o fator especificativo variável.

A seguir temos uma série específica das populações urbana e rural residentes no Brasil no ano de 2010. Percebam que os fatos analisados variam (população urbana x população rural); contudo, o tempo (2010) e o local de análise (Brasil) são onstantes.

\begin{table}[H]
    \centering
    \begin{tabular}{ll}
        \toprule
      Zona    &   População\\
      Urbano  &   93134\\
      Rural   &   119011\\
         \midrule
        \bottomrule
          \end{tabular}
    \caption{Censo 2010}
\end{table}

\subsection{Séries Mistas (ou Compostas)}

Muitas vezes, podemos ter a necessidade de apresentar, em uma única tabela, a variação de valores de mais de uma variável, isto é, combinar duas ou mais séries. As séries resultantes desse processo de combinação são chamadas de séries mistas (ou compostas) e apresentadas por meio de tabelas de dupla entrada.

O nome da nova série deve considerar pelo menos dois elementos. Assim, se for uma série mista de fato e tempo, denominaremos de série específico-temporal. A seguir temos uma série específicotemporal representando as populações de homens e mulheres residentes no Brasil, no período de 1970 a 2010, com variação decenal.

\begin{table}[H]
  \centering
  \caption{População residente no Brasil por sexo (1970--2010).}
  \label{tab:censo_sexo_1970_2010}
  \begin{tabular}{lrr}
    \toprule
      \multicolumn{1}{c}{} & \multicolumn{2}{c}{Sexo} \\
      \cmidrule(lr){2-3}
      \multicolumn{1}{c}{Ano} & \multicolumn{1}{c}{Homens} & \multicolumn{1}{c}{Mulheres} \\
    \midrule
    1970 & 46327 & 46807 \\
    1980 & 59142 & 59868 \\
    1991 & 72485 & 74340 \\
    2000 & 83602 & 86270 \\
    2010 & 93406 & 97348 \\
    \bottomrule
  \end{tabular}
\end{table}

Por sua vez, se tivermos uma série mista de local e tempo, denominaremos de série geográfica-temporal.
A seguir temos uma série geográfico-temporal representando as populações residentes em cada região brasileira, no período de 1970 a 2010, com variação decenal.

\begin{table}[H]
  \centering
  \begin{tabular}{lrrrrr}
    \toprule
      & \multicolumn{5}{c}{Regiões} \\
      \cmidrule(lr){2-6}
    Anos & N & NE & SE & S & CO \\
    \midrule
    1970 & 3105  & 28111 & 3980  & 16496 & 5072 \\
    1980 & 5880  & 34815 & 51737 & 19031 & 7545 \\
    1991 & 10030 & 42497 & 62470 & 22129 & 9427 \\
    2000 & 12900 & 47741 & 72412 & 25107 & 11636 \\
    2010 & 15864 & 53081 & 80364 & 27386 & 14058 \\
    \bottomrule
  \end{tabular}
  \caption{Censo 2010}
\end{table}

Por fim, devemos notar que podem existir séries compostas de três ou mais entradas, embora isso raramente aconteça, por conta da dificuldade de representação.
\section{Distribuição de Frequências}

Vimos anteriormente que, logo após a coleta de dados, temos o que denominamos de dados brutos. Como exemplo de dados brutos, citamos uma pesquisa de tempo médio de estudo diário, em minutos, envolvendo 50 alunos do Estratégia, onde os alunos foram escolhidos de maneira aleatória, não havendo nenhuma organização dos valores observados. Por serem apresentados na forma em que foram coletados, são denominados de dados brutos.

\begin{table}[hbt]
\centering
\resizebox{\textwidth}{!}{%
\begin{tabular}{l r r r r r r r r r}
\toprule
Aluno & Tempo & Aluno & Tempo & Aluno & Tempo & Aluno & Tempo & Aluno & Tempo\\
\midrule
1  & 143 & 11 & 113 & 21 & 170 & 31 & 124 & 41 & 105\\
2  & 142 & 12 & 143 & 22 & 158 & 32 & 137 & 42 & 154\\
3  & 161 & 13 & 159 & 23 & 123 & 33 & 153 & 43 & 99\\
4  & 126 & 14 & 168 & 24 & 96  & 34 & 129 & 44 & 114\\
5  & 134 & 15 & 123 & 25 & 98  & 35 & 148 & 45 & 161\\
6  & 137 & 16 & 135 & 26 & 135 & 36 & 173 & 46 & 128\\
7  & 171 & 17 & 135 & 27 & 129 & 37 & 126 & 47 & 175\\
8  & 85  & 18 & 175 & 28 & 126 & 38 & 104 & 48 & 137\\
9  & 155 & 19 & 115 & 29 & 103 & 39 & 157 & 49 & 165\\
10 & 171 & 20 & 89  & 30 & 171 & 40 & 127 & 50 & 115\\
\bottomrule
\end{tabular}%
}
\caption{Dados brutos.}
\label{tab:dados_brutos_1}
\end{table}

Normalmente, esses dados fornecem pouca informação ao leitor, sendo necessário organizá-los, com o propósito de aumentar sua capacidade informativa. A simples organização dos dados em um rol crescente já ajuda bastante nesse sentido. Com os dados organizados em rol, conseguimos verificar que o menor tempo observado foi de 85 minutos, e o maior, de 175 minutos, o que nos fornece uma amplitude total (\(AT = 175 - 85 = 90\)) de variação da ordem de 90 minutos.

\begin{table}[hbt]
    \centering
    \caption*{Rol em ordem crescente}
    \begin{tabular}{lrrrr}
        \toprule
        85  &115 &129 &143 &161\\
        89  &115 &129 &143 &165\\
        96  &123 &134 &148 &168\\
        98  &123 &135 &153 &170\\
        99  &124 &135 &154 &171\\
        104 &126 &137 &157 &171\\
        105 &126 &137 &158 &173\\
        113 &127 &137 &159 &175\\
        114 &128 &142 &161 &175\\
        \bottomrule
    \end{tabular}
\end{table}

Outra informação que conseguimos extrair dos dados organizados em rol crescente é que alguns tempos, como 126 min, 135 min, 137 min e 171 min, foram mais frequentes, ou seja, apareceram mais vezes durante a pesquisa.
Uma maneira mais concisa de mostrar os dados do rol é apresentar cada valor juntamente com o número de vezes em que ocorre, em vez de repeti-los. O número de ocorrências de um determinado valor recebe o nome de frequência. A tabela que contém todos os valores com suas respectivas frequências é denominada de distribuição de frequências.

Uma distribuição de frequências também pode ser definida como uma série estatística na qual permanecem constantes o fato, o local e a época. Ela pode ser classificada em dois tipos: distribuição de frequências pontual (ou discreta) e distribuição de frequências intervalar (ou contínua). Na distribuição de frequências pontual, são apresentados todos os dados coletados juntamente com suas respectivas frequências, não havendo perda de valores. Contudo, esse processo pode exigir muito espaço, especialmente quando o número de valores da variável tende a aumentar.

\begin{table}[H]
\centering
\setlength{\tabcolsep}{3pt}
\renewcommand{\arraystretch}{1.15}
\begin{tabular}{cc cc cc cc}
\toprule
\textbf{Tempo (min)} & \textbf{Freq.} &
\textbf{Tempo (min)} & \textbf{Freq.} &
\textbf{Tempo (min)} & \textbf{Freq.} &
\textbf{Tempo (min)} & \textbf{Freq.} \\
\midrule
85  & 1 & 114 & 1 & 135 & 3 & 158 & 1 \\
89  & 1 & 115 & 2 & 137 & 3 & 159 & 1 \\
96  & 1 & 123 & 2 & 142 & 1 & 161 & 2 \\
98  & 1 & 124 & 1 & 143 & 2 & 165 & 1 \\
99  & 1 & 126 & 3 & 148 & 1 & 168 & 1 \\
103 & 1 & 127 & 1 & 153 & 1 & 170 & 1 \\
104 & 1 & 128 & 1 & 154 & 1 & 171 & 3 \\
105 & 1 & 129 & 2 & 155 & 1 & 173 & 1 \\
113 & 1 & 134 & 1 & 157 & 1 & 175 & 2 \\
\bottomrule
\end{tabular}
\caption{Distribuição de frequências do tempo (min).}
\label{tab:tempo_freq}
\end{table}

Nesse caso, quando a variável é contínua, o mais recomendável é agrupar os valores por intervalos de classe. Desse modo, em vez de listar cada um dos valores que ocorrem, utilizamos uma distribuição de frequências intervalar, listando os intervalos de classe e as frequências correspondentes.

\begin{table}[H]
\centering
\caption{Distribuição de frequências do tempo médio.}
%\label{tab:tempo_medio_freq}
\setlength{\tabcolsep}{14pt}
\renewcommand{\arraystretch}{1.2}
\begin{tabular}{lc}
\toprule
\textbf{Tempo médio ($X_i$)} & \textbf{Frequência ($f_i$)} \\
\midrule
$85 \le x < 100$   & 5  \\
$100 \le x < 115$  & 5  \\
$115 \le x < 130$  & 12 \\
$130 \le x < 145$  & 10 \\
$145 \le x < 160$  & 7  \\
$160 \le x < 175$  & 9  \\
$175 \le x < 190$  & 2  \\
\bottomrule
\end{tabular}
\end{table}

Procedendo dessa forma, perdemos a informação detalhada dos tempos médios, mas ganhamos em termos de praticidade, o que simplifica o processo de análise de dados. Examinando a Tabela~\ref{tab:tempo_medio_freq}, percebemos facilmente que a maioria dos alunos estuda diariamente entre 115 e 130 minutos, enquanto uma minoria alcança entre 175 e 190 minutos.

Para identificar uma classe, temos que conhecer os valores dos limites inferior e superior da classe, que delimitam um intervalo de classe. Desse modo, precisamos definir a natureza do intervalo de classe, se aberto ou fechado. Portanto, temos as seguintes notações para os diferentes tipos de intervalos:

\begin{table}[H]
\centering
\caption{Tipos de intervalos e suas notações.}
\label{tab:intervalos}
\setlength{\tabcolsep}{11pt}
\renewcommand{\arraystretch}{1.5}
\begin{tabular}{p{3.6cm} p{2.5cm} p{2.4cm} p{4.8cm}}
\toprule
\textbf{\shortstack{Tipo de \\intervalo}} &
\textbf{\shortstack{Notação \\matemática}} &
\textbf{\shortstack{Notação \\estatística}} &
\textbf{Significado} \\
\midrule
Intervalo aberto &
$a < x < b$ &
$(a,b)$ &
Engloba todos os elementos entre $a$ e $b$, mas não engloba $a$ nem $b$. \\

Fechado à esquerda e aberto à direita &
$a \le x < b$ &
$[a,b)$ &
Engloba todos os elementos entre $a$ e $b$, inclusive $a$ mas não $b$. \\

Aberto à esquerda e fechado à direita &
$a < x \le b$ &
$(a,b]$ &
Engloba todos os elementos entre $a$ e $b$, inclusive $b$ mas não $a$. \\

Intervalo fechado &
$a \le x \le b$ &
$[a,b]$ &
Engloba todos os elementos entre $a$ e $b$, inclusive $a$ e $b$. \\
\bottomrule
\end{tabular}
\end{table}

Por fim, é importante salientarmos que, em análises estatísticas, constantemente encontramos distribuições de frequências intervalares, pois o objetivo da estatística é justamente fazer um apanhado geral das características de um conjunto de dados, sem adentrar em detalhes de casos particulares.

\subsection{Elementos de uma Distribuição de Frequências}
Agora, analisaremos detalhadamente cada elemento de uma distribuição de frequências. 

\begin{table}[H]
\centering
\caption{Distribuição de frequências do tempo médio.}
\label{tab:tempo_medio_freq}
\setlength{\tabcolsep}{14pt}
\renewcommand{\arraystretch}{1.2}
\begin{tabular}{lc}
\toprule
\textbf{Tempo médio ($X_i$)} & \textbf{Frequência ($f_i$)} \\
\midrule
$\,85 \le x < 100$   & 5  \\
$100 \le x < 115$  & 5  \\
$115 \le x < 130$  & 12 \\
$130 \le x < 145$  & 10 \\
$145 \le x < 160$  & 7  \\
$160 \le x < 175$  & 9  \\
$175 \le x < 190$  & 2  \\
\bottomrule
\end{tabular}
\end{table}

\subsection{Classe}
As classes são os intervalos nos quais o fenômeno é subdividido. Podemos dizer que as classes são os intervalos ou subdivisões dos elementos que compõem um conjunto de dados. Na Tabela~\ref{tab:tempo_medio_freq}, as classes são:
\(85 \le x < 100\),
\(100 \le x < 115\),
\(115 \le x < 130\),
\(130 \le x < 145\),
\(145 \le x < 160\),
\(160 \le x < 175\),
\(175 \le x < 190\).

Existem duas maneiras de determinar o número "ideal" de classes, \(k\), em função do número de dados da tabela, \(n\). A primeira consiste em utilizar a fórmula de Sturges:
\[
k = 1 + 3{,}3\log n.
\]
Outra abordagem, utilizada quando o número de dados é menor ou igual a 50, é:
\[
k=\sqrt{n}.
\]

\subsection{Limite de Classe}
Cada classe tem um limite inferior de classe \(l_{inf}\), sendo o menor número que pode pertencer à classe, e um limite superior de classe \(l_{sup}\), sendo o maior número que pode pertencer à classe. Os limites de uma classe são seus valores extremos.

\begin{table}[H]
\centering
\caption{Classes e respectivos limites inferior e superior.}
\label{tab:classes_limites}
\setlength{\tabcolsep}{12pt}
\renewcommand{\arraystretch}{1.2}
\begin{tabular}{lccc}
\toprule
\textbf{Classes} & \textbf& \textbf{\shortstack{Limite \\Inferior ($l_{\mathrm{inf}}$)}} & \textbf{\shortstack{Limite \\Superior ($l_{\mathrm{sup}}$)}} \\
\midrule
Pimeira classe  & $85  \le x < 100$   & 85  & 100 \\
Segunda Classe  & $100 \le x < 115$  & 100 & 115 \\
Terceira Classe & $115 \le x < 130$  & 115 & 130 \\
Quarta Classe   & $130 \le x < 145$  & 130 & 145 \\
Quinta Classe   & $145 \le x < 160$  & 145 & 160 \\
Sexta Classe    & $160 \le x < 175$  & 160 & 175 \\
Sétima Classe   & $175 \le x < 190$  & 175 & 190 \\
\bottomrule
\end{tabular}
\end{table}

\subsection{Amplitude de um Intervalo de Classe}
A amplitude de um intervalo de classe, ou simplesmente intervalo de classe, é a distância entre os limites inferiores (ou superiores) de classes consecutivas. Ela é obtida pela diferença entre dois limites inferiores (ou superiores) consecutivos:
\[
h=l_{sup}-l_{inf},
\]
em que \(l_{inf}\) é o limite inferior do intervalo de classe e \(l_{sup}\) é o limite superior do intervalo de classe.

\begin{table}[H]
\centering
\caption{Limites e amplitude das classes.}
\label{tab:classes_limites_amplitude}
\setlength{\tabcolsep}{8pt}
\renewcommand{\arraystretch}{1.2}
\begin{tabular}{l c c c}
\toprule
\textbf{Classes} &
\textbf{\shortstack{Limite Inferior\\($l_{\mathrm{inf}}$)}} &
\textbf{\shortstack{Limite Superior\\($l_{\mathrm{sup}}$)}} &
\textbf{\shortstack{Amplitude\\($h$)}} \\
\midrule
$85 \le x < 100$   & 85  & 100 & 15 \\
$100 \le x < 115$  & 100 & 115 & 15 \\
$115 \le x < 130$  & 115 & 130 & 15 \\
$130 \le x < 145$  & 130 & 145 & 15 \\
$145 \le x < 160$  & 145 & 160 & 15 \\
$160 \le x < 175$  & 160 & 175 & 15 \\
$175 \le x < 190$  & 175 & 190 & 15 \\
\bottomrule
\end{tabular}
\end{table}

Embora seja desejável, a amplitude do intervalo de classe nem sempre será constante ao longo de toda a distribuição de frequências intervalar.

\subsection{Amplitude Total}
A amplitude total é a diferença entre o limite superior da última classe (limite superior máximo) e o limite inferior da primeira classe (limite inferior mínimo). Portanto, corresponde à diferença entre o último e o primeiro elemento de um conjunto de dados ordenado crescentemente:
\[
AT=l_{max}-l_{min}.
\]
Note que, quando todas as classes possuem a mesma amplitude, também podemos determinar o valor da amplitude total multiplicando o valor do intervalo de classe \(h\) pela quantidade de classes da distribuição \(k\):
\[
AT=h\times k.
\]

\subsection{Ponto médio da classe}
O ponto médio é a média aritmética simples dos valores extremos de uma classe, ou seja, a soma dos limites inferior e superior dividida por dois. Esse ponto divide a classe em duas partes iguais. Também costuma ser chamado de marca ou representante da classe:
\[
PM=\frac{l_{sup}+l_{inf}}{2}.
\]

\begin{table}[H]
\centering
\caption{Limites e ponto médio das classes.}
\label{tab:Ponto_medio}
\setlength{\tabcolsep}{8pt}
\renewcommand{\arraystretch}{1.2}
\begin{tabular}{l c c c}
\toprule
\textbf{Classes} &
\textbf{\shortstack{Limite Inferior\\($l_{\mathrm{inf}}$)}} &
\textbf{\shortstack{Limite Superior\\($l_{\mathrm{sup}}$)}} &
\textbf{\shortstack{Ponto médio\\($PM$)}} \\
\midrule
$85 \le x < 100$   & 85  & 100 & 92,5 \\
$100 \le x < 115$  & 100 & 115 & 107,5 \\
$115 \le x < 130$  & 115 & 130 & 122,5 \\
$130 \le x < 145$  & 130 & 145 & 137,5 \\
$145 \le x < 160$  & 145 & 160 & 152,5 \\
$160 \le x < 175$  & 160 & 175 & 167,5 \\
$175 \le x < 190$  & 175 & 190 & 182,5 \\
\bottomrule
\end{tabular}
\end{table}

Veja que os pontos médios formaram uma progressão aritmética, pois a diferença entre dois pontos médios consecutivos foi constante e igual a 15. Isso ocorreu porque o intervalo de classe \(h\) também foi constante (e igual a 15) em toda a distribuição. Assim, quando o intervalo de classes é constante, a diferença entre os pontos médios também será constante e igual ao intervalo de classe.

Adicionalmente, sabendo que o ponto médio divide a classe em duas partes iguais, podemos derivar outras relações envolvendo o próprio ponto médio, a amplitude de classe e os limites inferior e superior.

\begin{figure}[hbt]
  \centering
  \begin{tikzpicture}[
  >=Stealth,
  br/.style={decorate, decoration={brace, amplitude=6pt}, thick, draw=blue!70!black},
  axis/.style={very thick, draw=blue!70!black},
  lab/.style={blue!70!black}
]
  \coordinate (A) at (0,0);
  \coordinate (B) at (10,0);
  \coordinate (M) at ($(A)!0.5!(B)$);

  \draw[axis, <->] (A) -- (B);

  \fill[blue!70!black] (M) circle (3pt);
  \node[lab, below=6pt] at (M) {PM};

  \node[lab, below=10pt] at (A) {$l_{\inf}$};
  \node[lab, below=10pt] at (B) {$l_{\sup}$};

  \draw[br] ($(A)+(0,0.9)$) -- ($(M)+(0,0.9)$)
    node[midway, above=8pt, lab] {$h/2$};
  \draw[br] ($(M)+(0,0.9)$) -- ($(B)+(0,0.9)$)
    node[midway, above=8pt, lab] {$h/2$};

  \draw[lab, thick] ($(A)!0.5!(M)+(0,0.9)$) -- ++(0,0.18);
  \draw[lab, thick] ($(M)!0.5!(B)+(0,0.9)$) -- ++(0,0.18);

  \draw[br, decoration={brace, mirror, amplitude=6pt}]
    ($(A)+(0,-0.9)$) -- ($(B)+(0,-0.9)$)
    node[midway, below=8pt, lab] {$h$};
\end{tikzpicture}
\end{figure}

Dada a figura anterior, podemos obter os limites de uma classe por meio das seguintes expressões:
\[
l_{inf}=PM-\frac{h}{2}
\qquad\text{e}\qquad
l_{sup}=PM+\frac{h}{2}.
\]
Além disso, podemos encontrar o ponto médio de uma classe a partir das seguintes relações:
\[
PM=l_{inf}+\frac{h}{2}
\qquad\text{e}\qquad
PM=l_{sup}-\frac{h}{2}.
\]

\subsection{Frequência}
Ao longo dessa aula, em várias oportunidades abordamos conceitos relacionados à frequência, isto é, ao número de ocorrências de um determinado valor ou de uma certa classe. Esse conceito é de grande relevância para a estatística descritiva e deve ser estudado de forma mais aprofundada. Nesse contexto, é importante sabermos que existem quatro tipos de frequência, os quais serão analisados nas subseções seguintes:

\begin{enumerate}[label=\alph*.]
  \item Frequência absoluta simples (\(f_i\))
  \item Frequência absoluta acumulada (\(f_{ac}\) e \(f_{ad}\))
  \item Frequência relativa simples (\(F_i\))
  \item Frequência relativa acumulada (\(F_{ac}\) e \(F_{ad}\))
\end{enumerate}

\subsection{Frequência Absoluta Simples}
A frequência absoluta simples corresponde ao número de observações correspondentes a uma determinada classe ou a um determinado valor.

\begin{table}[H]
  \centering
\begin{tabular}{llc}
  \toprule
i & \shortstack{Tempos \\(min)}  & \shortstack{Frequência \\\(f_i\)} \\
\midrule
1 & \(85\vdash100\)   & 5\\
2 & \(100\vdash115\)  & 5\\
3 & \(115\vdash130\)  & 12\\
4 & \(130\vdash145\)  & 10\\
5 & \(145\vdash160\)  & 7\\
6 & \(160\vdash175\)  & 9\\
\bottomrule
\end{tabular}
\caption{Frequência Absoluta Simples}
\label{tab:freq_abs-simp}
\end{table}

No exemplo, temos: \(f_1=5\), \(f_2=5\), \(f_3=12\), \(f_4=10\), \(f_5=7\), \(f_6=9\), \(f_7=2\).

A soma de todas as frequências é igual ao número total de dados analisados:

\[
  \sum_{i=1}^{k} f_i = n,
\]
em que a notação \(\sum_{i=1}^{k} f_i\) representa o somatório das frequências de cada uma das \(k\) classes.

\subsection{Frequência Absoluta Acumulada}
A frequência absoluta acumulada crescente \(f_{ac}\) é a soma das frequências de todos os valores inferiores ao limite superior do intervalo de uma determinada classe. Assim,
\[
  f_{ac,i}=f_1+f_2+\cdots+f_i.
\]

A frequência absoluta acumulada decrescente \(f_{ad}\) é a soma das frequências de todos os valores superiores ao limite inferior do intervalo de uma determinada classe. Assim,
\[
  f_{ad,i}=f_i+f_{i+1}+\cdots+f_k.
\]

A Tabela \ref{tab:freq_abs} apresenta a frequência absoluta simples e as frequências absolutas acumuladas (crescente e decrescente) para o exemplo.

\begin{table}[H]
  \centering
  \caption{Frequência absoluta simples e acumuladas (crescente e decrescente).}
  \label{tab:freq_abs}
  \begin{tabular}{lcccc}
    \toprule
    $\mathbf{i}$ &
    \textbf{\makecell{Tempos\\(min.)}} &
    \textbf{\makecell{Frequência\\($f_i$)}} &
    \textbf{\makecell{Frequência\\Acum. crescente ($f_{ac}$)}} &
    \textbf{\makecell{Frequência\\Acum. decrescente ($f_{ad}$)}} \\
    \midrule
    1 & $85 \le x < 100$   & 5  & 5  & 50 \\
    2 & $100 \le x < 115$  & 5  & 10 & 45 \\
    3 & $115 \le x < 130$  & 12 & 22 & 40 \\
    4 & $130 \le x < 145$  & 10 & 32 & 28 \\
    5 & $145 \le x < 160$  & 7  & 39 & 18 \\
    6 & $160 \le x < 175$  & 9  & 48 & 11 \\
    7 & $175 \le x < 190$  & 2  & 50 & 2  \\
    \midrule
    \multicolumn{2}{r}{\textbf{$n$}} & \textbf{50} & \textbf{50} & \textbf{50} \\
    \bottomrule
  \end{tabular}
  \label{tab:freq_absl}
\end{table}

\subsection{Frequência Relativa Simples}
A frequência relativa simples corresponde à proporção de dados existentes em uma determinada classe. Para calcular a frequência relativa de uma classe, dividimos a frequência absoluta simples \(f_i\) pela frequência total:
\[
  F_i=\frac{f_i}{n}.
\]
Para representar esses valores em termos de porcentagem, basta multiplicarmos por 100\%.

Repare que a soma de todas as frequências relativas deve ser igual a 100\%:
\[
  \sum_{i=1}^{k} F_i = 100\%.
\]

\subsection{Frequência Relativa Acumulada}
A frequência relativa acumulada crescente \(F_{ac}\) é a proporção de valores inferiores ao limite superior do intervalo de uma dada classe:
\[
  F_{ac,i}=F_1+F_2+\cdots+F_i.
\]
A frequência relativa acumulada decrescente \(F_{ad}\) é a proporção de valores superiores ao limite inferior do intervalo de uma dada classe:
\[
  F_{ad,i}=F_i+F_{i+1}+\cdots+F_k.
\]

A Tabela apresenta as frequências relativas simples e acumuladas (crescente e decrescente) para o exemplo.

\begin{table}[H]
  \centering
  \caption{Frequências relativas simples e acumuladas (em \%).}
  \label{tab:freq_rel}
  \begin{tabular}{lcccc}
    \toprule
    $\mathbf{i}$ &
    \textbf{\shortstack{Tempos\\(min.)}} &
    \textbf{\shortstack{Frequência\\Relativa ($F_i$)}} &
    \textbf{\shortstack{Frequência Rel.\\Acum. crescente ($F_{ac}$)}} &
    \textbf{\shortstack{Frequência Rel.\\Acum. decrescente ($F_{ad}$)}}\\
    \midrule
    1 & $85 \le x < 100$   & 10\% & 10\%  & 100\% \\
    2 & $100 \le x < 115$  & 10\% & 20\%  & 90\%  \\
    3 & $115 \le x < 130$  & 24\% & 44\%  & 80\%  \\
    4 & $130 \le x < 145$  & 20\% & 64\%  & 56\%  \\
    5 & $145 \le x < 160$  & 14\% & 78\%  & 36\%  \\
    6 & $160 \le x < 175$  & 18\% & 96\%  & 22\%  \\
    7 & $175 \le x < 190$  & 4\%  & 100\% & 4\%   \\
    \midrule
    \multicolumn{2}{r}{\(\sum f_i = 50\)} & \(\sum F_i = 100\%\) & & \\
    \bottomrule
  \end{tabular}
\end{table}

\section{Densidade de Frequência}
A densidade de frequência de uma classe consiste no quociente entre a frequência da classe (absoluta ou relativa) e sua amplitude:
\[
 \text{densidade}=\frac{\text{frequência}}{\text{amplitude}}.
\]
Usando a frequência absoluta e a amplitude do intervalo \(h_i\), temos:
\[
  d_i=\frac{f_i}{h_i}.
\]

\begin{table}[H]
  \centering
  \caption{Densidade de frequência (para $h_i=15$).}
  \label{tab:densidade_freq}
  \begin{tabular}{lccc}
    \toprule
    $\mathbf{i}$ &
    \textbf{\shortstack{Tempos\\(min.)}} &
    \textbf{\shortstack{Frequência\\($f_i$)}} &
    \textbf{\shortstack{Densidade de\\Frequência ($d_i$)}}\\
    \midrule
    1 & $85 \le x < 100$   & 5  & \(d_1=\frac{5}{15}=0{,}33\)\\
    2 & $100 \le x < 115$  & 5  & \(d_2=\frac{5}{15}=0{,}33\)\\
    3 & $115 \le x < 130$  & 12 & \(d_3=\frac{12}{15}=0{,}80\)\\
    4 & $130 \le x < 145$  & 10 & \(d_4=\frac{10}{15}=0{,}67\)\\
    5 & $145 \le x < 160$  & 7  & \(d_5=\frac{7}{15}=0{,}47\)\\
    6 & $160 \le x < 175$  & 9  & \(d_6=\frac{9}{15}=0{,}60\)\\
    7 & $175 \le x < 190$  & 2  & \(d_7=\frac{2}{15}=0{,}13\)\\
    \bottomrule
  \end{tabular}
\end{table}

\begin{table}[H]
\centering
\caption{Principais conceitos e notações da distribuição de frequências.}
\label{tab:conceitos_freq}
\setlength{\tabcolsep}{6pt}
\renewcommand{\arraystretch}{1.25}
\begin{tabular}{
  >{\raggedright\arraybackslash}p{3.1cm}
  >{\raggedright\arraybackslash}p{6.1cm}
  >{\raggedright\arraybackslash}p{6.0cm}
}
\toprule
\textbf{Item} & \textbf{Definição} & \textbf{Símbolos e fórmulas} \\
\midrule
Número de classes &
As classes são os intervalos nos quais o fenômeno é subdividido. &
\(\displaystyle
k = 1 + 3{,}3 \log n
\quad \text{ou} \quad
k = \sqrt{n}
\)
\\
Limites de classe &
Correspondem aos valores extremos de cada classe. &
\(\displaystyle
l_{\mathrm{inf}} \quad \text{e} \quad l_{\mathrm{sup}}
\)
\\
Amplitude de um intervalo de classe &
Distância entre os limites inferiores (ou superiores) de classes consecutivas. &
\(\displaystyle
h = l_{\mathrm{sup}} - l_{\mathrm{inf}}
\)
\\
Amplitude total &
Diferença entre o limite superior máximo e o limite inferior mínimo. &
\(\displaystyle
AT = l_{\max} - l_{\min}
\quad \text{ou} \quad
AT = h \times k
\)
\\
Ponto médio &
Média aritmética simples dos valores extremos de uma classe. &
\(\displaystyle
\begin{aligned}
PM &= \frac{l_{\mathrm{inf}} + l_{\mathrm{sup}}}{2}\\
PM &= l_{\mathrm{inf}} + \frac{h}{2}
     ,\quad
PM = l_{\mathrm{sup}} - \frac{h}{2}
\end{aligned}
\)
\\
Frequência absoluta simples &
Número de observações correspondentes a uma classe. &
\(\displaystyle f_i\)
\\
Frequência absoluta acumulada &
Soma das frequências até a classe considerada. &
\(\displaystyle
f_{ac,i} = f_1 + f_2 + \cdots + f_i
\)
\\
Frequência relativa simples &
Proporção de dados existentes em uma classe. &
\(\displaystyle
F_i = \frac{f_i}{n}
\)
\\
Frequência relativa acumulada &
Proporção acumulada até a classe considerada. &
\(\displaystyle
F_{ac,i} = F_1 + F_2 + \cdots + F_i
\)
\\
\bottomrule
\end{tabular}
\end{table}

\subsubsection{Quartil para dados agrupados em classes}

O cálculo do quartil para dados agrupados em classes será realizado por meio das seguintes etapas:


\section{Representação Gráfica das Distribuições de Frequências}
\subsection{Gráfico de Hastes ou Bastões}

O gráfico de hastes ou bastões é muito utilizado para representar dados não agrupados em classes, o que normalmente ocorre com dados discretos. Nesse caso, não há perda de informação pois os valores da variável aparecem individualmente, conforme constam da amostra. Com relação a sua construção, basta representarmos as frequências simples absolutas ou relativas de cada elemento do conjunto de dados.

\begin{figure}[H]
  \centering
  \begin{tikzpicture}
    \begin{axis}[
      xmin=-0.5, xmax=7,
      ymin=0, ymax=30,
      axis x line=middle,
      axis y line=middle,
      width=0.8 \textwidth,
      height=7cm,
      tick align=outside,
      grid=both,
      minor tick num=1,
      xlabel={$x$},
      ylabel={$y$},
      x axis line style={-{Stealth[length=3mm]}},
      y axis line style={-{Stealth[length=3mm]}},
    ]
      \filldraw[thick, fill=gray!30] (axis cs:0,4) circle[radius=1.5mm];
      \filldraw[thick, fill=gray!30] (axis cs:1,10) circle[radius=1.5mm];
      \filldraw[thick, fill=gray!30] (axis cs:2,25) circle[radius=1.5mm];
      \filldraw[thick, fill=gray!30] (axis cs:3,15) circle[radius=1.5mm];
      \filldraw[thick, fill=gray!30] (axis cs:4,10) circle[radius=1.5mm];
      \filldraw[thick, fill=gray!30] (axis cs:5,5) circle[radius=1.5mm];
      \filldraw[thick, fill=gray!30] (axis cs:6,1) circle[radius=1.5mm];
      \draw[blue, ultra thick] (0,0) -- (0,4);
      \draw[blue, ultra thick] (1,0) -- (1,10);
      \draw[blue, ultra thick] (2,0) -- (2,25);
      \draw[blue, ultra thick] (3,0) -- (3,15);
      \draw[blue, ultra thick] (4,0) -- (4,10);
      \draw[blue, ultra thick] (5,0) -- (5,5);
      \draw[blue, ultra thick] (6,0) -- (6,1);
    \end{axis}
  \end{tikzpicture}
  \caption{Gráfico de Hastes}
\end{figure}

O histograma é um gráfico destinado a representar dados agrupados em classe, sendo composto por um conjunto de retângulos contíguos (justapostos) cujas bases ficam localizadas sobre o eixo horizontal (eixo x), de forma que os seus pontos médios devem coincidir com os pontos médios dos intervalos de classe e seus limites devem coincidir com os limites da classe.

A quantidade de retângulos em um histograma é equivalente ao número de intervalos de classe. A largura de cada retângulo deve ser igual à amplitude do intervalo de classe, enquanto a altura precisa ser proporcional à frequência do intervalo de classe. Além disso, a área do histograma é proporcional ao somatório das frequências.

\begin{figure}[ht]
\centering
\begin{tikzpicture}
\begin{axis}[
  ybar,
  ymin=0,
  width=0.9\textwidth,
  height=7cm,
  axis lines=left,
  tick align=outside,
  grid=major,
  minor tick num=1,
  major grid style={opacity=0.2},
  xlabel={Classe},
  ylabel={Frequência},
  symbolic x coords={{85--100},{100--115},{115--130},{130--145},{145--160},{160--175},{175--190}},
  xtick=data,
  bar width=60pt,
  enlarge x limits=0
]
\addplot coordinates {
  ({85--100}, 5)
  ({100--115}, 5)
  ({115--130}, 12)
  ({130--145}, 10)
  ({145--160}, 7)
  ({160--175}, 9)
  ({175--190}, 2)
};
\end{axis}
\end{tikzpicture}
\caption{Histograma (classes e frequências).}
\end{figure}
A diferença básica entre um histograma e um gráfico de colunas (estudaremos na próxima seção) é a separação entre os retângulos adjacentes. Veja que não existe separação entre os retângulos no caso do histograma.

Dito isso, é importante mencionarmos a existência do gráfico denominado de poligonal
característica, que construímos utilizando somente os contornos do histograma.
\begin{figure}[hbt]
  \centering
  \begin{tikzpicture}
    \begin{axis}[
    xmin=80, xmax=200,
    ymin=0, ymax=14,
    axis x line=middle,
    axis y line=middle,
    width=0.8 \textwidth,
    height=7cm,
    %scale only axis,
    %scaled y ticks=true,
    %ytick scale label code/.code={$\times 10^{#1}$},
    %xtick distance=10, % ajuste: 5, 10, 20...
     x axis line style={-{Stealth[length=3mm]}},
     y axis line style={-{Stealth[length=3mm]}},
      tick align=outside,
      grid=major,
      minor tick num=1,
      xlabel={$x$},
      ylabel={$y$},
    ]
           \addplot[
        thick, red
      ] coordinates {
        (85,0)
        (85,5)
        (115,5)
        (115,12)
        (130,12)
        (130,10)
        (145,10)
        (145,7)
        (160,7)
        (175,7)
        (175,9)
        (190,9)
        (190,2  )
      };
    \end{axis}
  \end{tikzpicture}
  \caption{Contorno do Histograma.}
\end{figure}

\subsection{Polígono de Frequências}

    O polígono de frequências é um gráfico em linha obtido por meio da ligação, por segmentos de reta, dos pontos médios das bases superiores dos retângulos de um histograma. Também é necessário considerar a existência de uma classe anterior à primeira e outra posterior à última, ambas com a frequência nula.

    \begin{figure}[H]
        \centering
            \includegraphics[width=0.6 \linewidth]{histograma_poligono.png}
            \caption{Histograma com polígono}
        \end{figure}

\subsection{curva de Frequências}
A curva de frequências é obtida a partir do polimento de um polígono de frequências. Em sentido geométrico, o polimento corresponde à eliminação dos vértices (cantos) da linha poligonal. Esse processo suaviza os contornos do polígono de frequências, o que evidencia a verdadeira natureza dos dados em análise.

O polígono de frequências fornece a imagem real do fenômeno investigado, enquanto a curva de frequência mostra sua tendência. Naturalmente, quando o conjunto de dados é grande, a linha poligonal se torna curva. Por isso, podemos afirmar que a curva de frequência antecipa o comportamento da distribuição para um número maior de dados.

O processo de polimento é realizado por meio da seguinte fórmula:
\medskip
\[
    fc_i=\frac{f{ant}+2\times f_i+f_{post}}{4}
\]
\medskip
\begin{table}
    \centering
    \begin{tabular}{lccc}
        \toprule
        \shortstack{Tempo \\Médio\\\((X_i)\)}&\shortstack{Ponto \\Médio\\\((PM_i)\)} &Frequência\((f_i)\)&\shortstack{Frequência \\Calculada}\\
            \midrule
        \(70\leq\,x\,<85\)    &77,5   &0  &\(fc_0=\frac{0+2\times 0+5}{4}=1,25\)\\
        \(85\leq\,x\,<100\)   &92,2   &5  &\(fc_1=\frac{0+2\times 0+5}{4}=3,75\)\\
        \(100\leq\,x\,<115\)  &107,5  &5  &\(fc_2=\frac{0+2\times 0+5}{4}=6,75\)\\
        \(115\leq\,x\,<130\)  &122,5  &12 &\(fc_3=\frac{0+2\times 0+5}{4}=9,75\)\\
        \(130\leq\,x\,<145\)  &137,5  &10 &\(fc_4=\frac{0+2\times 0+5}{4}=9,75\)\\
        \(145\leq\,x\,<160\)  &152,5  &7  &\(fc_5=\frac{0+2\times 0+5}{4}=8,75\)\\
        \(160\leq\,x\,<175\)  &167,5  &9  &\(fc_6=\frac{0+2\times 0+5}{4}=6,75\)\\
        \(175\leq\,x\,<190\)  &192,5  &2  &\(fc_7=\frac{0+2\times 0+5}{4}=3,25\)\\
        \(190\leq\,x\,<205\)  &197,5  &0  &\(fc_8=\frac{0+2\times 0+5}{4}=0,50\)\\
        \bottomrule
    \end{tabular}
    \caption{Frequências calculadas}
\end{table}

\begin{figure}[H]
  \centering
  \includegraphics[width=0.8 \linewidth]{curva_de_frequencia.png}
  \caption{Curva de Frequência}
\end{figure}
\chapter{Medidas de Tendência Central}
\section{Noções Iniciais sobre Médias}
A média é um número que, de algum modo, resume as características de um grupo. Nosso contato
com a média surge ainda na escola, quando os professores calculam a média das nossas avaliações.
Suponha que um aluno tenha obtido as seguintes notas em determinada disciplina: 4; 10; 8; 10; 7; 9.
Com base nesses dados, podemos concluir que a média aritmética desse aluno nessa disciplina será 8.
Assim, a média consegue representar o conjunto dos valores observados.
\begin{figure}[H]
    \centering
    \includegraphics[width=0.4 \linewidth]{media.png}
    \caption{Média}
    \label{Fig: media}
\end{figure}

A média está presente em nosso cotidiano. Com certa frequência, as notícias abordam conceitos que
estão relacionados com a média: expectativa de vida dos brasileiros; idade média de uma população;
renda domiciliar per capita brasileira; consumo médio de combustível; tempo médio de deslocamento
em um trajeto.

Vamos analisar o primeiro exemplo. Quando o noticiário diz que expectativa de vida no Japão teve um
aumento recorde na última década, chegando a 84,2 anos. O que você entende diante dessa informação?

Essa informação reflete a qualidade de vida da população japonesa. Ela nos mostra que a população
japonesa está envelhecendo de forma saudável e que o sistema de saúde está sendo eficaz.

Se o noticiário também disser que a expectativa de vida em Angola gira em torno de 60 anos, você será
capaz comparar essas duas populações, certo? A resposta é sim. Quando comparados, os números
mostram que a qualidade de vida em Angola não é tão boa quanto a do Japão. Também nos dizem que
os angolanos tendem a viver, em média, 24 anos a menos que os japoneses.

Ao longo dessa aula, aprenderemos a calcular a média em diversas situações. Vamos ver que, a depender
de como os dados nos forem apresentados, o cálculo será feito de uma forma diferente. Além disso,
conheceremos algumas propriedades da média que facilitam a resolução de questões.
\section{Medidas de Posição}
Muitas vezes, queremos resumir um conjunto de dados apresentando um ou alguns valores que sejam representativos de uma série toda. As medidas de posição são estatísticas que caracterizam o comportamento dos elementos de uma série de dados, orientando quanto à posição da
distribuição em relação ao eixo horizontal do gráfico da curva de frequência.

Em outras palavras, podemos dizer que as medidas de posição indicam a tendência de concentração dos elementos de uma série, apontando o valor que melhor representa o conjunto de dados. Por exemplo, podemos ter uma medida para representar a posição de maior frequência de uma distribuição:
\medskip
\begin{tikzpicture}
\begin{axis}[
  width=0.8\linewidth,
  height=6cm,
  axis lines=middle,
  axis line style={-{Stealth[length=3mm,width=2mm]}},
  xmin=0, xmax=16,
  ymin=0,
  xtick=\empty,
  ytick=\empty,
  xlabel={Variável},
  ylabel={Frequência},
  enlargelimits=false,
  clip=false,
  % deixa o visual “limpo”
  every axis x label/.style={at={(axis description cs:1,0)}, anchor=west},
  every axis y label/.style={at={(axis description cs:0,1)}, anchor=south},
]

% --- (1) Barras (histograma estilizado) ---
% Cada coordinate é (centro da barra, altura)
\addplot[
  ybar,
  bar width=0.9,
  fill=blue!15,
  draw=blue!30,
] coordinates {
  (1,0.2) (2,0.9) (3,2.0) (4,3.0) (5,3.6)
  (6,3.9) (7,3.7) (8,3.2) (9,2.6) (10,2.0)
  (11,1.5) (12,1.1) (13,0.8) (14,0.55) (15,0.35)
};

% --- (2) Curva suave por cima (estilo “densidade”) ---
% Aqui eu estou desenhando uma curva manual (coordenadas suavizadas).
\addplot[
  smooth,
  very thick,
  blue!70!black,
] coordinates {
  (0,0.05) (1,0.2) (2,0.9) (3,2.0) (4,3.0) (5,3.6)
  (6,3.95) (7,3.7) (8,3.2) (9,2.6) (10,2.0)
  (11,1.5) (12,1.1) (13,0.8) (14,0.55) (15,0.35) (16,0.25)
};

% --- (3) Linhas: Moda, Mediana, Média ---
% Ajuste os x conforme sua distribuição.
\def\xModa{6}
\def\xMediana{8}
\def\xMedia{10}

% linhas verticais tracejadas até o topo do gráfico
\addplot[blue!70!black, dashed, thick] coordinates {(\xModa,0) (\xModa,4.1)};
%\addplot[blue!70!black, dashed, thick] coordinates {(\xMediana,0) (\xMediana,3.2)};
%\addplot[blue!70!black, dashed, thick] coordinates {(\xMedia,0) (\xMedia,2.0)};

% rótulos embaixo do eixo x
\node[font=\itshape] at (axis cs:\xModa,-0.35) {Medida Central};
%\node[font=\itshape] at (axis cs:\xMediana,-0.35) {Mediana};
%\node[font=\itshape] at (axis cs:\xMedia,-0.35) {Média};

\end{axis}
\end{tikzpicture}

As medidas de posição podem ser divididas em:
\begin{enumerate}[label=\alph*.]
    \item medidas de tendência central: representam o ponto central ou o valor típico de um conjunto de dados, indicando onde está localizada a maioria dos valores de uma distribuição.
    \begin{itemize}
        \item média aritmética: é a medida de posição mais utilizada, sendo o valor resultante da divisão entre a soma de todos os valores de uma série de observações e o número de observações;
        \item mediana: valor que ocupa a posição central de uma série de observações, quando organizadas em ordem crescente ou decrescente; e
        \item moda: valor mais frequente em uma série de observações,
    \end{itemize}
\end{enumerate}

Somente a título de exemplo, vejamos como as medidas de tendência central se posicionam em relação a
uma distribuição de frequências. Notem que essas medidas tendem a ocupar as posições centrais da
distribuição, sendo denominadas de medidas de tendência central.
\medskip

\begin{tikzpicture}
\begin{axis}[
  width=0.8\linewidth,
  height=6cm,
  axis lines=middle,
  axis line style={-{Stealth[length=3mm,width=2mm]}},
  xmin=0, xmax=16,
  ymin=0,
  xtick=\empty,
  ytick=\empty,
  xlabel={Variável},
  ylabel={Frequência},
  enlargelimits=false,
  clip=false,
  % deixa o visual “limpo”
  every axis x label/.style={at={(axis description cs:1,0)}, anchor=west},
  every axis y label/.style={at={(axis description cs:0,1)}, anchor=south},
]

% --- (1) Barras (histograma estilizado) ---
% Cada coordinate é (centro da barra, altura)
\addplot[
  ybar,
  bar width=0.9,
  fill=blue!15,
  draw=blue!30,
] coordinates {
  (1,0.2) (2,0.9) (3,2.0) (4,3.0) (5,3.6)
  (6,3.9) (7,3.7) (8,3.2) (9,2.6) (10,2.0)
  (11,1.5) (12,1.1) (13,0.8) (14,0.55) (15,0.35)
};

% --- (2) Curva suave por cima (estilo “densidade”) ---
% Aqui eu estou desenhando uma curva manual (coordenadas suavizadas).
\addplot[
  smooth,
  very thick,
  blue!70!black,
] coordinates {
  (0,0.05) (1,0.2) (2,0.9) (3,2.0) (4,3.0) (5,3.6)
  (6,3.95) (7,3.7) (8,3.2) (9,2.6) (10,2.0)
  (11,1.5) (12,1.1) (13,0.8) (14,0.55) (15,0.35) (16,0.25)
};

% --- (3) Linhas: Moda, Mediana, Média ---
% Ajuste os x conforme sua distribuição.
\def\xModa{6}
\def\xMediana{8}
\def\xMedia{10}

% linhas verticais tracejadas até o topo do gráfico
\addplot[blue!70!black, dashed, thick] coordinates {(\xModa,0) (\xModa,4.1)};
\addplot[blue!70!black, dashed, thick] coordinates {(\xMediana,0) (\xMediana,3.2)};
\addplot[blue!70!black, dashed, thick] coordinates {(\xMedia,0) (\xMedia,2.0)};

% rótulos embaixo do eixo x
\node[font=\itshape] at (axis cs:\xModa,-0.35) {Moda};
\node[font=\itshape] at (axis cs:\xMediana,-0.35) {Mediana};
\node[font=\itshape] at (axis cs:\xMedia,-0.35) {Média};

\end{axis}
\end{tikzpicture}

\begin{enumerate}[label=\alph*., start=2]
    \item medidas separatrizes: dividem (ou separam) uma série em duas ou mais partes, cada uma contendo a mesma quantidade de elementos. As medidas mais utilizadas são:
    \begin{itemize}
        \item mediana: divide uma série em duas partes iguais. Reparem que, além de ser uma medida separatriz, a mediana também é uma medida de tendência central;
        \item quartis: dividem uma série em quatro partes iguais;
        \item decis: dividem uma série em dez partes iguais; e
        \item percentis: dividem uma série em cem partes iguais.
    \end{itemize}
\end{enumerate}


\section{Notação de Somatório}

Com frequência, as fórmulas matemáticas exigem a adição de muitas variáveis, como a média aritmética. O somatório ou notação sigma é uma forma simples e conveniente de abreviação, usada para fornecer uma expressão concisa para a soma dos valores de uma variável. Por exemplo, se quisermos representar a soma de um número de termos tais como:
\[
    1+2+3+4+5
\]
ou,
\[
    1^2+2^2+3^2+5^2+5^2
\]
em que há um padrão evidente para os números envolvidos.

De modo geral, se tomarmos uma sequência de números \(x_1, x_2, x_3, \dots, x_n\) então podemos escrever a soma desses números como \(x_1, x_2, x_3, \dots, x_n\). Nesse conjunto, \(x_1\) representa o primeiro termo; \(x_2\) representa o segundo; \(x_3,\) o terceiro; e \(x_i\) o i-ésimo termo da soma.

Essa soma pode ser representada de uma forma mais simples e concisa, deixando que \(x_i\) represente o termo geral da sequência. Para isso, empregamos a seguinte notação:
\[
    \sum_{i=1}^n\,x_i
\]
Então, em vez de usarmos vários elementos para determinarmos o somatório, utilizamos somente o símbolo do somatório:

Essa notação envolve um símbolo de somatório, \(\sum\), sendo a letra grega maiúscula Sigma \(\Sigma\). Basicamente, esse símbolo está nos instruindo a somar determinados elementos de uma sequência. Os elementos típicos da sequência que está sendo somada aparecem à direita do símbolo de somatório:
\[
    {\sum}\,x_i
\]

Observe que essa notação também requer a definição de um índice, que fica localizado abaixo do símbolo de somatório. Esse índice é frequentemente representado por \(i\), embora também seja comum encontrarmos questões adotando \(j\) ou \(n\).
Esse índice normalmente aparece como uma expressão, por exemplo, \(i=a\), em que o índice assume um valor inicial atribuído no lado direito da equação, conhecido como limite inferior \(a\). Se considerarmos que \(i=1\), estamos dizendo que o primeiro elemento da sequência a ser considerado é o de índice igual a 1, isto é, o primeiro elemento da sequência. A condição de parada ou limite superior do somatório é o valor localizado acima do símbolo, no caso \(b\). A condição de parada indica o último elemento da sequência a ser considerado no somatório.
\[
    \sum_{i=a}^b\,x_i
\]
Então, se tivermos uma sequência de 10 valores, devemos interpretar a notação a seguir como a soma dos valores da sequência \(x_i\), com \(i\) variando de 1 a 10:
\[
    \sum_{i=1}^{10}\,x_i
\]
Vejamos alguns exemplos típicos de operações envolvendo somatórios. Para isso, tomaremos como exemplo a sequência {\(x_i\)} = {1, 2, 3, 4, 5, 6, 7, 8, 9, 10}:
\newpage
\begin{table}[H]
\centering
\renewcommand{\arraystretch}{1.3}
\setlength{\tabcolsep}{10pt}

\begin{tabularx}{\textwidth}{c X}
\toprule
\textbf{Notação} & \textbf{Interpretação e exemplo} \\
\midrule

$\displaystyle \sum_{i=3}^{10} x_i$
&
Representa a soma dos valores de $x$, começando em $x_3$ e terminando em $x_{10}$.
\[
\sum_{i=3}^{10} x_i = x_3 + x_4 + x_5 + x_6 + x_7 + x_8 + x_9 + x_{10}
\]
\[
= 3 + 4 + 5 + 6 + 7 + 8 + 9 + 10 = 55
\]
\\

$\displaystyle \sum x$
&
Quando os limites do somatório são omitidos, entende-se que a soma é feita de
$x_1$ até $x_n$.
\[
\sum x = x_1 + x_2 + x_3 + \dots + x_n
\]
\[
= 1 + 2 + 3 + \dots + 10 = 55
\]
\\

$\displaystyle \sum_{i=1}^{n} x_i^2$
&
Representa a soma dos quadrados dos valores de $x$, de $x_1$ até $x_n$.
\[
\sum_{i=1}^{n} x_i^2 = x_1^2 + x_2^2 + x_3^2 + \dots + x_n^2
\]
\[
= 1^2 + 2^2 + 3^2 + \dots + 10^2 = 385
\]
\\

$\displaystyle \sum_{k=1}^{4} (2x_k + 1)$
&
Representa a soma dos termos da sequência $(2n+1)$, com $k$ variando de 1 a 4.
\[
= (2\cdot1+1) + (2\cdot2+1) + (2\cdot3+1) + (2\cdot4+1)
\]
\[
= 3 + 5 + 7 + 9 = 24
\]
\\
$\displaystyle \sum_{i=3}^{5}\left(\frac{x_i}{x_i+1}\right)$
&
Representa a soma dos termos $\dfrac{x_i}{x_i+1}$, com $i$ começando em 3 e terminando em 5.
\[
\sum_{i=3}^{5}\left(\frac{x_i}{x_i+1}\right)
= \frac{x_3}{x_3+1} + \frac{x_4}{x_4+1} + \frac{x_5}{x_5+1}
\]
\[
= \frac{3}{3+1} + \frac{4}{4+1} + \frac{5}{5+1}
= \frac{3}{4} + \frac{4}{5} + \frac{5}{6}
\]
\[
= \frac{45 + 48 + 50}{60} = \frac{143}{60}
\]
\\

\bottomrule
\end{tabularx}

\caption{Interpretação de diferentes notações de somatório.}
\end{table}

Agora que já entendemos o funcionamento básico dessa notação, precisamos analisar outras operações aritméticas que também podem ser realizadas com as variáveis num somatório. Para tanto, vamos tomar como base as sequências {\(x_i\)} = {1, 2, 3, 4, 5, 6, 7, 8, 9, 10} e {\(y_i\)} = {3, 6, 9, 12, 15, 18, 21, 24, 27, 30}.

Por exemplo, na SOMA DOS PRODUTOS, multiplicamos \(x_1\) por \(y_1\); \(x_2\) por 
\(y_2\); e assim por diante, até \(x_n\)
por \(y_n\). Em seguida, somamos os resultados de cada multiplicação. A SOMA DOS PRODUTOS da variável
\(x\) pela variável \(y\), com \(i\) variando de 1 a 10, pode ser representada por meio da seguinte expressão:

\[
    \sum\limits_{i=1}^n\,x_i\times y_i=x_1\times y_1+x_2\times y_2+x_3\times y_3+\dots+x_n\times y_n
\]
\[
    \sum\limits_{i=1}^n\,x_i\times y_i=1\times 3+2\times 6+3\times 9+\dots+10\times 30 = 1155
\]
Observe que essa expressão é diferente de \(\sum_{i=1}^n\,x_i\times \sum_{i=1}^n\,y_i\times\), que representa o \textbf{PRODUTO DAS SOMAS} dessas duas variáveis. No \textbf{PRODUTO DAS SOMAS}, primeiro somamos toda a sequência \(x\), depois toda
a sequência \(y\) e, em seguida, multiplicamos o resultado das somas:

\[
    \sum\limits_{i=1}^{10}\,x_i \times\sum\limits_{i=1}^{10}\,y_i=(1+2+3+\dots+10)\times (3+6+9+\dots+30)
\]
\[
    \sum\limits_{i=1}^{10}\,x_i \times\sum\limits_{i=1}^{10}\,y_i=55\times 165=9075
\]

Dessa forma, temos que:
\begin{center}
    \textbf{SOMA DOS PRODUTOS} \(\neq\) \textbf{PRODUTO DAS SOMAS}
\end{center}

Também podemos utilizar a notação para representar o QUADRADO DA SOMA dos valores de \(x\), com \(i\) iniciando em 1 e terminando em 10. No QUADRADO DA SOMA, somamos toda a sequência e elevamos o resultado ao quadrado:

\begin{align*}
    \Bigg(\sum\limits_{i=1}^{n}\,x_i\Bigg)^2&=(x_1+x_2+x_3+\dots+x_n)^2\\
    \Bigg(\sum\limits_{i=1}^{n}\,x_i\Bigg)^2&=(1+2+3+\dots+10)^2\\
    \Bigg(\sum\limits_{i=1}^{n}\,x_i\Bigg)^2&=(55)^2=3025\\
\end{align*}

Veja que essa expressão é diferente de \(\sum_{i=n}^n\,x_i^2\), que representa a \textbf{SOMA DOS QUADRADOS}. Na \textbf{SOMA DOS QUADRADOS}, cada elemento da sequência é elevado ao quadrado e depois os resultados são somados:

\begin{align*}
      \sum\limits_{i=1}^{n}\,x_i^2&=1^2+2^2+3^2+\dots+10^2=385\\
\end{align*}

Por fim, ainda podemos representar o somatório de uma constante \(k\). Digamos que essa constante tenha valor igual a 3:

\begin{align*}
      \sum\limits_{i=1}^{n}\,k&=k+k+k+\dots+k=k\times n\\
      \sum\limits_{i=1}^{n}\,3&=3+3+3+\dots+3=3\times n\\
\end{align*}

\subsection{Propriedades do Somatório}
As propriedades apresentadas nesta seção facilitam o desenvolvimento de expressões algébricas com a notação de somatório.

\begin{enumerate}
    \item O somatório de uma constante \(k\) é igual ao produto do número de termos pela constante.
    \begin{align*}
      \sum\limits_{i=1}^{n}\,k&=k+k+k+\dots+k=k\times n\\
\end{align*}
Para demonstrar essa propriedade, consideraremos que cada constante está multiplicada pelo valor um, isto é:
\[
    \sum\limits_{i=1}^{n}\,k=k+k\times1+k\times1+\dots+k\times1
\]
Agora, colocaremos os novos valores em evidência:

\[
    \sum\limits_{i=1}^{n}\,k=k\,{\underbrace{(1+1+1+\dots+1)}_{n\,\text{termos}}}=k\times n
\]

Considere uma lista composta por noventa e nove elementos repetidos e iguais a 9:
\[
    \{\underbrace{{9, 9, 9, 9,\dots, 9}}_{\text{99 termos repetidos}}\}
\]

O somatório dos elementos dessa lista será:
\[
    \sum\limits_{i=1}^{99}\,9=\,{\underbrace{(9+9+9+\dots+9)}_{n\,\text{termos repetidos}}}=9\times \,{\underbrace{(9+9+9+\dots+9)}_{n\,\text{termos repetidos}}}=9\times99=891
\]

    \item O somatório do produto de uma constante por uma variável é igual ao
produto da constante pelo somatório da variável.

    Para demonstrarmos essa propriedade, colocaremos em evidência cada constante \(k\):
   \[
    \sum\limits_{i=1}^n\,k\times x_i = k\times x_1+ k\times x_2+k\times x_3 +\dots+k\times x_n=k\times\sum_{n=1}^{n}\,x_i
\] 
Para demonstrarmos essa propriedade, colocaremos em evidência cada constante \(k\):

   \[
    \sum\limits_{i=1}^n\,k\times x_i = k\times x_1+ k\times x_2+k\times x_3 +\dots+k\times x_n = k \times (x_1+x_2+x_3+\dots+x_n)
\] 

Já sabemos que \(\sum_{i=1}^{n}\,x_i=x_1+x_2+\dots+x_n\). Logo:

\[
\sum_{i=1}^{n}\,k\times x_i=k\times\sum_{i=1}^{n}x_i
\]

Portanto, a constante pode sair de dentro do somatório, passando a multiplicá-lo.

\end{enumerate}
\section{Média Aritmética Simples}
A média aritmética simples está muito presente em nosso cotidiano, seja no consumo médio de combustível, na temperatura média ou na renda per capita. Essa medida é definida como o \textbf{QUOCIENTE} entre a \textbf{SOMA DE TODOS OS ELEMENTOS} e o \textbf{NÚMERO DELES}. A propriedade principal da média é preservar a soma dos elementos de um conjunto de dados.

Podemos adotar o seguinte raciocínio para encontrarmos a fórmula da média aritmética. Dada uma lista de \(n\) números, \(x_1, x_2, \dots, x_n\) a soma de seus termos é igual a:

\[
    \overbrace{x_1+x_2+x_3+\dots+x_n}^{\text{n fatores}}
\]
A média aritmética dessa lista é um número , tal que, se todos os elementos forem substituídos por \(\bar{x}\), a soma da lista permanecerá preservada. Assim, substituindo todos os elementos por \(\bar{x}, \bar{x}, \dots, \bar{x}\), teremos uma nova lista, cuja soma é:

\[
    \overbrace{\bar{x}+\bar{x}+\dots+\bar{x}}^{\text{n fatores}}=n\times\bar{x}
\]
Como as somas das duas listas são iguais, temos:
\[
    n\times\bar{x}=\bar{x_1}+\bar{x_2}+\dots+\bar{x_n}
\]

Portanto, a média aritmética é:

\[
    \bar{x}=\frac{{x_1}+{x_2}+\dots+x_n}{n}
\]

\subsection{Propriedades da Média Aritmética}

Nessa seção, vamos estudar algumas propriedades importantes sobre a média aritmética.

\begin{propriedade}
Dado um conjunto com \(n\geq1\) elementos, a média aritmética sempre existirá e será única.
\end{propriedade}

 Desde que o conjunto tenha pelo menos um elemento, podemos afirmar que a média aritmética sempre existe, pois sempre conseguiremos calcular o quociente entre a soma dos elementos e o número deles. Além disso, como o somatório dos elementos resulta em um único número, o valor da média também sempre será único.

\begin{propriedade}
  A média aritmética \(\bar{x}\) de um conjunto de dados satisfaz a expressão \(m\leq\bar{x}\leq M\), em que \(m\) e \(M\) são, respectivamente, os elementos que representam o valor mínimo e o valor máximo desse conjunto.  
\end{propriedade}

\begin{center}
        \(\text{\textbf{Mínimo}}\leq\bar{x}\leq \text{\textbf{Máximo}}\)
    \end{center}

      Essa propriedade diz respeito ao fato da média aritmética sempre se encontrar entre os números mínimo e máximo de um conjunto.

\begin{propriedade}
    Somando-se (ou subtraindo-se) uma constante \(c\) de todos os valores de uma variável, a média do conjunto fica aumentada (ou diminuída) dessa constante.
\end{propriedade}

\begin{propriedade}
    Multiplicando-se (ou dividindo-se) uma constante c de todos os valores de uma variável, a média do conjunto fica multiplicada (ou dividida) por esta constante.
\end{propriedade}
\[
    \bar{y}=\bar{x}\times c \quad\text{ou}\quad \bar{y}=\bar{x}\div c
\]
 Portanto, não importa qual a sequência de números, a soma dos desvios em relação à média é sempre igual a zero.
\begin{propriedade}
    A soma algébrica dos desvios em relação à média é nula.
\end{propriedade}
    \[
    \sum_{i=1}^{n}\,(x_i-\bar{x})=0
    \]
\begin{propriedade}
    A soma dos quadrados dos desvios da sequência de números \({x_1}\), em relação a um número \(a\), é mínima se \(a\) for a média aritmética dos números.
\end{propriedade}
\[
    \sum_{i=1}^{n}(x_i-a)^2\geq \sum_{i=1}^{n}(x_i-\bar{x})^2
\]

    Essa propriedade afirma que, caso os desvios sejam calculados com relação a um número diferente da média, e os resultados de tais desvios sejam elevados ao quadrado e somados, teremos um número necessariamente maior do que obteríamos caso a mesma operação fosse realizada utilizando-se a média.
\section{Média Ponderada}

Muitas vezes, certos elementos de um conjunto de dados possuem relevância maior que os demais. Nessa situação, para calcular a média de tais conjuntos, devemos encontrar uma média ponderada. Uma média ponderada é a média de um conjunto de dados cujos valores possuem pesos variados.

Ela é calculada pela igualdade a seguir, em que \(p\) é o peso de cada valor de \(x\):
\medskip
\[
    \bar{x}=\frac{\sum_{i=1}^{n}(x_i\times p_1)}{\sum_{i=1}^{n}p_i}
\]

Observe que no numerador cada valor será multiplicado pelo seu respectivo peso, enquanto no denominador teremos a soma de todos os pesos.

Suponha que um candidato tenha prestado um concurso público para o cargo de Auditor Fiscal, alcançando as seguintes notas:

\begin{table}[H]
    \centering
\begin{tabular}{lr}
\toprule
Disciplina                  &Nota \(x_i\)\\
\midrule
Língua Portuguesa           &4,0\\
Direito Administrativo      &4,0\\
Direito Constitucional      &4,0\\
Direito Tributário          &7,0\\
Legislação Tributária       &7,0\\
Contabilidade               &8,0\\
Auditoria                   &8,0\\
\bottomrule
\end{tabular}
\caption{Notas.}
\end{table}

Considere, também, que o edital desse concurso previa que algumas disciplinas teriam importância maior do que outras, por isso foram atribuídos pesos diferentes às várias disciplinas. Digamos que os
pesos tenham sido distribuídos da seguinte forma:

\begin{table}[H]
    \centering
\begin{tabular}{lr}
\toprule
Disciplina                  &Peso \(p_i\)\\
\midrule
Língua Portuguesa           &1\\
Direito Administrativo      &2\\
Direito Constitucional      &2\\
Direito Tributário          &3\\
Legislação Tributária       &3\\
Contabilidade               &3\\
Auditoria                   &4\\
\bottomrule
\end{tabular}
\caption{Pesos.}
\end{table}

Agora, admita que o candidato deveria alcançar uma nota 7,0 ou superior na prova objetiva para que fosse convocado para a etapa discursiva. Se você fosse um dos avaliadores desse concurso, você consideraria o candidato aprovado na prova objetiva?

Para responder a esse questionamento, devemos calcular a média aritmética ponderada desse candidato, considerando os pesos de cada disciplina. Dessa forma, devemos multiplicar cada nota pelo seu respectivo peso, somar esses produtos e dividir pela soma dos pesos.

\begin{table}[H]
    \centering
\begin{tabular}{lccc}
\toprule
Disciplina                  &Notas \(x_i\)& Pesos \(p_i\) &\(x_i\times p_i\)\\
\midrule
Língua Portuguesa           &4,0         &1 & \(4,0\times 1\)=4,0\\
Direito Administrativo      &4,0         &2 & \(4,0\times 2\)=8,0\\
Direito Constitucional      &4,0         &2 & \(4,0\times 2\)=8,0\\
Direito Tributário          &6,0         &3 & \(7,0\times 3\)=18,0\\
Legislação Tributária       &6,0         &3 & \(7,0\times 3\)=18,0\\
Contabilidade               &7,0         &3 & \(8,0\times 3\)=21,0\\
Auditoria                   &7,0         &3 & \(8,0\times 3\)=21,0\\
\bottomrule
\end{tabular}
\caption{Notas.}
\end{table}

Nesse ponto, temos uma lista contendo todos os produtos de notas e pesos. Então, a média aritmética
ponderada é dada por:

\[
    \bar{x}=\frac{\sum_{i=1}^{n}(x_i\times p_1)}{\sum_{i=1}^{n}}=\frac{4,0+8,0+8,0+18,0+18,0+21,0+21,0}{1+2+2+3+3+3+3}=\frac{98}{17}=5,76
\]
\section{Média para Dados Agrupados}
Em estatística, os dados podem ser definidos como informações que representam os atributos qualitativos ou quantitativos de uma variável, ou de um conjunto de variáveis. Esses dados podem ser classificados em agrupados e não-agrupados. Normalmente, logo após a etapa de coleta, temos dados
não-agrupados ou dados brutos.

Por exemplo, suponha que o Estratégia Concursos esteja realizando um experimento com um grupo de dez alunos, para mensurar o tempo médio de resposta a uma questão de estatística. Logo após a coleta, os dados continuam brutos, pois não passaram por nenhuma análise nem foram agrupados de alguma forma. Então, teríamos uma tabela similar a seguinte:

\begin{table}[H]
    \centering
\begin{tabular}{lcccccccccc}
\toprule
Aluno       &1&2&3&4&5&6&7&8&9&10\\
\midrule
Tempo Médio &2&3&5&7&9&6&6&5&3&9\\
\bottomrule
\end{tabular}
\caption{Tempos.}
\end{table}

Por sua vez, os dados agrupados são aqueles que passaram por algum nível de análise, o que significa que já não são brutos. Os dados agrupados podem ser organizados por frequência de um determinado valor ou por intervalos de classes. Quando por frequência de valor, os dados são organizados de forma ascendente e suas ocorrências são contabilizadas:

\begin{table}[H]
    \centering
\begin{tabular}{cc}
\toprule
\shortstack{Tempo \\Médio \(X_i\)} &Frequência \(f_i\)\\
\midrule
1      &1\\
3      &2\\
5      &2\\
6      &2\\
7      &1\\
9      &2\\
\bottomrule
\end{tabular}
\caption{Frequências.}
\end{table}

Quando por intervalos de classes, os dados também são organizados de forma ascendente, porém, em classes preestabelecidas, e as ocorrências de cada classe são contabilizadas:

\begin{table}[H]
    \centering
\begin{tabular}{lc}
\toprule
\shortstack{Tempo \\Médio \(X_i\)} &Frequência \(f_i\)\\
\midrule
0\(\leq x \le 2\)      &1\\
2\(\leq x \le 4\)      &2\\
4\(\leq x \le 6\)      &2\\
6\(\leq x \le 8\)      &3\\
8\(\leq x \le 10\)     &2\\
\bottomrule
\end{tabular}
\caption{Frequências.}
\end{table}

Para dados agrupados e apresentados como diagramas ou tabelas, a definição da média permanece inalterada, então, tudo o que estudamos até o momento permanece válido, mas teremos métodos específicos para obtenção da média. A seguir, veremos como proceder em cada caso.

\subsection{Média para Dados Agrupados por Valor}

Dando continuidade ao nosso exemplo, vamos a calcular a média aritmética de dados agrupados por valor. Os dados foram organizados na tabela a seguir:

\begin{table}[H]
    \centering
\begin{tabular}{cc}
\toprule
\shortstack{Tempo \\Médio \(X_i\)} &Frequência \(f_i\)\\
\midrule
1      &1\\
3      &2\\
5      &2\\
6      &2\\
7      &1\\
9      &2\\
\bottomrule
\end{tabular}
\caption{Frequências.}
\end{table}

Como podemos interpretar essa tabela? Basta você saber que as frequências refletem o número de repetições de cada valor da nossa variável tempo médio. Isto é, um aluno conseguiu responder à questão em 1 minuto, dois alunos conseguiram em 3 minutos, dois alunos conseguiram em 5 minutos, e assim
sucessivamente.

Para calcularmos a média a partir de uma tabela de frequências como esta, devemos utilizar a seguinte fórmula:

\[
    \bar{x}=\frac{\sum_{i=1}^{n}(X_i\times f_i)}{\sum_{i=1}^{n}\,f_i}
\]

A aplicação dessa fórmula é bem simples. O raciocínio é o mesmo adotado para a média ponderada, sendo que, agora, o peso é representado pela frequência. Desse modo, vamos multiplicar cada valor por sua respectiva frequência, somar tudo e dividir pela soma das frequências:

\begin{table}[H]
    \centering
\begin{tabular}{ccl}
\toprule
\shortstack{Tempo \\Médio \(X_i\)} &Frequência \(f_i\)  &\(f_i\times X_i\)\\
\midrule
1      &1   &\(1\times 1 =1\)\\
3      &2   &\(2\times 3 =2\)\\
5      &2   &\(2\times 5 =10\)\\
6      &2   &\(2\times 6 =12\)\\
7      &1   &\(1\times 7 =7\)\\
9      &2   &\(2\times 9 =18\)\\
\bottomrule
\end{tabular}
\caption{\text{\(f_i\times X_i\)}}
\end{table}

Após isso, somaremos todos os valores da coluna \(X_i\times f_i\), obtendo o termo \(\sum_{i=1}^{n} X_i\times f_i\), e também somaremos os termos da coluna \(f_1\) , obtendo o termo \(\sum_{i=1}^{n} f_i\). Veja a última linha da tabela:

\begin{table}[H]
    \centering
\begin{tabular}{lll}
\toprule
\shortstack{Tempo \\Médio \(X_i\)} &Frequência \(f_i\)  &\(f_i\times X_i\)\\
\midrule
1      &1   &\(1\times 1 =1\)\\
3      &2   &\(2\times 3 =2\)\\
5      &2   &\(2\times 5 =10\)\\
6      &2   &\(2\times 6 =12\)\\
7      &1   &\(1\times 7 =7\)\\
9      &2   &\(2\times 9 =18\)\\
\midrule
Totais &10  &54\\
\bottomrule
\end{tabular}
\caption{\text{\(f_i\times X_i\)}}
\end{table}

Agora, basta dividirmos um valor pelo outro, obtendo:

\[
    \bar{x}=\frac{\sum_{i=1}^{n}(X_i\times f_i)}{\sum_{i=1}^{n}\,f_i}=\frac{54}{10}=5,4
\]

Portanto, a média dos dados apresentados na tabela é 5,4.

\subsection{Média para Dados Agrupados por Classe}

Retomando nosso exemplo, vamos calcular a média aritmética de dados agrupados por classe. Os dados foram organizados na tabela a seguir:

\begin{table}[H]
    \centering
\begin{tabular}{lc}
\toprule
\shortstack{Tempo \\Médio \(X_i\)} &Frequência \(f_i\)\\
\midrule
0\(\leq x \le 2\)      &1\\
2\(\leq x \le 4\)      &2\\
4\(\leq x \le 6\)      &2\\
6\(\leq x \le 8\)      &3\\
8\(\leq x \le 10\)     &2\\
\bottomrule
\end{tabular}
\caption{Separação por classe.}
\end{table}

Como podemos interpretar essa tabela? Basta sabermos que as frequências refletem o número de ocorrências em cada um dos intervalos definidos para a variável tempo médio. Isto é, um aluno respondeu à questão com tempo médio abaixo de 2 minutos, dois responderam com tempo médio entre 2 e 4 minutos, dois com tempo médio entre 4 e 6 minutos, e assim sucessivamente.

Ao agruparmos os dados em classes, precisaremos fazer uma modificação em relação ao cálculo anterior: substituir os intervalos pelos seus respectivos pontos médios. Como assim? Ao invés de considerarmos o intervalo de 0 a 2 minutos, por exemplo, substituiremos pelo valor de 1 minuto.

Em nosso exemplo, a identificação dos pontos médios é relativamente fácil. Mas é possível que você encontre situações em que isso não seja tão trivial. Como fazer nesses casos? Devemos calcular a média dos dois extremos do intervalo. Assim, o ponto médio \(PM\) é calculado pela seguinte expressão:

\[
  PM=\frac{l_{sup}+l_{inf}}{2}
\]

em que \(l_inf\) e \(l_sup\) são, respectivamente, os limites inferior e superior do intervalo considerado.

Na tabela abaixo, repare que foi incluída uma nova coluna para o cálculo dos pontos médios:

\begin{table}[H]
    \centering
\begin{tabular}{lcc}
\toprule
\shortstack{Tempo \\Médio \(X_i\)} & \shortstack{Ponto Médio \\\(PM_i\)} &Frequência \(f_i\)\\
\midrule
0\(\leq x \le 2\)    &1  &1\\
2\(\leq x \le 4\)    &3  &2\\
4\(\leq x \le 6\)    &5  &2\\
6\(\leq x \le 8\)    &7  &3\\
8\(\leq x \le 10\)   &9  &2\\
\bottomrule
\end{tabular}
\caption{Separação por classe.}
\end{table}

O próximo passo consiste em calcular os valores das multiplicações \(PM_i\times f_i\) multiplicando essas duas colunas. Vamos ver:
\begin{table}[H]
    \centering
\begin{tabular}{lccc}
\toprule
\shortstack{Tempo \\Médio \(X_i\)} & \shortstack{Ponto Médio \\\(PM_i\)} &Frequência \(f_i\)    & \(PM_i\times f_i\)\\
\midrule
0\(\leq x \le 2\)    &1  &1 &\(1\times1=1\)\\
2\(\leq x \le 4\)    &3  &2 &\(3\times2=6\)\\
4\(\leq x \le 6\)    &5  &2 &\(5\times2=10\)\\
6\(\leq x \le 8\)    &7  &3 &\(7\times3=21\)\\
8\(\leq x \le 10\)   &9  &2 &\(9\times2=18\)\\
\bottomrule
\end{tabular}
\caption{Separação por classe.}
\end{table}

Após isso, somaremos todos os valores da coluna \(PM_i\times f_i\), obtendo o termo \(\sum_{i=1}^{n}\,PM_i\times f_i\) , e também somaremos os termos da coluna \(f_i\), obtendo o termo \(\sum_{i=1}^{n}\,f_i\). Veja a última linha da tabela:

\begin{table}[H]
    \centering
\begin{tabular}{lccc}
\toprule
\shortstack{Tempo \\Médio \(X_i\)} & \shortstack{Ponto Médio \\\(PM_i\)} &Frequência \(f_i\)    & \(PM_i\times f_i\)\\
\midrule
0\(\leq x \le 2\)    &1  &1 &\(1\times1=1\)\\
2\(\leq x \le 4\)    &3  &2 &\(3\times2=6\)\\
4\(\leq x \le 6\)    &5  &2 &\(5\times2=10\)\\
6\(\leq x \le 8\)    &7  &3 &\(7\times3=21\)\\
8\(\leq x \le 10\)   &9  &2 &\(9\times2=18\)\\
\midrule
Total & & 10 & 56\\
\bottomrule
\end{tabular}
\caption{Separação por classe.}
\end{table}

Agora, basta dividirmos um valor pelo outro, obtendo:

\[
    \bar{x}=\frac{\sum_{i=1}^{n}(PM_i\times f_i)}{\sum_{i=1}^{n}f_i}=\frac{56}{10}=5,6
\]

Finalmente, note que a média de dados agrupados por classe (5,6) foi diferente da média de dados agrupados por valor (5,4). Por que isso ocorreu? Isso ocorreu porque, ao agruparmos os valores da
variável em classes, perdemos detalhes que eram relevantes para o cálculo exato da média, embora a forma de apresentação tenha sido simplificada.
\section{Média Geométrica}%

A média geométrica é uma medida estatística muito utilizada em situações de acumulação de percentuais, fato muito comum em problemas financeiros. Também é encontrada na geometria plana,
quando, por exemplo, devemos fazer com que a área de um quadrado seja igual à área de um retângulo.

Essa medida é definida, para o conjunto de números positivos, como a raiz \textit{n}-ésima do produto de \textit{n} elementos de um conjunto de dados. A propriedade principal dessa média é preservar o
produto dos elementos de um conjunto de dados.

O raciocínio para encontrarmos a fórmula da média geométrica é análogo ao adotado para a média aritmética. Dada uma lista de \textit{n} números, \({x_1, x_2, \dots, x_n}\), o produto de seus termos é igual a:

\[
    \underbrace{{x_1\times x_2\times \dots\times x_n}}_{\text{n fatores}}
\]

A média geométrica dessa lista é um número \textit{G}, tal que, se todos os elementos forem substituídos por textit{G}, o produto da lista permanecerá preservado. Assim, substituindo todos os elementos por \textit{G}, teremos
uma nova lista, \({G_1, G_2, \dots, G_n}\), cujo produto é:

\[
    \underbrace{{G\times G\times \dots\times G}}_{\text{n fatores}}=G^n
\]

Como os produtos das duas listas são iguais, temos:

\[
    G^n=x_1\times x_2\times \dots\times x_n=\sqrt[n]{x_1\times x_2\times \dots\times x_n}
\]

Note que a raiz varia conforme a quantidade de elementos da lista de números, isto é, se a lista contém dois números, teremos uma raiz quadrada; se a lista contém três números, teremos uma raiz cúbica.
\section{Média Harmônica}

A média harmônica é muito utilizada quando precisamos trabalhar com grandezas inversamente proporcionais. É o caso de problemas clássicos, como o cálculo da velocidade média de um automóvel ou da vazão de torneiras (quanto tempo duas ou mais torneiras levam para encher um tanque).

Essa medida é definida, para o conjunto de números positivos, como o inverso da média
aritmética dos inversos. A propriedade principal dessa média é preservar a soma dos inversos dos elementos de um conjunto de números.

O raciocínio para encontrarmos a fórmula da média harmônica é similar ao adotado para as médias aritmética e geométrica. Dada uma lista de \textit{x} números, \({x_1, x_2, \dots, x_n}\), a soma dos inversos de seus termos é igual a:

\[
    \underbrace{\frac{1}{x_1}+\frac{1}{x_2}+\dots+\frac{1}{x_n}}_{\text{n fatores}}
\]

A média harmônica dessa lista é um número \textit{H}, tal que, se todos os elementos forem substituídos por \textit{H}, a soma dos inversos permanecerá preservada. Assim, substituindo todos os elementos por H, teremos uma lista, \(H, H, \dots, H\), cuja soma dos inversos é:

\[
    \underbrace{\frac{1}{H}+\frac{1}{H}+\dots+\frac{1}{H}}_{\text{n fatores}}=\frac{n}{H}
\]

Como as somas dos inversos das duas listas são iguais, temos:

\[
    \frac{n}{H}=\frac{1}{x_1}+\frac{1}{x_2}+\dots+\frac{1}{n}
\]

\[
    n=H\times \Bigg(\frac{1}{x_1}+\frac{1}{x_2}+\dots+\frac{1}{n}\Bigg)
\]

\[
    H=\frac{n}{\frac{1}{x_1}+\frac{1}{x_2}+\dots+\frac{1}{n}}
\]

Como vimos no início, muitas vezes, a média harmônica é descrita como o inverso da média
aritmética dos inversos. Isso porque a fórmula acima também pode ser escrita na forma mostrada a
seguir, em que \((\frac{1}{x_1}+\frac{1}{x_2}+\dots+\frac{1}{n})/n\) corresponde à média aritmética dos inversos.

\section{Desigualdade das Médias}   
Dada uma lista de .. números positivos, \({x_1, x_2, \dots, x_n}\), podemos afirmar que:

\[
    \boxed{\bar{x}\geq G \geq H}
\]

em que \(\bar{x}\) é a média aritmética; \(G\) é a média geométrica e \(H\) é a média harmônica.
Significa dizer que a média aritmética será sempre maior ou igual à média geométrica que, por sua vez, será sempre maior ou igual à média harmônica. A igualdade ocorrerá quando os números da lista forem todos iguais.

Tomemos como exemplo os números 4, 12 e 20. Como sabemos, a média aritmética será:

\[
    \bar{x}=\frac{4+12+20}{3}=12
\]

A média geométrica é

\[
    G=\sqrt[3]{4\times 12\times 20}\approx9,86
\]

E a média harmônica é:

\[
    H=\frac{3}{\frac{1}{4}+\frac{1}{12}+\frac{1}{20}}\approx2,61
\]
Obtivemos, portanto, uma média aritmética \(\bar{x} = 12\) maior que a média geométrica \(G = 9,86\) que, por sua vez, é maior que a média harmônica \(H = 2,61\).

Agora, analisaremos um caso em que as três médias são iguais: considere uma lista composta pelos números 5, 5 e 5. Nesse caso, temos que \(\bar{x}, G e H\) são, respectivamente:

\[
    \bar{x}=\frac{5+5+5}{3}=5
\]

A média geométrica é

\[
    G=\sqrt[3]{5\times 5\times 5}=5
\]

E a média harmônica é:

\[
    H=\frac{3}{\frac{1}{5}+\frac{1}{5}+\frac{1}{5}}=5
\]

Portanto, quando todos os números da lista são iguais, as média aritmética \(\bar{x}\), geométrica \(G\) e harmônica \(H\) também são iguais.
\section{Mediana}

\subsection{Medidas Separatrizes}
As separatrizes são medidas que dividem (ou separam) uma série ordenada em duas ou mais partes, cada uma contendo a mesma quantidade de elementos. Nesse caso, o nome da medida separatriz é definido conforme a quantidade de partes onde a série é dividida:

\begin{itemize}
    \item mediana: divide uma série ordenada em duas partes iguais, cada uma contendo 50\% dos valores da sequência;
    \item quartis: dividem uma série ordenada em quatro partes iguais, cada uma contendo 25\% dos valores da sequência;
    \item quintis: dividem uma série ordenada em cinco partes iguais, cada uma contendo 20\% dos valores da sequência;
    \item decis: dividem uma série ordenada em dez partes iguais, cada uma contendo 10\% dos valores da sequência; e
    \item percentis: dividem uma série ordenada em cem partes iguais, cada uma contendo 1\% dos valores da sequência.
\end{itemize}

Ao longo da aula, vamos estudar a mediana, os quartis, os decis e os percentis. Os quintis, por não serem tão explorados em provas de concurso, não serão abordados.

\subsection{Mediana}

A mediana é, simultaneamente, uma MEDIDA DE POSIÇÃO, de TENDÊNCIA CENTRAL e SEPARATRIZ. Ela caracteriza a posição central de uma série de valores. Além disso, também separa uma série de valores em duas partes de tamanhos iguais, cada uma contendo o mesmo número de elementos. Muitas vezes, a mediana é designada como valor mediano, sendo representada pelos símbolos \(M_d\) ou, em menor frequência, \(\tilde{x}\)

\subsubsection{Mediana para Dados não Agrupados}
O método para determinação da mediana envolve a realização de uma etapa anterior, que consiste na ordenação do conjunto de dados. Feito isso, a mediana é o elemento que ocupa a POSIÇÃO CENTRAL de uma série de observações ORDENADA segundo suas grandezas (isto é, dados brutos organizados em rol crescente ou decrescente).

Por exemplo, vamos determinar a mediana da seguinte série de valores:

\[
    \{3, 17, 13, 19, 2, 5, 7, 1, 8, 21, 9\}
\]

Conforme a definição da mediana, a primeira etapa consiste na ordenação (crescente ou decrescente) da série de valores. Desse modo, obtemos:

\[
    \{1, 2, 3, 5, 7, 8, 9, 13, 17, 19, 21\}
\]

Agora, determinaremos o elemento que ocupa a posição central desse conjunto de dados. Para isso, devemos encontrar o termo que possui o mesmo número de elementos tanto à sua esquerda quanto à sua direita. Em nosso exemplo, esse valor é o 8, pois existem cinco elementos antes dele e cinco após ele.

\[
    \underbrace{1, 2, 3, 5, 7}_{\text{5 elementos antes}}, \underbrace{8}_{\text{Elemento central}}, \underbrace{9, 13, 17, 19, 21}_{\text{5 elementos depois}}
\]

É importante notarmos que essa série possui um número ímpar de elementos. Quando isso acontece, isto é, quando uma série possui um NÚMERO ÍMPAR de elementos, a MEDIANA SEMPRE COINCIDE com o ELEMENTO CENTRAL do conjunto de dados. Portanto, temos:

\begin{center}
    \(M_d=8\)
\end{center}
Contudo, se porventura a série tivesse um número par de elementos, POR CONVENÇÃO, a MEDIANA seria a MÉDIA ARITMÉTICA dos dois termos centrais. Assim, caso dicionássemos o número 23 ao conjunto de dados apresentado anteriormente, teríamos a seguinte situação:

\[
    \underbrace{1, 2, 3, 5, 7}_{\text{5 elementos antes}}, \underbrace{8, 9}_{\text{Elemento central}}, \underbrace{ 13, 17, 19, 21, 23}_{\text{5 elementos depois}}
\]
Nesse caso, em que temos um número par de elementos, a mediana é definida como a média aritmética dos termos centrais, sendo os números 8 e 9. Assim,
\[
    M_d=\frac{8+9}{2}=8,5
\]
Note que, quando o número é ímpar, o termo central sempre ocupa a posição \(\frac{n+1}{2}\). Por outro lado, quando o número de termos é par, existem dois termos centrais, sendo que o primeiro ocupa a posição \(\frac{n}{2}\); e o
segundo ocupa a posição imediatamente seguinte, ou seja, \(\frac{n}{2}+1\).

Essas relações são importantes porque nem sempre conseguiremos identificar a posição central tão facilmente. Por exemplo, se tivermos uma série composta por 501 elementos, podemos afirmar que o termo central será o elemento ocupando a posição \(\frac{n+1}{2}=\frac{501+1}{2}=251\), sem precisar recorrer a qualquer outro
método. Logo, a mediana terá o mesmo valor do termo central:
\[
    M_d=x_{251}
\]
Vejamos a disposição do termo central em relação aos demais elementos da série:
\[
    \underbrace{x_1, x_2, \dots, x_{250}}_{\text{250 elementos antes}}, \underbrace{x_{251}}_{\text{Termo central}}, \underbrace{ x_{252}, x_{253}, \dots, x_{501}}_{\text{250 elementos depois}}
\]
Agora, se tivermos uma série composta por 500 elementos, os termos centrais serão os elementos ocupando as posições:

\[
    \frac{n}{2}=\frac{500}{2}=250\,\text{e}\,\frac{n}{2}+1=\frac{500}{2}+1=251
\]
Vejamos a disposição dos termos centrais em relação aos demais elementos da série:
\[
    \underbrace{x_1, x_2, \dots, x_{249}}_{\text{250 elementos antes}}, \underbrace{x_{250}, x_{251}}_{\text{Termo central}}, \underbrace{ x_{252}, x_{253}, \dots, x_{500}}_{\text{250 elementos depois}}
\]
Nessa situação, por convenção, a mediana será a média aritmética entre os termos centrais,
\[
    M_d=\frac{x_{250}+x_{251}}{2},
\]
Portanto, podemos estabelecer que a mediana de um conjunto composto por \(n\)elementos ordenados de forma crescente ou decrescente será:
\begin{enumerate}[label=\alph *.]
    \item se \(n\) for impar, o termo de ordem \(\frac{n+1}{2}\), isto é:
    \[
        M_d=\frac{x_n+1}{2}
    \]
    \item se \(n\) for par, a média aritmética dos termos de ordem \(\frac{n}{2}\) e \(\frac{n}{2}+1\), isto é:
    \[
        M_d=\frac{\frac{x_n}{2}+\frac{x_n}{2}+1}{2}
    \]
\end{enumerate}
Como vimos, a mediana depende somente do termo que ocupa a posição central em um conjunto de dados, e não dos valores de todos os elementos que compõem a série. Por isso, dizemos que a mediana não sofre tanta influência pela presença de valores extremos/discrepantes quanto à média. Essa é, inclusive, uma das principais diferenças entre essas duas medidas.

Podemos constatar essa propriedade da mediana por meio de um exemplo. Considere que tenhamos inicialmente a seguinte série:
\[
    \{1, 2, 4, 6, 7, 9, 10, 11, 13\}
\]
A média aritmética desses valores é:

\[
    \bar{x}=\frac{1+2+4+6+7+9+10+11-13}{9}=\frac{63}{9}=7
\]
Como o número de elementos é ímpar, \(n\)= 9, a mediana será o elemento que ocupa a posição:

\[
    \frac{n}{2}+1=\frac{9}{2}+1=\frac{10}{2}=5
\]

\[
    M_d=x_5\Rightarrow M_d=7
\]

Agora, considere que o elemento de valor 13 tenha sido alterado para 130.000. Veja o que acontece com a média aritmética desse conjunto:

\[
    \bar{x}=\frac{1+2+4+6+7+9+10+11-130.000}{9}=14.450
\]
Mas como o número de elementos é o mesmo assim como o elemento central, a mediada continua sendo 7.

Portanto, a alteração no valor de um único elemento do conjunto de dados causou um impacto significativo na média, ao passo que a mediana permaneceu inalterada. Por isso, dizemos que a média é mais influenciada pela presença de valores extremos que a mediana.

\subsubsection{Mediana para dados Agrupados sem Intervalos de Classe}

O raciocínio adotado no cálculo da mediana para dados agrupados por valor (sem intervalos de classe) é similar ao empregado no caso dos dados não-agrupados. Basicamente, teremos que encontrar um valor que dividirá a distribuição de frequências em duas partes contendo o mesmo número de elementos.

Considere a seguinte situação hipotética: uma empresa realizou uma pesquisa para medir o nível de satisfação dos clientes com relação ao seu atendimento. Os clientes puderam atribuir notas de 0 a 5 no que diz respeito ao nível de satisfação, resultando na seguinte distribuição de frequências:
\pgfplotstableread{
Xi fi
0  5
1  5
2  8
3  10
4  13
5  10
}\Satisf

% ===== TABELA (gerada dos dados) =====
\begin{table}[H]
  \centering
  \pgfplotstabletypeset[
    columns={Xi,fi},
    columns/Xi/.style={column name=\shortstack{Nível de\\Satisfação \(X_i\)}},
    columns/fi/.style={column name=\shortstack{Frequência\\\(f_i\)}},
    every head row/.style={before row=\toprule, after row=\midrule},
    every last row/.style={after row=\bottomrule},
    col sep=space,
  ]{\Satisf}
  \caption{Frequência de níveis de Satisfação}
\end{table}


O total de clientes entrevistados foi de:
\[
    5+5+8+10+13+10=49
\]
Como o número de entrevistados é ímpar, \(n\) = 49, a mediana será o termo que ocupa a posição de ordem:
\[
    \frac{n}{2}+1=\frac{49}{2}+1=\frac{50}{2}=25
\]
Em outros termos, a mediana será o elemento que ocupa a vigésima quinta posição. Para chegarmos a esse elemento, precisamos percorrer cada um dos níveis de satisfação. Reparem que três clientes atribuíram a nota 0 (zero); cinco atribuíram a nota 1 (um); e oito atribuíram a nota 2 (dois). Portanto, até esse ponto, temos um total de 16 avaliações:
\[
    3+5+8=16
\]
Vejam que ainda não chegamos na posição desejada, isto é, na vigésima quinta. Contudo, sabemos que o próximo nível de satisfação, referente à nota 3 (três), teve frequência absoluta igual a 10. Se somarmos essas dez novas avaliações com o total obtido anteriormente, chegaremos a um valor que ultrapassa a posição
procurada (16 + 10 = 26). Assim, descobrimos que a mediana está localizada nessa faixa de avaliação. Portanto,
\[
    M_d=x_25=3
\]
Esse procedimento pode ser simplificado com a introdução de uma coluna adicional para armazenar as frequências acumuladas. Já vimos que, para calcularmos a frequência acumulada, devemos repetir a primeira frequência e somar as frequências subsequentes, exibindo os resultados a cada linha. Observem:

\begin{table}
    \centering
    \begin{tabular}{cccl}
        \toprule
        \shortstack{Nivel de \\Satisfação \\\(X_i\)} & \shortstack{Frequênia \\\(f_i\)} & \shortstack{Frequência \\Acumulada} &\shortstack{Memória de \\Cálculo}\\
        \midrule
        0 & 3&  3&  0+3=3\\
        1 & 5&  8&  3+5=8\\
        2 & 8&  16& 8+8=16\\
        3 & 10& 26& 16+10=26\\
        4 & 13& 39& 26+13=39\\
        5 & 10& 49& 39+10=49\\
        \bottomrule
    \end{tabular}
    \caption{Frequência Acumulada}
\end{table}

Vamos remover a memória de cálculo para simplificar a tabela:

\begin{table}
    \centering
    \begin{tabular}{ccc}
        \toprule
        \shortstack{Nivel de \\Satisfação \\\(X_i\)} & \shortstack{Frequênia \\\(f_i\)} & \shortstack{Frequência \\Acumulada}\\
        \midrule
        0 & 3&  3\\
        1 & 5&  8\\
        2 & 8&  16\\
        3 & 10& 26\\
        4 & 13& 39\\
        5 & 10& 49\\
        \bottomrule
    \end{tabular}
    \caption{Frequência Acumulada}
\end{table}
Reparem que o número 16, na terceira linha da coluna de frequências acumuladas, representa a soma das frequências absolutas simples das três primeiras linhas, isto é, 3 + 5 + 8. Assim, concluímos que 16 clientes avaliaram o atendimento da empresa com nota igual ou inferior a 2. De forma análoga, como 49 clientes
participaram da pesquisa, podemos afirmar que 33 avaliaram o atendimento com nota igual ou superior a 3.

Observem que a introdução da coluna de frequências acumuladas permite calcularmos a mediana de forma praticamente imediata. Nesse sentido, se n for ímpar, basta identificarmos o valor da variável correspondente à primeira frequência acumulada imediatamente igual ou superior à posição de ordem \(\frac{n+1}{2}\) ; e, se n for par, basta identificarmos os dois valores correspondentes às frequências acumuladas imediatamente iguais ou superiores às posições de ordens\(\frac{n}{2}\) e \(\frac{n+1}{2}\), respectivamente, e tirarmos a média
aritmética desses dois valores.

Em nosso exemplo, como a frequência total é ímpar, teremos que buscar pela posição \(\frac{n+1}{2}=\frac{49+1}{2}=25\)A mediana será o valor da variável correspondente à primeira frequência acumulada maior ou igual a essa posição, portanto, \(M_d=3\). Vejamos:

\begin{table}
    \centering
    \begin{tabular}{ccc}
        \toprule
        \shortstack{Nivel de \\Satisfação \\\(X_i\)} & \shortstack{Frequênia \\\(f_i\)} & \shortstack{Frequência \\Acumulada}\\
        \midrule
        0 & 3&  3\\
        1 & 5&  8\\
        2 & 8&  16\\
        3 & 10& 26\,\ge\,25\\
        4 & 13& 39\\
        5 & 10& 49\\
        \bottomrule
    \end{tabular}
    \caption{Frequência Acumulada}
\end{table}
Assim, podemos estabelecer que a mediana de uma tabela de frequências composta por um total de \(n\) elementos será:

\begin{itemize}[label=\alph*.]
    \item se \(n\) for ímpar, o valor identificado na tabela correspondente à frequência acumulada imediatamente igual ou superior à posição de ordem \(\frac{n+1}{2}\), isto é,
    \[
        M-d=\frac{X_n+1}{2}
    \]
    \item se \(n\) for par, a média aritmética dos valores identificados na tabela correspondentes às frequências acumuladas imediatamente iguais ou superiores às posições de ordens \(\frac{n}{2}\) e \(\frac{n}{2}+1\), isto é,
     \[
        M_d=\frac{\frac{x_n}{2}+\frac{x_n}{2}+1}{2}
    \]
\end{itemize}

\subsubsection{Mediana para dados agrupados em classes}

O raciocínio adotado no cálculo da mediana para dados agrupados em classes é muito similar ao empregado no tópico anterior. Agora, contudo, não nos importaremos com o número de elementos da série.
Adotaremos um único procedimento de cálculo, independentemente de termos um número par ou ímpar de elementos.

Considere a distribuição de frequências descrita a seguir, que resume as idades de um grupo de 50 pessoas:

\begin{table}[H]
    \centering
    \begin{tabular}{lc}
        \toprule
        Idade & \shortstack{Frequência \(f_i\)}\\
        \(23\vdash16\)  &3\\
        \(26\vdash29\)  &4\\
        \(29\vdash32\)  &10\\
        \(32\vdash35\)  &13\\
        \(35\vdash38\)  &10\\
        \(38\vdash41\)  &6\\
        \(41\vdash44\)  &4\\
        \midrule
        Total           &50\\
        \bottomrule     
    \end{tabular}
    \caption{Distribuição de Frequência de Idades}
    \label{Tab:~dist_freq_idades}
\end{table}
A etapa inicial do cálculo da mediana consiste na construção da coluna de frequências a partir da tabela %\ref{Tab:~dist_freq_idades} 
das frequências acumuladas:

\begin{table}[H]
    \centering
    \begin{tabular}{lcc}
\toprule
\makecell{Idade} &
\makecell{Frequência\\ \(f_i\)} &
\makecell{Frequência\\ Acumulada \(f_{ac}\)}\\
\midrule
        \(23\vdash16\)  &3  & 3\\
        \(26\vdash29\)  &4  & 7\\
        \(29\vdash32\)  &10 & 17\\
        \(32\vdash35\)  &13 & 30\\
        \(35\vdash38\)  &10 & 40\\
        \(38\vdash41\)  &6  & 46\\
        \(41\vdash44\)  &4  & 50\\
        \midrule
        Total           &50\\
        \bottomrule     
    \end{tabular}
    \caption{Cáculo da mediana}
    \label{Tab:~calc_mediana_I}
\end{table}
Para calcular a mediana de dados agrupados por intervalo de classes, precisamos identificar a lasse em que se encontra a mediana, a chamada classe mediana, que corresponde à frequência acumulada imediatamente igual ou superior à metade da frequência total, ou seja, metade da soma das frequências simples, \(\sum \frac{f_i}{2}\). Em nosso exemplo, temos:
\[
    \frac{\sum f_i}{2}=\frac{50}{2}=25
\]
Agora, devemos comparar o valor encontrado com os valores presentes na coluna de frequências acumuladas, percorrendo-os de cima para baixo. A classe mediana será a primeira classe onde a frequência acumulada for igual ou superior a 25. Assim, teremos que analisar o seguinte:

\begin{itemize}
    \item a primeira frequência acumulada (3) é maior ou igual a 25? Não;
    \item a segunda frequência acumulada (7) é maior ou igual a 25? Não;
    \item a terceira frequência acumulada (17) é maior ou igual a 25? Não;
    \item a quarta frequência acumulada (30) é maior ou igual a 25? Sim.
\end{itemize}
Pronto, encontramos a classe mediana. Nesse ponto, paramos a comparação e verificamos que a mediana se encontra na quarta classe, isto é, no intervalo entre 32 e 35.

Conhecendo a classe mediana, podemos aplicar a fórmula da mediana, a seguir:

\[
    M_d=l_{inf}+\Bigg[\frac{\Big(\frac{\sum{f_i}}{2}\Big)-f_{ac_{ant}}}{f_i}\Bigg]\times h
\]

onde:
\begin{itemize}
    \item \(l_{inf}\) é o limite inferior da classe mediana;
    \item \(l_{fac_{ant}}\) é a frequência acumulada da classe anterior à classe mediana;
    \item \(l_{i}\) é a frequência simples da classe mediana; e
    \item \textit{h} é a amplitude do intervalo da classe mediana.
    \end{itemize}

    Já sabemos que a amplitude é a diferença entre os limites da classe. Assim, temos:
    \[
        h=35-32=3
    \]

    \begin{figure}[H]
        \centering
        \includegraphics[width=0.8 \linewidth]{limites_de_classes.png}
        \caption{classe mediana}
    \end{figure}

    Após identificarmos os elementos, precisamos aplicá-los na fórmula evidenciada anteriormente:
\begin{align*}
   &M_d=l_{inf}+\Bigg[\frac{\Big(\frac{\sum{f_i}}{2}\Big)-}{f_i}\Bigg]\times h\\
   &M_d=32+\Bigg[\frac{\Big(\frac{50}{2}\Big)-17}{13}\Bigg]\times 3\\
   &M_d=32+\frac{8}{13}\times 3\approx 33,85\\
\end{align*}

\subsubsection{Propriedades da Mediana}

\begin{propriedade}
    Somando-se (ou subtraindo-se) uma constante \textit{c} a todos os valores de uma variável, a mediana do conjunto fica aumentada (ou diminuída) dessa constante.
\end{propriedade}
\begin{propriedade}
    Multiplicando-se (ou dividindo-se) todos os valores de uma variável por uma constante \textit{c}, a mediana do conjunto fica multiplicada (ou dividida) por esta constante.
\end{propriedade}
\begin{propriedade}
    A soma dos desvios absolutos de uma sequência de números, em relação a um número \textit{a}, é mínima quando \textit{a} é a mediana dos números.
\end{propriedade}
\section{Quartis, Decis e Percentis}

Já vimos que a mediana separa uma série em duas partes iguais, cada uma contendo o mesmo número de elementos. Contudo, uma série também pode ser dividida em um número maior de partes, todas compostas por quantidades iguais de elementos. Nesse caso, o nome da medida separatriz é atribuído conforme a quantidade de partes onde a série é dividida:

\begin{itemize}
    \item quartil: divide uma série em quatro partes iguais \(Q_1 , Q_2 , Q_3\);
    \item quintil: divide uma série em cinco partes iguais \(Qt_1 , Qt_2 , \dots, Qt_5\);
    \item decil: divide uma série em dez partes iguais \(D_1 , D_2 , \dots, D_9\);
    \item percentil: divide uma série em cem partes iguais \(P_1 , P_2 , \dots, P_{99}\).
\end{itemize}
\subsection{Quartis}

Denominamos de quartis os valores de uma série que a dividem em quatro partes iguais, isto é, quatro partes contendo o mesmo número de elementos (25\%). A imagem a seguir mostra os quartis de uma
distribuição hipotética:
\medskip
\begin{figure}[h]
    \centering
\begin{tikzpicture}
\begin{axis}[
  width=12cm, height=6cm,
  axis lines=left,
  xlabel={$x$}, ylabel={$f(x)$},
  axis line style={-{Stealth[length=3mm,width=2mm]}},
  domain=-4:4, samples=300,
  ymin=0,
  ymax=0.6,
  grid=both
]

% linhas verticais
\addplot [blue, thick] {(1/sqrt(2*pi) * exp(-x^2/2))+0.05};
\draw [red, thick] (-2,0) -- (-2,0.103);
\draw [red, thick] (0,0) -- (0,0.448);
\draw [red, thick] (2,0) -- (2,0.103);


% rótulos embaixo do eixo x
\node[font=\itshape] at (0,0.5) {Md=\(Q_2\)};
\node[font=\itshape] at (2,0.16) {\(Q_3\)};
\node[font=\itshape] at (-2,0.16) {\(Q_1\)};
\end{axis}
\end{tikzpicture}
\caption{Quartis}
\end{figure}

Temos, então, 3 quartis \(Q_1\,, Q_2 \,\text{e}\, Q_3\) para dividir uma série em quatro partes iguais:
\begin{itemize}[label=\diamond]
    \item o primeiro quartil corresponde à separação dos primeiros 25\% de elementos da série;
    \item o segundo quartil corresponde à separação de metade dos elementos da série, coincidindo com a mediana \(Q_2= M_d\);
    \item o terceiro quartil corresponde à separação dos primeiros 75\% de elementos da série, ou dos últimos 25\% de elementos da série.
\end{itemize}

Para o cálculo dos quartis, empregaremos a mesma fórmula adotada no cálculo da mediana, somente substituindo a expressão \(\frac{\sum f_i}{2}\) por \(\frac{k\times \sum f_i}{4}\) em que \textit{k} indica a ordem do quartil e assume valores inteiros no intervalo de 1 a 3.

\subsubsection{Quartil para dados Não Agrupados}

O cálculo do quartil para dados não-agrupados é realizado, aproximadamente, por meio das etapas descritas a seguir:

\begin{itemize}
    \item \textbf{primeira etapa}: determinamos a posição do quartil, por meio da expressão:
    \[
        P_{Q_k}=\frac{k\times n}{4}\quad (k=1,\,2,\,3)
    \]
    \item \textbf{segunda etapa}: identificamos a posição do quartil na coluna de frequências acumuladas, isto é, a frequência acumulada imediatamente igual ou superior à posição do quartil;
    \item \textbf{terceira etapa}: verificamos o valor da variável que corresponde a essa posição.
\end{itemize}

Sempre que houver necessidade, teremos que incluir uma coluna de frequências acumuladas.

\fbox{\parbox{.9\linewidth}{
Embora fórmula anterior possa ser utilizada para o cálculo da posição de \(Q_2\), por depender de uma aproximação, nem sempre o valor do segundo quartil  resultará no valor convencionado para a mediana. Por isso, para o cálculo de \(Q_2\), vamos adotar o procedimento utilizado para encontrar a mediana. Isto é, se o número de elementos for ímpar, \(Q_2\) será representado pelo elemento que
ocupar a posição central, \(\frac{n}{2}\). Se o número de elementos do conjunto for par, \(Q_2\) será representado pela média aritmética entre os elementos que ocuparem as posições centrais \(\frac{n}{2}\) e \(\frac{n}{2}+1\)}}

\begin{align*}
    Q_k=l_{inf_{Q_k}}+\Bigg[\frac{\frac{k\times \sum f_i}{4}-f_{ac_{ant}}}{f_{Q_k}}\Bigg]\times h_{Q_k}
\end{align*}
onde:
\begin{itemize}
    \item \(l_{inf_{Q_k}}=\,\text{limite inferior da classe do quartil considerado}\);
    \item \(f_{ac_{ant}}\) =  frequência acumulada da classe anterior à classe do quartil considerado;
    \item \({Q_k}\) = amplitude do intervalo de classe do quartil considerado;
    \item \(f_{Q_k}\) = frequência simples da classe do quartil considerado.
\end{itemize}

\subsection{Decis}
Denominamos de decis os valores de uma série que a dividem em dez partes iguais, isto é, dez partes contendo o mesmo número de elementos (10\%). A imagem a seguir mostra os decis de uma distribuição hipotética:

\begin{figure}[hbt]
    \centering
\begin{tikzpicture}
\begin{axis}[
  width=12cm, height=6cm,
  axis lines=left,
  xlabel={$x$}, ylabel={$f(x)$},
  axis line style={-{Stealth[length=3mm,width=2mm]}},
  domain=-4:4, samples=300,
  ymin=0,
  ymax=0.6,
  grid=both
]
\addplot [blue, thick] {(1/sqrt(2*pi) * exp(-x^2/2)+0.05)};
\draw [red, thick] (-0.8,0) -- (-0.8,0.3396);
\draw [red, thick] (-1.6,0) -- (-1.6,0.16);
\draw [red, thick] (-2.4,0) -- (-2.4,0.07);
\draw [red, thick] (-3.2,0) -- (-3.2,0.052);
\draw [red, thick] (0,0) -- (0,0.448);
\draw [red, thick] (0.8,0) -- (0.8,0.3396);
\draw [red, thick] (1.6,0) -- (1.6,0.16);
\draw [red, thick] (2.4,0) -- (2.4,0.07);
\draw [red, thick] (3.2,0) -- (3.2,0.052);



% rótulos embaixo do eixo x
\node[font=\itshape] at (3.6,0.1) {\(D_{10}\)};
\node[font=\itshape] at (-3.6,0.1) {\(D_1\)};

\node[font=\itshape] at (2.7,0.12) {\(D_9\)};
\node[font=\itshape] at (-2.7,0.12) {\(D_2\)};

\node[font=\itshape] at (2,0.16) {\(D_8\)};
\node[font=\itshape] at (-2,0.16) {\(D_3\)};

\node[font=\itshape] at (1.28,0.3) {\(D_7\)};
\node[font=\itshape] at (-1.28,0.3) {\(D_4\)};

\node[font=\itshape] at (0,0.48) {\(D_5=Md\)};

\end{axis}
\end{tikzpicture}
\caption{Decis}
\end{figure}

Temos, então, 9 decis \(D_1 , D_2,\dots, D_9\) para dividir uma série em dez partes iguais:

\begin{itemize}
    \item \(D_1\): o primeiro decil corresponde à separação dos primeiros 10\% de elementos da série;
    \item \(D_5\): o quinto decil corresponde à separação de metade dos elementos da série, coincidindo com a mediana \(D_5 = Md\);
    \item \(D_9\): o nono decil corresponde à separação dos primeiros 90\% de elementos da série, ou dos últimos 10\% de elementos da série.
\end{itemize}

Para o cálculo dos decis, empregaremos a mesma fórmula adotada no cálculo da mediana, somente substituindo a expressão \(\sum \frac{\Sigma f_i}{2}\) por \(k \times\sum \frac{\Sigma f_i}{10}\), em que k indica a ordem do decil e assume valores inteiros no intervalo de 1 a 9.

\subsubsection{Decis para dados Não Agrupados}

O cálculo do decil segue o mesmo raciocínio empregado no cálculo do quartil para dados não-agrupados. A primeira tarefa que devemos realizar, se houver necessidade, é organizar o conjunto de valores por ordem
de magnitude. Depois disso, procedemos conforme as seguintes etapas:

\begin{itemize}
    \item \textbf{primeira etapa}: determinados a posição do decil conforme a seguinte fórmula:
    \[
        P_{D_k}=\frac{k\times n}{10}\,(k=1,2,\dots, 9);
    \]
    \item \textbf{segunda etapa}: identificamos a posição mais próxima ao rol;
    \item \textbf{terceira etapa}: verificamoso o valor que está ocupando essa posição.
\end{itemize}

\textbf{Exemplo}

Calcular os decís \(D_1\,\text{e}\,  D_8\) para o seguinte conjunto de valores:
\begin{center}
    (5, 12, 15, 20, 2, 3, 4, 18, 10, 22)
\end{center}

Note que os valores não estão organizados. Portanto, nossa primeira tarefa será colocá-los em ordem de magnitude (rol):

    \begin{center}
        (\textbf{2}, 3, 4, 5, 10, 12, 15, 18, 20, 22)
    \end{center}
\begin{enumerate}[label=\alph*.]
    \item     Cálculo de \(D_1\):
    \[
     P_{D_1}=\frac{1\times 10}{10}=1  
    \]
    Depois, identificamos a posição mais próxima no rol. Como o resultado foi um número inteiro, a posição mais próxima coincidirá com o valor encontrado, não havendo necessidade de aproximação.
    \begin{table}[H]
        \centering
        \begin{tabular}{llllllllll}
            \toprule
            $x_1$&$x_2$&$x_3$&$x_4$&$x_5$&$x_6$&$x_7$&$x_8$&$x_9$&$x_{10}$\\
            \midrule
            2& 3& 4& 5& 10& 12& 15& 18& 20& 22\\
            \bottomrule
            \end{tabular}
            \caption{Rol em ordem crescente}
    \end{table}

    Portanto, o valor 2 corresponde a 10\% do rol.

    \item Calcular o decil \(D_8\)
O rol é o mesmo:
\begin{center}
    (2, 3, 4, 5, 10, 12, 15, 18, 20, 22)
\end{center}
    
Cálculo de \(D_1\):
    \[
     P_{D_8}=\frac{8\times 10}{10}=8 
    \]
    Depois, identificamos a posição mais próxima no rol. Como o resultado foi um número inteiro, a posição mais próxima coincidirá com o valor encontrado, não havendo necessidade de aproximação.
    \begin{table}[H]
        \centering
        \begin{tabular}{llllllllll}
            \toprule
            $x_1$&$x_2$&$x_3$&$x_4$&$x_5$&$x_6$&$x_7$&$x_8$&$x_9$&$x_{10}$\\
            \midrule
            2& 3& 4& 5& 10& 12& 15& \textbf{18}& 20& 22\\
            \bottomrule
            \end{tabular}
            \caption{Rol em ordem crescente}
    \end{table}

    Portanto, o valor 8 corresponde a 80\% do rol.
\end{enumerate}

\subsubsection{Decil para dados agrupados sem intervalos de classe.}

O cálculo do decil para dados agrupados sem intervalos de classe será realizado por meio das etapas descritas a seguir:

 em que \(\sum f_i\) é a soma das frequências simples;

\begin{itemize}
    \item \textbf{primeira etapa}:etapa: determinamos a posição do decil, por meio da expressão:
   \[
        P_{D_k}=\frac{k\times n}{10}\,(k=1,2,\dots, 9);
    \]
    \item \textbf{segunda etapa}: identificamos a posição do decil na coluna de frequências acumuladas, isto é, a frequência acumulada imediatamente igual ou superior à posição do decil;
    \item \textbf{terceira etapa}: verificamos o valor da variável que corresponde a essa posição.
\end{itemize}

Sempre que houver necessidade, teremos que incluir uma coluna de frequências acumuladas.

Vamos calcular \(D_3\) e  \(D_8\) da tabela de frequências a seguir, que representa a quantidade de filhos de um grupo de pessoas:

\begin{table}[hbt]
    \centering
\begin{tabular}{ccc}
\toprule
\makecell{Filhos} &
\makecell{Frequência \\\(f_i\)} &
\makecell{Frequência \\ Acumulada\\\(f_{ac}\)}\\
\midrule
0 & 18 & 18  \\ 
\arrayrulecolor{red}\hline
1 & 35 & \(53(\ge 51)\)  \\
\arrayrulecolor{red}\hline
2 & 46 & 99  \\
3 & 28 & 127 \\
4 & 25 & 152 \\
5 & 10 & 162 \\
6 & 5  & 167 \\
7 & 3  & 170 \\
\arrayrulecolor{black}\hline
Total & &170\\
\bottomrule
\end{tabular}
\caption{Cálculo de decis}
\label{tab: Calculo_D3_ }
\end{table}
\begin{enumerate}[label=\alph*.]
    \item Determinar a posição de \(D_3\);
    \[
        P_{D_3}=\frac{3\times 170}{10}=51
    \]

    Portanto, a quantidade de 1 filho corresponde a 30\% do rol.

    \item Cálculo de \(D_8\):
  \[
        P_{D_8}=\frac{8\times 170}{10}=136
    \]
\begin{table}[H]
    \centering
\begin{tabular}{ccc}
\toprule
\makecell{Filhos} &
\makecell{Frequência \\\(f_i\)} &
\makecell{Frequência \\ Acumulada\\\(f_{ac}\)}\\
\midrule
0 & 18 & 18  \\ 
1 & 35 & 53  \\
2 & 46 & 99  \\
3 & 28 & 127 \\
\arrayrulecolor{red}\hline
4 & 25 & \(152\ge 136\) \\
\arrayrulecolor{red}\hline
5 & 10 & 162 \\
6 & 5  & 167 \\
7 & 3  & 170 \\
\arrayrulecolor{black}\hline
Total & &170\\
\bottomrule
\end{tabular}
\caption{Cálculo de decis}
\label{tab: Calculo_D8 }
\end{table}
\end{enumerate}

\subsubsection{Decil para dados agrupados em classes}

O cálculo do decil para dados agrupados em classes será realizado por meio das seguintes etapas:

\begin{itemize}
    \item \textbf{primeira etapa}: determinamos a posição do decil, por meio da expressão:
   \[
        P_{D_k}=\frac{k\times n}{10}\,(k=1,2,\dots, 9);
    \]
    em que:

    \(\sum f_i =\) somatório das frequências simples.\\
    \(k=\) índice do decil.
\medskip
    \item \textbf{segunda etapa}: identificamos a posição do decil na coluna de frequências acumuladas, isto é, a frequência acumulada imediatamente igual ou superior à posição do decil;
    \item \textbf{terceira etapa} verificamos as informações referentes à classe correspondente a essa posição; e
    \item \textbf{quarta etapa} calculamos o valor do decil por meio da fórmula apresentada a seguir, que consiste em uma variação da fórmula da mediana para dados agrupados em classes, mudando-se somente o \(\frac{k\times \sum f_i}{10}\)

\end{itemize}

\begin{align*}
    Q_k=l_{inf_{Q_k}}+\Bigg[\frac{\frac{k\times \sum f_i}{4}-f_{ac_{ant}}}{f_{Q_k}}\Bigg]\times h_{Q_k}
\end{align*}

em que:
\begin{itemize}
    \item \(l_{inf_{Q_k}}\) = limite inferior da classe do quartil considerado;
    \item \(f_{ac_{ant}}\) = frequência acumulada da classe anterior à classe do quartil;
    \item \(h_{Q_k}\) = amplitude do intervalo de classe do quartil considerado;
    \item \(f_{Q_k}\) = frequência simples da classe do quartil considerado.
\end{itemize}

Sempre que houver necessidade, teremos que incluir uma coluna de frequências acumuladas.
\begin{table}[hbt]
    \centering
    \begin{tabular}{ccccc}
        \toprule
        \makecell {i} & 
        \makecell{Altura \\(cm)} & 
        \makecell {Frequência \\\(f_i\)} &
        \makecell{Frequência \\ Acumulada \\\(f_{ac}\)} &
        \makecell{Comentários}\\
                \midrule
                1 & 120 \vdash 128 &  6  &  \(6\ge 5,4\)   & Classe considerada\\
                2 & 128 \vdash 136 & 12  & 18   &\\
                3 & 136 \vdash 144 & 16  & 34   &\\
                4 & 144 \vdash 152 & 13  & 47   &\\
                5 & 152 \vdash 160 &  7  & 54   &\\
                \midrule
                    Total & & & 54\\
            \bottomrule
    \end{tabular}
\end{table}

\begin{enumerate}[label=\alph*.]
    \item Calcular \(Q_1\)
    Calcular a posição de \(Q_1\) por:
    \[
        P_{D_1}=\frac{1\times 54}{10}=5,4
     \]

onde o limite inferior é 120 cm e a frequência é 6.

\begin{align*}
   & D_1=l_{inf_{Q_1}}+\Bigg[\frac{\frac{k\times \sum f_i}{10}-f_{ac_{ant}}}{f_{Q_1}}\Bigg]\times h_{Q_1}\\\\
   &D_1=120+\frac{5,4-0}{6}\times 8 = \SI{127,5}{\centi\meter}\\
\end{align*}

\item Calcular \(D_2\):

Para começar vamos achar a posição de \(D_2\)

    \[
        P_{D_2}=\frac{2\times 54}{10}=10,5
    \]
\begin{table}[hbt]
    \centering
    \begin{tabular}{ccccc}
        \toprule
        \makecell {i} & 
        \makecell{Altura \\(cm)} & 
        \makecell {Frequência \\\(f_i\)} &
        \makecell{Frequência \\ Acumulada \\\(f_{ac}\)} &
        \makecell{Comentários}\\
                \midrule
                1 & 120 \vdash 128 &  6  &  6   & F. acumulada anterior\\
                \arrayrulecolor{red}\hline
                2 & 128 \vdash 136 & 12  & \(18\ge 10,5\) & Classe considerada\\
                \arrayrulecolor{red}\hline
                3 & 136 \vdash 144 & 16  & 34   &\\
                4 & 144 \vdash 152 & 13  & 47   &\\
                5 & 152 \vdash 160 &  7  & 54   &\\
                \arrayrulecolor{black}\hline
                Total & & $\longrightarrow$ & 54\\
            \bottomrule
    \end{tabular}
\end{table}

\begin{align*}
   & D_2=l_{inf_{Q_2}}+\Bigg[\frac{\frac{k\times \sum f_2}{10}-f_{ac_{ant}}}{f_{Q_2}}\Bigg]\times h_{Q_2}\\\\
   &D_1=128+\frac{10,8-6}{12}\times 8 = \SI{131,2}{\centi\meter}\\
\end{align*}

\item Cálculo para \(D_7\):

Para começar vamos achar a posição de \(D_7\)

    \[
        P_{D_7}=\frac{7\times 54}{10}=37,8
    \]
\begin{table}[hbt]
    \centering
    \begin{tabular}{ccccc}
        \toprule
        \makecell {i} & 
        \makecell{Altura \\(cm)} & 
        \makecell {Frequência \\\(f_i\)} &
        \makecell{Frequência \\ Acumulada \\\(f_{ac}\)} &
        \makecell{Comentários}\\
                \midrule
                1 & 120 \vdash 128 &  6  &  6   & \\
                2 & 128 \vdash 136 & 12  &  18  & \\
                3 & 136 \vdash 144 & 16  &  34   &F. acumulada anterior\\
                \arrayrulecolor{red}\hline
                4 & 144 \vdash 152 & 13  &  47   & Classe considerada\\
                \arrayrulecolor{red}\hline
                5 & 152 \vdash 160 &  7  &  54   &\\
                \arrayrulecolor{black}\hline
                Total & & $\longrightarrow$ & 54\\
            \bottomrule
    \end{tabular}
\end{table}

\begin{align*}
   & D_7=l_{inf_{D_7}}+\Bigg[\frac{\frac{k\times \sum f_7}{10}-f_{ac_{ant}}}{f_{D_7}}\Bigg]\times h_{D_7}\\\\
   &D_1=144+\frac{37,8-34}{13}\times 8 = \SI{146,3}{\centi\meter}\\
\end{align*}

\end{enumerate}

\subsection{Percentis}

Denominamos de percentis os valores de uma série que a dividem em cem partes iguais, isto é, cem partes
contendo o mesmo número de elementos (1\%). A imagem a seguir mostra os percentis de uma distribuição
hipotética:
\begin{figure}[H]
    \centering
        \includegraphics[width=0.8 \linewidth]{percentis.png}
\end{figure}

Temos, então, 99 percentis \(P_1 , P_2, \dots, P_{99}\) para dividir uma série em cem partes iguais:
\begin{itemize}
    \item \(P_1\): o primeiro percentil corresponde à separação do primeiro 1\% de elementos da série;
    \item \(P_50\): o quinquagésimo percentil corresponde à separação de metade dos elementos da série, coincidindo com a mediana \(P_{𝟓𝟎 = Md}\);
    \item \(P_99\): o nonagésimo nono percentil corresponde à separação dos primeiros 99\% de elementos da série, ou do último 1\% de elementos da série.
\end{itemize}

Para o cálculo dos percentis, empregaremos a mesma fórmula adotada no cálculo da mediana, somente substituindo a expressão \(\frac{sum f_i}{2}\) por \(\frac{k\times sum f_i}{100}\), em que 𝑘 indica a ordem do percentil e assume valores inteiros no
intervalo de 1 a 99.

\subsubsection{Percentil para dados não-agrupados}

O cálculo do percentil segue o mesmo raciocínio empregado nos cálculos do quartil e do decil para dados não-agrupados. A primeira tarefa que devemos realizar, se houver necessidade, é organizar o conjunto de valores por ordem de magnitude. Depois disso, colocamos em prática as seguintes etapas:
    \begin{itemize}
    \item \textbf{primeira etapa}: determinamos a posição do percentil, por meio da expressão:
\[
    P{P_k}=\frac{k\times n}{100}\, (k= 1,\,2,\,\dots.\,99);
\]
    \item \textbf{segunda etapa}: identificamos a posição mais próxima do rol;
    \item \textbf{terceira etapa}: verificamos o valor que está ocupando essa posição.
\end{itemize}

\textbf{Exemplo}
Calcularemos os percentis \(P_{27} \text{e} P_{83}\) para o seguinte conjunto de valores:
\begin{center}
    {\{15, 2, 4, 6, 10, 12, 13, 7, 21, 18, 20\}}
\end{center}
Reparem que os valores não estão organizados. Portanto, nossa primeira tarefa será colocá-los em ordem de magnitude (rol):
\begin{center}
    \{{2, 4, 6, 7, 10, 12, 13, 15, 18, 20, 21}\}
\end{center}

\begin{enumerate} [label=\alph*.]
    \item Cálculo de \(P_{27}\)

    Primeiramente determina-se a posição de \(P_27\)
    \[
        P_27=\frac{27\times 11}{100}=2,97
    \]
    Depois, identificamos a posição mais próxima no rol:
    \[
        P_{27}=3
    \]
    Em seguida, verificamos o valor que está ocupando essa posição.

    Em seguida, seguida, verificamos o valor que está ocupando essa posição.

        \begin{table}[H]
        \centering
        \begin{tabular}{lllllllllll}
            \toprule
            $x_1$&$x_2$&$x_3$&$x_4$&$x_5$&$x_6$&$x_7$&$x_8$&$x_9$&$x_{10}$&$x_11$\\
            \midrule
            2& 4& \textbf{6}& 7& 10& 12& 13& 15& 18& 20& 21\\
            \bottomrule
            \end{tabular}
            \caption{Rol em ordem crescente}
    \end{table}
    Portanto, o valor 6 corresponde a 27\% do rol.

    \item Determinar a posição \(P_{83}\)
    
    Determina-se Primeiramente a posição de \(P_83\)
    \[
        P_83=\frac{83\times 11}{100}=9,13
    \]

    Depois, identificamos a posição mais próxima no rol:
    \[
        P_{83}=9
    \]

    Em seguida, verificamos o valor que está ocupando essa posição.

            \begin{table}[H]
        \centering
        \begin{tabular}{lllllllllll}
            \toprule
            $x_1$&$x_2$&$x_3$&$x_4$&$x_5$&$x_6$&$x_7$&$x_8$&$x_9$&$x_{10}$&$x_11$\\
            \midrule
            2& 4& 6& 7& 10& 12& 13& 15& \textbf{18}& 20& 21\\
            \bottomrule
            \end{tabular}
            \caption{Rol em ordem crescente}
    \end{table}
\end{enumerate}

\subsubsection{Percentil para dados agrupados em classes}

O cálculo do percentil para dados agrupados em classes será realizado por meio das seguintes etapas:

\begin{itemize}
    \item \textbf{primeira etapa}: determinamos a posição do percentil, por meio da expressão:

\[
    P{P_k}=\frac{k\times n}{100}\, (k= 1,\,2,\,\dots.\,99);
\]
em que:

    \(k=\text{índice do percentil}\)\\
    \(\sum f_i=\text{somatório das frequencias simples}\)

    \item \textbf{segunda etapa}: identificamos a posição do percentil na coluna de frequências acumuladas, isto é, a
frequência acumulada imediatamente igual ou superior à posição do percentil;
    \item \textbf{terceira etapa}: verificamos as informações referentes à classe correspondente a essa posição; e
    \item \textbf{quarta etapa}: calculamos o valor do percentil por meio da fórmula apresentada a seguir, que consiste
em uma variação da fórmula da mediana para dados agrupados em classes, mudando-se somente o \(\frac{k\times \sum f_i}{100}\)
\[
    P_k=l_{inf_{Pk}}+
    \Bigg[
    \frac{\frac{k\times \sum f_i}{100}-f_{ac_{ant}}}{f{P_k}}
    \Bigg]
    \times h_{P_k}
\]
\end{itemize}

em que:
\begin{itemize}
    \item \(l_{inf_{Pk}}\) = limite inferior da classe do percentil considerado;
    \item \(f_{ac_{ant}}\) = frequência acumulada;
    \item \(f{P_k}\) = amplitude do intervalo de classe do percentil considerado;
    \item \(h_{P_k}\) = frequência simples da classe do percentil considerado. 
\end{itemize}

Sempre que houver necessidade, teremos que incluir uma coluna de frequências acumuladas.

\section{Box Plot}
%========================
%Box Plot
%========================
Um boxplot (também chamado de box-and-whisker plot) é uma ferramenta gráfica frequentemente utilizada na análise exploratória de dados que permite visualizar a distribuição dos dados e os valores discrepantes (outliers), assim como a distância dos valores extremos em relação à maioria dos dados. Essa ferr amenta resume cinco medidas descritivas de um conjunto de dados, incluindo: o valor mínimo, o primeiro quartil, a mediana, o terceiro quartil e o valor máximo. 

Para construir um gráfico de boxplot, usamos uma haste horizontal ou vertical e uma caixa retangular (box). O local onde a haste começa (da esquerda para a direita) indica o valor mínimo e o ponto em que a haste termina indica o valor máximo.

A caixa retangular, localizada no meio da haste, em geral, possui três linhas. A primeira linha, na extremidade esquerda da caixa, indica o primeiro quartil. A terceira linha, na extremidade direita, indica o terceiro quartil. A linha do meio, no interior da caixa, indica o segundo quartil ou a mediana. O segundo quartil pode estar entre o primeiro e o terceiro quartis, ou pode coincidir com um, ou outro, ou ambos. 

\begin{figure}[H]
    \begin{tikzpicture}[
         font=\sffamily,
  >=Latex,
  thick
]

% =========================
% AJUSTE AQUI os 5 valores
% (na mesma escala horizontal)
% =========================
\def\xmin{0}   % Valor mínimo
\def\xqone{4}  % Q1
\def\xmed{7}   % Mediana (Q2)
\def\xqthr{10} % Q3
\def\xmax{14}  % Valor máximo

% Geometria da caixa
\def\y0{0}
\def\boxH{2.2}
\def\ybot{\y0 - \boxH/2}
\def\ytop{\y0 + \boxH/2}

% Cores (próximas da sua figura)
\definecolor{cminmax}{RGB}{0,153,0}   % verde
\definecolor{cqone}{RGB}{40,90,200}   % azul
\definecolor{cmed}{RGB}{220,0,0}      % vermelho
\definecolor{cqthr}{RGB}{235,120,0}   % laranja

% =========================
% ELEMENTOS DO BOXPLOT
% =========================

% Bigodes (linhas)
\draw[thick] (\xmin,\y0) -- (\xqone,\y0);
\draw[thick] (\xqthr,\y0) -- (\xmax,\y0);

% Caixa (Q1 a Q3)
\draw[thick] (\xqone,\ybot) rectangle (\xqthr,\ytop);

% Mediana
\draw[very thick, cmed] (\xmed,\ybot) -- (\xmed,\ytop);

% Linhas Q1 e Q3 (bordas coloridas como na imagem)
\draw[very thick, cqone] (\xqone,\ybot) -- (\xqone,\ytop);
\draw[very thick, cqthr] (\xqthr,\ybot) -- (\xqthr,\ytop);

% Marcas verticais mínimo / máximo
\draw[very thick, cminmax] (\xmin,\y0-1.0) -- (\xmin,\y0+1.0);
\draw[very thick, cminmax] (\xmax,\y0-1.0) -- (\xmax,\y0+1.0);

% =========================
% RÓTULOS
% =========================

% Valor mínimo / máximo
\node[cminmax, align=center, font=\bfseries] at (\xmin, \y0+2.0) {Valor\\Mínimo};
\node[cminmax, align=center, font=\bfseries] at (\xmax, \y0+2.0) {Valor\\Máximo};

% Quartis e mediana
\node[cqone, align=center, font=\bfseries] at (\xqone, \y0+2.0) {Quartil\\Inferior\\$Q_1$};
\node[cmed,  align=center, font=\bfseries] at (\xmed,  \y0+2.0) {Mediana\\$Q_2$};
\node[cqthr, align=center, font=\bfseries] at (\xqthr, \y0+2.0) {Quartil\\Superior\\$Q_3$};

% "whisker = bigode"
\node[align=center] at ({(\xmin+\xqone)/2}, \y0+0.55) {whisker = bigode};
\node[align=center] at ({(\xqthr+\xmax)/2}, \y0+0.55) {whisker = bigode};

% "box = caixa"
\node[align=center] at ({(\xqone+\xqthr)/2}, \y0-1.5) {box = caixa};

    \end{tikzpicture}
\end{figure}
 
Além disso, há dois traços, chamados de whiskers(ou bigodes), ligando o valor mínimo à extremidade esquerda da caixa e o valor máximo à extremidade direita da caixa. Cada um desses traços comporta, aproximadamente, 25\% dos dados. O restante, cerca de 50\%, está distribuído no interior da caixa. 

Também podemos encontrar gráficos de box plot com pontos ou asteriscos marcando valores discrepantes(outliers). Nesses casos, os whiskers não se estendem aos valores mínimo e máximo do conjunto de dados, mas ficam limitados a um comprimento máximo de 1,5 \(\times\), em que DIQ é a distância interquartílica. A distância interquartílica (ou amplitude interquartílica, ou intervalo interquartílico) é calculada pela fórmula: 
\[
    DIQ=Q3-Q1
\]
\begin{figure}[hbt]
    \begin{tikzpicture}[
         font=\sffamily,
  >=Latex,
  thick
]
% =========================
% AJUSTE AQUI os 5 valores
% (na mesma escala horizontal)
% =========================
\def\xmin{0}   % Valor mínimo
\def\xqone{4}  % Q1
\def\xmed{7}   % Mediana (Q2)
\def\xqthr{10} % Q3
\def\xmax{14}  % Valor máximo

% Geometria da caixa
\def\y0{0}
\def\boxH{2.2}
\def\ybot{\y0 - \boxH/2}
\def\ytop{\y0 + \boxH/2}

% Cores (próximas da sua figura)
\definecolor{cminmax}{RGB}{0,153,0}   % verde
\definecolor{cqone}{RGB}{40,90,200}   % azul
\definecolor{cmed}{RGB}{220,0,0}      % vermelho
\definecolor{cqthr}{RGB}{235,120,0}   % laranja

% =========================
% ELEMENTOS DO BOXPLOT
% =========================

% Bigodes (linhas)
\draw[thick] (\xmin-0.5,\y0) -- (\xqone+1,\y0);
\draw[thick] (\xqthr-1,\y0) -- (\xmax+0.5,\y0);

\fill[red!70!black] (\xmin-0.5,\y0) circle (3pt);
\fill[red!70!black] (\xmax+0.5,\y0) circle (3pt);

% Caixa (Q1 a Q3)
\draw[thick] (\xqone+1,\ybot) rectangle (\xqthr-1,\ytop);

% Mediana
\draw[line width=4pt, cmed] (\xmed,\ybot) -- (\xmed,\ytop);

% Linhas Q1 e Q3 (bordas coloridas como na imagem)
\draw[very thick, cqone] (\xqone+1,\ybot) -- (\xqone+1,\ytop);
\draw[very thick, cqthr] (\xqthr-1,\ybot) -- (\xqthr-1,\ytop);

% cotas da DIQ
\draw[thick, black] (\xmed+2,-1) -- (\xmed+2,-2.6);
\draw[thick, black] (\xmed-2,-1) -- (\xmed-2,-2.6);
\draw[thick, black, <->, >=Stealth] (5,-2.5) -- (9,-2.5);


% Marcas verticais mínimo / máximo +
\draw[very thick, cminmax, ->, >=Stealth] (\xmin+1,\y0-1) -- (\xmin+1,\y0+1.0);
\draw[very thick, cminmax, ->, >=Stealth] (\xmax-1,\y0-1) -- (\xmax-1,\y0+1.0);

\draw[very thick, cqthr, <-, >=Stealth] (\xmin+0.5,\y0-1) -- (\xmin+0.5,\y0+1.0);
\draw[very thick, cqthr, <-, >=Stealth] (\xmax-0.5,\y0-1) -- (\xmax-0.5,\y0+1.0);

% =========================
% RÓTULOS
% =========================

% Valor mínimo / máximo
\node[cminmax, align=center, font=\bfseries] at (\xmin+1, \y0+2.0) {Menor Valor\\Acima do\\ Limite Inferior};
\node[cminmax, align=center, font=\bfseries] at (\xmax-1, \y0+2.0) {Maior Valor\\Abaixo do\\ Limite Superior};

\node[cqthr, align=center, font=\bfseries] at (\xmin+0.5, \y0-2.0) {Limite \\Inferior};
\node[cqthr, align=center, font=\bfseries] at (\xmax-0.5, \y0-2.0) {Limite \\Superior};
%==========================
%Legendas de Outliers 
%==========================
\node[cmed, align=center, font=\bfseries, rotate=90] at (\xmin-0.5, \y0-1) {Outlier};
\node[cmed, align=center, font=\bfseries, rotate=90] at (\xmax+0.5, \y0-1) {Outlier};

% Quartis e mediana
\node[cqone, align=center, font=\bfseries] at (\xqone+0.5, \y0+2.0) {Quartil\\Inferior\\$Q_1$};
\node[cmed,  align=center, font=\bfseries] at (\xmed,  \y0+2.0) {Mediana\\$Q_2$};
\node[cqthr, align=center, font=\bfseries] at (\xqthr-0.5, \y0+2.0) {Quartil\\Superior\\$Q_3$};

% "whisker = bigode"
\node[align=center] at ({(\xmin+\xqone)/1.4}, \y0-0.5)
{$\overbrace{\text{whisker = bigode}}$};
\node[align=center] at ({(\xqthr+\xmax)/2.15}, \y0-0.5) {$\overbrace{\text{whisker = bigode}}$};

% distância interquartílica
\node[align=center] at ({(\xqone+\xqthr)/2}, \y0-2)
{Distância \\Interquartílica};
\node[align=center] at ({(\xqone+\xqthr)/2}, \y0-2.8)
{DIQ};

%Q1-1,5DIQ

\draw[thick, black, dashed] (\xmin+0.5,-2.8) -- (\xmin+0.5,-5);
\draw[thick, black, dashed] (\xmin+5,-2.8) -- (\xmin+5,-3.7);

\draw[thick, black, dashed] (\xmax-5,-2.8) -- (\xmax-+5,-3.7);
\draw[thick, black, dashed] (\xmax-0.5,-2.8) -- (\xmax-0.5,-5);

\draw[thick, black, <->, >=Stealth] (\xmin+0.5,-3.5) -- (\xmin+5,-3.5);
\draw[thick, black, <->, >=Stealth] (\xmax-0.5,-3.5) -- (\xmax-5,-3.5);
\draw[thick, black, <->, >=Stealth] (\xmin+0.5,-4.7) -- (\xmax-0.5,-4.7);

\node[align=center] at ({(\xqthr+\xmax)/9}, \y0-3.1) {$Q_1-1,5\times DIQ$};
\node[align=center] at ({(\xqthr+\xmax)/2.15}, \y0-3.1) {$Q_1+1,5\times DIQ$};
\node[align=center] at ({(\xqone+\xqthr)/2}, \y0-4.5)
{Limites dos Outliers };
    \end{tikzpicture}
\end{figure}

Dessa forma, valores menores que \(Q_1-1,5\times DIQ\) ou maiores que \(Q_3+1,5\times DIQ\) são considerados VALORES DISCREPANTES (OUTLIERS) e representados por PONTOS ou ASTERISCOS.

É importante salientarmos que a fórmula da distância interquartílica se parece muito com a do desvio quartílico (ou amplitude semi-interquartílica), podendo ser facilmente confundida. O desvio quartílico é calculado pela expressão apresentada a seguir:
\[
    \boxed{D_q=\frac{Q_3-Q_1}{2}}
\]

\section{Moda}
Nessa aula, aprenderemos outra importante medida descritiva: a moda estatística. A moda é uma medida de posição e de tendência central que descreve o valor mais frequente de um conjunto de observações, ou seja, o valor de maior ocorrência dentre os valores observados.

\begin{figure}[ht]
    \centering
        \includegraphics[width=0.8\linewidth]{Moda.png}
\end{figure}

Na estatística, o termo moda foi introduzido por Karl Pearson, em 1895, influenciado, muito provavelmente, pela forma com que as pessoas se referiam àquilo que estava em destaque, em evidência, com o significado de coisa mais frequente.

A definição evidencia que um conjunto de valores pode possuir uma ou mais modas, ou não possuir nenhuma. Assim, dizemos que um conjunto é unimodal, bimodal, trimodal ou plurimodal, conforme o número de modas que apresenta. A ausência de uma moda caracteriza o conjunto como amodal.

\begin{figure}[ht]
    \centering
        \includegraphics[width=0.8\linewidth]{conjunto_amodal.png}
\end{figure}

Em geral, a moda é utilizada em distribuições nas quais o valor mais frequente é o mais importante da distribuição. A moda também é útil para a determinação da medida de posição de variáveis qualitativas nominais, ou seja, variáveis não-numéricas que não podem ser ordenadas.

O cálculo da moda ocorre de diferentes formas, a depender de como os dados estão organizados. Nesse contexto, aprenderemos a calcular a moda para as seguintes situações:
\begin{itemize}
    \item dados não-agrupados;
    \item dados agrupados sem intervalos de classe (ou por valores); e
    \item dados agrupados em classes.
\end{itemize}
 
\subsection{Moda para dados não-agrupados}

Para determinarmos a moda de um conjunto ordenado de valores não agrupados em classes, basta identificarmos o elemento (ou elementos) de maior frequência no conjunto.

Com relação ao número de modas, o conjunto de pode ser classificado como:

\begin{itemize}
    \item amodal: quando todos os elementos apresentam a mesma frequência, isto é, quando todosaparecem o mesmo número de vezes:
    
    \begin{center}
    (1, 2, 3, 5, 6, 8, 10)
    \end{center}

\item unimodal: quando a frequência de um elemento é maior que as frequências dos demais elementos. Assim, um único elemento se destaca entre os demais. Isto é, quando o conjunto tem uma única moda. No conjunto a seguir, o elemento 2 repete-se cinco vezes enquanto o elemento 3 aparece duas vezes. Logo, $Mo$ = 2.

    \begin{center}
    $\Bigg
    (\underbrace{2, 2, 2, 2, 2,}_{5\; elementos} 
    \underbrace{3,3}_{2\; elementos}\Bigg)$
    \end{center}

\item bimodal: quando as frequências de dois elementos são iguais e maiores que as frequências dos demais elementos. Isto é, quando o conjunto tem duas modas. No conjunto a seguir, os elementos 2 e 3 repetem-se cinco vezes, enquanto o elemento 4 aparece duas vezes. Logo, $Mo = 2$ e $Mo=3$.

    \begin{center}
    $\Bigg(\underbrace{2, 2, 2, 2, 2,}_{5\; elementos}\, \underbrace{3, 3, 3, 3, 3}_{5\; elementos}
    \underbrace{4, 4}_{2\; elementos}\Bigg)$
    \end{center}

\item multimodal ou plurimodal: quando as frequências de três ou mais elementos são iguais e maiores que as frequências dos demais elementos. Isto é, quando o conjunto tem três ou mais modas. No conjunto a seguir, os elementos 2, 3 e 4 repetem-se cinco vezes, enquanto o elemento 5 aparece duas vezes. Logo, $Mo = 2$  $Mo=3$ e $Mo=4$.

    \begin{center}
    $\Bigg(\underbrace{2, 2, 2, 2, 2,}_{5\; elementos}\, \underbrace{3, 3, 3, 3, 3}_{5\; elementos}
    \underbrace{4, 4, 4, 4, 4}_{5\; elementos}
    \underbrace{5, 5}_{2\; elementos}\Bigg)$
    \end{center}
\end{itemize}

\subsection{Moda para dados agrupados sem intervalos de classe}

Quando os dados estão agrupados por valores, isto é, quando não agrupados em intervalos de classe, o cálculo da moda também é realizado de maneira simples e rápida. Para tanto, devemos identificar o valor que apresenta a maior frequência absoluta. Vejamos um exemplo.

\begin{example}
    Considere que o Estratégia Concursos tenha realizado um simulado, contendo 50 questões, com 100 estudantes da área fiscal, obtendo a seguinte distribuição de acertos:

    \begin{table}[ht]
        \centering
\begin{tabular}{cc}
    \toprule
\multicolumn{1}{c}{\begin{tabular}[c]{@{}c@{}}Nº de \\      Acertos ($X_i$)\end{tabular}} & \multicolumn{1}{c}{\begin{tabular}[c]{@{}c@{}}Frequência \\      Absoluta ($f_i$)\end{tabular}} \\
\midrule
45 & 8 \\
46 & 12 \\
47 & 28 \\
48 & \textcolor{red}{32} \\
49 & 17 \\
50 & 3\\
\bottomrule
\end{tabular}
\caption{Número de Acertos}
\label{tab:moda_agrupados}
\end{table}
\newpage
Para calcular a moda dessa distribuição, devemos identificar o maior valor existente na coluna de frequências. Veja novamente a tabela \ref{tab:moda_agrupados}, o valor da maior frequência é 32 que está marcado em vermelho. Logo, a moda da distribuição é o resultado de 48 questões corretas, pois corresponde a maior frequência. Portanto, podemos concluir que a maior concentração dos participantes errou somente duas questões:

\begin{center}
    $Mo=48$
\end{center}

\begin{figure}[H]
    \centering
    \begin{center}
        \includegraphics[width=0.8\linewidth]{histo_acertos.png}
    \end{center}
    \caption{Histograma do número de acertos.}
    \label{Fig:histo_acertos}
\end{figure}
Perceba que a maior barra do gráfico, referente à frequência 32, corresponde à moda da distribuição, isto é, um total de 48 questões corretas.
\end{example}

\subsection{Moda para dados agrupados em classes}

Quando os dados estão agrupados em classes de mesma amplitude, a moda será o valor dominante da classe que apresenta a maior frequência, a qual é denominada classe modal. Como já vimos, a amplitude de classe é a diferença entre os limites superior e inferior de uma determinada classe. Assim, quando as amplitudes são todas iguais, a moda estará contida na classe de maior frequência. A seguir, veremos os principais métodos empregados no cálculo da moda de distribuições agrupadas por intervalos de classe: moda bruta, moda de Pearson, moda de Czuber e moda de King.

\subsubsection{Moda Bruta}

A maneira mais simples de calcular a moda é tomar o ponto médio da classe modal. Esse valor, ao qual denominamos de moda bruta, é determinado pela seguinte fórmula:
\[
    \boxed{M_o=\frac{l_{inf}+l_{sup}}{2}}
\]
em que $l_{inf}$ é o limite inferior da classe modal; e
$l_{sup}$ é o limite superior da classe modal.

\begin{example}
Assim podemos exemplificar:    

\begin{table}[ht]
    \centering
\begin{tabular}{ccc}
\hline
\multicolumn{2}{c}{Faixa etária \$(X\_i)} & Frequência $(f_i)$ \\ \hline
\multicolumn{2}{c}{10 \vdash 20} & 30 \\
\multicolumn{2}{c}{20 \vdash 30} & 50 \\
\multicolumn{2}{c}{\textcolor{red}{30 \vdash 40}} & \textcolor{red}{70} \\
\multicolumn{2}{c}{40 \vdash 50} & 60 \\
\multicolumn{2}{c}{50 \vdash 60} & 10 \\ \hline
\multicolumn{2}{c}{Total 220} &  \\ \hline
\end{tabular}
\caption{Moda agrupada por classes}
\label{tab:moda_classes}
\end{table}

Como todas as classes possuem a mesma amplitude, a classe modal é aquela com maior frequência simples. No caso, trata-se da terceira classe marcada em vermelho na tabela \ref{tab:moda_classes}.
\medskip
Temos, então, as seguintes informações: 
\begin{itemize}
    \item limite inferior da classe modal: $l_{inf}=30$; e
    \item limite superior da classe modal: $l_{sup}=40$.
\end{itemize}
Aplicando a fórmula da moda bruta, temos:
\begin{align*}
    &M_o=\frac{l_{inf}+l_{sup}}{2}\\
    &M_o=\frac{30+40}{2}=35\\
\end{align*}
\end{example}

\subsubsection{Moda de Pearson}
O matemático Karl Pearson observou a existência de uma relação empírica que permite calcular a moda quando são conhecidas a média $(\bar{x})$  e a mediana $(M_d)$ de uma distribuição moderadamente assimétrica. Quando essas condições são satisfeitas, podemos aplicar a relação denominada de moda de Pearson:
\[
    M_o=3\times M_d-2\times \bar{x}
\]
Onde:
$\bar{x}$ é a média e
$M_d$ é a mediana da distribuição.

\subsubsection{Moda de Czuber}

O matemático Emanuel Czuber elaborou um processo gráfico capaz de aproximar o cálculo da moda. Para determinar graficamente a moda, Czuber partiu de um histograma, utilizando os três retângulos correspondentes à classe modal e às classes adjacentes (anterior e posterior).
\[
    M_o=l_{inf}+x
\]
Nesse caso, o valor de $x$ é determinado pela intersecção dos segmentos $\overline{AB}$ (que une o limite superior da
classe que antecede a classe modal ao limite superior da classe modal) e $\overline{CD}$ (que une o limite inferior da
classe modal ao inferior da classe posterior).

\begin{figure}[ht]
    \centering
    \includegraphics[width=0.8\linewidth]{Fig_czuber.png}
    \caption{Moda de Czber}
    \label{Fig:czuber_1}
    \end{figure}

Note os triângulos (I) e (II), indicados na figura a seguir:

    \begin{figure}[ht]
    \centering
    \includegraphics[width=0.8\linewidth]{Fig_czube_2r.png}
    \caption{Moda de Czber (geometria)}
    \label{Fig:czuber_2}
    \end{figure}

Por semelhança de triângulos temos:
    \[
        \frac{\Delta_1}{x}=\frac{\Delta_2}{h-x}
    \]
    Fazendo a multiplicação cruzada das frações, temos que:
    \begin{align*}
        &\Delta_1(h-x)=\Delta_2(x)\\
        &h\Delta_1-x\Delta_1=x\Delta_2\\
        &h\Delta_1=x\Delta_2+x\Delta_1\\
        &h\Delta_1=x(\Delta_2+x\Delta_1)\\
    \end{align*}

Então temos:
    \[
        x=\Bigg[\frac{\Delta_1}{\Delta_1+\Delta_2}\Bigg]\times h
    \]
    Como a moda é igual ao limite inferior da classe modal adicionado de 𝑥, temos que:
    \[
        M_o=l_{inf}+x
    \]
    e portanto,
    \[
        M_o=l_{inf}+\Bigg[\frac{\Delta_1}{\Delta_1+\Delta_2}\Bigg]\times h
    \]
    ou
    \[
          M_o=l_{inf}+\Bigg[\frac{f_{Mo}-f_{ant}}{2\times f{M_o}-f_{post}}\Bigg]\times h
    \]

    \subsection{Propriedades da Moda}

São propriedades da Moda:
\begin{itemize}
    \item Somando-se (ou subtraindo-se) uma constante $c$ a todos os valores de uma variável, a moda do conjunto fica aumentada (ou diminuída) dessa constante.
    \item Multiplicando-se (ou dividindo-se) uma constante $c$ a todos os valores de uma variável, a moda do conjunto fica aumentadamultiplicada (ou dividida) dessa constante.
\end{itemize}

\end{document}