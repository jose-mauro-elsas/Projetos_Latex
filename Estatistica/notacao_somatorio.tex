
Com frequência, as fórmulas matemáticas exigem a adição de muitas variáveis, como a média aritmética. O somatório ou notação sigma é uma forma simples e conveniente de abreviação, usada para fornecer uma expressão concisa para a soma dos valores de uma variável. Por exemplo, se quisermos representar a soma de um número de termos tais como:
\[
    1+2+3+4+5
\]
ou,
\[
    1^2+2^2+3^2+5^2+5^2
\]
em que há um padrão evidente para os números envolvidos.

De modo geral, se tomarmos uma sequência de números \(x_1, x_2, x_3, \dots, x_n\) então podemos escrever a soma desses números como \(x_1, x_2, x_3, \dots, x_n\). Nesse conjunto, \(x_1\) representa o primeiro termo; \(x_2\) representa o segundo; \(x_3,\) o terceiro; e \(x_i\) o i-ésimo termo da soma.

Essa soma pode ser representada de uma forma mais simples e concisa, deixando que \(x_i\) represente o termo geral da sequência. Para isso, empregamos a seguinte notação:
\[
    \sum_{i=1}^n\,x_i
\]
Então, em vez de usarmos vários elementos para determinarmos o somatório, utilizamos somente o símbolo do somatório:

Essa notação envolve um símbolo de somatório, \(\sum\), sendo a letra grega maiúscula Sigma \(\Sigma\). Basicamente, esse símbolo está nos instruindo a somar determinados elementos de uma sequência. Os elementos típicos da sequência que está sendo somada aparecem à direita do símbolo de somatório:
\[
    {\sum}\,x_i
\]

Observe que essa notação também requer a definição de um índice, que fica localizado abaixo do símbolo de somatório. Esse índice é frequentemente representado por \(i\), embora também seja comum encontrarmos questões adotando \(j\) ou \(n\).
Esse índice normalmente aparece como uma expressão, por exemplo, \(i=a\), em que o índice assume um valor inicial atribuído no lado direito da equação, conhecido como limite inferior \(a\). Se considerarmos que \(i=1\), estamos dizendo que o primeiro elemento da sequência a ser considerado é o de índice igual a 1, isto é, o primeiro elemento da sequência. A condição de parada ou limite superior do somatório é o valor localizado acima do símbolo, no caso \(b\). A condição de parada indica o último elemento da sequência a ser considerado no somatório.
\[
    \sum_{i=a}^b\,x_i
\]
Então, se tivermos uma sequência de 10 valores, devemos interpretar a notação a seguir como a soma dos valores da sequência \(x_i\), com \(i\) variando de 1 a 10:
\[
    \sum_{i=1}^{10}\,x_i
\]
Vejamos alguns exemplos típicos de operações envolvendo somatórios. Para isso, tomaremos como exemplo a sequência {\(x_i\)} = {1, 2, 3, 4, 5, 6, 7, 8, 9, 10}:
\newpage
\begin{table}[H]
\centering
\renewcommand{\arraystretch}{1.3}
\setlength{\tabcolsep}{10pt}

\begin{tabularx}{\textwidth}{c X}
\toprule
\textbf{Notação} & \textbf{Interpretação e exemplo} \\
\midrule

$\displaystyle \sum_{i=3}^{10} x_i$
&
Representa a soma dos valores de $x$, começando em $x_3$ e terminando em $x_{10}$.
\[
\sum_{i=3}^{10} x_i = x_3 + x_4 + x_5 + x_6 + x_7 + x_8 + x_9 + x_{10}
\]
\[
= 3 + 4 + 5 + 6 + 7 + 8 + 9 + 10 = 55
\]
\\

$\displaystyle \sum x$
&
Quando os limites do somatório são omitidos, entende-se que a soma é feita de
$x_1$ até $x_n$.
\[
\sum x = x_1 + x_2 + x_3 + \dots + x_n
\]
\[
= 1 + 2 + 3 + \dots + 10 = 55
\]
\\

$\displaystyle \sum_{i=1}^{n} x_i^2$
&
Representa a soma dos quadrados dos valores de $x$, de $x_1$ até $x_n$.
\[
\sum_{i=1}^{n} x_i^2 = x_1^2 + x_2^2 + x_3^2 + \dots + x_n^2
\]
\[
= 1^2 + 2^2 + 3^2 + \dots + 10^2 = 385
\]
\\

$\displaystyle \sum_{k=1}^{4} (2x_k + 1)$
&
Representa a soma dos termos da sequência $(2n+1)$, com $k$ variando de 1 a 4.
\[
= (2\cdot1+1) + (2\cdot2+1) + (2\cdot3+1) + (2\cdot4+1)
\]
\[
= 3 + 5 + 7 + 9 = 24
\]
\\
$\displaystyle \sum_{i=3}^{5}\left(\frac{x_i}{x_i+1}\right)$
&
Representa a soma dos termos $\dfrac{x_i}{x_i+1}$, com $i$ começando em 3 e terminando em 5.
\[
\sum_{i=3}^{5}\left(\frac{x_i}{x_i+1}\right)
= \frac{x_3}{x_3+1} + \frac{x_4}{x_4+1} + \frac{x_5}{x_5+1}
\]
\[
= \frac{3}{3+1} + \frac{4}{4+1} + \frac{5}{5+1}
= \frac{3}{4} + \frac{4}{5} + \frac{5}{6}
\]
\[
= \frac{45 + 48 + 50}{60} = \frac{143}{60}
\]
\\

\bottomrule
\end{tabularx}

\caption{Interpretação de diferentes notações de somatório.}
\end{table}

Agora que já entendemos o funcionamento básico dessa notação, precisamos analisar outras operações aritméticas que também podem ser realizadas com as variáveis num somatório. Para tanto, vamos tomar como base as sequências {\(x_i\)} = {1, 2, 3, 4, 5, 6, 7, 8, 9, 10} e {\(y_i\)} = {3, 6, 9, 12, 15, 18, 21, 24, 27, 30}.

Por exemplo, na SOMA DOS PRODUTOS, multiplicamos \(x_1\) por \(y_1\); \(x_2\) por 
\(y_2\); e assim por diante, até \(x_n\)
por \(y_n\). Em seguida, somamos os resultados de cada multiplicação. A SOMA DOS PRODUTOS da variável
\(x\) pela variável \(y\), com \(i\) variando de 1 a 10, pode ser representada por meio da seguinte expressão:

\[
    \sum\limits_{i=1}^n\,x_i\times y_i=x_1\times y_1+x_2\times y_2+x_3\times y_3+\dots+x_n\times y_n
\]
\[
    \sum\limits_{i=1}^n\,x_i\times y_i=1\times 3+2\times 6+3\times 9+\dots+10\times 30 = 1155
\]
Observe que essa expressão é diferente de \(\sum_{i=1}^n\,x_i\times \sum_{i=1}^n\,y_i\times\), que representa o \textbf{PRODUTO DAS SOMAS} dessas duas variáveis. No \textbf{PRODUTO DAS SOMAS}, primeiro somamos toda a sequência \(x\), depois toda
a sequência \(y\) e, em seguida, multiplicamos o resultado das somas:

\[
    \sum\limits_{i=1}^{10}\,x_i \times\sum\limits_{i=1}^{10}\,y_i=(1+2+3+\dots+10)\times (3+6+9+\dots+30)
\]
\[
    \sum\limits_{i=1}^{10}\,x_i \times\sum\limits_{i=1}^{10}\,y_i=55\times 165=9075
\]

Dessa forma, temos que:
\begin{center}
    \textbf{SOMA DOS PRODUTOS} \(\neq\) \textbf{PRODUTO DAS SOMAS}
\end{center}

Também podemos utilizar a notação para representar o QUADRADO DA SOMA dos valores de \(x\), com \(i\) iniciando em 1 e terminando em 10. No QUADRADO DA SOMA, somamos toda a sequência e elevamos o resultado ao quadrado:

\begin{align*}
    \Bigg(\sum\limits_{i=1}^{n}\,x_i\Bigg)^2&=(x_1+x_2+x_3+\dots+x_n)^2\\
    \Bigg(\sum\limits_{i=1}^{n}\,x_i\Bigg)^2&=(1+2+3+\dots+10)^2\\
    \Bigg(\sum\limits_{i=1}^{n}\,x_i\Bigg)^2&=(55)^2=3025\\
\end{align*}

Veja que essa expressão é diferente de \(\sum_{i=n}^n\,x_i^2\), que representa a \textbf{SOMA DOS QUADRADOS}. Na \textbf{SOMA DOS QUADRADOS}, cada elemento da sequência é elevado ao quadrado e depois os resultados são somados:

\begin{align*}
      \sum\limits_{i=1}^{n}\,x_i^2&=1^2+2^2+3^2+\dots+10^2=385\\
\end{align*}

Por fim, ainda podemos representar o somatório de uma constante \(k\). Digamos que essa constante tenha valor igual a 3:

\begin{align*}
      \sum\limits_{i=1}^{n}\,k&=k+k+k+\dots+k=k\times n\\
      \sum\limits_{i=1}^{n}\,3&=3+3+3+\dots+3=3\times n\\
\end{align*}

\subsection{Propriedades do Somatório}
As propriedades apresentadas nesta seção facilitam o desenvolvimento de expressões algébricas com a notação de somatório.

\begin{enumerate}
    \item O somatório de uma constante \(k\) é igual ao produto do número de termos pela constante.
    \begin{align*}
      \sum\limits_{i=1}^{n}\,k&=k+k+k+\dots+k=k\times n\\
\end{align*}
Para demonstrar essa propriedade, consideraremos que cada constante está multiplicada pelo valor um, isto é:
\[
    \sum\limits_{i=1}^{n}\,k=k+k\times1+k\times1+\dots+k\times1
\]
Agora, colocaremos os novos valores em evidência:

\[
    \sum\limits_{i=1}^{n}\,k=k\,{\underbrace{(1+1+1+\dots+1)}_{n\,\text{termos}}}=k\times n
\]

Considere uma lista composta por noventa e nove elementos repetidos e iguais a 9:
\[
    \{\underbrace{{9, 9, 9, 9,\dots, 9}}_{\text{99 termos repetidos}}\}
\]

O somatório dos elementos dessa lista será:
\[
    \sum\limits_{i=1}^{99}\,9=\,{\underbrace{(9+9+9+\dots+9)}_{n\,\text{termos repetidos}}}=9\times \,{\underbrace{(9+9+9+\dots+9)}_{n\,\text{termos repetidos}}}=9\times99=891
\]

    \item O somatório do produto de uma constante por uma variável é igual ao
produto da constante pelo somatório da variável.

    Para demonstrarmos essa propriedade, colocaremos em evidência cada constante \(k\):
   \[
    \sum\limits_{i=1}^n\,k\times x_i = k\times x_1+ k\times x_2+k\times x_3 +\dots+k\times x_n=k\times\sum_{n=1}^{n}\,x_i
\] 
Para demonstrarmos essa propriedade, colocaremos em evidência cada constante \(k\):

   \[
    \sum\limits_{i=1}^n\,k\times x_i = k\times x_1+ k\times x_2+k\times x_3 +\dots+k\times x_n = k \times (x_1+x_2+x_3+\dots+x_n)
\] 

Já sabemos que \(\sum_{i=1}^{n}\,x_i=x_1+x_2+\dots+x_n\). Logo:

\[
\sum_{i=1}^{n}\,k\times x_i=k\times\sum_{i=1}^{n}x_i
\]

Portanto, a constante pode sair de dentro do somatório, passando a multiplicá-lo.

\end{enumerate}