%!TEX program = lualatex
% Arquivo: Template_Estat_Texto.tex

\documentclass[a4paper,12pt]{article}

% =========================================================
% Idioma e tipografia (LuaLaTeX)
% =========================================================
\usepackage[portuguese]{babel}
\usepackage{fontspec}
\setmainfont{Latin Modern Roman}

% =========================================================
% Layout
% =========================================================
\usepackage[top=3cm,bottom=2.5cm,left=3cm,right=2.5cm]{geometry}
\usepackage{setspace}
\onehalfspacing
\usepackage{indentfirst}
\setlength{\parindent}{1.25cm}

% =========================================================
% Matemática (align*, etc.)
% =========================================================
\usepackage{amsmath}
\usepackage{amssymb}

% =========================================================
% Tabelas, listas, figuras
% =========================================================
\usepackage{booktabs}
\usepackage{array}
\usepackage{enumitem}
\setlist[itemize]{noitemsep,topsep=3pt}
\setlist[enumerate]{noitemsep,topsep=3pt}

\usepackage{graphicx}
\usepackage{float}
\usepackage{pdflscape}
\graphicspath{{./}{./imagens/}{./figuras/}}

% =========================================================
% Links
% =========================================================
\usepackage{hyperref}
\usepackage{xurl}
\hypersetup{
  colorlinks=true,
  linkcolor=black,
  urlcolor=blue,
  citecolor=black,
  pdfauthor={},
  pdftitle={}
}

% =========================================================
% Texto de preenchimento (lorem)
% =========================================================
\usepackage{lipsum}

% =========================================================
% Dados da capa
% =========================================================
\newcommand{\Instituicao}{Universidade / Instituição}
\newcommand{\Curso}{Disciplina / Curso}
\newcommand{\Professor}{Professor: Nome Sobrenome}
\newcommand{\Aluno}{Aluno: José Mauro Xavier Elsas}
\newcommand{\Cidade}{Rio de Janeiro}
\newcommand{\Ano}{2026}

\title{\textbf{Título do Documento}}
\author{\Aluno}
\date{\Cidade --- \Ano}

\begin{document}

% =========================================================
% Capa (página de rosto)
% =========================================================
\begin{titlepage}
\centering
{\large \Instituicao \par}
\vspace{0.5cm}
{\large \Curso \par}
\vspace{1.0cm}

{\LARGE \textbf{Título do Documento} \par}
\vspace{0.6cm}
{\Large Subtítulo (opcional) \par}

\vfill

\begin{flushright}
\Aluno\\
\Professor
\end{flushright}

\vspace{1.0cm}
{\Cidade --- \Ano \par}
\end{titlepage}

% =========================================================
% Sumário
% =========================================================
\tableofcontents
\newpage

% =========================================================
% Corpo do texto (exemplos)
% =========================================================
\section{Introdução}
\lipsum[1-2]

\section{Fundamentação teórica}
\lipsum[3]

\subsection{Um exemplo com \texttt{align*}}
\begin{align*}
f(x) &= -0.5x^2 + 2x - 3 \\
g(x) &= -0.5x - 0.5
\end{align*}

\lipsum[4]

\section{Metodologia}
\lipsum[5]

\section{Resultados}
\lipsum[6]

% Exemplo de tabela
\begin{table}[H]
\centering
\caption{Exemplo de tabela com \texttt{booktabs}.}
\label{tab:exemplo}
\setlength{\tabcolsep}{12pt}
\renewcommand{\arraystretch}{1.2}
\begin{tabular}{lc}
\toprule
\textbf{Intervalo} & \textbf{Frequência} \\
\midrule
$85 \le x < 100$  & 5 \\
$100 \le x < 115$ & 5 \\
\bottomrule
\end{tabular}
\end{table}

\section{Discussão}
\lipsum[7]

\section{Conclusão}
\lipsum[8]

% =========================================================
% Referências (entra no sumário)
% =========================================================
\newpage
\phantomsection
\addcontentsline{toc}{section}{Referências}
\begin{thebibliography}{99}

\bibitem{exemplo1}
SOBRENOME, Nome.
\textbf{Título do livro}.
Cidade: Editora, 2020.

\bibitem{exemplo2}
AUTOR, Nome.
Título do artigo.
\textit{Nome do Periódico}, v. 10, n. 2, p. 1--20, 2021.
Disponível em: \url{https://exemplo.com}. Acesso em: 30 jan. 2026.

\end{thebibliography}

\end{document}
