Exercício 5 da segunda lista.

\subsubsection*{Retas tangentes nas direções $x$ e $y$ e plano tangente}

Considere a superfície
\[
z = f(x,y) = 3x^2 + \frac{3}{2}y^2 - \frac{7}{2}.
\]

No ponto
\[
P = (x_0,y_0,z_0) = (1,1,1),
\]
temos as derivadas parciais
\[
f_x(1,1) = 6
\qquad\text{e}\qquad
f_y(1,1) = 3.
\]

\subsection*{1. Reta tangente no corte paralelo ao plano $xz$}

Fixando $y=1$, obtemos uma curva espacial dada por
\[
z = f(x,1) = 3x^2 - 2.
\]

A inclina\c{c}\~ao dessa curva no ponto $x=1$ (isto é, a taxa de variação de $z$ quando andamos na dire\c{c}\~ao de $x$ e mantemos $y$ constante) é
\[
\frac{\partial z}{\partial x}\Big|_{(1,1)} = f_x(1,1) = 6.
\]

Logo, a reta tangente a essa curva em $P=(1,1,1)$ pode ser escrita de forma paramétrica como
\[
\ell_x(t) =
\bigl(x(t),y(t),z(t)\bigr)
=
(1+t,\; 1,\; 1 + 6t),
\quad t \in \mathbb{R}.
\]

Isto está contido num plano paralelo ao plano $xz$ (pois $y=1$ é constante).

\subsection*{2. Reta tangente no corte paralelo ao plano $yz$}

Agora fixamos $x=1$, obtendo a curva
\[
z = f(1,y) = \frac{3}{2}y^2 - \frac{1}{2}.
\]

A inclina\c{c}\~ao dessa curva no ponto $y=1$ é
\[
\frac{\partial z}{\partial y}\Big|_{(1,1)} = f_y(1,1) = 3.
\]

Assim, a reta tangente em $P=(1,1,1)$ nesta direção é
\[
\ell_y(s) =
\bigl(x(s),y(s),z(s)\bigr)
=
(1,\; 1+s,\; 1 + 3s),
\quad s \in \mathbb{R}.
\]

Isto está contido num plano paralelo ao plano $yz$ (pois $x=1$ é constante).

\subsection*{3. Plano tangente à superfície em $P$}

A equação geral do plano tangente a $z=f(x,y)$ no ponto $(x_0,y_0)$ é
\[
z - z_0
=
f_x(x_0,y_0)\,(x-x_0)
+
f_y(x_0,y_0)\,(y-y_0).
\]

Substituindo $x_0=1$, $y_0=1$, $z_0=1$, $f_x(1,1)=6$ e $f_y(1,1)=3$, obtemos
\[
z - 1
=
6(x-1) + 3(y-1).
\]

Equivalentemente,
\[
z = 6x + 3y - 8.
\]

Observe que as duas retas tangentes $\ell_x$ e $\ell_y$ est\~ao contidas neste plano tangente.

\bigskip

\section*{Visualiza\c{c}\~ao em 3D}

A figura abaixo mostra:
\begin{itemize}
    \item A superf\'icie $z = 3x^2 + \tfrac{3}{2}y^2 - \tfrac{7}{2}$;
    \item O ponto $P=(1,1,1)$;
    \item As duas retas tangentes $\ell_x$ e $\ell_y$.
\end{itemize}

  \[f(x,y)= xy\frac{(x^2-y^2)}{x^2+y^2} \] é melhor usar a recra do produto considerando
  \[ u=xy \] e o \[ v=\frac{(x^2-y^2)}{x^2+y^2} \] ou a regra do quociente sendo \[uu= xy(x^2-y^2)
  \] e \[ v=(x^2+y^2) \]? ou há ainda alguma coisa melhor?
% --- Passo a passo para reta tangente via plano tangente (sem preâmbulo) ---

\subsection*{Reta tangente por interseção com o plano tangente (uma equação, duas ``partes'')}

\paragraph{Ideia-chave.}
Calcule o \textbf{plano tangente} à superfície no ponto pedido. \emph{Ele é único.}
Depois, para cada item do exercício, você \textbf{impõe o plano de corte} correspondente
e obtém a \textbf{reta tangente da curva de interseção} como a interseção desses dois planos.

\paragraph{Plano tangente (equação-base).}
Para \(z=f(x,y)\) no ponto \(P=(x_0,y_0,z_0)\) com \(z_0=f(x_0,y_0)\),
\[
\boxed{\,z - z_0 \;=\; f_x(x_0,y_0)\,(x-x_0) \;+\; f_y(x_0,y_0)\,(y-y_0)\, }.
\]