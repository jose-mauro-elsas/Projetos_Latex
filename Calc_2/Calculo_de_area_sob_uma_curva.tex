% conteudo/teoria/area_sob_a_curva.tex
% Interpretação geométrica da área sob o gráfico de uma função

Seja $f:[a,b] \to \mathbb{R}$ uma função definida em um intervalo fechado e limitado, tal que:
\[
f(x) \ge 0 \quad \text{para todo } x \in [a,b].
\]

Nessas condições, pode-se associar à função $f$ uma região do plano limitada pelo gráfico de $y=f(x)$, pelo eixo $x$ e pelas retas verticais $x=a$ e $x=b$. A medida dessa região é chamada de \textbf{área sob a curva} de $f$ no intervalo $[a,b]$.

\begin{figure}[H]
\centering
\begin{tikzpicture}[x=1cm,y=1cm,>=Latex]

  % --- parâmetros (ajuste aqui) ---
  \def\a{2.2}   % posição de a
  \def\b{7.6}   % posição de b

  % --- moldura ---
  \draw[line width=0.6pt] (-0.6,-0.6) rectangle (9.5,4.8);

  % --- eixos com setas ---
  \draw[->, line width=0.8pt] (0,0) -- (8.8,0) node[below right] {$x$};
  \draw[->, line width=0.8pt] (0,0) -- (0,4.2) node[above] {$y$};
  
  % --- função (curva com oscilações suaves) ---
  \def\fx(#1){ 2.55 + 0.95*sin(20*#1) - 0.45*cos(70*#1) - 0.08*(#1-5.0) }

  % --- região sombreada S entre a e b ---
  \fill[gray!25]
    (\a,0)
    -- plot[domain=\a:\b, samples=200] (\x,{ \fx(\x) })
    -- (\b,0) -- cycle;

  % --- linhas verticais em a e b (bordas da região) ---
  \draw[line width=0.6pt] (\a,0) -- (\a,{ \fx(\a) });
  \draw[line width=0.6pt] (\b,0) -- (\b,{ \fx(\b) });

  % --- curva completa (um pouco antes e depois) ---
  \draw[line width=0.9pt]
    plot[domain=0.6:8.6, samples=240] (\x,{ \fx(\x) });

  % --- rótulos ---
  \node at ({0.5*(\a+\b)},1.5) {$\textbf{S}$};
  \node[above right] at (8.15,{ \fx(8.15) }) {$f(x)$};

  % marcações a e b no eixo x
  \node[below] at (\a,0) {$a$};
  \node[below] at (\b,0) {$b$};

\end{tikzpicture}
\caption{Região $S$ sob o gráfico de $f(x)$ entre $a$ e $b$.}
\end{figure}

Do ponto de vista geométrico, essa área é o objeto que se deseja calcular. Entretanto, não existe, em geral, uma fórmula elementar direta para essa medida. Por isso, recorre-se a métodos de aproximação.

\subsubsection{Aproximação por retângulos}

Para estimar a área sob a curva, divide-se o intervalo $[a,b]$ em subintervalos:
\[
a=x_0 < x_1 < \cdots < x_n=b.
\]

Em cada subintervalo $[x_{i-1}, x_i]$, constrói-se um retângulo de base:
\[
\Delta x_i = x_i - x_{i-1}
\]
e altura dada por um valor da função nesse respectivo subintervalo.

A soma das áreas desses retângulos fornece uma aproximação da área total. Quanto maior o número de retângulos (ou menor a largura dos subintervalos), melhor tende a ser essa aproximação.

\subsubsection{Área e sinal da função}

A interpretação geométrica apresentada só é válida quando $f(x) \ge 0$. Quando a função assume valores negativos, a região correspondente situa-se abaixo do eixo $x$, e a noção de área geométrica deve ser tratada com cuidado, pois a integral pode resultar em valores negativos.

Nesse caso, a ferramenta matemática adequada para tratar o problema de forma sistemática é a integral definida, que será formalmente introduzida na próxima seção.
