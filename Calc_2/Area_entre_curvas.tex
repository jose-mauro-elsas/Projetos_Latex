%!TEX root = C:/LaTex_Projects/Calc_2/Caderno_de_Calc_II.tex
%!TEX program = lualatex
% LTeX: language=pt-BR

Nesta seção, "área entre curvas" significa \textbf{área geométrica} (sempre não-negativa), calculada por integrais definidas. Em todos os casos, a ideia central é somar a área de tiras (verticais ou horizontais) que preenchem a região.
O cálculo de áreas utilizando integrais baseia-se no princípio fundamental de somar infinitos retângulos de altura infinitesimal. A altura de cada retângulo é definida pela diferença entre o "teto" (limite superior) e o "chão" (limite inferior) da região. Abaixo, detalhamos os quatro cenários principais.
\newpage
\begin{example}
  Segue um exemplo gráfico de área entre curvas:
  \begin{figure}[H]
  \centering
    \includegraphics[width=0.8\linewidth]{area_entre_y=x+4_e_y=raiz_de_0.5x.png}
    \caption{Gráfico de área entre curvas}
    \label{graf_area_entre_curvas_ex}
\end{figure}
\end{example}

\subsection{Área sob uma Curva (Acima do Eixo X)}
Conforme visto na seção\, \ref{subsec:integral-riemann} sobre integral de Riemann, quando uma função $f(x)$ é contínua e positiva em um intervalo $[a,b]$, a área é calculada entre a curva e o eixo $x$. Neste caso, o eixo $x$ atua como o limite inferior ($y = 0$).

\begin{equation}
A = \int_{a}^{b} f(x) \, dx
\end{equation}

\begin{figure}[hbt]
\centering
\begin{tikzpicture}
\begin{axis}[
  axis x line=middle,
  axis y line=middle,
  axis line style={-{Stealth[length=3mm,width=2mm]}},
  xlabel={$x$},
  ylabel={$y$},
  xmin=0, xmax=4,
  ymin=0, ymax=5,
  title={Caso 1: $f(x) \ge 0$ em $[a,b]$},
]
  \def\a{0.5} \def\b{3.5}
  \addplot[name path=curva, thick, blue, domain=\a:\b, samples=200] {0.5*(x-2)^2+1};
  \addplot[name path=eixo, draw=none, domain=\a:\b] {0};
  \addplot[fill=gray!20, draw=none] fill between[of=curva and eixo];
  \draw[line width=0.6pt] (axis cs:\a,0) -- (axis cs:\a,{0.5*(\a-2)^2+1});
  \draw[line width=0.6pt] (axis cs:\b,0) -- (axis cs:\b,{0.5*(\b-2)^2+1});
  \node[below] at (axis cs:\a,0) {$a$};
  \node[below] at (axis cs:\b,0) {$b$};
\end{axis}
\end{tikzpicture}
\caption{Se $f(x)\ge 0$, então $A=\int_a^b f(x)\,dx$.}
\end{figure}

\subsection{Área entre Duas Curvas (integralmente no primeiro quadraente)}
Quando temos duas funções $f(x)$ e $g(x)$ em um intervalo $[a, b]$, onde $f(x) \geq g(x)$ para todo $x$, a área da região delimitada por elas é a integral da diferença entre a função superior e a inferior.
\begin{equation}
A = \int_{a}^{b} [f(x) - g(x)] \, dx
\end{equation}

\begin{figure}[H]
  \centering
  \begin{tikzpicture}
    \begin{axis}[
      axis x line=middle,
      axis y line=middle,
      axis line style={-{Stealth[length=3mm,width=1.5mm]}},
      xlabel={$x$},
      ylabel={$y$},
      xmin=0, xmax=2,
      ymin=0, ymax=6,
      title={$\sin(x)$ e $e^x$ em $[0,\pi/2]$},
      xtick={0, 1.57079}, xticklabels={$0$, $\frac{\pi}{2}$},
    ]
      \addplot[name path=A, thick, blue, domain=0:pi/2, samples=200] {sin(deg(x))};
      \addplot[name path=B, thick, red,  domain=0:pi/2, samples=200] {exp(x)};
      \addplot[fill=gray!20, draw=none] fill between[of=B and A];
      \node[anchor=west] at (axis cs:0.9,3.2) {$e^x$};
      \node[anchor=west] at (axis cs:1.35,1.0) {$\sin(x)$};
    \end{axis}
  \end{tikzpicture}
  \caption{Área entre $e^x$ e $\sin(x)$ no intervalo $[0,\pi/2]$.}
\end{figure}

Se \(f(x)=e^x\) e \(g(x)=\sin(x)\) em \([0,\frac{\pi}{2}]\), então a área entre
essas duas curvas é dada por:
\begin{align*}
A
&=\int_0^{\frac{\pi}{2}}\bigl(e^x-\sin(x)\bigr)\,dx\\
&=\int_0^{\frac{\pi}{2}}e^x\,dx-\int_0^{\frac{\pi}{2}}\sin(x)\,dx.
\end{align*}

Primeira integral:
\begin{align*}
\int_0^{\frac{\pi}{2}}e^x\,dx
&=\Big[e^x\Big]_0^{\frac{\pi}{2}}
=e^{\frac{\pi}{2}}-1.
\end{align*}

Segunda integral:
\begin{align*}
-\int_0^{\frac{\pi}{2}}\sin(x)\,dx
&=-\Big[-\cos(x)\Big]_0^{\frac{\pi}{2}}
=\Big[\cos(x)\Big]_0^{\frac{\pi}{2}}
=\cos\Big(\frac{\pi}{2}\Big)-\cos(0)\\
&=0-1=-1.
\end{align*}

Assim, a área entre as curvas é:
\begin{align*}
A
&=\bigl(e^{\frac{\pi}{2}}-1\bigr)+(-1)
=e^{\frac{\pi}{2}}-2
\approx 2{,}81\ \text{u.a.}
\end{align*}

\begin{exercicio}\label{Ex_curvas_cruzadas}
  sejam \(f(x)=4x-x^2\) e \(g(x)=x^2\), calcular a área entre as curvas.
  Embora não seja obrigatório, vou fazer o gráfico para ter uma noção maior do problema.

  \begin{figure}
    \centering
    \tikzsetnextfilename{area_entre_curvas_1}
    \begin{tikzpicture}
      \begin{axis}
        [
          xmin=0, xmax=5,
          ymin=0, ymax=10,
          axis x line=middle,
          axis y line=middle,
          xlabel=\(x\),
          ylabel=\(y\),
          grid=both,
          tick align=outside,
          minor tick num=1,
          axis line style={-{Stealth[length=3mm,width=1.5mm]}},
          title={Funções \(y=4x-x^2\) e \(y=x^2\)},
        ]
          \addplot[red, domain=0:3, samples=200, thick, name path=A] {4*x-x^2}; 
          \addplot[azulmarinho, domain=0:3, samples=200, thick, name path=B] {x^2};
          \node at (axis cs:3.5,3.5) {(3,3)};
          \node at (axis cs:2.5,9) {(3,9)};
          \draw[red, dashed] (3,3) -- (3,9);  
          %0..2: A acima de B
          \addplot[fill=gray!20, draw=none]
          fill between[of=A and B, soft clip={domain=0:2}];

          % 2..3: B acima de A
          \addplot[fill=gray!20, draw=none]
          fill between[of=B and A, soft clip={domain=2:3}];
      \end{axis}
    \end{tikzpicture}
    \caption{Área entre as curvas que se cruzam}
  \end{figure}
  Pelo gráfico, a interseção entre as curvas parece ser em (0,0 e)(2,4) mas isso precisa ser demontrado algebricamente. As interseções são no pontos onde as funções são iguais.

  Então:
  \begin{align*}
    &f(x)=g(x)\\
    &4x-x^2=x^2\\
    &-2x^2+4x=0\\
    &x(-2x+4)=0\\
    &x'=0\\
    &2x=4\,\therefore \, x"=2\\
  \end{align*}

  conforme mostrado no gráfico, podemos dizer que os limites da primeira integral é dado por: [0,2] e para a segunda integral é [2,2,5]

  Primeira Integral
  \begin{align*}
    &\int_{0}^{2} (4x-x^2)\,dx-\int_{0}^{2}x^2\,dx\\
    &=\Bigg[2x^2-\frac{x^3}{3}\Bigg]_0^2-\Bigg[\frac{x^3}{3}\Bigg]_0^2\\
    &=\Bigg[2(2^2)-\frac{2^3}{3}\Bigg]-0-\Bigg[\frac{2^3}{3}\Bigg]-0\\
    &=\Bigg[8-\frac{8}{3}\Bigg]-\Bigg[\frac{8}{3}\Bigg]=\frac{24}{3}-\frac{8}{3}=\boxed{\frac{16}{3}}\\
  \end{align*}

  Segunda Integral
  \begin{align*}
    & \int_{2}^{3}x^2\,dx-\int_{2}^{3}(4x-x^2)\,dx\\
    &=\Bigg[\frac{x^3}{3}\Bigg]_2^{3}-\Bigg[2x^2-\frac{x^3}{3}\Bigg]_2^{3}\\
    & =\Bigg[\frac{(3)^3}{3}-\frac{2^3}{3}\Bigg]-
    \Bigg[\Big[2({3}^2)-\frac{3^3}{3}\Big]-\Big[2({2}^2)-\frac{2^3}{3}\Big]\Bigg]\\
    &=\frac{19}{3}-\Big[18-\frac{27}{3}-\Big(8-\frac{8}{3}\Big)\Big]\\
    &=\frac{19}{3}-\Big[\frac{54}{3}-\frac{27}{3}-\frac{16}{3}\Big]\\
    &=\frac{19}{3}-\frac{11}{3}=\frac{8}{3}\\
  \end{align*}
  
  então a área total é,

  \begin{align*}
    \frac{16}{3}+\frac{8}{3}=\frac{24}{3}=\boxed{8\text{u.a.}}
  \end{align*}
\end{exercicio}

\subsection{Tipo 1: Uma função sempre acima da outra (posição relativa fixa)}
Considere duas funções contínuas \(f\) e \(g\) em \([a,b]\) tais que, para todo \(x\in[a,b]\), vale
\[
f(x)\ge g(x).
\]
Nesse caso, a região entre as curvas \(y=f(x)\) (curva superior) e \(y=g(x)\) (curva inferior), limitada pelas retas \(x=a\) e \(x=b\), tem área dada por
\[
A=\int_a^b \big(f(x)-g(x)\big)\,dx.
\]
\textbf{Exemplo (sem números):} área limitada superiormente por \(y=f(x)\), inferiormente por \(y=g(x)\) e lateralmente por \(x=a\) e \(x=b\), sem que as curvas se cruzem no intervalo.

\medskip
\textbf{Observação importante.} Não importa se \(f\) e \(g\) estão acima ou abaixo do eixo \(x\). O que importa é somente qual delas está \emph{acima da outra}. Mesmo que \(f(x)\) e \(g(x)\) sejam negativos, a área continua sendo \(\int ( \text{superior} - \text{inferior})\,dx\).

\subsection{Tipo 2: As curvas se entrecruzam (troca de quem está por cima)}
Suponha que \(f\) e \(g\) sejam contínuas em \([a,b]\) e que existam pontos de interseção:\\
 \(c_1,\, c\,_2, \dots,c_n\) no intervalo, isto é,
\[
f(c_k)=g(c_k)\quad (k=1,2,\dots,n),
\]
de modo que a posição relativa (quem está acima) pode mudar de um subintervalo para outro. Nesse caso, a área geométrica deve ser calculada \textbf{particionando o intervalo} nos pontos de interseção e somando as contribuições:
\[
A=\sum_{k=0}^{n}\int_{c_k}^{c_{k+1}} \big(\text{curva superior} - \text{curva inferior}\big)\,dx,
\]
onde \(c_0=a\) e \(c_{n+1}=b\), e em cada subintervalo \([c_k,c_{k+1}]\) escolhe-se corretamente qual função é a superior.

Uma forma compacta (quando só há duas curvas) é
\[
A=\int_a^b \big|f(x)-g(x)\big|\,dx,
\]
mas, na prática, a forma particionada é a mais segura, pois explicita a troca de posição.

\textbf{Exemplo (sem números):} duas curvas que se cruzam uma vez dentro de \([a,b]\), sendo \(f\) superior em \([a,c]\) e \(g\) superior em \([c,b]\). A área é a soma de duas integrais, uma em cada trecho.

\subsection{Tipo 3: Área em relação ao eixo \texorpdfstring{\(x\)}{x} (curva(s) acima/abaixo do eixo)}

Quando o problema envolve o eixo \(x\) como fronteira (por exemplo, "área entre a curva e o eixo \(x\)"), a mudança de sinal passa a ser essencial. Para uma função contínua \(f\) em \([a,b]\):

\subsubsection*{(a) Curva toda acima do eixo}
Se \(f(x)\ge 0\) em \([a,b]\), então a área entre \(y=f(x)\) e o eixo \(x\) é
\[
A=\int_a^b f(x)\,dx.
\]
\textbf{Exemplo (sem números):} região limitada por \(y=f(x)\), pelo eixo \(x\) e pelas retas \(x=a\) e \(x=b\), com \(f\) sempre positiva.

\subsubsection*{(b) Curva toda abaixo do eixo}
Se \(f(x)\le 0\) em \([a,b]\), a área geométrica é
\[
A=\int_a^b |f(x)|\,dx \;=\; -\int_a^b f(x)\,dx.
\]
\textbf{Exemplo (sem números):} região limitada por \(y=f(x)\), pelo eixo \(x\) e por \(x=a\), \(x=b\), com \(f\) sempre negativa.

\subsubsection*{(c) Curva cruza o eixo (muda de sinal)}
Se existem pontos \(r_1,r_2,\dots,r_m\) em \([a,b]\) tais que \(f(r_j)=0\), então deve-se \textbf{particionar} o intervalo nesses zeros e somar as áreas positivas:
\[
A=\sum_{j=0}^{m}\int_{r_j}^{r_{j+1}} |f(x)|\,dx,
\]
com \(r_0=a\) e \(r_{m+1}=b\).
\textbf{Exemplo (sem números):} uma curva que cruza o eixo \(x\) uma vez: calcula-se a integral em dois trechos e somam-se os módulos.

\subsection{Resumo operacional}
\begin{itemize}
  \item \textbf{Entre duas curvas:} em cada trecho, use \(\text{superior} - \text{inferior}\). Se cruzarem, \textbf{parta} nos pontos de interseção.
  \item \textbf{Com o eixo \(x\):} use \(|f(x)|\) ou parta nos zeros de \(f\) e some áreas positivas.
  \item O eixo \(x\) \textbf{só importa} quando ele é fronteira do problema. \newline Caso contrário, "acima/abaixo do eixo" não altera a fórmula: o critério é sempre \textbf{quem está acima de quem}.
\end{itemize}


\subsection{Curvas que se Cruzam}
Se as funções $f(x)$ e $g(x)$ se cruzam em um ponto $c$, a posição relativa de "teto" e "chão" se inverte. Para calcular a área total, devemos dividir a integral nos pontos de interseção.
\begin{equation}
A = \int_{a}^{c} [f(x) - g(x)] \, dx + \int_{c}^{b} [g(x) - f(x)] \, dx
\end{equation}

\begin{figure}[hbt]
\centering
\begin{tikzpicture}
\begin{axis}[
    xmin=-1, xmax=5,
    ymin=0.5, ymax=5.5,
    axis x line=middle,
    axis y line=middle,
    samples=200,
    xlabel={$x$},
    ylabel={$y$},
    axis line style={-{Stealth[length=3mm,width=1.5mm]}},
    axis on top
]
    \addplot[blue, thick, domain=-1:4.5, name path=curva] {(x-2)^2+1};
    \addplot[red,  thick, domain=-1:4.5, name path=reta ] {x+1};

    \addplot [
        pattern=north east lines,
        pattern color=gray,
    ] fill between [
        of=curva and reta,
        split,
        every segment no 0/.style={transparent},
        every segment no 1/.style={pattern=north east lines},
        every segment no 2/.style={transparent}
    ];
\end{axis}
\end{tikzpicture}
\caption{Área entre curvas que se cruzam.}
\label{Fig_funcoes_cruzadas}
\end{figure}

\begin{exercicio}
  
No caso da área da Figura \ref{Fig_funcoes_cruzadas}, as funções \(f(x)\) e \(g(x)\) são:
\begin{align*}
f(x)&=(x-2)^2+1,\\
g(x)&=x+1.
\end{align*}

Nesse caso, diferentemente do que já foi feito no Exercício \ref{Ex_curvas_cruzadas}o cálculo será feito entre os pontos de interseção das funções, sem incluir parte entre 0 e o primeiro cruzamento das funções. (Esse exercício pode ser revisado porsteriormente.)

\begin{align*}
      f(x)&=g(x)\\
      (x-2)^2+1&=x+1\\
      (x-2)^2&=x\\
      x^2-4x+4&=x\\
      x^2-5x+4&=0\\
      x^2-5x&=-4\\
      x^2-5x+\frac{25}{4}&=-4+\frac{25}{4}\\
      \left(x-\frac{5}{2}\right)^2&=\frac{9}{4}\\
      x-\frac{5}{2}&=\pm\frac{3}{2}\\
      x&=\frac{5}{2}\pm\frac{3}{2}
    \ \Rightarrow\
      \begin{cases}
      x_1=1\\
      x_2=4
      \end{cases}
\quad \text{\LARGE$\checkmark$}
\end{align*}

\begin{figure}[H]
  \centering
  \begin{tikzpicture}
    \begin{axis}[
      xmin=0, xmax=6,
      ymin=0, ymax=6,
      axis x line=middle,
      axis y line=middle,
      axis line style={-{Stealth[length=3mm,width=2mm]}},
      tick align=outside,
      grid=none,
      minor tick num=1,
      xlabel={$x$},
      ylabel={$y$},
      xtick=\empty,
      ytick=\empty,
    ]
    \draw[red,  thick] (axis cs:1,1) -- (axis cs:1,4);
    \draw[red,  thick] (axis cs:1,4) -- (axis cs:4,4);
    \draw[red,  thick] (axis cs:4,4) -- (axis cs:4,1);
    \draw[red,  thick] (axis cs:4,1) -- (axis cs:1,1);

    \draw[blue, thick] (axis cs:1,4) -- (axis cs:1,5);
    \draw[blue, thick] (axis cs:1,5) -- (axis cs:4,5);
    \draw[blue, thick] (axis cs:4,5) -- (axis cs:4,4);

    \draw[black, thick] (axis cs:4,1) -- (axis cs:5,1);
    \draw[black, thick] (axis cs:5,1) -- (axis cs:5,4);
    \draw[black, thick] (axis cs:5,4) -- (axis cs:4,4);

    \draw[black, dashed, thick] (axis cs:5,4) -- (axis cs:5,5);
    \draw[black, dashed, thick] (axis cs:5,5) -- (axis cs:4,5);

    \fill[pattern=north east lines]
      (axis cs:1,1) -- (axis cs:1,4) -- (axis cs:4,4) -- (axis cs:4,1) -- cycle;

    \node at (axis cs:2.6,2.6) {\Large{$x^2$}};
    \node at (axis cs:2.6,5.2) {\Large{$x$}};
    \node at (axis cs:5.2,2.6) {\Large{$x$}};
    \node at (axis cs:4.5,0.5) {\Large{$\frac{5}{2}$}};
    \node at (axis cs:0.5,4.5) {\Large{$\frac{5}{2}$}};
    \end{axis}
  \end{tikzpicture}
  \caption{Completar quadrados.}
\end{figure}

Assim, fica definido que a área entre os pontos de interseção das curvas é calculada por:
\[
  \int_{1}^{4}(x+1)\,dx-\int_{1}^{4}\bigl((x-2)^2+1\bigr)\,dx
\]

\begin{align*}
  &\int_{1}^{4}(x+1)\,dx-\int_{1}^{4}\bigl((x-2)^2+1\bigr)\,dx\\
  &=\int_{1}^{4}x\,dx+\int_{1}^{4}1\,dx-\left(\int_{1}^{4}(x-2)^2\,dx+\int_{1}^{4}1\,dx \right)\\
  &=\int_{1}^{4}x\,dx-\int_{1}^{4}(x-2)^2\,dx\\
  &=\left[\frac{x^2}{2}\right]_{1}^{4}-\left[\frac{x^3}{3}-2x^2+4x\right]_{1}^{4}\\
  &=\left(\frac{16}{2}-\frac{1}{2}\right)-\left(\frac{64}{3}-\frac{1}{3}\right)
  +\left(2\cdot16-2\cdot1\right)-(16-4)\\
  &=\frac{15}{2}-21+18=\frac{9}{2}.
\end{align*}

\begin{figure}[H]
\centering
\begin{tikzpicture}
\begin{axis}[
    xmin=0.5, xmax=4,
    ymin=-3,  ymax=.5,
    axis x line=middle,
    axis y line=middle,
    axis line style={-{Stealth[length=3mm,width=2mm]}},
    samples=200,
    xlabel={$x$},
    ylabel={$y$}
]
    \addplot[blue, thick, domain=1:3.75, samples=200, name path=curva] {-(0.5*(x-2)^2+1)};
    \addplot[red,  thick, domain=1:3.75, samples=200, name path=reta ] {-(.5*x+0.5)};

    \addplot[
        pattern=north east lines,
        pattern color=gray,
    ] fill between[
        of=curva and reta,
        split,
        every segment no 0/.style={transparent},
        every segment no 1/.style={pattern=north east lines},
        every segment no 2/.style={transparent}
    ];
\end{axis}
\end{tikzpicture}
\caption{Área hachurada entre as intersecções.}
\end{figure}

No segundo caso, as funções são:
\begin{align*}
f(x)&=-\bigl(0{,}5(x-2)^2+1\bigr)=-0{,}5x^2+2x-3,\\
g(x)&=-\bigl(0{,}5x+0{,}5\bigr).
\end{align*}

Cálculo dos pontos de interseção:
\begin{align*}
f(x)&=g(x)\\
-0{,}5x^2+2x-3&=-(0{,}5x+0{,}5)\\
-0{,}5x^2+2{,}5x&=2{,}5\\
0{,}5x^2-2{,}5x&=-2{,}5\\
x^2-5x&=-5\\
x^2-5x+\frac{25}{4}&=-5+\frac{25}{4}\\
\left(x-\frac{5}{2}\right)^2&=\frac{5}{4}\\
x&=\frac{5}{2}\pm\sqrt{\frac{5}{4}}=\frac{5\pm\sqrt5}{2}.
\end{align*}

Assim, a área é calculada por:
\begin{align*}
A
&=\int_{x_1}^{x_2}\bigl(f(x)-g(x)\bigr)\,dx\\
&=\int_{\frac{5-\sqrt5}{2}}^{\frac{5+\sqrt5}{2}}
\left[\left(-\frac12x^2+2x-3\right)-\left(-\frac12x-\frac12\right)\right]dx\\
&=\int_{\frac{5-\sqrt5}{2}}^{\frac{5+\sqrt5}{2}}
\left(-\frac12x^2+\frac52x-\frac52\right)dx\\
&=\left[-\frac{x^3}{6}+\frac{5x^2}{4}-\frac{5x}{2}\right]_{\frac{5-\sqrt5}{2}}^{\frac{5+\sqrt5}{2}}\\
&=\frac{5\sqrt5}{12}\approx 0{,}932.
\end{align*}

\subsection*{Resumo Prático}
A fórmula será sempre:
\[
  \int_{\text{esquerda}}^{\text{direita}} (\text{Teto} - \text{Chão}) \,dx
\]

\end{exercicio}

\begin{exercicio}
Calcular a área entre a curva e o eixo \(x\) para o intervalo \([2,5]\).
\[
f(x)=4x-x^2
\]

\begin{figure}[H]
  \centering
      \begin{tikzpicture}
        \begin{axis}[
          xmin=1.5, xmax=5.5,
          ymin=-6,  ymax=5,
          axis x line=middle,
          axis y line=middle,
          axis line style={-{Stealth[length=3mm,width=1.5mm]}},
          xlabel={$x$},
          ylabel={$y$},
          title={$f(x)=4x-x^2$},
          samples=200,
        ]
          \addplot[red, domain=2:5, thick, samples=200] {4*x-x^2};

          % Sombreado acima do eixo (2 a 4)
          \addplot[fill=gray!20, opacity=0.5, domain=2:4] {4*x-x^2} \closedcycle;

          % Sombreado abaixo do eixo (4 a 5)
          \addplot[fill=gray!40, opacity=0.5, domain=4:5] {4*x-x^2} \closedcycle;

          \draw[dashed] (axis cs:2,0) -- (axis cs:2,4);
          \draw[dashed] (axis cs:5,0) -- (axis cs:5,-5);
        \end{axis}
  \end{tikzpicture}
  \caption{Área entre $f(x)$ e o eixo $x$ no intervalo $[2,5]$.}
  \label{area1}
\end{figure}

Como parte da curva fica abaixo do eixo \(x\), separamos a área em dois trechos:
\([2,4]\) (acima do eixo) e \([4,5]\) (abaixo do eixo). Assim:
\[
A=\int_{2}^{4}(4x-x^2)\,dx-\int_{4}^{5}(4x-x^2)\,dx.
\]

Primeira integral:
\begin{align*}
\int_{2}^{4}(4x-x^2)\,dx
&=\Big[2x^2-\frac{x^3}{3}\Big]_{2}^{4}
=\left(32-\frac{64}{3}\right)-\left(8-\frac{8}{3}\right)
=\frac{16}{3}.
\end{align*}

Segunda integral (com sinal de área):
\begin{align*}
-\int_{4}^{5}(4x-x^2)\,dx
&=-\Big[2x^2-\frac{x^3}{3}\Big]_{4}^{5}
=-\left(\left(50-\frac{125}{3}\right)-\left(32-\frac{64}{3}\right)\right)\\
&=-\left(\frac{25}{3}-\frac{32}{3}\right)
=\frac{7}{3}.
\end{align*}

Logo, a área total sob a curva no intervalo \([2,5]\) é:
\[
A=\frac{16}{3}+\frac{7}{3}=\boxed{\frac{23}{3}\,\text{u.a.}}
\]
\end{exercicio}


\begin{exercicio}
  Calcule a área entre as curvas:
  \[
    \begin{cases*}
    g(x)=-x^2-2x\\% em azul
    f(x)=x^2-4 \\% em vermelho
        \end{cases*}
  \]    
As interseções são calculadas igualando-se a duas funções:
\begin{align*}
  &x^2-4=-x^2-2x\\
  &2x^2+2x-4=0\\
  &\underbrace{x^2+\underbrace{x}_{\frac{1}{2}\rightarrow {(\frac{1}{2})^2}}+\frac{1}{4}=2+\frac{1}{4}}_{\text{completando quadrados}}\\
  &x^2+x+\frac{1}{4}=\frac{9}{4}\\
  &\Big(x+\frac{1}{2}\Big)^2=\frac{9}{4}\\
  &\sqrt{\Big(x+\frac{1}{2}\Big)^2}=\pm \sqrt{\frac{9}{4}}\\
  &x=\pm\frac{3}{2}-\frac{1}{2}\\
  & \begin{cases*}
        x'=\frac{3}{2}-\frac{1}{2}=1\\
        x''=-\frac{3}{2}-\frac{1}{2}=-2\\
    \end{cases*}
  \end{align*}

  \begin{figure}[H]
  \centering
    \begin{tikzpicture}
      \begin{axis}
        [xmin=-3, xmax=1.5,
         ymin=-5, ymax=2,
         axis x line=middle,
         axis y line=middle,
         tick align=outside,
         minor tick num=1,
         grid=none,
        axis line style={-{Stealth[length=3mm,width=1.5mm]}},
         ]
         \addplot[red, thick, domain=-2.5:1.5, samples=200, name path=vermelho]{x^2-4};
         \addplot[blue, thick, domain=-2.5:1.5, samples=200, name path=azul]{-x^2-2*x};

          \addplot[fill=gray!20, opacity=0.5, draw=none]
          fill between[of=vermelho and azul, soft clip={domain=-2:1}];
      
      \end{axis}
    \end{tikzpicture}
\end{figure}

Assim, como não há entrecruzamento de curvas, podemos integrar direto tendo como limites de integração os pontos de interseção.

\begin{align*}
  &\int_{-2}^{1}(g(x)-f(x))\, dx\\
  &=\int_{-2}^{1}(-x^2-2x)-(x^2-4)\, dx\\
  &=\int_{-2}^{1}(-2x^2-2x+4)\, dx\\
  &=\Big[-2\frac{x^3}{3}-x^2+4x\Big]_{-2}^1\\
  &=\Big(-\frac{2}{3}-1+4\Big)-\Big(\frac{16}{3}-4-8\Big)\\
  &=\Big(-\frac{2}{3}-\frac{3}{3}+\frac{12}{3}\Big)-\Big(\frac{16}{3}-\frac{12}{3}-\frac{24}{3}\Big)\\
  &=\Big(\frac{7}{3}\Big)-\Big(\frac{20}{3}\Big)=\frac{27}{3}=\boxed{9\,\text{u.a.}}\\
\end{align*}
\end{exercicio}





