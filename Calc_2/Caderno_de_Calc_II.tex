%!TEX root = Caderno_de_Calc_II.tex
%!TEX program = lualatex
% LTeX: language=pt-BR
% \tikzexternalize[prefix=externalized/]

\documentclass[a4paper,12pt]{article}

% ===== Básico =====
%\usepackage[T1]{fontenc}
% \usepackage[utf8]{inputenc}    % (REMOVER no LuaLaTeX)
\usepackage[portuguese]{babel}
\usepackage[margin=3cm,tmargin=3cm,rmargin=2cm,bmargin=2cm]{geometry}
\usepackage{indentfirst}
\usepackage{setspace}
\usepackage{enumitem}
\usepackage{booktabs}
\usepackage{textcomp} 
\usepackage[official]{eurosym}
\usepackage{adjustbox}
\usepackage{lipsum}

% ===== Cores =====
\usepackage[dvipsnames]{xcolor}
% Define a cor azul escuro (DarkBlue ou customizado)
\definecolor{azulmarinho}{RGB}{0, 0, 139} % Opção 1: Customizado

% ===== Matemática / Fontes =====
\usepackage{lmodern}
\usepackage{amsmath,amsthm,amsfonts,amssymb,dsfont,mathtools}
\usepackage{cancel}
\usepackage{cases}

\renewcommand{\CancelColor}{\color{red}}
\newtheorem{theorem}{Teorema}
\newtheorem*{definicao}{Definição}
\newtheorem{example}{Exemplo}[section]

\newtheorem*{classificacao}{Classificação de pontos críticos}


\newcommand{\dif}{\,\mathrm{d}}
\newcommand{\sen}{\operatorname{sen}}

% ===== Figuras / Gráficos =====
\usepackage{graphicx}
\graphicspath{{Images/}{./}}
\usepackage{float}
\usepackage{tabularx}

%\usepackage{showframe}

% ===== PGFPlots/TikZ =====
\usepackage{tikz}
\usepackage{pgfplots}
\usetikzlibrary{angles,quotes}
\usetikzlibrary{external, arrows.meta,intersections,patterns,calc,positioning,shapes.geometric}

\pgfplotsset{compat=1.18}
\usepgfplotslibrary{fillbetween}

% ===== PGFPlots/TikZ =====
\usepackage{tikz}
\usepackage{pgfplots}
\pgfplotsset{compat=1.18}

% --- TikZ externalization (cache) ---
\usetikzlibrary{external}
\tikzexternalize[
  prefix=externalized/,
  up to date check=diff
]

% (opcional, mas eu recomendo no Windows)
\tikzset{external/system call={
  lualatex -shell-escape -halt-on-error -interaction=batchmode -jobname "\image" "\texsource"
}}


% Chamada externa explícita (evita variações do ambiente)
\tikzset{external/system call={
  lualatex -shell-escape -halt-on-error -interaction=batchmode -jobname "\image" "\texsource"
}}


% ===== Estilos para gráficos de área =====
\newcommand{\CorCurvaSup}{red}          
\newcommand{\CorCurvaInf}{blue}         
\newcommand{\CorSombra}{fill=gray!20}   
\newcommand{\MarcarAB}[4]{%
  \draw[line width=0.6pt] (axis cs:#1,#3) -- (axis cs:#1,#4);
  \draw[line width=0.6pt] (axis cs:#2,#3) -- (axis cs:#2,#4);
  \node[below] at (axis cs:#1,#3) {$a$};
  \node[below] at (axis cs:#2,#3) {$b$};
}
\pgfplotsset{
  AreaAxisStyle/.style={
    axis x line=middle,
    axis y line=middle,
    axis line style={->},
    tick align=outside,
    minor tick num=1,
    grid=none,
    clip=false,
    xlabel={$x$},
    ylabel={$y$},
  }
}

% ===== Hyperref por último =====
\usepackage{hyperref}
\hypersetup{hypertexnames=false}
\usepackage{bookmark}


\title{\textbf{Notas e Exercícios}}
\author{Jose Mauro Xavier Elsas}
\date{outubro de 2025}

\newcommand{\leftdisplay}[1]{%
  \[
  \makebox[\linewidth][l]{%
    \begin{aligned}
      #1
    \end{aligned}}
  \]%
}
\theoremstyle{definition}
\newtheorem{definition}{Definição}[section]
\newtheorem{exercicio}{Exercício}[subsection]

% ===== MACROS GLOBAIS =====
% uso: \Setor[<estilo>]{(x,y)}{raio}{ang_ini}{ang_fim}
\newcommand{\Setor}[5][]{%
  \begin{scope}
    \path[#1] #2 -- ++(#4:#3) arc[start angle=#4, end angle=#5, radius=#3] -- cycle;
  \end{scope}%
}

\begin{document}
    
% ----------------------------------------------------------
% PÁGINA DE ROSTO
% ----------------------------------------------------------
\begin{titlepage}
  \centering
  {\Large\textbf{Universidade do Estado do Rio de Janeiro}}\\[0.2cm]
  {\large Instituto de Oceanografia}\\[1.5cm]

  {\huge\bfseries CÁLCULO II}\\[0.2cm]
  {\Large Notas, Exercícios e Aplicações}\\[2cm]

  \vfill
  \textbf{J. Mauro Xavier Elsas}\\[0.3cm]
\end{titlepage}

% ----------------------------------------------------------
% SUMÁRIO AUTOMÁTICO
% ----------------------------------------------------------
\tableofcontents
\newpage



\onehalfspacing
\setlist{nosep}
\setlength{\parindent}{0pt}

\section*{Ementa de Cálculo II}
\begin{enumerate}[label=\arabic*.]
  \item Integral Definida: Aplicações
    \begin{enumerate}[label=\arabic{enumi}.\arabic*.]
      \item Cálculo de área sob uma curva
      \item Integral de Riemann
      \item Teorema fundamental do cálculo
      \item Área entre duas curvas
      \item Volume de sólido de revolução
      \item Comprimento de arco
      \item Integral imprópria
    \end{enumerate}

  \item Derivadas Parciais
    \begin{enumerate}[label=\arabic{enumi}.\arabic*.]
      \item Funções de duas ou mais variáveis, limite, continuidade
      \item Curva de nível
      \item Derivada direcional e gradiente
      \item Plano tangente
      \item A "Regra da Cadeia"
      \item Diferencial total
      \item Derivadas de ordem superior
      \item Máximos e Mínimos; Multiplicadores de Lagrange
      \item Integrais Múltiplas
      \item Integral dupla; Área
      \item Coordenadas polares
    \end{enumerate}
\end{enumerate}

\newpage
\section{Integral Definida: Aplicações}
\section{Cálculo de área sob uma curva}
% conteudo/teoria/area_sob_a_curva.tex
% Interpretação geométrica da área sob o gráfico de uma função

Seja $f:[a,b] \to \mathbb{R}$ uma função definida em um intervalo fechado e limitado, tal que:
\[
f(x) \ge 0 \quad \text{para todo } x \in [a,b].
\]

Nessas condições, pode-se associar à função $f$ uma região do plano limitada pelo gráfico de $y=f(x)$, pelo eixo $x$ e pelas retas verticais $x=a$ e $x=b$. A medida dessa região é chamada de \textbf{área sob a curva} de $f$ no intervalo $[a,b]$.

\begin{figure}[H]
\centering
\begin{tikzpicture}[x=1cm,y=1cm,>=Latex]

  % --- parâmetros (ajuste aqui) ---
  \def\a{2.2}   % posição de a
  \def\b{7.6}   % posição de b

  % --- moldura ---
  \draw[line width=0.6pt] (-0.6,-0.6) rectangle (9.5,4.8);

  % --- eixos com setas ---
  \draw[->, line width=0.8pt] (0,0) -- (8.8,0) node[below right] {$x$};
  \draw[->, line width=0.8pt] (0,0) -- (0,4.2) node[above] {$y$};
  
  % --- função (curva com oscilações suaves) ---
  \def\fx(#1){ 2.55 + 0.95*sin(20*#1) - 0.45*cos(70*#1) - 0.08*(#1-5.0) }

  % --- região sombreada S entre a e b ---
  \fill[gray!25]
    (\a,0)
    -- plot[domain=\a:\b, samples=200] (\x,{ \fx(\x) })
    -- (\b,0) -- cycle;

  % --- linhas verticais em a e b (bordas da região) ---
  \draw[line width=0.6pt] (\a,0) -- (\a,{ \fx(\a) });
  \draw[line width=0.6pt] (\b,0) -- (\b,{ \fx(\b) });

  % --- curva completa (um pouco antes e depois) ---
  \draw[line width=0.9pt]
    plot[domain=0.6:8.6, samples=240] (\x,{ \fx(\x) });

  % --- rótulos ---
  \node at ({0.5*(\a+\b)},1.5) {$\textbf{S}$};
  \node[above right] at (8.15,{ \fx(8.15) }) {$f(x)$};

  % marcações a e b no eixo x
  \node[below] at (\a,0) {$a$};
  \node[below] at (\b,0) {$b$};

\end{tikzpicture}
\caption{Região $S$ sob o gráfico de $f(x)$ entre $a$ e $b$.}
\end{figure}

Do ponto de vista geométrico, essa área é o objeto que se deseja calcular. Entretanto, não existe, em geral, uma fórmula elementar direta para essa medida. Por isso, recorre-se a métodos de aproximação.

\subsection{Aproximação por retângulos}

Para estimar a área sob a curva, divide-se o intervalo $[a,b]$ em subintervalos:
\[
a=x_0 < x_1 < \cdots < x_n=b.
\]

Em cada subintervalo $[x_{i-1}, x_i]$, constrói-se um retângulo de base:
\[
\Delta x_i = x_i - x_{i-1}
\]
e altura dada por um valor da função nesse respectivo subintervalo.

A soma das áreas desses retângulos fornece uma aproximação da área total. Quanto maior for o número de retângulos (ou menor a largura dos subintervalos), melhor tende a ser essa aproximação.

\subsection{Área e sinal da função}

A interpretação geométrica apresentada só é válida quando $f(x) \ge 0$. Quando a função assume valores negativos, a região correspondente situa-se abaixo do eixo $x$, e a noção de área geométrica deve ser tratada com cuidado, pois a integral pode resultar em valores negativos.

Nesse caso, a ferramenta matemática adequada para tratar o problema de forma sistemática é a integral definida, que será formalmente introduzida na próxima seção.



\subsection*{Como identificar integrais resolvidas por mudança de variável}

A técnica de \textbf{mudança de variável} (substituição) é usada quando o integrando tem a estrutura:

\[
f'(x)\, g(f(x))
\]

ou seja, aparece uma função composta e, multiplicando-a, algo proporcional à sua derivada.

\subsubsection*{1. Interpretação conceitual (Regra da cadeia ao contrário)}

Sabemos que:

\[
\frac{d}{dx} F(f(x)) = F'(f(x))\, f'(x)
\]

Logo, se a integral tem a forma

\[
f'(x)\, F'(f(x)),
\]

ela pode ser vista como a regra da cadeia ``ao contrário''.

\subsubsection*{2. Exemplo motivador}

Considere:

\[
\int \sin(2x)\, e^{\sin^2 x}\, dx
\]

Observe que:

\[
\frac{d}{dx}(\sin^2 x)=2\sin x \cos x=\sin(2x).
\]

Portanto, fazendo:

\[
u=\sin^2 x \quad \Rightarrow \quad du=\sin(2x)\,dx,
\]

a integral se transforma em:

\[
\int e^u\,du.
\]

\subsubsection*{3. Checklist prático}

Ao analisar uma integral, pergunte:

\begin{enumerate}
    \item Existe uma função dentro de outra (composição)?
    \item A derivada da função interna aparece multiplicando?
    \item A expressão lembra a regra da cadeia invertida?
\end{enumerate}

Se a resposta for positiva, tente a substituição \(u=f(x)\).

\subsubsection*{4. Casos típicos de substituição}

\begin{itemize}
    \item Exponencial composta:
    \[
    \int e^{x^2} 2x\,dx
    \]

    \item Potência composta:
    \[
    \int (3x^2+1)^5 \cdot 6x\,dx
    \]

    \item Função trigonométrica composta:
    \[
    \int \cos(5x)\,dx
    \]

    \item Expressões com raiz:
    \[
    \int \frac{x}{\sqrt{1+x^2}}\,dx
    \]
\end{itemize}

\subsubsection*{5. Ajuste por constante}

Às vezes a derivada da parte interna não aparece exatamente, mas pode ser ajustada.

Exemplo:

\[
\int x e^{x^2}\,dx
\]

Observe que:

\[
\frac{d}{dx}(x^2)=2x.
\]

Então:

\[
\int x e^{x^2}\,dx
=
\frac{1}{2} \int 2x e^{x^2}\,dx.
\]

Agora a substituição fica imediata.

\subsubsection*{6. Regra prática para decisão}

Em exercícios de Cálculo:

\begin{itemize}
    \item Se há composição evidente → tente substituição.
    \item Se há produto polinômio × exponencial/trigonométrica → teste substituição.
    \item Se há raiz envolvendo expressão algébrica → teste substituição.
    \item Se a substituição não funcionar → considere integração por partes.
\end{itemize}

\subsubsection*{7. Regra de ouro}

Sempre que aparecer:

\[
\text{função complicada dentro de outra}
\]

ou

\[
(\text{expressão})^{n}
\quad \text{ou} \quad
e^{(\text{expressão})},
\]

procure imediatamente a derivada da parte interna.  
Se ela estiver presente (mesmo que multiplicada por constante), a técnica adequada é mudança de variável.





\section{Integral de Riemann}\label{subsec:integral-riemann}
% conteudo/teoria/integral_riemann.tex
% Definição formal da integral de Riemann

A integral de Riemann é um conceito matemático que formaliza o processo de
passagem ao limite das somas usadas para aproximar áreas.
Ela independe de interpretações geométricas sendo definida de forma puramente analítica.

\subsubsection{Partições e somas de Riemann}

A área do \textbf{círculo} (disco) pode ser aproximada particionando-se o disco em \(n\) triângulos,
cujos vértices estão no centro \(O\) e cujas bases são \textbf{cordas} consecutivas da
\textbf{circunferência}. Essa construção equivale a considerar um \textbf{polígono inscrito}
de \(n\) lados e decompor sua área em \(n\) triângulos.


À medida que tomamos mais triângulos (isto é, quando \(n\) aumenta), as cordas passam a aproximar
pequenos arcos da circunferência e a soma das áreas dos triângulos aproxima a área do círculo.

\medskip

Em cada triângulo, a \textbf{apótema} (segmento perpendicular do centro à base) coincide com a
\textbf{altura} do triângulo. Assim, a área do \(i\)-ésimo triângulo é
\[
A_i = \frac{b_i \, h_i}{2},
\]
em que \(b_i\) é o comprimento da corda (base) e \(h_i\) é a apótema correspondente.
Somando as áreas dos \(n\) triângulos, obtemos a aproximação
\[
A_n = \sum_{i=1}^{n} \frac{b_i \, h_i}{2}.
\]

\medskip

Quando \(n\) é grande, as apótemas \(h_i\) tornam-se cada vez mais próximas do raio \(r\).
Além disso, a soma das bases \(\sum_{i=1}^{n} b_i\) é o \textbf{perímetro} do polígono inscrito,
que tende ao comprimento da circunferência. Logo,
\[
A_n \approx \frac{1}{2} \left( \sum_{i=1}^{n} b_i \right) r
\quad\longrightarrow\quad
\frac{1}{2} \, (2\pi r) \, r = \pi r^2.
\]
Portanto, a área do círculo é
\[
A = \pi r^2.
\]

\medskip

\noindent\textbf{Nota (opção sem índices).}
Se tomarmos os \(n\) triângulos como \textbf{congruentes} (mesma base \(b\) e mesma apótema \(h\)),
então
\[
A_n = n \cdot \frac{b \, h}{2}.
\]
Nesse caso, \(n \cdot b\) é o perímetro do polígono inscrito e, quando \(n \to \infty\), temos
\(n b \to 2\pi r\) e \(h \to r\), concluindo novamente \(A = \pi r^2\).

% ===== DESENHO DO CÍRCULO =====
\begin{figure}[H]
  \centering
  \begin{tikzpicture}
    \def\R{3}
    \coordinate (O) at (0,0);

    % círculo (circunferência)
    \draw[thick] (O) circle (\R);

    % pontos nas extremidades dos raios
    \coordinate (A) at (60:\R);
    \coordinate (B) at (90:\R);
    \coordinate (C) at (120:\R);
    \coordinate (D) at (150:\R);

    % raios
    \draw[thick] (O)--(A);
    \draw[thick] (O)--(B);
    \draw[thick] (O)--(C);
    \draw[thick] (O)--(D);

    % cordas (lados do polígono inscrito)
    \draw[thick] (A)--(B)--(C)--(D);

    % marcar e rotular A,B,C,D
    \fill (A) circle (1.2pt) node[above right] {$A$};
    \fill (B) circle (1.2pt) node[above]       {$B$};
    \fill (C) circle (1.2pt) node[above left]  {$C$};
    \fill (D) circle (1.2pt) node[left]        {$D$};

    % --- apótemas (alturas) ---
    \coordinate (HAB) at ($(A)!(O)!(B)$);
    \coordinate (HBC) at ($(B)!(O)!(C)$);
    \coordinate (HCD) at ($(C)!(O)!(D)$);

    % hachura de um dos triângulos
    \fill[pattern=north east lines, pattern color=blue, opacity=0.3] (O)--(A)--(B)--cycle;

    % desenhar as apótemas
    \draw[thick, blue] (O)--(HAB);
    \draw[thick, blue] (O)--(HBC);
    \draw[thick, blue] (O)--(HCD);

    % marcar os pés e rotular
    \fill (HAB) circle (1.0pt) node[below right, xshift=-4pt, yshift=15pt] {$h_{AB}$};
    \fill (HBC) circle (1.0pt) node[below right, xshift=-15pt, yshift=20pt] {$h_{BC}$};
    \fill (HCD) circle (1.0pt) node[below right, xshift=-22pt, yshift=20pt] {$h_{CD}$};

  \end{tikzpicture}
  \caption{Triangulação por cordas e apótemas: a soma das áreas aproxima a área do círculo quando \(n\) cresce.}
\end{figure}

Se a mesma ideia for usada para uma \(f(x)\) no intervalo \([a,b]\), temos:


Dada uma partição $P$ do intervalo $[a,b]$ em $n$ subintervalos $[x_{i-1}, x_i]$, definimos a largura de cada intervalo como
\[
\Delta x_i = x_i - x_{i-1}.
\]
Para cada subintervalo, escolhe-se um ponto amostral $C_i \in [x_{i-1}, x_i]$. A soma
\[
\sum_{i=1}^{n} f(C_i) \, \Delta x_i
\]
é chamada de \textbf{soma de Riemann} associada à partição $P$.

Define-se ainda a \textbf{norma da partição} por
\[
\|P\| = \max_{1 \le i \le n} \Delta x_i.
\]
\begin{figure}[H]
  \centering
  \begin{tikzpicture}
    \begin{axis}[
      xmin=0.3, xmax=2.25,
      ymin=0, ymax=3.25,
      axis x line=middle,
      axis y line=middle,
      axis line style={->},
      xlabel={$x$},
      ylabel={$y$},
      title={Soma de Riemann},
      clip=false,
      xticklabels={}
    ]
      % curva
      \addplot[red, thick, domain=0.5:2, samples=200] {2*x - x^2 + 2};

      % retângulos da soma
      \draw[dash pattern=on 5pt off 3pt, line width=0.6pt]  (axis cs:0.5,0) -- (axis cs:0.5,{2*0.75 - 0.75^2 + 2});
      \draw[blue, line width=0.6pt] (axis cs:0.75,0) -- (axis cs:0.75,{2*0.75 - 0.75^2 + 2});
      \draw[line width=0.6pt] (axis cs:1,0)    -- (axis cs:1,{2*0.75 - 0.75^2 + 2});
      \draw[blue, line width=0.6pt] (axis cs:1.25,0) -- (axis cs:1.25,{2*1.25 - 1.25^2 + 2});
      \draw[line width=0.6pt] (axis cs:1.5,0)  -- (axis cs:1.5,{2*1.25 - 1.25^2 + 2});
      \draw[blue, line width=0.6pt] (axis cs:1.75,0) -- (axis cs:1.75,{2*1.75 - 1.75^2 + 2});
      \draw[dash pattern=on 5pt off 3pt, line width=0.6pt]  (axis cs:2,0) -- (axis cs:2,{2*1.75- 1.75^2 + 2});
      % linhas de fechamento do topo
      \draw[blue, line width=0.6] (axis cs:0.5,2*0.75-0.75^2+2) -- (1,2*0.75-0.75^2+2);
      \draw[blue, line width=0.6] (axis cs:1,2*1.25-1.25^2+2) -- (1.5,2*1.25-1.25^2+2);
      \draw[blue, line width=0.6] (axis cs:1.5,2*1.75-1.75^2+2) -- (2,2*1.75-1.75^2+2);

      % marcas no eixo x
      \node[below, text=red] at (axis cs:0.5,-0.11) {$a=x_0$};
      \node[below, text=black] at (axis cs:0.75,-0.1) {$C_1$};
      \node[below, text=black] at (axis cs:0.75,-0.5) {$\underbrace{\hspace{4em}}_{\Delta_1}$};
      \node[below, text=red] at (axis cs:1,-0.1) {$x_1$};
      \node[below, text=black] at (axis cs:1.25,-0.1) {$C_2$};
      \node[below, text=black] at (axis cs:1.25,-0.5) {$\underbrace{\hspace{4em}}_{\Delta_2}$};
      \node[below, text=red] at (axis cs:1.5,-0.1) {$x_2$};
      \node[below, text=black] at (axis cs:1.75,-0.1) {$C_3$};
      \node[below, text=black] at (axis cs:1.75,-0.5) {$\underbrace{\hspace{4em}}_{\Delta_3}$};
      \node[below, text=red] at (axis cs:2,-0.1) {$b=x_n$};
          \end{axis}
  \end{tikzpicture}
  \caption{Aproximação da área sob $f(x)=2x-x^2+2$ por retângulos.}
\end{figure}
\subsubsection{Definição de integral de Riemann}

Dizemos que a função $f$ é \textbf{integrável no sentido de Riemann} em $[a,b]$
se existe o limite
\[
\lim_{\|P\| \to 0} \sum_{i=1}^{n} f(C_i) \, \Delta x_i,
\]
e esse limite é o mesmo para qualquer escolha dos pontos $C_i$ nos subintervalos.
Quando esse limite existe, ele é chamado de \textbf{integral definida de Riemann}
de $f$ em \([a,b]\) e é denotado por
\[
\int_a^b f(x) \, dx.
\]

\subsubsection{Condições suficientes de integrabilidade}

\begin{theorem}[Condições Suficientes de Integrabilidade]
Seja $f:[a,b] \to \mathbb{R}$ uma função definida em um intervalo fechado e limitado.
Então $f$ é integrável no sentido de Riemann em $[a,b]$ se satisfaz pelo menos uma
das condições a seguir:
\begin{itemize}
  \item $f$ é contínua em $[a,b]$;
  \item $f$ é monótona em $[a,b]$;
  \item $f$ é limitada em $[a,b]$ e possui somente um número finito de pontos de
  descontinuidade.
\end{itemize}
\end{theorem}

\newpage
\section{Teorema Fundamental do Cálculo}
Seja \(f[a,b]\longrightarrow\mathbb{R}\) contínua, F é a prmitiva de f.
\[
  \int_{a}^{b}\,f(x)\,dx=F\big|_a^b=F(b)-F(a)
\]

\begin{definition}
Seja $P$ uma partição de $[a,b]$ que subdivide o intervalo em $n$ subintervalos

$[x_{i-1},x_i]$ de mesmo comprimento 
\[
  \Delta x = \frac{b-a}{n}.
\]
\end{definition}

\begin{theorem} (TMV -- Teorema do Valor Médio.)\label{TVM}
  Se $f$ é contínua no intervalo fechado $[a,b]$ e derivável no intervalo aberto $(a, b)$, então existe pelo menos um número $c \in (a, b)$ tal que:
\[
f'(c) = \frac{f(b) - f(a)}{b - a}
\]
\end{theorem}

Aplicando o TVM \ref{TVM} acima à função \(F\, \text{em}\, (x_{i-1}, x_{i})\exists \quad c_i \quad]\, x_{x-1,x_i \,[}\) tal que
\[
  \sum_{i=1}^{n}\,F_{x_i}-F_{(x-1,x_i)}=\sum_{i=1}^{n}f(c_i)\Delta x = F(b)-F(a)=\int_{a}^{b}f(x)\,dx
\]

Mais uma vez, seja \(f[a,b]\longrightarrow\mathbb{R}\) contínua, F é a prmitiva de f.
\[
  \int_{a}^{x}\,f(t)\,dt\,\text{é primitiva de}\,f
\]
ou seja, 
\[
  \frac{dF(x)}{dx}=\frac{d}{dx}\,\int_{a}^{x}f(t)\,dt=f(x)
\]
\begin{figure}[H]
  \centering
  \begin{tikzpicture}
    \begin{axis}[
      clip=false,
      axis x line=middle,
      axis y line=middle,
      axis line style={-{Stealth[length=3mm,width=1.5mm]}},
      xmin=0, xmax=2*pi,
      ymin=-0.5, ymax=1.2, % <- se quiser zoom, use ymin=-0.05 em vez de 0.4
      grid=both,
      xtick=\empty,
      %{0,pi/2,pi,3*pi/2,2*pi}
      xticklabels=\empty, 
      % {$0$,$\frac{\pi}{2}$,$\pi$,$\frac{3\pi}{2}$,$2\pi$}
    ]

      % Curva
      \addplot[
        thick, blue, samples=200,
        domain=pi/2:3*pi/2,
        name path=curva
      ] {sin(deg(0.5*x))};

      % Caminho do eixo x (y=0) para o fillbetween
      \path[name path=xaxis] (axis cs:{pi/2},0) -- (axis cs:{3*pi/2},0);

      % Linhas verticais pontilhadas
      \draw[blue, thick, dashed] (axis cs:{pi/2},0) -- (axis cs:{pi/2},{sqrt(2)/2});
      \draw[blue, thick, dashed] (axis cs:{pi},0)   -- (axis cs:{pi},1);
      \draw[blue, thick, dashed] (axis cs:{3*pi/2},0) -- (axis cs:{3*pi/2},{sqrt(2)/2});

      % Área preenchida só de pi/2 até pi
      \addplot[fill=gray!35, draw=none, opacity=.5]
        fill between[of=curva and xaxis, soft clip={domain=pi/2:pi}];
      \node[anchor=north, yshift=-2pt] at (axis cs:{pi/2},0) {$a$};
      \node[anchor=north, yshift=-2pt] at (axis cs:{pi},0) {$b$};
      \node[anchor=north, yshift=-2pt] at (axis cs:{3*pi/2},0) {$C$};
    \end{axis}
  \end{tikzpicture}
\end{figure}

\subsubsection{Exemplos clássicos}
\begin{itemize}
  \item \(f(x)=x\) em \([0,1]\): contínua \(\Rightarrow\) integrável e 
  \[
  \int_0^1 x\,dx = \frac{x^2}{2}\Bigg|_0^1=\frac{1}{2}
  \].
  \item \textbf{Dirichlet:} \(f(x)=1\) se \(x\) é racional e \(0\) se irracional em \([0,1]\). Não é integrável no sentido de Riemann.
  \item \textbf{Thomae ("pipoca"):} 
  \(f(x)=\begin{cases}1/q, & x=p/q
    \ (\gcd(p,q)=1)\\[2pt] 0, & x\ \text{irracional}\end{cases}\).

    É integrável no sentido de Riemann e 
    \[
      \int_0^1 f\,dx = 0.
    \]
\end{itemize}

\subsubsection{Exemplo por somas superior/inferior}
Para \(f(x)=x\) em \([0,1]\), use a partição uniforme \(x_i=i/n\).
Temos \(m_i=x_{i-1}=\tfrac{i-1}{n}\) e \(M_i=x_i=\tfrac{i}{n}\). Logo,
\[
L(P_n,f)=\sum_{i=1}^n \frac{i-1}{n}\cdot\frac{1}{n}
=\frac{1}{n^2}\sum_{k=0}^{n-1}k
=\frac{n-1}{2n},
\]
\[
U(P_n,f)=\sum_{i=1}^n \frac{i}{n}\cdot\frac{1}{n}
=\frac{1}{n^2}\sum_{i=1}^{n}i
=\frac{n+1}{2n}.
\]
Como \(L(P_n,f), U(P_n,f) \to \tfrac{1}{2}\), conclui-se que \(\displaystyle 
\int_0^1 x\,dx = \tfrac{1}{2}\).

\subsubsection{Relação do TFC com as integrais definidas}
Se \(F\) é uma primitiva de \(f\) (\(F'=f\)), então:
\[
\int_a^b f(x)\,dx = F(b)-F(a).
\]

\subsubsection{Integrais impróprias (observação)}
Se \(f\) não é limitada em \([a,b]\) ou se o intervalo é infinito, define-se por limites:
\[
\int_a^b f(x)\,dx
=\lim_{t\to b^-}\int_a^t f(x)\,dx,\qquad
\int_a^{+\infty} f(x)\,dx
=\lim_{t\to +\infty}\int_a^{t} f(x)\,dx.
\]

\subsubsection{Propriedades úteis}
Para funções integráveis \(f, g\) e escalares \(\alpha, \beta\):
\begin{align*}
&\int_a^b (\alpha f + \beta g)\,dx = \alpha\int_a^b f\,dx + \beta\int_a^b g\,dx,\\
&\int_a^b f\,dx = \int_a^c f\,dx + \int_c^b f\,dx,\\
&\int_a^b f\,dx = -\int_b^a f\,dx.
\end{align*}

\[
f: [a,b] \to \mathbb{R}, \quad f \geq 0
\]

\begin{figure}[H]
  \centering
  \begin{tikzpicture}
    \begin{axis}[
      xmin=0.5, xmax=3,
      ymin=0.6, ymax=1.1,
      axis x line=middle,
      axis y line=middle,
      axis line style={-{Stealth[length=3mm,width=1.5mm]}},
      tick align=outside,
      minor tick num=3,
      xlabel={$x$}, 
      ylabel={$y$},
      xlabel style={at={(axis description cs:1,-0.08)},anchor=north},
      title={$f(x) = 0,9 \operatorname{sen}(0,8x)$}
    ]
      % Intervalo de sombreamento
      \def\a{1}
      \def\b{2.5}

      \addplot[name path=curva, domain=\a:\b, samples=250, thick]
        {0.9*sin(deg(0.8*x))};
      \addplot[name path=base, domain=\a:\b, samples=2] {0};

      \addplot[fill=gray!20, draw=none, opacity=.5]
        fill between[of=curva and base];

      \draw[line width=0.6pt] (axis cs:\a,0) -- (axis cs:\a,{0.9*sin(deg(0.8*\a))});
      \draw[line width=0.6pt] (axis cs:\b,0) -- (axis cs:\b,{0.9*sin(deg(0.8*\b))});
    \end{axis}
  \end{tikzpicture}
  \caption{Área sob o gráfico.}
\end{figure}

Se, por hipótese, deseja-se calcular a área sob a curva, calcula-se a integral da função entre os limites de integração $[a,b]$, ou seja, a integral de Riemann.
\medskip
 
A área de \(S\) é:
\[
\int_a^b f(x)\,dx = \lim_{\|\Delta x\| \to 0} \sum_{i=1}^n f(c_i )(x_i - x_{i-1}).
\]

Seja \(f:[a,b] \to \mathbb{R}\) contínua. Se \(F\) é uma primitiva de \(f\), então:
\[
\int_{a}^{b}f(x)\,dx = F(x)\Big|_{a}^{b} = F(b)-F(a).
\]

\begin{exercicio}\label{ex:int_x2_0_5}
Calcule a integral definida:
\begin{align*}
\int_{0}^{5}x^2\,dx
&= \frac{x^3}{3}\bigg|_{0}^{5}\\
&= \frac{5^3}{3} - \frac{0^3}{3}\\
&= \boxed{\frac{125}{3}\,u.a.}
\end{align*}
\end{exercicio}

\begin{exercicio}\label{ex:int_e_-1_1_}
Calcule a integral definida:
\begin{align*}
 &\int_{-1}^{1}e^x\,dx = e^x\bigg|_{-1}^{1}\\
 &= e^1 - e^{-1}\\
 &= \boxed{e - \frac{1}{e}}\,u.a.
\end{align*}
\end{exercicio}

\begin{exercicio}\label{int_x2_-2a}
  Calcule a integral definida:
 \begin{align*}
    &\int_{0}^{2}(x^2 - 2x)\,dx\\
    &= \int_{0}^{2}x^2\,dx - 2\int_{0}^{2}x\,dx \\
    &= \frac{x^3}{3}\bigg|_{0}^{2} - x^2\bigg|_{0}^{2} \\
    &= \frac{8}{3} - 4 = \boxed{-\frac{4}{3}}\,u.a.
\end{align*}
\end{exercicio}

\begin{exercicio}\label{int_4x_x2}
  Calcule a Intedgral definida:
  \begin{align*}
    & \int_{2}^{4}(4x-x^2)  \,dx \\
    &=\Big[2x^2-\frac{x^3}{3}\Big]_2^4\\
    &=(2\cdot 4^2-\frac{4^3}{3})-(2\cdot 2^2-\frac{2^3}{3})\\
    &=\Big(32-\frac{64}{3}\Big)-\Big(8-\frac{8}{3}\Big)\\
    &=\frac{32}{3}-\frac{16}{3}= \boxed{\frac{16}{3}}
  \end{align*}
\end{exercicio}
    
\begin{figure}[H]
  \centering
  \begin{tikzpicture}
    \begin{axis}[
      xmin=-0.5, xmax=3,
      ymin=-1.5, ymax=2,
      axis x line=middle,
      axis y line=middle,
      axis line style={-{Stealth[length=3mm,width=1.5mm]}},
      xlabel={$x$},
      ylabel={$y$},
    ]
      \addplot[red, domain=-0.5:2.5, samples=200, thick] {x^2-2*x};
      \addplot[domain=0:2, samples=200, draw=none, fill=gray!20, opacity=.35] {x^2-2*x} \closedcycle;
    \end{axis}
  \end{tikzpicture}
  \caption{Função $f(x)=x^2-2x$ e a área (abaixo do eixo $x$).}
\end{figure}

\subsubsection{O Teorema Fundamental do Cálculo (Parte 1)}

O TFC estabelece que a derivada de uma função definida por uma integral é o próprio integrando avaliado no limite de integração superior. Quando o limite superior é uma função $g(x)$, aplicamos a Regra da Cadeia:

\[ 
\frac{d}{dx} \int_{a}^{g(x)} f(t) \, dt = f(g(x)) \cdot g'(x)
\]

\begin{exercicio}
      Encontre a derivada da função:
      \[
        F(x) = \int_{3}^{x^2} \ln(t + 1) \, dt.
      \]
      \medskip
      \begin{enumerate}[label=\alph*.]
        \item Identificamos o integrando $f(t) = \ln(t + 1)$ e o limite superior $g(x) = x^2$.
        \item Substituímos $t$ por $g(x)$ na função:
        $$f(g(x)) = \ln(x^2 + 1)$$
        \item Multiplicamos pela derivada de $g(x)$, sendo 
              \begin{align*}
                &g'(x) = 2x:\\
                &F'(x) = \ln(x^2 + 1) \cdot 2x\\
          \end{align*}
      \end{enumerate}
      \[
      F'(x) = 2x \ln(x^2 + 1)
      \]
\end{exercicio}

\subsubsection{Caso Geral -- Ambos os limites são funções}

Se a função possui variáveis em ambos os limites de integração:
\[
  F(x) = \int_{h(x)}^{g(x)} f(t) \, dt
\]
A derivada é dada por:
  \[
    F'(x) = f(g(x)) \cdot g'(x) - f(h(x)) \cdot h'(x)
  \]
  \begin{exercicio}
    Seja \(f(x)=x^2-4\) e \(g(x)=(-x^2-2x)\) calcule a área sob as curvas entre os limites [-3,0]

    \begin{figure}[H]
  \centering
      \begin{tikzpicture}
          \begin{axis}[
            xmin=-3.5, xmax=0.5,
            ymin=0, ymax=5.5,
            axis x line=middle,
            axis y line=middle,
            axis line style={-{Stealth[length=3mm,width=1.5mm]}},
            xlabel={$x$},
            ylabel={$y$},
                ]
            \addplot[red, domain=-3:-2, samples=200, thick] {x^2-4};
            \addplot[blue, domain=-2.5:0, samples=200, thick] {-x^2-2*x};
            \draw[azulmarinho, dashed] (-3,0) -- (-3,5);
            \addplot[domain=-3:-2, samples=200, draw=none, fill=gray!20, opacity=.35] {x^2-4} \closedcycle;
            \addplot[domain=-2:0, samples=200, draw=none, fill=gray!20, opacity=.35] {-x^2-2*x} \closedcycle;
          \end{axis}
      \end{tikzpicture}
  \caption{Função $f(x)=x^2-4$ e $g(x)=-x^2-2x$.}
\end{figure}
\begin{align*}
    &\text{Primeira Integral}\\
    &\int_{-3}^{-2}(x^2-4)\,dx\\
    &=\Bigg[\frac{x^3}{3}-4x\Bigg]_{-3}^{-2}\\
    &= \Bigg[\frac{(-2)^3}{3}-(4\cdot(-2))\Bigg]-\Bigg[\frac{(-3)^3}{3}-(4\cdot(-3))\Bigg]\\
    &-\frac{8}{3}-(-8)-(-\frac{27}{3}-(-12))\\
    &-\frac{8}{3}+\frac{24}{3}-\Big(-\frac{27}{3}+\frac{36}{3}\Big)\\
    &\frac{16}{3}-\frac{9}{3}=\frac{7}{3}\\\\
    &\text{Segunda Integral}\\
    &\int_{-2}^{0}(-x^2-2x)\,dx\\
    &\Bigg[\Big(-\frac{x^3}{3}\Big)-\Big(x^2\Big)\Bigg]_{-2}^0\\
    &0-\Bigg[\Big(-\frac{-2^3}{3}\Big)-\Big(-2^2\Big)\Bigg]\\
    &0-\Big(\frac{8}{3}-4\Big)=0-\Big(-\frac{4}{3}\Big)=\frac{4}{3}\\\\
    &\text{Área total}\\
    &\frac{7}{3}+\frac{4}{3}=\boxed{\frac{11}{3}}\,\text{u.a.}
\end{align*}
  \end{exercicio}



\section{Área entre curvas}
%!TEX root = C:/LaTex_Projects/Calc_2/Caderno_de_Calc_II.tex
%!TEX program = lualatex
% LTeX: language=pt-BR

Nesta seção, "área entre curvas" significa \textbf{área geométrica} (sempre não-negativa), calculada por integrais definidas. Em todos os casos, a ideia central é somar a área de tiras (verticais ou horizontais) que preenchem a região.
O cálculo de áreas utilizando integrais baseia-se no princípio fundamental de somar infinitos retângulos de altura infinitesimal. A altura de cada retângulo é definida pela diferença entre o "teto" (limite superior) e o "chão" (limite inferior) da região. Abaixo, detalhamos os quatro cenários principais.
\newpage
\begin{example}
  Segue um exemplo gráfico de área entre curvas:
  \begin{figure}[H]
  \centering
    \includegraphics[width=0.8\linewidth]{area_entre_y=x+4_e_y=raiz_de_0.5x.png}
    \caption{Gráfico de área entre curvas}
    \label{graf_area_entre_curvas_ex}
\end{figure}
\end{example}

\subsection{Área sob uma Curva (Acima do Eixo X)}
Conforme visto na seção\, \ref{subsec:integral-riemann} sobre integral de Riemann, quando uma função $f(x)$ é contínua e positiva em um intervalo $[a,b]$, a área é calculada entre a curva e o eixo $x$. Neste caso, o eixo $x$ atua como o limite inferior ($y = 0$).

\begin{equation}
A = \int_{a}^{b} f(x) \, dx
\end{equation}

\begin{figure}[hbt]
\centering
\begin{tikzpicture}
\begin{axis}[
  axis x line=middle,
  axis y line=middle,
  axis line style={-{Stealth[length=3mm,width=2mm]}},
  xlabel={$x$},
  ylabel={$y$},
  xmin=0, xmax=4,
  ymin=0, ymax=5,
  title={Caso 1: $f(x) \ge 0$ em $[a,b]$},
]
  \def\a{0.5} \def\b{3.5}
  \addplot[name path=curva, thick, blue, domain=\a:\b, samples=200] {0.5*(x-2)^2+1};
  \addplot[name path=eixo, draw=none, domain=\a:\b] {0};
  \addplot[fill=gray!20, draw=none] fill between[of=curva and eixo];
  \draw[line width=0.6pt] (axis cs:\a,0) -- (axis cs:\a,{0.5*(\a-2)^2+1});
  \draw[line width=0.6pt] (axis cs:\b,0) -- (axis cs:\b,{0.5*(\b-2)^2+1});
  \node[below] at (axis cs:\a,0) {$a$};
  \node[below] at (axis cs:\b,0) {$b$};
\end{axis}
\end{tikzpicture}
\caption{Se $f(x)\ge 0$, então $A=\int_a^b f(x)\,dx$.}
\end{figure}

\subsection{Área entre Duas Curvas (integralmente no primeiro quadraente)}
Quando temos duas funções $f(x)$ e $g(x)$ em um intervalo $[a, b]$, onde $f(x) \geq g(x)$ para todo $x$, a área da região delimitada por elas é a integral da diferença entre a função superior e a inferior.
\begin{equation}
A = \int_{a}^{b} [f(x) - g(x)] \, dx
\end{equation}

\begin{figure}[H]
  \centering
  \begin{tikzpicture}
    \begin{axis}[
      axis x line=middle,
      axis y line=middle,
      axis line style={-{Stealth[length=3mm,width=1.5mm]}},
      xlabel={$x$},
      ylabel={$y$},
      xmin=0, xmax=2,
      ymin=0, ymax=6,
      title={$\sin(x)$ e $e^x$ em $[0,\pi/2]$},
      xtick={0, 1.57079}, xticklabels={$0$, $\frac{\pi}{2}$},
    ]
      \addplot[name path=A, thick, blue, domain=0:pi/2, samples=200] {sin(deg(x))};
      \addplot[name path=B, thick, red,  domain=0:pi/2, samples=200] {exp(x)};
      \addplot[fill=gray!20, draw=none] fill between[of=B and A];
      \node[anchor=west] at (axis cs:0.9,3.2) {$e^x$};
      \node[anchor=west] at (axis cs:1.35,1.0) {$\sin(x)$};
    \end{axis}
  \end{tikzpicture}
  \caption{Área entre $e^x$ e $\sin(x)$ no intervalo $[0,\pi/2]$.}
\end{figure}

Se \(f(x)=e^x\) e \(g(x)=\sin(x)\) em \([0,\frac{\pi}{2}]\), então a área entre
essas duas curvas é dada por:
\begin{align*}
A
&=\int_0^{\frac{\pi}{2}}\bigl(e^x-\sin(x)\bigr)\,dx\\
&=\int_0^{\frac{\pi}{2}}e^x\,dx-\int_0^{\frac{\pi}{2}}\sin(x)\,dx.
\end{align*}

Primeira integral:
\begin{align*}
\int_0^{\frac{\pi}{2}}e^x\,dx
&=\Big[e^x\Big]_0^{\frac{\pi}{2}}
=e^{\frac{\pi}{2}}-1.
\end{align*}

Segunda integral:
\begin{align*}
-\int_0^{\frac{\pi}{2}}\sin(x)\,dx
&=-\Big[-\cos(x)\Big]_0^{\frac{\pi}{2}}
=\Big[\cos(x)\Big]_0^{\frac{\pi}{2}}
=\cos\Big(\frac{\pi}{2}\Big)-\cos(0)\\
&=0-1=-1.
\end{align*}

Assim, a área entre as curvas é:
\begin{align*}
A
&=\bigl(e^{\frac{\pi}{2}}-1\bigr)+(-1)
=e^{\frac{\pi}{2}}-2
\approx 2{,}81\ \text{u.a.}
\end{align*}

\begin{exercicio}\label{Ex_curvas_cruzadas}
  sejam \(f(x)=4x-x^2\) e \(g(x)=x^2\), calcular a área entre as curvas.
  Embora não seja obrigatório, vou fazer o gráfico para ter uma noção maior do problema.

  \begin{figure}
    \centering
    \tikzsetnextfilename{area_entre_curvas_1}
    \begin{tikzpicture}
      \begin{axis}
        [
          xmin=0, xmax=5,
          ymin=0, ymax=10,
          axis x line=middle,
          axis y line=middle,
          xlabel=\(x\),
          ylabel=\(y\),
          grid=both,
          tick align=outside,
          minor tick num=1,
          axis line style={-{Stealth[length=3mm,width=1.5mm]}},
          title={Funções \(y=4x-x^2\) e \(y=x^2\)},
        ]
          \addplot[red, domain=0:3, samples=200, thick, name path=A] {4*x-x^2}; 
          \addplot[azulmarinho, domain=0:3, samples=200, thick, name path=B] {x^2};
          \node at (axis cs:3.5,3.5) {(3,3)};
          \node at (axis cs:2.5,9) {(3,9)};
          \draw[red, dashed] (3,3) -- (3,9);  
          %0..2: A acima de B
          \addplot[fill=gray!20, draw=none]
          fill between[of=A and B, soft clip={domain=0:2}];

          % 2..3: B acima de A
          \addplot[fill=gray!20, draw=none]
          fill between[of=B and A, soft clip={domain=2:3}];
      \end{axis}
    \end{tikzpicture}
    \caption{Área entre as curvas que se cruzam}
  \end{figure}
  Pelo gráfico, a interseção entre as curvas parece ser em (0,0 e)(2,4) mas isso precisa ser demontrado algebricamente. As interseções são no pontos onde as funções são iguais.

  Então:
  \begin{align*}
    &f(x)=g(x)\\
    &4x-x^2=x^2\\
    &-2x^2+4x=0\\
    &x(-2x+4)=0\\
    &x'=0\\
    &2x=4\,\therefore \, x"=2\\
  \end{align*}

  conforme mostrado no gráfico, podemos dizer que os limites da primeira integral é dado por: [0,2] e para a segunda integral é [2,2,5]

  Primeira Integral
  \begin{align*}
    &\int_{0}^{2} (4x-x^2)\,dx-\int_{0}^{2}x^2\,dx\\
    &=\Bigg[2x^2-\frac{x^3}{3}\Bigg]_0^2-\Bigg[\frac{x^3}{3}\Bigg]_0^2\\
    &=\Bigg[2(2^2)-\frac{2^3}{3}\Bigg]-0-\Bigg[\frac{2^3}{3}\Bigg]-0\\
    &=\Bigg[8-\frac{8}{3}\Bigg]-\Bigg[\frac{8}{3}\Bigg]=\frac{24}{3}-\frac{8}{3}=\boxed{\frac{16}{3}}\\
  \end{align*}

  Segunda Integral
  \begin{align*}
    & \int_{2}^{3}x^2\,dx-\int_{2}^{3}(4x-x^2)\,dx\\
    &=\Bigg[\frac{x^3}{3}\Bigg]_2^{3}-\Bigg[2x^2-\frac{x^3}{3}\Bigg]_2^{3}\\
    & =\Bigg[\frac{(3)^3}{3}-\frac{2^3}{3}\Bigg]-
    \Bigg[\Big[2({3}^2)-\frac{3^3}{3}\Big]-\Big[2({2}^2)-\frac{2^3}{3}\Big]\Bigg]\\
    &=\frac{19}{3}-\Big[18-\frac{27}{3}-\Big(8-\frac{8}{3}\Big)\Big]\\
    &=\frac{19}{3}-\Big[\frac{54}{3}-\frac{27}{3}-\frac{16}{3}\Big]\\
    &=\frac{19}{3}-\frac{11}{3}=\frac{8}{3}\\
  \end{align*}
  
  então a área total é,

  \begin{align*}
    \frac{16}{3}+\frac{8}{3}=\frac{24}{3}=\boxed{8\text{u.a.}}
  \end{align*}
\end{exercicio}

\subsection{Tipo 1: Uma função sempre acima da outra (posição relativa fixa)}
Considere duas funções contínuas \(f\) e \(g\) em \([a,b]\) tais que, para todo \(x\in[a,b]\), vale
\[
f(x)\ge g(x).
\]
Nesse caso, a região entre as curvas \(y=f(x)\) (curva superior) e \(y=g(x)\) (curva inferior), limitada pelas retas \(x=a\) e \(x=b\), tem área dada por
\[
A=\int_a^b \big(f(x)-g(x)\big)\,dx.
\]
\textbf{Exemplo (sem números):} área limitada superiormente por \(y=f(x)\), inferiormente por \(y=g(x)\) e lateralmente por \(x=a\) e \(x=b\), sem que as curvas se cruzem no intervalo.

\medskip
\textbf{Observação importante.} Não importa se \(f\) e \(g\) estão acima ou abaixo do eixo \(x\). O que importa é somente qual delas está \emph{acima da outra}. Mesmo que \(f(x)\) e \(g(x)\) sejam negativos, a área continua sendo \(\int ( \text{superior} - \text{inferior})\,dx\).

\subsection{Tipo 2: As curvas se entrecruzam (troca de quem está por cima)}
Suponha que \(f\) e \(g\) sejam contínuas em \([a,b]\) e que existam pontos de interseção:\\
 \(c_1,\, c\,_2, \dots,c_n\) no intervalo, isto é,
\[
f(c_k)=g(c_k)\quad (k=1,2,\dots,n),
\]
de modo que a posição relativa (quem está acima) pode mudar de um subintervalo para outro. Nesse caso, a área geométrica deve ser calculada \textbf{particionando o intervalo} nos pontos de interseção e somando as contribuições:
\[
A=\sum_{k=0}^{n}\int_{c_k}^{c_{k+1}} \big(\text{curva superior} - \text{curva inferior}\big)\,dx,
\]
onde \(c_0=a\) e \(c_{n+1}=b\), e em cada subintervalo \([c_k,c_{k+1}]\) escolhe-se corretamente qual função é a superior.

Uma forma compacta (quando só há duas curvas) é
\[
A=\int_a^b \big|f(x)-g(x)\big|\,dx,
\]
mas, na prática, a forma particionada é a mais segura, pois explicita a troca de posição.

\textbf{Exemplo (sem números):} duas curvas que se cruzam uma vez dentro de \([a,b]\), sendo \(f\) superior em \([a,c]\) e \(g\) superior em \([c,b]\). A área é a soma de duas integrais, uma em cada trecho.

\subsection{Tipo 3: Área em relação ao eixo \texorpdfstring{\(x\)}{x} (curva(s) acima/abaixo do eixo)}

Quando o problema envolve o eixo \(x\) como fronteira (por exemplo, "área entre a curva e o eixo \(x\)"), a mudança de sinal passa a ser essencial. Para uma função contínua \(f\) em \([a,b]\):

\subsubsection*{(a) Curva toda acima do eixo}
Se \(f(x)\ge 0\) em \([a,b]\), então a área entre \(y=f(x)\) e o eixo \(x\) é
\[
A=\int_a^b f(x)\,dx.
\]
\textbf{Exemplo (sem números):} região limitada por \(y=f(x)\), pelo eixo \(x\) e pelas retas \(x=a\) e \(x=b\), com \(f\) sempre positiva.

\subsubsection*{(b) Curva toda abaixo do eixo}
Se \(f(x)\le 0\) em \([a,b]\), a área geométrica é
\[
A=\int_a^b |f(x)|\,dx \;=\; -\int_a^b f(x)\,dx.
\]
\textbf{Exemplo (sem números):} região limitada por \(y=f(x)\), pelo eixo \(x\) e por \(x=a\), \(x=b\), com \(f\) sempre negativa.

\subsubsection*{(c) Curva cruza o eixo (muda de sinal)}
Se existem pontos \(r_1,r_2,\dots,r_m\) em \([a,b]\) tais que \(f(r_j)=0\), então deve-se \textbf{particionar} o intervalo nesses zeros e somar as áreas positivas:
\[
A=\sum_{j=0}^{m}\int_{r_j}^{r_{j+1}} |f(x)|\,dx,
\]
com \(r_0=a\) e \(r_{m+1}=b\).
\textbf{Exemplo (sem números):} uma curva que cruza o eixo \(x\) uma vez: calcula-se a integral em dois trechos e somam-se os módulos.

\subsection{Resumo operacional}
\begin{itemize}
  \item \textbf{Entre duas curvas:} em cada trecho, use \(\text{superior} - \text{inferior}\). Se cruzarem, \textbf{parta} nos pontos de interseção.
  \item \textbf{Com o eixo \(x\):} use \(|f(x)|\) ou parta nos zeros de \(f\) e some áreas positivas.
  \item O eixo \(x\) \textbf{só importa} quando ele é fronteira do problema. \newline Caso contrário, "acima/abaixo do eixo" não altera a fórmula: o critério é sempre \textbf{quem está acima de quem}.
\end{itemize}


\subsection{Curvas que se Cruzam}
Se as funções $f(x)$ e $g(x)$ se cruzam em um ponto $c$, a posição relativa de "teto" e "chão" se inverte. Para calcular a área total, devemos dividir a integral nos pontos de interseção.
\begin{equation}
A = \int_{a}^{c} [f(x) - g(x)] \, dx + \int_{c}^{b} [g(x) - f(x)] \, dx
\end{equation}

\begin{figure}[hbt]
\centering
\begin{tikzpicture}
\begin{axis}[
    xmin=-1, xmax=5,
    ymin=0.5, ymax=5.5,
    axis x line=middle,
    axis y line=middle,
    samples=200,
    xlabel={$x$},
    ylabel={$y$},
    axis line style={-{Stealth[length=3mm,width=1.5mm]}},
    axis on top
]
    \addplot[blue, thick, domain=-1:4.5, name path=curva] {(x-2)^2+1};
    \addplot[red,  thick, domain=-1:4.5, name path=reta ] {x+1};

    \addplot [
        pattern=north east lines,
        pattern color=gray,
    ] fill between [
        of=curva and reta,
        split,
        every segment no 0/.style={transparent},
        every segment no 1/.style={pattern=north east lines},
        every segment no 2/.style={transparent}
    ];
\end{axis}
\end{tikzpicture}
\caption{Área entre curvas que se cruzam.}
\label{Fig_funcoes_cruzadas}
\end{figure}

\begin{exercicio}
  
No caso da área da Figura \ref{Fig_funcoes_cruzadas}, as funções \(f(x)\) e \(g(x)\) são:
\begin{align*}
f(x)&=(x-2)^2+1,\\
g(x)&=x+1.
\end{align*}

Nesse caso, diferentemente do que já foi feito no Exercício \ref{Ex_curvas_cruzadas}o cálculo será feito entre os pontos de interseção das funções, sem incluir parte entre 0 e o primeiro cruzamento das funções. (Esse exercício pode ser revisado porsteriormente.)

\begin{align*}
      f(x)&=g(x)\\
      (x-2)^2+1&=x+1\\
      (x-2)^2&=x\\
      x^2-4x+4&=x\\
      x^2-5x+4&=0\\
      x^2-5x&=-4\\
      x^2-5x+\frac{25}{4}&=-4+\frac{25}{4}\\
      \left(x-\frac{5}{2}\right)^2&=\frac{9}{4}\\
      x-\frac{5}{2}&=\pm\frac{3}{2}\\
      x&=\frac{5}{2}\pm\frac{3}{2}
    \ \Rightarrow\
      \begin{cases}
      x_1=1\\
      x_2=4
      \end{cases}
\quad \text{\LARGE$\checkmark$}
\end{align*}

\begin{figure}[H]
  \centering
  \begin{tikzpicture}
    \begin{axis}[
      xmin=0, xmax=6,
      ymin=0, ymax=6,
      axis x line=middle,
      axis y line=middle,
      axis line style={-{Stealth[length=3mm,width=2mm]}},
      tick align=outside,
      grid=none,
      minor tick num=1,
      xlabel={$x$},
      ylabel={$y$},
      xtick=\empty,
      ytick=\empty,
    ]
    \draw[red,  thick] (axis cs:1,1) -- (axis cs:1,4);
    \draw[red,  thick] (axis cs:1,4) -- (axis cs:4,4);
    \draw[red,  thick] (axis cs:4,4) -- (axis cs:4,1);
    \draw[red,  thick] (axis cs:4,1) -- (axis cs:1,1);

    \draw[blue, thick] (axis cs:1,4) -- (axis cs:1,5);
    \draw[blue, thick] (axis cs:1,5) -- (axis cs:4,5);
    \draw[blue, thick] (axis cs:4,5) -- (axis cs:4,4);

    \draw[black, thick] (axis cs:4,1) -- (axis cs:5,1);
    \draw[black, thick] (axis cs:5,1) -- (axis cs:5,4);
    \draw[black, thick] (axis cs:5,4) -- (axis cs:4,4);

    \draw[black, dashed, thick] (axis cs:5,4) -- (axis cs:5,5);
    \draw[black, dashed, thick] (axis cs:5,5) -- (axis cs:4,5);

    \fill[pattern=north east lines]
      (axis cs:1,1) -- (axis cs:1,4) -- (axis cs:4,4) -- (axis cs:4,1) -- cycle;

    \node at (axis cs:2.6,2.6) {\Large{$x^2$}};
    \node at (axis cs:2.6,5.2) {\Large{$x$}};
    \node at (axis cs:5.2,2.6) {\Large{$x$}};
    \node at (axis cs:4.5,0.5) {\Large{$\frac{5}{2}$}};
    \node at (axis cs:0.5,4.5) {\Large{$\frac{5}{2}$}};
    \end{axis}
  \end{tikzpicture}
  \caption{Completar quadrados.}
\end{figure}

Assim, fica definido que a área entre os pontos de interseção das curvas é calculada por:
\[
  \int_{1}^{4}(x+1)\,dx-\int_{1}^{4}\bigl((x-2)^2+1\bigr)\,dx
\]

\begin{align*}
  &\int_{1}^{4}(x+1)\,dx-\int_{1}^{4}\bigl((x-2)^2+1\bigr)\,dx\\
  &=\int_{1}^{4}x\,dx+\int_{1}^{4}1\,dx-\left(\int_{1}^{4}(x-2)^2\,dx+\int_{1}^{4}1\,dx \right)\\
  &=\int_{1}^{4}x\,dx-\int_{1}^{4}(x-2)^2\,dx\\
  &=\left[\frac{x^2}{2}\right]_{1}^{4}-\left[\frac{x^3}{3}-2x^2+4x\right]_{1}^{4}\\
  &=\left(\frac{16}{2}-\frac{1}{2}\right)-\left(\frac{64}{3}-\frac{1}{3}\right)
  +\left(2\cdot16-2\cdot1\right)-(16-4)\\
  &=\frac{15}{2}-21+18=\frac{9}{2}.
\end{align*}

\begin{figure}[H]
\centering
\begin{tikzpicture}
\begin{axis}[
    xmin=0.5, xmax=4,
    ymin=-3,  ymax=.5,
    axis x line=middle,
    axis y line=middle,
    axis line style={-{Stealth[length=3mm,width=2mm]}},
    samples=200,
    xlabel={$x$},
    ylabel={$y$}
]
    \addplot[blue, thick, domain=1:3.75, samples=200, name path=curva] {-(0.5*(x-2)^2+1)};
    \addplot[red,  thick, domain=1:3.75, samples=200, name path=reta ] {-(.5*x+0.5)};

    \addplot[
        pattern=north east lines,
        pattern color=gray,
    ] fill between[
        of=curva and reta,
        split,
        every segment no 0/.style={transparent},
        every segment no 1/.style={pattern=north east lines},
        every segment no 2/.style={transparent}
    ];
\end{axis}
\end{tikzpicture}
\caption{Área hachurada entre as intersecções.}
\end{figure}

No segundo caso, as funções são:
\begin{align*}
f(x)&=-\bigl(0{,}5(x-2)^2+1\bigr)=-0{,}5x^2+2x-3,\\
g(x)&=-\bigl(0{,}5x+0{,}5\bigr).
\end{align*}

Cálculo dos pontos de interseção:
\begin{align*}
f(x)&=g(x)\\
-0{,}5x^2+2x-3&=-(0{,}5x+0{,}5)\\
-0{,}5x^2+2{,}5x&=2{,}5\\
0{,}5x^2-2{,}5x&=-2{,}5\\
x^2-5x&=-5\\
x^2-5x+\frac{25}{4}&=-5+\frac{25}{4}\\
\left(x-\frac{5}{2}\right)^2&=\frac{5}{4}\\
x&=\frac{5}{2}\pm\sqrt{\frac{5}{4}}=\frac{5\pm\sqrt5}{2}.
\end{align*}

Assim, a área é calculada por:
\begin{align*}
A
&=\int_{x_1}^{x_2}\bigl(f(x)-g(x)\bigr)\,dx\\
&=\int_{\frac{5-\sqrt5}{2}}^{\frac{5+\sqrt5}{2}}
\left[\left(-\frac12x^2+2x-3\right)-\left(-\frac12x-\frac12\right)\right]dx\\
&=\int_{\frac{5-\sqrt5}{2}}^{\frac{5+\sqrt5}{2}}
\left(-\frac12x^2+\frac52x-\frac52\right)dx\\
&=\left[-\frac{x^3}{6}+\frac{5x^2}{4}-\frac{5x}{2}\right]_{\frac{5-\sqrt5}{2}}^{\frac{5+\sqrt5}{2}}\\
&=\frac{5\sqrt5}{12}\approx 0{,}932.
\end{align*}

\subsection*{Resumo Prático}
A fórmula será sempre:
\[
  \int_{\text{esquerda}}^{\text{direita}} (\text{Teto} - \text{Chão}) \,dx
\]

\end{exercicio}

\begin{exercicio}
Calcular a área entre a curva e o eixo \(x\) para o intervalo \([2,5]\).
\[
f(x)=4x-x^2
\]

\begin{figure}[H]
  \centering
      \begin{tikzpicture}
        \begin{axis}[
          xmin=1.5, xmax=5.5,
          ymin=-6,  ymax=5,
          axis x line=middle,
          axis y line=middle,
          axis line style={-{Stealth[length=3mm,width=1.5mm]}},
          xlabel={$x$},
          ylabel={$y$},
          title={$f(x)=4x-x^2$},
          samples=200,
        ]
          \addplot[red, domain=2:5, thick, samples=200] {4*x-x^2};

          % Sombreado acima do eixo (2 a 4)
          \addplot[fill=gray!20, opacity=0.5, domain=2:4] {4*x-x^2} \closedcycle;

          % Sombreado abaixo do eixo (4 a 5)
          \addplot[fill=gray!40, opacity=0.5, domain=4:5] {4*x-x^2} \closedcycle;

          \draw[dashed] (axis cs:2,0) -- (axis cs:2,4);
          \draw[dashed] (axis cs:5,0) -- (axis cs:5,-5);
        \end{axis}
  \end{tikzpicture}
  \caption{Área entre $f(x)$ e o eixo $x$ no intervalo $[2,5]$.}
  \label{area1}
\end{figure}

Como parte da curva fica abaixo do eixo \(x\), separamos a área em dois trechos:
\([2,4]\) (acima do eixo) e \([4,5]\) (abaixo do eixo). Assim:
\[
A=\int_{2}^{4}(4x-x^2)\,dx-\int_{4}^{5}(4x-x^2)\,dx.
\]

Primeira integral:
\begin{align*}
\int_{2}^{4}(4x-x^2)\,dx
&=\Big[2x^2-\frac{x^3}{3}\Big]_{2}^{4}
=\left(32-\frac{64}{3}\right)-\left(8-\frac{8}{3}\right)
=\frac{16}{3}.
\end{align*}

Segunda integral (com sinal de área):
\begin{align*}
-\int_{4}^{5}(4x-x^2)\,dx
&=-\Big[2x^2-\frac{x^3}{3}\Big]_{4}^{5}
=-\left(\left(50-\frac{125}{3}\right)-\left(32-\frac{64}{3}\right)\right)\\
&=-\left(\frac{25}{3}-\frac{32}{3}\right)
=\frac{7}{3}.
\end{align*}

Logo, a área total sob a curva no intervalo \([2,5]\) é:
\[
A=\frac{16}{3}+\frac{7}{3}=\boxed{\frac{23}{3}\,\text{u.a.}}
\]
\end{exercicio}


\begin{exercicio}
  Calcule a área entre as curvas:
  \[
    \begin{cases*}
    g(x)=-x^2-2x\\% em azul
    f(x)=x^2-4 \\% em vermelho
        \end{cases*}
  \]    
As interseções são calculadas igualando-se a duas funções:
\begin{align*}
  &x^2-4=-x^2-2x\\
  &2x^2+2x-4=0\\
  &\underbrace{x^2+\underbrace{x}_{\frac{1}{2}\rightarrow {(\frac{1}{2})^2}}+\frac{1}{4}=2+\frac{1}{4}}_{\text{completando quadrados}}\\
  &x^2+x+\frac{1}{4}=\frac{9}{4}\\
  &\Big(x+\frac{1}{2}\Big)^2=\frac{9}{4}\\
  &\sqrt{\Big(x+\frac{1}{2}\Big)^2}=\pm \sqrt{\frac{9}{4}}\\
  &x=\pm\frac{3}{2}-\frac{1}{2}\\
  & \begin{cases*}
        x'=\frac{3}{2}-\frac{1}{2}=1\\
        x''=-\frac{3}{2}-\frac{1}{2}=-2\\
    \end{cases*}
  \end{align*}

  \begin{figure}[H]
  \centering
    \begin{tikzpicture}
      \begin{axis}
        [xmin=-3, xmax=1.5,
         ymin=-5, ymax=2,
         axis x line=middle,
         axis y line=middle,
         tick align=outside,
         minor tick num=1,
         grid=none,
        axis line style={-{Stealth[length=3mm,width=1.5mm]}},
         ]
         \addplot[red, thick, domain=-2.5:1.5, samples=200, name path=vermelho]{x^2-4};
         \addplot[blue, thick, domain=-2.5:1.5, samples=200, name path=azul]{-x^2-2*x};

          \addplot[fill=gray!20, opacity=0.5, draw=none]
          fill between[of=vermelho and azul, soft clip={domain=-2:1}];
      
      \end{axis}
    \end{tikzpicture}
\end{figure}

Assim, como não há entrecruzamento de curvas, podemos integrar direto tendo como limites de integração os pontos de interseção.

\begin{align*}
  &\int_{-2}^{1}(g(x)-f(x))\, dx\\
  &=\int_{-2}^{1}(-x^2-2x)-(x^2-4)\, dx\\
  &=\int_{-2}^{1}(-2x^2-2x+4)\, dx\\
  &=\Big[-2\frac{x^3}{3}-x^2+4x\Big]_{-2}^1\\
  &=\Big(-\frac{2}{3}-1+4\Big)-\Big(\frac{16}{3}-4-8\Big)\\
  &=\Big(-\frac{2}{3}-\frac{3}{3}+\frac{12}{3}\Big)-\Big(\frac{16}{3}-\frac{12}{3}-\frac{24}{3}\Big)\\
  &=\Big(\frac{7}{3}\Big)-\Big(\frac{20}{3}\Big)=\frac{27}{3}=\boxed{9\,\text{u.a.}}\\
\end{align*}
\end{exercicio}







\section{Volume de Sólido de Revolução}

\begin{example}
        Seja o gráfico da curva e área \ref{graf:area_para_volume} se rotacionada em torno do eixo de x forma um sólido de revolução, que tem como volume:
        \[
            S=\int_{a}^{b}\pi (f(x))^2\,dx
        \]
    \begin{figure}[ht]
  \centering
  \begin{tikzpicture}
    \begin{axis}[
      xmin=0, xmax=3,
      ymin=-0.5, ymax=2.5,
      axis x line=middle,
      axis y line=middle,
      tick align=outside,
      xtick=\empty,
      minor tick num=1,
      grid=both,
      axis line style={-{Stealth[length=3mm,width=1.5mm]}}
      ]
      % curva
      \addplot[red, thick, domain=0.5:2.5, samples=200]{0.5*(x-2)^2+1};
      \node at (0.5,-0.25) {a};  
      \node at (2.5,-0.25) {b};  
      \draw[blue, thick, dashed] (axis cs:0.5,0) -- (axis cs:0.5,2.1);
      \draw[blue, thick, dashed] (axis cs:2.5,0) -- (axis cs:2.5,1.1);
    \end{axis}
  \end{tikzpicture}
  \caption{Curva e área para volume}
  \label{graf:area_para_volume}
\end{figure}

Se for em torno do eixo de y temos:

\begin{figure}[ht]
  \centering
    \includegraphics[width=\linewidth]{Sólido_de_Revolução.png}
    \caption{Sólido gerado a partir da geratriz da figura \ref{graf:area_para_volume}}
\end{figure}

\begin{figure}[H]
  \centering
  \begin{tikzpicture}
    \begin{axis}[
      xmin=-1, xmax=4,
      ymin=-1, ymax=3,
      axis x line=middle,
      axis y line=middle,
      axis line style={-},
      tick align=outside,
      grid=both,
      minor tick num=1,
      xlabel={$x$},
      ylabel={$y$},
      title={Superfície Geratriz}
    ]
      % Retângulo (fechando o caminho voltando ao primeiro ponto)
      \addplot[
        thick
      ] coordinates {
        (0.5,0.5)
        (3,0.5)
        (3,2)
        (0.5,2)
        (0.5,0.5)
      };

      % Preenchimento opcional
      \addplot[
        fill=blue!15,
        draw=none
      ] coordinates {
        (0.5,0.5)
        (3,0.5)
        (3,2)
        (0.5,2)
        (0.5,0.5)
      };

      % Pontos marcados
      \addplot[only marks, mark=*] coordinates {
        (0.5,0.5)
        (3,0.5)
        (3,2)
        (0.5,2)
        (0.5,0.5)
      };
    \end{axis}
  \end{tikzpicture}
  \caption{Superfície geratriz de um volume em torno de y}
  \label{fig:geratriz_da_superfície_y}
\end{figure}

quando a superfície é girada em torno do eixo y \ref{fig:geratriz_da_superfície_y} temos.

\begin{figure}[H]
  \centering
    \includegraphics[width=0.8\linewidth]{Cilindro_3d.png}
    \caption{Cilindro em torno de y}
\end{figure}

\end{example}
  
\section{Funções no espaço tridimensional}

São funções do tipo: $f: \mathbb{R}^3 \longrightarrow \mathbb{R}$.

Superfícies de nível são um conjunto de pontos no espaço tridimensional onde uma função de três variáveis atinge um valor constante. Elas são obtidas ao igualar a função a uma constante, como $f(x,y,z) = k$. Essa visualização tridimensional é análoga às curvas de nível usadas em mapas topográficos para representar a altitude.

\medskip
\begin{figure}[ht]
  \centering
  \begin{tikzpicture}
    \begin{axis}[
      view={60}{30},
      axis lines=middle,
      xlabel={$x$}, ylabel={$y$}, zlabel={$z$},
      domain=0:360,          % theta em graus
      y domain=0:2.2,        % r (raio)
      samples=50,            % reduzido levemente para compilação mais rápida
      samples y=20,
      z buffer=sort,
      grid=major,
      axis equal,
    ]

      % 1) "Sino" (paraboloide) com domínio circular
      \addplot3[
        surf,
        shader=interp,
        opacity=0.75,
        draw=none
      ]
      (
        {y*cos(x)},           % x = r cos(theta)
        {y*sin(x)},           % y = r sin(theta)
        {y^2}                 % z = r^2
      );

      % 2) "Boca plana": disco no topo z = R^2
      \addplot3[
        surf,
        shader=interp,
        opacity=0.30,
        draw=none
      ]
      (
        {y*cos(x)},
        {y*sin(x)},
        { (2.2)^2 }
      );

    \end{axis}
  \end{tikzpicture}
  \caption{parabolóide}
\end{figure}

\begin{enumerate}
  \item \textbf{Equação do Plano:}
  \[
    z = ax + by + c
  \]
  \[
    \Big\{ (x,y,z)\in \mathrm{Dom}(f)\ \big|\ f(x,y,z)=k \Big\}
  \]

  \item \textbf{Paraboloide Elíptico.} Para $a, b > 0$:
  \[
    z = ax^2 + by^2
  \]

  \item \textbf{Paraboloide Hiperbólico.}
  \[
    z = ax^2 - by^2
  \]

  \item \textbf{Cilindros.}
  \begin{enumerate}
    \item \textbf{Cilindro Parabólico.} Cada corte $y=c$ é uma parábola.
    \[
      z = x^2
    \]
    \item \textbf{Cilindro Elíptico.}
    \[
      x^2 + y^2 = 1
    \]
  \end{enumerate}

  \item \textbf{Superfícies Quádricas.}
  \begin{enumerate}
    \item \textbf{Elipsoide.}
    \[
      \frac{x^2}{a^2} + \frac{y^2}{b^2} + \frac{z^2}{c^2} = 1
    \]
    \item \textbf{Hiperboloide de uma folha.}
    \[
      \frac{x^2}{a^2} + \frac{y^2}{b^2} - \frac{z^2}{c^2} = 1
    \]
    \item \textbf{Hiperboloide de duas folhas.}
    \[
      -\frac{x^2}{a^2} - \frac{y^2}{b^2} + \frac{z^2}{c^2} = 1
    \]
  \end{enumerate}

  \bigskip
  \begin{table}[H]
    \centering
    \caption{Sinais dos coeficientes das superfícies quádricas usuais}
    \label{tab:quadricas}
    \begin{tabular}{lccc}
      \toprule
      \textbf{Superfície} & \textbf{a} & \textbf{b} & \textbf{c} \\
      \midrule
      Elipsoide             & + & + & + \\
      Hiperboloide 1 folha  & + & + & - \\
      Hiperboloide 2 folhas & - & - & + \\
      \bottomrule
    \end{tabular}
  \end{table}

  \item \textbf{Superfícies Esféricas.}
  A esfera não é uma função de $(x,y)$.

  \item \textbf{Superfícies Compostas.}
  \begin{enumerate}
    \item \textbf{Cone.}
    \[
      z = \sqrt{x^2+y^2}
    \]
    \item \textbf{Superfície Helicoidal.}
    \[
      \begin{cases}
        x = r \cos(\theta) \\
        y = r \sin(\theta) \\
        z = c\,\theta
      \end{cases}
    \]
    onde: $r$ é o raio; $\theta$ é o parâmetro angular (rad); $c$ controla o passo da hélice.
  \end{enumerate}
\end{enumerate}

Para funções de uma variável $y=f(x)$, temos:
\[
  \lim_{x \to x_0} f(x)
\]

\textbf{Exemplo.}
\[
  f(x,y,z) = x^2 + y^2 + z^2
\]
Domínio: $D(f)=\mathbb{R}^3$ e Imagem: $I(f)=[0,\infty)$.

As \textbf{superfícies de nível} são dadas por:
\[
  x^2 + y^2 + z^2 = c
\]
\begin{itemize}
  \item Se $c>0$: esfera de centro $(0,0,0)$ e raio $\sqrt{c}$.
  \item Se $c=0$: reduz-se ao ponto $(0,0,0)$.
  \item Se $c<0$: não existe superfície de nível (conjunto vazio).
\end{itemize}

\textbf{Exemplo.}
Seja a equação $z^2 - 3(x^2 + y^2) = 0$.
Isolando os termos:
\[
  \frac{z^2}{3} = x^2 + y^2
\]
Isso representa um \textbf{cone circular duplo}.

\begin{figure}[H]
  \centering
  \begin{tikzpicture}
    \begin{axis}[
      view={60}{30},
      axis lines=middle,
      xlabel={$x$}, ylabel={$y$}, zlabel={$z$},
      domain=0:360, y domain=0:1.6,
      samples=50, samples y=20,
      z buffer=sort,
      grid=major,
    ]
      % Parametrização: x = r cos(theta), y = r sin(theta), z = ±sqrt(3) r
      \addplot3[surf, opacity=0.85, draw=none]
        ({y*cos(x)}, {y*sin(x)}, {sqrt(3)*y});
      \addplot3[surf, opacity=0.45, draw=none]
        ({y*cos(x)}, {y*sin(x)}, {-sqrt(3)*y});
    \end{axis}
  \end{tikzpicture}
  \caption{Cone circular duplo: $z^2 - 3(x^2+y^2)=0$.}
\end{figure}

Se $c=1$ na equação $\frac{z^2}{3} - x^2 - y^2 = c$, temos um \textbf{hiperboloide de duas folhas}. Se $c=-1$, o gráfico será um \textbf{hiperboloide de uma folha}.



\subsection{Interpretação Geométrica}
\subsection*{Modelo geral}
Para qualquer \(f(x,y)\) e \(x(t),y(t)\), temos:
\[
\boxed{\frac{d}{dt}f\big(x(t),y(t)\big)
= f_x\big(x(t),y(t)\big)\,x'(t) + f_y\big(x(t),y(t)\big)\,y'(t)}.
\]
 
\textbf{Exercício.}
Suponha que o seu peso ($z$) em kg segue a função $z = f(c, n)$, onde $c$ é o número de calorias que você consome diariamente e $n$ é o número de minutos de exercícios físicos diários.

(FALTA COMPLETAR O EXERCÍCIO.)

\bigskip

Páginas referenciadas de A a D.

\[ f(x,y)= \begin{cases}
  \dfrac{x^3 - y^2}{x^2 + y^2}, & \quad (x,y) \neq (0,0) \\
  0, & \quad (x,y) = (0,0)
\end{cases} \]

\subsubsection*{Cálculo fora da origem $(x,y) \neq (0,0)$}

Derivada em relação a $x$ pela regra do quociente:
\begin{align*}
& \left( \frac{u}{v} \right)' = \frac{u'v - uv'}{v^2} \\
& u = x^3 - y^2 \quad \Rightarrow \quad u_x = 3x^2 \\
& v = x^2 + y^2 \quad \Rightarrow \quad v_x = 2x
\end{align*}
\begin{align*}
  f_x &= \frac{3x^2(x^2 + y^2) - 2x(x^3 - y^2)}{(x^2 + y^2)^2} \\
  f_x &= \frac{3x^4 + 3x^2y^2 - 2x^4 + 2xy^2}{(x^2 + y^2)^2} \\
  f_x &= \frac{x^4 + 3x^2y^2 + 2xy^2}{(x^2 + y^2)^2}
\end{align*}

Derivada em relação a $y$:
\begin{align*}
& u = x^3 - y^2 \quad \Rightarrow \quad u_y = -2y \\
& v = x^2 + y^2 \quad \Rightarrow \quad v_y = 2y
\end{align*}
\begin{align*}
  f_y &= \frac{-2y(x^2 + y^2) - 2y(x^3 - y^2)}{(x^2 + y^2)^2} \\
  f_y &= \frac{-2yx^2 - 2y^3 - 2yx^3 + 2y^3}{(x^2 + y^2)^2} \\
  f_y &= \frac{-2x^2y - 2x^3y}{(x^2 + y^2)^2} = \frac{-2x^2y(1+x)}{(x^2 + y^2)^2}
\end{align*}

\subsubsection*{Cálculo na origem $(0,0)$ pela definição}

Para $f_x(0,0)$:
\begin{align*}
  f_x(0,0) &= \lim_{h \to 0} \frac{f(0+h, 0) - f(0,0)}{h} \\
  &= \lim_{h \to 0} \frac{\frac{h^3 - 0}{h^2 + 0} - 0}{h} = \lim_{h \to 0} \frac{h}{h} = 1
\end{align*}

Para $f_y(0,0)$:
\begin{align*}
  f_y(0,0) &= \lim_{k \to 0} \frac{f(0, 0+k) - f(0,0)}{k} \\
  &= \lim_{k \to 0} \frac{\frac{0 - k^2}{0 + k^2} - 0}{k} = \lim_{k \to 0} \frac{-1}{k}
\end{align*}
Como $k \to 0$, o limite tende a $\pm \infty$. Logo, $\nexists f_y(0,0)$.

\textbf{Nota:} Uma função de várias variáveis pode ter derivadas parciais em um ponto e não ser contínua nesse ponto. Contudo, se uma das derivadas parciais tende ao infinito, a função não é diferenciável naquele ponto.

\subsubsection{Exercício: Inclinação de Superfície}

Determine a inclinação da superfície dada por:
\[ f(x,y) = -\frac{x^2}{2} - y^2 + \frac{25}{8} \]
no ponto $\left( \frac{1}{2}, \frac{1}{2} \right)$.

\begin{enumerate}
  \item \textbf{Na direção do eixo $x$:}
  \[ f_x = -x \quad \Rightarrow \quad f_x\left(\frac{1}{2}, \frac{1}{2}\right) = -\frac{1}{2} \]
  \item \textbf{Na direção do eixo $y$:}
  \[ f_y = -2y \quad \Rightarrow \quad f_y\left(\frac{1}{2}, \frac{1}{2}\right) = -2\left(\frac{1}{2}\right) = -1 \]
\end{enumerate}

\subsubsection{Curvas de Nível}

\begin{figure}[H]
\centering
\begin{tikzpicture}
    \begin{axis}[
        axis equal,
        xmin=-2.5, xmax=2.5,
        ymin=-2.5, ymax=2.5,
        axis lines=middle,
        xlabel={$x$},
        ylabel={$y$},
        grid=both,
        minor tick num=1,
        title={Curvas de nível de $f(x,y)=\sqrt{x^2+y^2}$}
    ]

    % círculo c = 2
    \addplot[domain=0:360, samples=100, very thick, color=blue]
    ({2*cos(x)},{2*sin(x)});
    \node[anchor=west, blue] at (axis cs:2,0.2) {$c=2$};

    % círculo c = 1
    \addplot[domain=0:360, samples=100, thick, color=red]
    ({1*cos(x)},{1*sin(x)});
    \node[anchor=south west, red] at (axis cs:0.7,0.7) {$c=1$};

    % ponto c = 0
    \addplot[mark=*, only marks, mark size=2pt, color=black]
    coordinates {(0,0)};
    \node[anchor=north east] at (axis cs:0,0) {$c=0$};

    \end{axis}
\end{tikzpicture}
\caption{Curvas de nível para $c=0, 1, 2$ (Cones).}
\end{figure}

\subsubsection{Plano Tangente}
Seja a função:
\[ f(x,y) = \begin{cases}
  \frac{x^2 - 4y}{x^2 + y^2}, & \quad (x,y) \neq (0,0) \\
  0, & \quad (x,y) = (0,0)
\end{cases} \]


\subsection{Exercício 5 da Lista 2}

\section {Derivadas Parciais}
\subsection {Funções de duas ou mais variáveis, limite, continuidade}
\section {Curva de nível}
\section {Derivada direcional e gradiente}
\section {Plano tangente}
\section {A Regra da Cadeia}
\section {Diferencial total}
\section {Derivadas de ordem superior}
\section {Máximos e Mínimos; Multiplicadores de Lagrange}
\section{Otimização Condicionada}
\subsection{Multiplicadores de Lagrange}

Maximizar $f(x,y)$ quando $(x,y)$ satisfazem a condição $g(x,y)=0$:
\[
\begin{cases}
\text{Maximizar } f(x,y)\\
\text{s.a. } g(x,y)=0
\end{cases}
\]

\begin{figure}[htbp]
  \centering
  \begin{tikzpicture}
    \begin{axis}[
      width=\linewidth,
      height=0.45\textwidth,
      xmin=0, xmax=3.5,
      ymin=0, ymax=2.5,
      axis x line=bottom,
      axis y line=left,
      axis line style={-},
      tick align=outside,
      grid=both,
      minor tick num=1
    ]
      % C1: 0.7/x
      \addplot[
        domain=0.3:3,
        samples=200
      ] {0.7/x}
        coordinate[pos=1] (Cone);
      \node[anchor=west, xshift=2pt] at (Cone) {$C_1$};

      % C2: 1/x
      \addplot[
        domain=0.3:3,
        samples=200
      ] {1/x}
        coordinate[pos=1] (Ctwo);
      \node[anchor=west, xshift=2pt, yshift=2pt] at (Ctwo) {$C_2$};

      % C3: 1.5/x
      \addplot[
        domain=0.3:3,
        samples=200
      ] {1.5/x}
        coordinate[pos=1] (Cthree);
      \node[anchor=west, xshift=2pt, yshift=2pt] at (Cthree) {$C_3$};

      % C4: 2/x
      \addplot[
        domain=0.3:3,
        samples=200
      ] {2/x}
        coordinate[pos=1] (Cfour);
      \node[anchor=west, xshift=2pt, yshift=2pt] at (Cfour) {$C_4$};

      % curva tangente (exemplo)
      \addplot[domain=0:3, samples=200] {-2*x^2 + 3*x};

    \end{axis}
  \end{tikzpicture}
  \caption{Curvas de nível e tangência (exemplo ilustrativo).}
\end{figure}

\[
\nabla g(P) \perp \text{curva } g=0,
\qquad
\nabla f(P) \perp \text{curva } f=C.
\]

No ponto $P$ que fornece $f_{\max}$ (ou $f_{\min}$) sob a condição $g=0$, as curvas
de nível de $g$ e de $f$ são tangentes. Logo, $\nabla f(P)$ e $\nabla g(P)$ têm a mesma direção,
isto é, são múltiplos.

Assim, existe um número real $\lambda$ tal que
\[
\nabla f(P) = \lambda \, \nabla g(P),
\]
onde $\lambda$ é chamado de \textit{multiplicador de Lagrange}.

Para encontrar máximos e mínimos de $f(x,y)$ sob a restrição $g(x,y)=0$, devemos resolver:
\[
\begin{cases}
\nabla f(x,y) = \lambda \, \nabla g(x,y)\\
g(x,y) = 0
\end{cases}
\]
Isto é:
\[
\begin{cases}
\dfrac{\partial f}{\partial x}(x,y) = \lambda \, \dfrac{\partial g}{\partial x}(x,y) \\[6pt]
\dfrac{\partial f}{\partial y}(x,y) = \lambda \, \dfrac{\partial g}{\partial y}(x,y) \\[6pt]
g(x,y) = 0
\end{cases}
\]

\textbf{Exercício 1.} Determinar, dentre todos os retângulos de 
perímetro $12$, aquele que possui a maior área.

\begin{figure}[htbp]
  \centering
  \begin{tikzpicture}
    \begin{axis}[
      width=\linewidth,
      height=0.45\textwidth,
      xmin=0, xmax=5,
      ymin=0, ymax=4,
      axis x line=bottom,
      axis y line=left,
      axis line style={-},
      tick align=outside,
      grid=both,
      minor tick num=1
    ]
      % retângulo exemplo
      \addplot[thick, solid] coordinates {(0,0) (4,0) (4,3) (0,3) (0,0)};
      \addplot[fill=black!10, draw=none]
        coordinates {(0,0) (4,0) (4,3) (0,3)};
    \end{axis}
  \end{tikzpicture}
  \caption{Retângulo com lados $x$ e $y$ (ilustrativo).}
\end{figure}


\[
\begin{cases}
\text{Maximizar } f(x,y) = xy \\
\text{s.a. } g(x,y) = 2x + 2y - 12 = 0
\end{cases}
\]
\textbf{Nota.} $g(x,y)=0$ representa o perímetro fixado em $12$.

Como $\nabla f = (y,x)$ e $\nabla g = (2,2)$, a condição $\nabla f = \lambda \nabla g$ resulta em:
\[
\begin{cases}
y = 2\lambda \\
x = 2\lambda \\
2x + 2y - 12 = 0
\end{cases}
\Rightarrow x = y.
\]
Substituindo na restrição: $2x + 2x = 12 \Rightarrow 4x = 12 \Rightarrow x = 3$, logo $y = 3$.
O retângulo de área máxima é um quadrado de lado $3$, resultando em $x=3$ e $y=3$.

\textbf{Exercício 2.} Calcule os valores máximo ($\operatorname{Max}$) e mínimo ($\operatorname{Min}$) de
$f(x,y) = x^2 - xy + y^2$ sob a restrição $g(x,y) = x^2 + y^2 - 1 = 0$.

Calculando os gradientes:
\[
\nabla f(x,y) = (2x-y, \, -x+2y),
\qquad
\nabla g(x,y) = (2x, \, 2y).
\]
O sistema $\nabla f = \lambda \nabla g$ fornece:
\[
\begin{cases}
2x - y = 2\lambda x \\
-x + 2y = 2\lambda y \\
x^2 + y^2 = 1
\end{cases}
\]

Da primeira equação: $2x - y = 2\lambda x \Rightarrow -y = 2x(\lambda-1) \Rightarrow \boxed{y = 2x(1-\lambda)}$.
Da segunda equação: $-x + 2y = 2\lambda y \Rightarrow -x = 2y(\lambda-1) \Rightarrow \boxed{x = 2y(1-\lambda)}$.

Dividindo as equações (assumindo $xy \neq 0$):
\[
\frac{y}{x} = 2(1-\lambda) \quad \text{e} \quad \frac{x}{y} = 2(1-\lambda) \Rightarrow \frac{y}{x} = \frac{x}{y} \Rightarrow x^2 = y^2.
\]
Com $x^2 + y^2 = 1$, segue que:
\[
x^2 = y^2 = \frac{1}{2} \Rightarrow x = \pm\frac{\sqrt{2}}{2}, \quad y = \pm\frac{\sqrt{2}}{2}.
\]
Os pontos candidatos são $(\pm\sqrt{2}/2, \pm\sqrt{2}/2)$. Deve-se avaliar $f$ em cada ponto para determinar o máximo e o mínimo.

\medskip
\textbf{Exercício 3.} Achar o máximo ($\operatorname{Max}$) e o mínimo ($\operatorname{Min}$) de
$f(x,y) = 2x + 2y - x^2 - y^2$ sob a restrição $(x-2)^2 + y^2 = 2$.

Derivadas parciais:
\[
\begin{cases}
f_x = 2 - 2x \\
f_y = 2 - 2y
\end{cases}
\qquad
\begin{cases}
g_x = 2(x-2) \\
g_y = 2y
\end{cases}
\]

Sistema de Lagrange:
\[
\begin{cases}
2 - 2x = \lambda \, 2(x-2) \Rightarrow (1-x) = \lambda(x-2) \\
2 - 2y = \lambda \, 2y \Rightarrow 1-y = \lambda y \\
(x-2)^2 + y^2 = 2
\end{cases}
\]

Da segunda equação: $\lambda = \frac{1-y}{y}$ (para $y \neq 0$).
Substituindo na primeira:
\[
(1-x) = \frac{1-y}{y}(x-2) \Rightarrow y(1-x) = (1-y)(x-2).
\]
Expandindo:
\[
y - xy = x - 2 - xy + 2y \Rightarrow y = x - 2 + 2y \Rightarrow y = 2 - x.
\]

Aplicando na restrição:
\[
(x-2)^2 + y^2 = 2 \Rightarrow (x-2)^2 + (2-x)^2 = 2 \Rightarrow 2(x-2)^2 = 2 \Rightarrow (x-2)^2 = 1.
\]
Portanto, $x = 2 \pm 1$, resultando em:
\[
x_1 = 3, \, y_1 = -1 \quad \text{e} \quad x_2 = 1, \, y_2 = 1.
\]

Avaliação da função:
\[
f(1,1) = 2+2-1-1 = 2, \qquad f(3,-1) = 6-2-9-1 = -6.
\]
Logo, $f_{\max} = 2$ em $(1,1)$ e $f_{\min} = -6$ em $(3,-1)$.
\newpage


\begin{figure}[p]
  \centering
  \begin{adjustbox}{max width=\linewidth, max totalheight=0.78\textheight, keepaspectratio, center}
    \begin{tikzpicture}[
      node distance=1.0cm,
      >=Stealth,
      every node/.style={font=\scriptsize},
      block/.style={
        rectangle, draw, rounded corners, align=center,
        minimum width=5.6cm, minimum height=0.70cm
      },
      startstop/.style={
        ellipse, draw, align=center,
        minimum width=2.4cm, minimum height=0.65cm
      }
    ]
      \node[startstop] (start) {Início};

    \node[block, below=of start] (def) {Definir \\[2pt]
      $f(x,y) = 2x + 2y - x^2 - y^2$ \\[2pt]
      $g(x,y) = (x-2)^2 + y^2 - 2 = 0$
    };

    \node[block, below=of def] (grad) {Calcular os gradientes \\[2pt]
      $\nabla f = (2-2x, \; 2-2y)$ \\[2pt]
      $\nabla g = (2(x-2), \; 2y)$
    };

    \node[block, below=of grad] (sist) {Montar o sistema de Lagrange \\[2pt]
      $\nabla f = \lambda \nabla g$ \\[2pt]
      $\Rightarrow
        \begin{cases}
          1 - x = \lambda(x-2)\\
          1 - y = \lambda y\\
          (x-2)^2 + y^2 = 2
        \end{cases}$
    };

    \node[block, below=of sist] (solve) {Resolver o sistema \\[2pt]
      $\Rightarrow y = 2 - x$ e $(x-2)^2 = 1$ \\[2pt]
      Pontos candidatos: $(1,1)$ e $(3,-1)$
    };

    \node[block, below=of solve] (eval) {Avaliar $f$ nos pontos \\[2pt]
      $f(1,1) = 2$, \quad $f(3,-1) = -6$
    };

    \node[block, below=of eval] (class) {Classificar \\[2pt]
      Máximo em $(1,1)$ com $f_{\max}=2$ \\[2pt]
      Mínimo em $(3,-1)$ com $f_{\min}=-6$
    };

    \node[startstop, below=of class] (end) {Fim};

    \draw[->] (start) -- (def);
    \draw[->] (def) -- (grad);
    \draw[->] (grad) -- (sist);
    \draw[->] (sist) -- (solve);
    \draw[->] (solve) -- (eval);
    \draw[->] (eval) -- (class);
    \draw[->] (class) -- (end);

    \end{tikzpicture}
  \end{adjustbox}
  \caption{Fluxograma do método dos multiplicadores de Lagrange.}
\end{figure}



\section {Integrais Múltiplas}
\section {Integral dupla; Área}
\section {Coordenadas polares}
\end{document}
